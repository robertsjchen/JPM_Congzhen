\includepdf[pages={39,40},fitpaper=false]{tst.pdf}
\chapter*{第二十回 傻幫閑趨奉鬧華筵 癡子弟爭鋒毀花院}
\addcontentsline{toc}{chapter}{第二十回 傻幫閑趨奉鬧華筵 癡子弟爭鋒毀花院}
\markboth{{\titlename}卷之二}{第二十回 傻幫閑趨奉鬧華筵 癡子弟爭鋒毀花院}


詞曰:

\begin{myquote} 
步花徑,闌干狹。防人覷,常驚嚇。荊刺抓裙釵,倒閃在荼䕷架。勾引嫩枝咿啞,討歸路,尋空罅,被舊家巢燕,引入窓紗。

\raggedleft{——右調《歸洞仙》\rightquadmargin}
\end{myquote} 

話說西門慶在房中,被李瓶兒柔情軟語,{\meipi{四字銷盡古今多少英雄氣骨。}}感觸的回嗔作喜,拉他起來,穿上衣裳,兩個相摟相抱,極盡綢繆。一面令春梅進房放桌兒,往後邊取酒去。且說金蓮和玉樓,從西門慶進他房中去,站在角門首竅聽訊息。他這邊門又閉着,止春梅一人在院子裡伺候。金蓮同玉樓兩個打門縫兒往裡張覷,只見房中掌着燈燭,裡邊說話,都聽不見。金蓮道:「俺到不如春梅賊小肉兒,他倒聽的伶俐。」那春梅在窓下潛聽了一回,又走過來。金蓮悄問他房中怎的動靜,春梅便隔門告訴與二人說:「俺爹怎的教他脫衣裳跪着,他不脫。爹惱了,抽了他幾馬鞭子。」金蓮道:「打了他,他脫了不曾?」春梅道:「他見爹惱了,纔慌了,就脫了衣裳,跪在地平上。爹如今問他話哩。」玉樓恐怕西門慶聽見,便道:「五姐,咱過那邊去罷。」{\pangpi{寫出玉樓膽小。}}拉金蓮來西角門首。

此時是八月二十頭,月色纔上來。兩個站立在黑頭裡,一處說話,等着春梅出來問他話。潘金蓮向玉樓道:「我的姐姐,只說好食菓子,一心只要來這裡。頭兒沒過動,下馬威早討了這幾下在身上。俺這個好不順臉的貨兒,你若順順兒他倒罷了。屬扭孤兒糖的,你扭扭兒也是錢,不扭也是錢。想着先前吃小婦奴才壓枉造舌,我陪下十二分小心,還吃他奈何得我那等哭哩。姐姐,你來了幾時,還不知他性格哩!」

二人正說話之間,只聽開的角門響,春梅出來,一直逕往後邊走。不防他娘站在黑影處叫他,問道:「小肉兒,那去?」春梅笑着只顧走。{\pangpi{畫。}}金蓮道:「怪小肉兒,你過來,我問你話。慌走怎的?」那春梅方纔立住了脚,方說:「他哭着對俺爹說了許多話。爹喜歡抱起他來,令他穿上衣裳,教我放了桌兒,如今往後邊取酒去。」金蓮聽了,向玉樓說道:「賊沒廉恥的貨!頭裡那等雷聲大雨點小,打哩亂哩。及到其間,也不怎麼的。我猜,也沒的想,管情取了酒來,教他遞。{\pangpi{又從經歷處着想,妙甚。}}賊小肉兒,沒他房裡丫頭?你替他取酒去!到後邊,又叫雪娥那小婦奴才𣭈聲浪顙,{\pangpi{觸起舊恨。}}我又聽不上。」春梅道:「爹使我,管我事!」於是笑嘻嘻去了。金蓮道:「俺這小肉兒,正經使着他,死了一般,懶待動旦。{\pangpi{又為春梅洗髮。}}若干貓兒頭差事,鑽頭覓縫幹辦了要去,去的那快!現他房裡兩個丫頭,你替他走,管你腿事!賣蘿葡的跟着鹽担子走——好個閑嘈心的小肉兒!」玉樓道:「可不怎的!俺大丫頭蘭香,我正使他做活兒,他便有要沒緊的。爹使他,行鬼頭兒,聽人的話兒,你看他走的那快!」正說着,只見玉簫自後邊驀地走來,便道:「三娘還在這裡?我來接你來了。」玉樓道:「怪狗肉,唬我一跳!」{\pangpi{又映膽小。}}因問:「你娘知道你來不曾?」玉簫道:「我打發娘睡下這一日了,我來前邊瞧瞧,剛纔看見春梅後邊要酒菓去了。」因問:「俺爹到他屋裡,怎樣個動靜兒?」金蓮接過來伸着手道:「進他屋裡去,齊頭故事。」{\pangpi{妙語。}}玉簫又問玉樓,玉樓便一一對他說。玉簫道:「三娘,眞個教他脫了衣裳跪着,打了他五馬鞭子來?」玉樓道:「你爹因他不跪,纔打他。」玉簫道:「帶着衣服打來,去了衣裳打來?虧他那瑩白的皮肉兒上,怎麼捱得?」{\meipi{癡丫頭問語,酷肖。}}玉樓笑道:「怪小狗肉兒,你倒替古人耽憂!」正說着,只見春梅拏着酒,小玉拏着方盒,逕往李瓶兒那邊去。金蓮道:「賊小肉兒,不知怎的,聽見幹恁勾當兒,雲端裡老鼠——天生的耗。」分付:「快送了來,教他家丫頭伺候去。你不要管他,我要使你哩!」那春梅笑嘻嘻同小玉進去了。一面把酒菜擺在桌上,就出來了,只是綉春、迎春在房答應。玉樓、金蓮問了他話。玉簫道:「三娘,咱後邊去罷。」二人一路去了。金蓮叫春梅關上角門,歸進房來,獨自宿歇,不在話下。正是:

\begin{myquote} 
可惜團圓今夜月,清光咫尺別人圓。
\end{myquote} 

不說金蓮獨宿,單表西門慶與李瓶兒兩個相憐相愛,飲酒說話到半夜,方纔被伸翡翠,枕設鴛鴦,上床就寢。燈光掩映,不啻鏡中鸞鳳和鳴;香氣薰籠,好似花間蝴蝶對舞。正是:今宵勝把銀缸照,只恐相逢是夢中。有詞為證:

\begin{myquote} 
淡畫眉兒斜插梳,不忻拈弄倩工夫。雲窓霧閣深深許,蕙性蘭心款款呼。相憐愛,倩人扶,神仙標格世間無。從今罷卻相思調,美滿恩情錦不如。
\end{myquote} 

兩個睡到次日飯時。李瓶兒恰待起來臨鏡梳頭,只見迎春後邊拏將飯來。婦人先漱了口,陪西門慶吃了半盞兒,又教迎春:「將昨日剩的金華酒篩來。」拏甌子陪着西門慶,每人吃了兩甌子,方纔洗臉梳粧。一面開箱子,打點細軟首飾衣服,與西門慶過目。拏出一百顆西洋珠子,{\pangpi{應。}}與西門慶看,原是昔日梁家帶來之物。又拏出一件金鑲鴉青帽頂子,說是過世老公公的。起下來上等子秤,四錢八分重。李瓶兒教西門慶拏與銀匠,替他做一對墜子。又拏出一頂金絲鬏髻,重九兩。因問西門慶:「上房他大娘衆人,有這鬏髻沒有?」西門慶道:「他們銀絲鬏髻倒有兩三頂,只沒編這鬏髻。」{\pangpi{口角妙甚。}}婦人道:「我不好戴出來的。你替我拏到銀匠家毀了,打一件金九鳳墊根兒,每個鳳嘴啣一溜珠兒,剩下的再替我打一件,照依他大娘正面戴的金鑲玉觀音滿池嬌分心。」西門慶收了,一面梳頭洗臉,穿了衣服出門。李瓶兒又說道:「那邊房裡沒人,你好歹委付個人兒看守,替了小厮天福兒來家使喚。那老馮老行貨子,啻啻磕磕的,獨自在那裡,我又不放心。」西門慶道:「我知道了。」袖着鬏髻和帽頂子,一直往外走。不妨金蓮鬅着頭,站在東角門首,{\pangpi{偏有心。}}叫道:「哥,你往那去?這咱纔出來?」{\meipi{忽又播弄一番,風情無限。}}西門慶道:「我有勾當去。」婦人道:「怪行貨子,慌走怎的?我和你說話。」那西門慶見他叫的緊,只得回來。被婦人引到房中,婦人便坐在椅子上,把他兩隻手拉着說道:「我不好罵出來的,怪火燎腿三寸貨,那個拏長鍋鑊吃了你!慌往外搶的是些甚的?你過來,我且問你。」西門慶道:「罷麼,小淫婦兒,只顧問甚麼!我有勾當哩,等我回來說。」說着,往外走。婦人摸見袖子裡重重的,{\pangpi{偏細密。}}道:「是甚麼?拏出來我瞧瞧。」西門慶道:「是我的銀子包。」{\pangpi{瞞得妙。}}婦人不信,伸手進袖子裡就掏,掏出一頂金絲鬏髻來,說道:「這是他的鬏髻,{\pangpi{偏認得。}}你拏那去?」西門慶道:「他問我,知你每沒有,說不好戴的,教我到銀匠家替他毀了,打兩件頭面戴。」金蓮問道:「這鬏髻多少重?他要打甚麼?」西門慶道:「這鬏髻重九兩,他要打一件九鳳甸兒,一件照依上房娘的正面那一件玉觀音滿池嬌分心。」金蓮道:「一件九鳳甸兒,滿破使了三兩五六錢金子勾了。{\pangpi{偏曉得。}}大姐姐那件分心,我秤只重一兩六錢,{\pangpi{偏記得。}}把剩下的,好歹你替我照依他也打一件九鳳甸兒。」西門慶道:「滿池嬌他要揭實枝梗的。」金蓮道:「就是揭實枝梗,使了三兩金子滿頂了。還落他二三兩金子,{\pangpi{偏會算。}}勾打個甸兒了。」西門慶笑罵道:「你這小淫婦兒!單管愛小便宜兒,隨處也捏個尖兒。」金蓮道:「我兒,娘說的話,你好歹記着。你不替我打將來,我和你答話!」那西門慶袖了鬏髻,笑着出門。金蓮戲道:「哥兒,你幹上了。」西門慶道:「我怎的幹上了?」金蓮道:「你既不幹上,昨日那等雷聲大雨點小,要打着教他上弔。今日拏出一頂鬏髻來,使的你狗油嘴鬼推磨,不怕你不走。」{\meipi{一味嘴不讓人,使人愛,亦使人憎。}}西門慶笑道:「這小淫婦兒,單只管胡說!」說着往外去了。

卻說吳月娘和孟玉樓、李嬌兒在房中坐的,忽聽見外邊小厮一片聲尋來旺兒,尋不着。只見平安來掀簾子,月娘便問:「尋他做甚麼?」平安道:「爹緊等着哩。」月娘半日纔說:「我使他有勾當去了。」原來月娘早晨分付下他,往王姑子庵裡送香油白米去了。平安道:「小的回爹,只說娘使他有勾當去了。」月娘罵道:「怪奴才,隨你怎麼回去!」平安慌的不敢言語,往外走了。{\meipi{二人不說話合氣情景,偏在沒要沒緊處畫出。}}月娘便向玉樓衆人說道:「我開口,又說我多管。不言語,我又憋的慌。一個人也拉剌將來了,那房子賣弔了就是了。平白扯淡,搖鈴打鼓的,看守甚麼?左右有他家馮媽媽子,再派一個沒老婆的小厮,同在那裡就是了,怕走了那房子也怎的?巴巴叫來旺兩口子去!他媳婦子七病八痛,{\pangpi{伏宋蕙蓮。}}一時病倒了在那裡,誰扶侍他?」玉樓道:「姐姐在上,不該我說。你是個一家之主,不爭你與他爹兩個不說話,就是俺們不好主張的,下邊孩子每也沒投奔。他爹這兩日隔二騙三的,也甚是沒意思。姐姐依俺每一句話兒,與他爹笑開了罷。」月娘道:「孟三姐,你休要起這個意。我又不曾和他兩個嚷鬧,他平白的使性兒。那怕他使的那臉,休想我正眼看他一眼兒!他背地對人罵我不賢良的淫婦,我怎的不賢良?如今聳七八個在屋裡,纔知道我不賢良!自古道:『順情說好話,幹直惹人嫌。』我當初說着攔你,也只為好來。你既收了他許多東西,又買他房子,今日又圖謀他老婆,就着官兒也看喬了。何況他孝服不滿,你不好娶他的。誰知道人在背地裡,把圈套做的成成的,每日行茶過水,只瞞我一個兒,把我合在缸底下。今日也推在院裡歇,明日也推在院裡歇,誰想他只當把個人兒歇了家裡來,端的好在院裡歇!他自吃人在他跟前那等花麗狐哨,喬龍畫虎的,{\pangpi{明指金蓮。}}兩面刀哄他,就是千好萬好了。似俺每這等依老實,苦口良言,着他理你理兒!你不理我,我想求你?一日不少我三頓飯,我只當沒漢子,守寡在這裡。{\meipi{月娘與西門慶相好時何等賢慧,今稍冷落,便有許多牢騷不平之言。可見處敗局、冷局之難。}}隨我去,你每不要管他。」幾句話說的玉樓衆人訕訕的。

良久,只見李瓶兒梳粧打扮,上穿大紅遍地金對襟羅衫兒,翠蓋拖泥粧花羅裙,迎春抱着銀湯瓶,綉春拏着茶盒,走來上房,與月娘衆人遞茶。月娘叫小玉安放座兒與他坐。落後孫雪娥也來到,都遞了茶,一處坐地。潘金蓮嘴快,便叫道:「李大姐,你過來,與大姐姐下個禮兒。實和你說了罷,大姐姐和他爹好些時不說話,都為你來!俺每剛纔替你勸了恁一日。你改日安排一席酒兒,央及央及大姐姐,教他兩個老公婆笑開了罷。」李瓶兒道:「姐姐分付,奴知道。」於是向月娘面前插燭也似磕了四個頭。月娘道:「李大姐,他哄你哩。」又道:「五姐,你每不要來攛掇。我已是賭下誓,就是一百年也不和他在一答兒哩。」以此衆人再不敢復言。金蓮在旁拏把抿子與李瓶兒抿頭,見他頭上戴着一副金玲瓏草蟲兒頭面,並金累絲松竹梅歲寒三友梳背兒,因說道:「李大姐,你不該打這碎草蟲頭面,有些抓頭髮,不如大姐姐戴的金觀音滿池嬌,是揭實枝梗的好。」{\pangpi{尖甚。}}這李瓶兒老實,就說道:「奴也照樣兒要教銀匠打恁一件哩!」落後小玉、玉簫來遞茶,都亂戲他。先是玉簫問道:「六娘,你家老公公當初在皇城內那衙門來?」李瓶兒道:「先在惜薪司掌廠。」玉簫笑道:「嗔道你老人家昨日捱得好柴!」小玉又道:「去年許多里長老人,好不尋你,教你往東京去。」婦人不省,說道:「他尋我怎的?」小玉笑道:「他說你老人家會告的好水災。」玉簫又道:「你老人家鄉里媽媽拜千佛,昨日磕頭磕勾了。」小玉又說道:「昨日朝廷差四個夜不收,請你往口外和番,端的有這話麼?」李瓶兒道:「我不知道。」小玉笑道:「說你老人家會叫的好達達!」把玉樓、金蓮笑的不了。月娘罵道:「怪臭肉每,幹你那營生去,只顧奚落他怎的?」於是把個李瓶兒羞的臉上一塊紅、一塊白,站又站不得,坐又坐不住,{\pangpi{虧瓶兒禁得起。}}半日回房去了。

良久,西門慶進房來,回他顧銀匠家打造生活。就計較發柬,二十五日請官客吃會親酒,少不的請請花大哥。李瓶兒道:「他娘子三日來,再三說了。也罷,你請他請罷。」李瓶兒又說:「那邊房子左右有老馮看守,你這裡再教一個和天福兒輪着上宿就是,不消叫旺官去罷。上房姐姐說,他媳婦兒有病,去不的。」{\meipi{又急急挽回,是瓶兒之為人。若金蓮則定要來旺去矣。}}西門慶道:「我不知道。」即叫平安,分付:「你和天福兒兩個輪,一遞一日,獅子街房子裡上宿。」不在言表。

不覺到二十五日,西門慶家中吃會親酒,安排插花筵席,一起雜耍步戲。四個唱的,李桂姐、吳銀兒、董玉仙、韓金釧兒,從晌午就來了。官客在捲棚內吃了茶,等到齊了,然後大廳上坐席。頭一席花大舅、吳大舅;第二席吳二舅、沈姨夫;第三席應伯爵、謝希大;第四席祝實念、孫天化;第五席常峙節、吳典恩;第六席雲裡守、白賚光。西門慶主位,其餘傅自新、賁第傳、女婿陳敬濟兩邊列坐。樂人撮弄雜耍數回,就是笑樂院本。下去,李銘、吳惠兩個小優上來彈唱,間着清吹。下去,四個唱的出來,筵外遞酒。應伯爵在席上先開言說道:「今日哥的喜酒,是兄弟不當斗膽,請新嫂子出來拜見拜見,足見親厚之情。俺每不打緊,花大尊親,並二位老舅、沈姨丈在上,今日為何來?」西門慶道:「小妾醜陋,不堪拜見,免了罷。」謝希大道:「哥,這話難說。當初有言在先,不為嫂子,俺每怎麼兒來?何況見有我尊親花大哥在上,先做友,後做親,又不同別人。請出來見見怕怎的?」西門慶笑不動身。應伯爵道:「哥,你不要笑,俺每都拏着拜見錢在這裡,不白教他出來見。」西門慶道:「你這狗才,單管胡說。」吃他再三逼迫不過,叫過玳安來,教他後邊說去。

半日,玳安出來回說:「六娘道,免了罷。」應伯爵道:「就是你這小狗骨禿兒的鬼!你幾時往後邊去,就來哄我?」玳安道:「小的莫不哄應二爹!二爹進去問不是?」伯爵道:「你量我不敢進去?左右花園中熟徑,好不好我走進去,連你那幾位娘都拉了出來。」玳安道:「俺家那大猱獅狗,好不利害。倒沒有把應二爹下半截撕下來。」伯爵故意下席,趕着玳安踢兩脚,笑道:「好小狗骨禿兒,你傷的我好!趁早與我後邊請去。請不將來,打二十欄杆。」把衆人、四個唱的都笑了。玳安走到下邊立着,把眼只看着他爹不動身。西門慶無法可處,只得叫過玳安近前,分付:「對你六娘說,收拾了出來見見罷。」那玳安去了半日出來,復請了西門慶進去。然後纔把脚下人趕出去,關上儀門。孟玉樓、潘金蓮百方攛掇,替他抿頭,戴花翠,打發他出來。廳上鋪下錦氊繡毯,四個唱的,都到後邊彈樂器,導引前行。麝蘭靉靆,絲竹和鳴。婦人身穿大紅五彩通袖羅袍,下着金枝線葉沙綠百花裙,腰裡束着碧玉女帶,腕上籠着金壓袖。胸前纓落繽紛,裙邊環佩叮噹,頭上珠翠堆盈,鬢畔寶釵半卸,粉面宜貼翠花鈿,湘裙越顯紅鴛小。正是:

\begin{myquote} 
恍似姮嫦離月殿,猶如神女到筵前。
\end{myquote} 

當下四個唱的,琵琶箏弦,簇擁婦人,花枝招展,繡帶飄搖,望上朝拜。慌的衆人都下席來,還禮不迭。卻說孟玉樓、潘金蓮、李嬌兒簇擁着月娘,都在大廳軟壁後聽覷,聽見唱「喜得功名遂」,唱到「天之配合一對兒,如鸞似鳳」,直至「永團圓,世世夫妻」。金蓮向月娘說道:「大姐姐,你聽唱的!小老婆今日不該唱這一套,{\pangpi{輸身跌妙。}}他做了一對魚水團圓,世世夫妻,把姐姐放到那裡?」{\meipi{從曲中挑撥,又聰明,又微冷。}}那月娘雖故好性兒,聽了這兩句,未免有幾分惱在心頭。又見應伯爵、謝希大這夥人,見李瓶兒出來上拜,恨不得生出幾個口來誇獎奉承,說道:「我這嫂子,端的寰中少有,蓋世無雙!休說德性溫良,舉止沉重,自這一表人物,普天之下,也尋不出來。那裡有哥這樣大福?俺每今日得見嫂子一面,明日死也得好處。」{\meipi{一班花面口角,妙甚。}}因喚玳安兒:「快請你娘回房裡,只怕勞動着,倒値了多的。」吳月娘衆人聽了,罵「扯淡輕嘴的囚根子」不絕。良久,李瓶兒下來。四個唱的見他手裡有錢,都亂趨奉着他,娘長娘短,替他拾花翠,疊衣裳,無所不至。月娘歸房,甚是不樂。只見玳安、平安接了許多拜錢,也有尺頭、衣服並人情禮,盒子盛着,拏到月娘房裡。月娘正眼也不看,罵道:「賊囚根子!拏送到前頭就是了,平白拏到我房裡來做甚麼?」玳安道:「爹分付拏到娘房裡來。」{\pangpi{猶正景。}}月娘叫玉簫接了,掠在床上去。不一時,吳大舅吃了第二道湯飯,走進後邊來見月娘。月娘見他哥進房來,連忙與他哥哥行禮畢,坐下。吳大舅道:「昨日你嫂子在這裡打攪,又多謝姐夫送了桌面去。到家對我說,你與姐夫兩下不說話。我執着要來勸你,不想姐夫今日又請。姐姐,你若這等,把你從前一場好都沒了。自古癡人畏婦,賢女畏夫。三從四德,乃婦道之常。今後他行的事,你休要攔他,料姐夫他也不肯差了。落的做好好先生,纔顯出你賢德來。」{\meipi{夫妻之間,大倫所繫,乃以好好先生為賢德,可勝嘆息。}}月娘道:「早賢德好來,不教人這般憎嫌。他有了他富貴的姐姐,把我這窮官兒家丫頭,只當忘故了的算帳。你也不要管他,左右是我,隨他把我怎麼的罷!賊強人,從幾時這等變心來?」說着,月娘就哭了。吳大舅道:「姐姐,你這個就差了。你我不是那等人家,快休如此。你兩口兒好好的,俺每走來也有光輝些!」{\meipi{此數語必不可少,不然則與路人何異?}}勸月娘一回。小玉拏茶來。吃畢茶,只見前邊使小厮來請,吳大舅便作辭月娘出來。當下衆人吃至掌燈以後,就起身散了。四個唱的,李瓶兒每人都是一方銷金汗巾兒,五錢銀子,歡喜回家。自此西門慶連在瓶兒房裡歇了數夜。別人都罷了,只有潘金蓮惱的要不的,背地唆調吳月娘與李瓶兒合氣。對着李瓶兒,又說月娘容不的人。李瓶兒尚不知墮他計中,每以姐姐呼之,與他親厚尤密。正是:

\begin{myquote} 
逢人且說三分話,未可全拋一片心。
\end{myquote} 

西門慶自娶李瓶兒過門,又兼得了兩三場橫財,家道營盛,外庄內宅,煥然一新。米麥陳倉,騾馬成羣,奴僕成行。把李瓶兒帶來小厮天福兒,改名琴童。又買了兩個小厮,一名來安兒,一名棋童兒。把金蓮房中春梅、上房玉簫、李瓶兒房中迎春、玉樓房中蘭香,一般兒四個丫頭,衣服首飾粧束起來,在前廳西廂房,教李嬌兒兄弟樂工李銘來家,教演習學彈唱。{\meipi{富便奢侈,此見一斑。}}春梅琵琶,玉簫學箏,迎春學弦子,蘭香學胡琴。每日三茶六飯,管待李銘,一月與他五兩銀子。又開啟門面兩間,兌出二千兩銀子來,委傅夥計、賁第傳開解當鋪。女婿陳敬濟只掌鑰匙,出入尋討。賁第傳只寫帳目,秤發貨物。傅夥計便督理生藥、解當兩個鋪子,看銀色,做買賣。潘金蓮這邊樓上,堆放生藥。李瓶兒那邊樓上,廂成架子,擱解當庫衣服、首飾、古董、書畫、玩好之物。一日也當許多銀子出門。陳敬濟每日起早睡遲,帶着鑰匙,同夥計查點出入銀錢,收放寫算皆精。西門慶見了,喜歡的要不的。一日在前廳與他同桌兒吃飯,說道:「姐夫,你在我家這等會做買賣,就是你父親在東京知道,他也心安,我也得托了。常言道:有兒靠兒,無兒靠婿。我若久後沒出,這分兒家當,都是你兩口兒的。」{\meipi{數語往往釀成大禍。}}那敬濟說道:「兒子不幸,家遭官事,父母遠離,投在爹娘這裡。蒙爹娘擡舉,莫大之恩,生死難報。只是兒子年幼,不知好歹,望爹娘耽待便了,豈敢非望。」西門慶聽見他說話兒聰明乖覺,越發滿心歡喜。但凡家中大小事務、出入書柬、禮帖,都教他寫。但凡客人到,必請他席側相陪。吃茶吃飯,一時也少不的他。誰知道這小夥兒綿裡之針,肉裡之刺。常向繡簾窺賈玉,每從綺閣竊韓香。光陰似箭,不覺又是十一月下旬。西門慶在常峙節家會茶散的早,未掌燈就起身,同應伯爵、謝希大、祝實念三個並馬而行。剛出了門,只見天上彤雲密佈,又早紛紛揚揚飄下一天雪花來。應伯爵便道:「哥,咱這時候就家去,家裡也不收。我每許久不曾進裡邊看看桂姐,今日趁着落雪,只當孟浩然踏雪尋梅,望他望去。」祝實念道:「應二哥說的是。你每月風雨不阻,出二十銀子包錢包着他,{\meipi{點出。}}你不去,落的他自在。」西門慶吃三人你一言我一句,說的把馬逕往東街抅欄來了。來到李桂姐家,已是天氣將晚。只見客位裡掌着燈,丫頭正掃地。老媽並李桂卿出來,見禮畢,上面列四張交椅,四人坐下。老虔婆便道:「前者桂姐在宅裡來晚了,多有打攪。又多謝六娘,賞汗巾花翠。」西門慶道:「那日空過他。我恐怕晚了他們,客人散了,就打發他來了。」說着,虔婆一面看茶吃了,丫鬟就安放桌兒,設放案酒。西門慶道:「怎麼桂姐不見?」虔婆道:「桂姐連日在家伺候姐夫,不見姐夫來。今日是他五姨媽生日,拏轎子接了與他五姨媽做生日去了。」原來李桂姐也不曾往五姨家做生日去。近日見西門慶不來,又接了杭州販紬絹的丁相公兒子丁二官人,號丁雙橋,販了千兩銀子紬絹在客店裡,瞞着他父親來院中闝。頭上拏十兩銀子、兩套杭州重絹衣服請李桂姐,一連歇了兩夜。適纔正和桂姐在房中吃酒,不想西門慶到。老虔婆忙教桂姐陪他到後邊第三層一間僻靜小房坐去了。當下西門慶聽信虔婆之言,便道:「既是桂姐不在,老媽快看酒來,俺每慢慢等他。」這老虔婆在下面一力攛掇,酒餚蔬菜齊上,須臾,堆滿桌席。李桂卿不免箏排雁柱,歌按新腔,衆人席上猜枚行令。

正飲酒時,不妨西門慶往後邊更衣去。也是合當有事,忽聽東耳房有人笑聲。西門慶更畢衣,走至窓下偸眼觀覷,正見李桂姐在房內陪着一個戴方巾的蠻子飲酒。{\meipi{此書妙在處處破敗,寫出世情之假。}}繇不的心頭火起,走到前邊,一手把吃酒桌子掀翻,碟兒盞兒打的粉碎。喝令跟馬的平安、玳安、畫童、琴童四個小厮上來,把李家門窓戶壁床帳都打碎了。應伯爵、謝希大、祝實念向前拉勸不住。西門慶口口聲聲只要採出蠻囚來,和粉頭一條繩子墩鎖在門房內。那丁二官又是個小膽之人,見外邊嚷鬬起來,慌的藏在裡間床底下,只叫:「桂姐救命!」桂姐道:「呸!好不好,還有媽哩!這是俺院中人家常有的,不妨事,隨他發作叫嚷,你只休要出來。」老虔婆見西門慶打的不象模樣,還要架橋兒說謊,上前分辨。西門慶那裡還聽他,只是氣狠狠呼喝小厮亂打,險些不曾把李老媽打起來。多虧了應伯爵、謝希大、祝實念三人死勸,活喇喇拉開了手。西門慶大鬧了一場,賭誓再不踏他門來,大雪裡上馬回家。正是:

\begin{myquote} 
宿盡閑花萬萬千,不如歸家伴妻眠。\\雖然枕上無情趣,睡到天明不要錢。
\end{myquote} 

