\includepdf[pages={13,14},fitpaper=false]{tst.pdf}
\chapter*{第七回 薛媒婆說娶孟三兒 楊姑娘氣罵張四舅}
\addcontentsline{toc}{chapter}{第七回 薛媒婆說娶孟三兒 楊姑娘氣罵張四舅}
\markboth{{\titlename}卷之一}{第七回 薛媒婆說娶孟三兒 楊姑娘氣罵張四舅}


詩曰:

\begin{myquote} 
我做媒人實自能,全憑兩腿走慇懃。\\唇鎗慣把鰥男配,舌劍能調烈女心。\\利市花常頭上帶,喜筵餅錠袖中撐。\\只有一件不堪處,半是成人半敗人。
\end{myquote} 

話說西門慶家中,一個賣翠花的薛嫂兒,提着花廂兒,一地裡尋西門慶不着。因見西門慶貼身使的小厮玳安兒,便問道:「大官人在那裡?」玳安道:「俺爹在鋪子裡和傅二叔算帳。」原來西門慶家開生藥鋪,主管姓傅名銘,字自新,排行第二,因此呼他做傅二叔。這薛嫂聽了,一直走到鋪子門首,掀開簾子,見西門慶正與主管算帳,便點點頭兒,喚他出來。西門慶見是薛嫂兒,連忙撇了主管出來,兩人走在僻靜處說話。西門慶問道:「有甚話說?」薛嫂道:「我有一件親事,來對大官人說,管情中你老人家意,就頂死了的三娘的窩兒,{\pangpi{入情。}}何如?」西門慶道:「你且說這件親事是那家的?」薛嫂道:「這位娘子,說起來你老人家也知道,就是南門外販布楊家的正頭娘子。手裡有一分好錢。南京拔步床也有兩張。四季衣服,插不下手去,也有四五隻箱子。金鐲銀釧不消說,手裡現銀子也有上千兩,好三梭布也有有三二百筒。不料他男子漢去販布,死在外邊。他守寡了一年多,身邊又沒子女,止有一個小叔兒,纔十歲。青春年少,守他什麼!有他家一個嫡親姑娘,要主張着他嫁人。這娘子今年不上二十五六歲,{\pangpi{瞞四、五歲,妙。}}生的長挑身材,一表人物,打扮起來就是個燈人兒。風流俊俏,百伶百俐,當家立紀,針指女工,雙陸棋子,不消說。不瞞大官人說,{\pangpi{好頓挫。}}他娘家姓孟,排行三姐,就住在臭水巷。{\meipi{小小一地名,亦下得恰好。}}又會彈一手好月琴,大官人若見了,管情一箭就上垜。」西門慶聽見婦人會彈月琴,便可在他心上,就問薛嫂兒:「既是這等,幾時相會看去?」薛嫂道:「相看到不打緊。我且和你老人家計議:{\meipi{引入彀,卻纔勒住,細細商量,鬆緊合宜。}}如今他家一家子,只是姑娘大。雖是他娘舅張四,山核桃——差着一槅兒哩。這婆子原嫁與北邊半邊街徐公公房子裡住的孫歪頭。{\meipi{孫歪頭三字寫得活現,恰象眞有其人。}}歪頭死了,這婆子守寡了三四十年,男花女花都無,只靠姪男姪女養活。大官人只倒在他身上求他。這婆子愛的是錢財,明知姪兒媳婦有東西,隨問什麼人家他也不管,只指望要幾兩銀子。大官人家裡有的是那囂段子,拏一段,買上一担禮物,明日親去見他,再許他幾兩銀子,一拳打倒他。{\meipi{段子日囂,禮物曰買上一担,銀子曰許他幾兩,只數虛字,說得毫不費事,想見立言之妙。}}隨問旁邊有人說話,這婆子一力張主,誰敢怎的!」這薛嫂兒一席話,說的西門慶歡從額角眉尖出,喜向腮邊笑臉生。正是:

\begin{myquote} 
媒妁慇懃說始終,孟姬愛嫁富家翁。\\有緣千里能相會,無緣對面不相逢。
\end{myquote} 

西門慶當日與薛嫂相約下了,明日是好日期,就買禮往他姑娘家去。薛嫂說畢話,提着花廂兒去了。西門慶進來和傅夥計算帳。一宿晚景不題。到次日,西門慶早起,打選衣帽整齊,拏了一段尺頭,買了四盤羹菓,裝做一盒担,叫人擡了。薛嫂領着,西門慶騎着頭口,小厮跟隨,逕來楊姑娘家門首。薛嫂先入去通報姑娘,說道:「近邊一個財主,{\pangpi{先入。}}要和大娘子說親。我說一家只姑奶奶是大,先來覿面,親見過你老人家,講了話,{\pangpi{遞局。}}然後纔敢去門外相看。今日小媳婦領來,見在門首伺候。」婆子聽見,便道:「阿呀,{\pangpi{傳神。}}保山,你如何不先來說聲!」一面分付丫鬟頓下好茶,一面道:「有請。」這薛嫂一力攛掇,先把盒担擡進去擺下,打發空盒担出去,就請西門慶進來相見。這西門慶頭戴纏綜大帽,一撒鉤縧,粉底皁靴,進門見婆子拜四拜。婆子拄着拐,慌忙還下禮去。西門慶那裡肯,一口一聲只叫:「姑娘請受禮。」讓了半日,婆子受了半禮。分賓主坐下,薛嫂在旁邊打橫。婆子便道:「大官人貴姓?」薛嫂道:「便是咱清河縣數一數二的財主,西門大官人。在縣前開個大生藥鋪,家中錢過北斗,米爛陳倉,沒個當家立紀的娘子。聞得咱家門外大娘子要嫁,特來見姑奶奶講說親事。」婆子道:「官人倘然要說俺姪兒媳婦,自恁來閑講罷了,何必費煩又買禮來,使老身卻之不恭,受之有愧。」西門慶道:「姑娘在上,沒的禮物,惶恐。」那婆子一面拜了兩拜謝了,收過禮物去,拏茶上來。吃畢,婆子開口道:「老身當言不言謂之懦。{\pangpi{開口訣。}}我姪兒在時,掙了一分錢財,不幸先死了,如今都落在他手裡,說少也有上千兩銀子東西。官人做小做大我不管你,只要與我姪兒念上個好經。老身便是他親姑娘,又不隔從,就與上我一個棺材本,也不曾要了你家的。我破着老臉,和張四那老狗做臭毛鼠,替你兩個硬張主。娶過門時,遇生辰時節,官人放他來走走,就認俺這門窮親戚,也不過上你窮。」{\meipi{先入念經,故正題目,然後說到自己,說自己卻提出張四一段,說得有條理,有斤兩,有拏手。}}西門慶笑道:「你老人家放心,所說的話,我小人都知道了。只要你老人家主張得定,休說一個棺材本,就是十個,小人也來得起。」說着,便叫小厮拏過拜匣來,取出六錠三十兩雪花官銀,放在面前,說道:「這個不當甚麼,先與你老人家買盞茶吃,到明日娶過門時,還你七十兩銀子、兩疋段子,與你老人家為送終之資。其四時八節,只管上門行走。」這老虔婆黑眼珠見了二三十兩白晃晃的官銀,滿面堆下笑來,說道:「官人在上,不是老身意小,自古先斷後不亂。」薛嫂在旁插口說:「你老人家忒多心,那裡這等計較!我這大官人不是這等人,只恁還要掇着盒兒認親。你老人家不知,如今知縣知府相公也都來往,好不四海。你老人家能吃他多少?」一席話說的婆子屁滾尿流。吃了兩道茶,西門慶便要起身,婆子挽留不住。薛嫂道:「今日既見了姑奶奶,明日便好往門外相看。」婆子道:「我家姪兒媳婦不用大官人相,保山,你就說我說,不嫁這樣人家,再嫁甚樣人家!」西門慶作辭起身。婆子道:「老身不知大官人下降,匆忙不曾預備,空了官人,休怪。」拄拐送出。送了兩步,西門慶讓回去了。薛嫂打發西門慶上馬,因說道:「我主張的有理麼?你老人家先回去罷,我還在這裡和他說句話。明日須早些往門外去。」西門慶便拏出一兩銀子來,與薛嫂做驢子錢。薛嫂接了,西門慶便上馬來家。他還在楊姑娘家說話飲酒,到日暮纔歸家去。

話休饒舌。到次日,西門慶打選衣帽齊整,袖着插戴,騎着匹白馬,玳安、平安兩個小厮跟隨,薛嫂兒騎着驢子,出的南門外來。不多時,到了楊家門首。卻是坐南朝北一間門樓,粉青照壁。薛嫂請西門慶下了馬,同進去。裡面儀門照墻,竹槍籬影壁,院內擺設榴樹盆景,臺基上靛缸一溜,打布櫈兩條。{\pangpi{好映帶。}}薛嫂推開朱紅槅扇,三間倒坐客位,上下椅桌光鮮,簾櫳瀟灑。薛嫂請西門慶坐了,一面走入裡邊。片晌出來,向西門慶耳邊說:「大娘子梳粧未了,你老人家請坐一坐。」只見一個小厮兒拏出一盞福仁泡茶來,西門慶吃了。這薛嫂一面指手畫脚與西門慶說:「這家中除了那頭姑娘,只這位娘子是大。雖有他小叔,還小哩,不曉得什麼。當初有過世的官人在鋪子裡,一日不算銀子,銅錢也賣兩大簸籮。毛青鞋面布,{\pangpi{異想。}}俺每問他買,定要三分一尺。一日常有二三十染的吃飯,都是這位娘子主張整理。{\meipi{偏在沒要緊處寫照。}}手下使着兩個丫頭,一個小厮。大丫頭十五歲,弔起頭去了,名喚蘭香。小丫頭名喚小鸞,纔十二歲。到明日過門時,都跟他來。我替你老人家說成這親事,指望典兩間房兒住哩。」西門慶道:「這不打緊。」薛嫂道:「你老人家去年買春梅,{\meipi{無意中點出春梅,冷甚,妙甚。}}許我幾疋大布,還沒與我。到明日不管一總謝罷了。」

正說着,只見使了個丫頭來叫薛嫂。不多時,只聞環佩叮咚,蘭麝馥郁,薛嫂忙掀開簾子,婦人出來。西門慶睜眼觀那婦人,但見:

\begin{myquote} 
月畫烟描,粉粧玉琢。俊龐兒不肥不瘦,俏身材難減難增。素額逗幾點微麻,天然美麗;緗裙露一雙小脚,周正堪憐。行過處花香細生,坐下時淹然百媚。
\end{myquote} 

西門慶一見滿心歡喜。婦人走到堂下,望上不端不正道了個萬福,就在對面椅子上坐下。西門慶眼不轉睛看了一回,婦人把頭低了。西門慶開言說:「小人妻亡已久,欲娶娘子管理家事,未知尊意如何?」那婦人偸眼看西門慶,見他人物風流,心下已十分中意,遂轉過臉來,問薛婆道:「官人貴庚?沒了娘子多少時了?」西門慶道:「小人虛度二十八歲,不幸先妻沒了一年有餘。不敢請問,娘子青春多少?」婦人道:「奴家是三十歲。」西門慶道:「原來長我二歲。」薛嫂在旁插口道:「妻大兩,黃金日日長。妻大三,黃金積如山。」{\meipi{雖套語,用在此處恰好。}}說着,只見小丫鬟拏出三盞蜜餞金柳丁泡茶來。婦人起身,先取頭一盞,用纖手抹去盞邊水漬,{\pangpi{舉止俏甚。}}遞與西門慶,道個萬福。薛嫂見婦人立起身,就趁空兒{\pangpi{有竅。}}輕輕用手掀起婦人裙子來,正露出一對剛三寸、恰半杈、尖尖趫趫金蓮脚來,{\meipi{賣弄脚好處,妙在都不開口,只俏俏畫出。}}穿着雙大紅遍地金雲頭白綾高低鞋兒。{\pangpi{動人。}}西門慶看了,滿心歡喜。婦人取第二盞茶來遞與薛嫂。他自取一盞陪坐。吃了茶,西門慶便叫玳安用方盒呈上錦帕二方、寶釵一對、金戒指六個,放在托盤內送過去。薛嫂一面叫婦人拜謝了。因問官人行禮日期:「奴這裡好做預備。」西門慶道:「既蒙娘子見允,今月二十四日,有些微禮過門來。六月初二準娶。」婦人道:「既然如此,奴明日就使人對姑娘說去。」薛嫂道:「大官人昨日已到姑奶奶府上講過話了。」婦人道:「姑娘說甚來?」薛嫂道:「姑奶奶聽見大官人說此樁事,好不喜歡!說道,不嫁這等人家,再嫁那樣人家!我就做硬主媒,保這門親事。」婦人道:「既是姑娘恁般說,又好了。」{\meipi{滿肚皮要嫁,只三字。}}薛嫂道:「好大娘子,莫不俺做媒敢這等搗謊。」說畢,西門慶作辭起身。薛嫂送出巷口,向西門慶說道:「看了這娘子,你老人家心下如何?」西門慶道:「薛嫂,其實累了你。」{\pangpi{寫出中意。}}薛嫂道:「你老人家先行一步,我和大娘子說句話就來。」西門慶騎馬進城去了。薛嫂轉來向婦人說道:「娘子,你嫁得這位官人也罷了。」婦人道:「但不知房裡有人沒有人?{\pangpi{有含蓄。}}見作何生理?」薛嫂道:「好奶奶,就有房裡人,那個是成頭腦的?我說是謊,你過去就看出來。{\meipi{說得活活落落,絕有意味,卻又妙在斬釘截鐵,模寫處眞匪夷所思。}}他老人家名目,誰不知道,清河縣數一數二的財主,有名賣生藥放官吏債西門慶大官人。知縣知府都和他來往。近日又與東京楊提督結親,都是四門親家,誰人敢惹他!」婦人安排酒飯,與薛嫂兒正吃着,只見他姑娘家使個小厮安童,盒子裡盛着四塊黃米麵棗兒糕、兩塊糖、幾十個艾窩窩,就來問:「曾受了那人家插定不曾?奶奶說來:這人家不嫁,待嫁甚人家。」婦人道:「多謝你奶奶掛心。今已留下插定了。」薛嫂道:「天麼,天麼!早是俺媒人不說謊,姑奶奶早說將來了。」{\meipi{口角宛然。}}婦人收了糕,取出盒子,裝了滿滿一盒子點心臘肉,又與了安童五六十文錢,說:「到家多拜上奶奶。那家日子定在二十四日行禮,出月初二日準娶。」小厮去了。薛嫂道:「姑奶奶家送來什麼?與我些,包了家去孩子吃。」婦人與了他一塊糖、十個艾窩窩,方纔出門,不在話下。

且說他母舅張四,倚着他小外甥楊宗保,要圖留婦人東西,一心舉保大街坊尚推官兒子尚舉人為繼室。若小可人家,還有話說,不想聞得是西門慶定了,知他是把持官府的人,遂動不得了。尋思千方百計,不如破為上計。即走來對婦人說:「娘子不該接西門慶插定,還依我嫁尚舉人的是。他是詩禮人家,又有庄田地土,頗過得日子,強如嫁西門慶。那厮積年把持官府,刁徒潑皮。{\meipi{句句良言,可惜為破親而發。}}他家見有正頭娘子,乃是吳千戶家女兒,你過去做大是,做小是?況他房裡又有三四個老婆,除沒上頭的丫頭不算。你到他家,人多口多,還有的惹氣哩!」婦人聽見話頭,明知張四是破親之意,{\meipi{先被婦人看破,後便語言無味。}}便佯說道:「自古船多不礙路。若他家有大娘子,我情願讓他做姐姐。雖然房裡人多,只要丈夫作主,若是丈夫喜歡,多亦何妨。丈夫若不喜歡,便只奴一個也難過日子。況且富貴人家,那家沒有四五個?你老人家不消多慮,奴過去自有道理,料不妨事。」張四道:「不獨這一件。他最慣打婦煞妻,又管挑販人口,稍不中意,就令媒婆賣了。{\meipi{破語雖毒,卻嫌太直。}}你受得他這氣麼?」婦人道:「四舅,你老人家差矣。男子漢雖利害,不打那勤謹省事之妻。我到他家,把得家定,裡言不出,外言不入,他敢怎的奴?」張四道:「不是我打聽的,他家還有一個十四歲未出嫁的閨女,誠恐去到他家,三窩兩塊惹氣怎了?」{\meipi{此一破尤不動人。}}婦人道:「四舅說那裡話,奴到他家,大是大,小是小,待得孩兒們好,不怕男子漢不歡喜,不怕女兒們不孝順。休說一個,便是十個也不妨事。」張四道:「還有一件最要緊的事,此人行止欠端,專一在外眠花臥柳。又裡虛外實,少人家債負。只怕坑陷了你。」婦人道:「四舅,你老人家又差矣。他少年人,就外邊做些風流勾當,也是常事。奴婦人家,那裡管得許多?{\meipi{護局中夾出喜愛眞情,妙甚。}}惹說虛實,常言道:世上錢財儻來物,那是長貧久富家?況姻緣事皆前生分定,你老人家到不消這樣費心。」張四見說不動婦人,到吃他搶白了幾句,好無顏色,吃了兩盞清茶,{\meipi{一清字傳冷落之神,令人絕倒。}}起身去了。有詩為證:

\begin{myquote} 
張四無端散楚言,姻緣誰想是前緣。\\佳人心愛西門慶,說破咽喉總是閑。
\end{myquote} 

張四羞慚歸家,與婆子商議,{\pangpi{伏後罵句,細甚。}}單等婦人起身,指着外甥楊宗保,要攔奪婦人箱籠。話休饒舌。到二十四日,西門慶行了禮。到二十六日,請十二位素僧念經燒靈,都是他姑娘一力張主。張四到婦人將起身頭一日,請了幾位街坊衆隣,來和婦人說話。此時薛嫂正引着西門慶家小厮伴當,並守備府裡討的一二十名軍牢,正進來搬擡婦人床帳、嫁粧箱籠。被張四攔住說道:「保山且休擡!有話講。」一面同了街坊隣舍進來見婦人。坐下,張四先開言說:「列位高隣聽着:大娘子在這裡,不該我張龍說,{\pangpi{酷肖。}}你家男子漢楊宗錫與你這小叔楊宗保,都是我外甥。今日不幸大外甥死了,空掙一場錢。有人主張着你,{\pangpi{暗指姑娘。}}這也罷了。爭奈第二個外甥楊宗保年幼,一個業障都在我身上。他是你男子漢一母同胞所生,莫不家當沒他的份兒?今日對着列位高隣在這裡,只把你箱籠開啟,眼同衆人看一看,有東西沒東西,大家見個明白。」婦人聽言,一面哭起來,說道:「衆位聽着,你老人家差矣!奴不是歹意謀死了男子漢,今日添羞臉又嫁人。他手裡有錢沒錢,人所共知,就是積攢了幾兩銀子,都使在這房子上。{\pangpi{好出脫。}}房子我沒帶去,都留與小叔。家活等件,分毫不動。就是外邊有三四百兩銀子欠帳,文書合同已都交與你老人家,陸續討來家中盤纏。再有甚麼銀兩來?」張四道:「你沒銀兩也罷。如今只對着衆位開啟箱籠看一看。就有,你還拏了去,我又不要你的。」婦人道:「莫不奴的鞋脚也要瞧不成?」正亂着,只見姑娘拄拐自後而出。{\meipi{先讓張四與婦人鬧一陣,然後姑娘慢慢走出來。絕有情景。}}衆人便道:「姑娘出來。」都齊聲唱喏。姑娘還了萬福,陪衆人坐下。姑娘開口道:「列位高隣在上,我是他是親姑娘,又不隔從,莫不沒我說處?死了的也是姪兒,活着的也是姪兒,十個指頭咬着都疼。如今休說他男子漢手裡沒錢,他就有十萬兩銀子,你只好看他一眼罷了。他身邊又無出,少女嫩婦的,你攔着不教他嫁人,做什麼?」衆街隣高聲道:「姑娘見得有理!」婆子道:「難道他娘家陪的東西,也留下他的不成?他背地又不曾自與我什麼,{\meipi{此處無銀。}}說我護他,也要公道。不瞞列位說,我這姪兒媳婦平日有仁義,老身捨不得他,好溫克性兒。不然,老身也不管着他。」那張四在旁,把婆子瞅了一眼,{\pangpi{逼眞。}}說道:「你好公平心兒!鳳凰無寶處不落。」只這一句話,道着婆子眞病,登時怒起,紫漲了面皮,指定張四大罵道:「張四,你休胡言亂語!我雖不能是楊家正頭香主,你這老油嘴,是楊家那膫子㒲的?」{\meipi{罵得妙,纔象孫歪頭的婆子。}}張四道:「我雖是異姓,兩個外甥是我姐姐養的,你這老咬蟲,女生外嚮,怎一頭放火,又一頭放水?」姑娘道:「賤沒廉恥老狗骨頭!他少女嫩婦的,你留他在屋裡,有何算計?既不是圖色慾,便欲起謀心,將錢肥己。」張四道:「我不是圖錢,只恐楊宗保後來大了,過不得日子。不似你這老殺才,搬着大引着小,黃貓兒黑尾。」姑娘道:「張四,你這老花根,老奴才,老粉嘴,你恁騙口張舌的好淡扯,到明日死了時,不使了繩子扛子。」張四道:「你這嚼舌頭老淫婦,掙將錢來焦尾靶,怪不得你無兒無女。」姑娘急了,罵道:「張四,賊老蒼根,老豬狗,我無兒無女,強似你家媽媽子穿寺院,養和尚,㒲道士,你還在睡夢裡。」當下兩個差些兒不曾打起來,多虧衆隣舍勸住,說道:「老舅,你讓姑娘一句兒罷。」薛嫂兒見他二人嚷做一團,領率西門慶家小厮伴當,併發來衆軍牢,趕人鬧裡,七手八脚將婦人床帳、粧奩、箱籠,扛的扛,擡的擡,一陣風都搬去了。{\meipi{收煞得妙。若等講清日子再扛擡,便呆矣。}}那張四氣的眼大睜着,半晌說不出話來。衆隣舍見不是事,安撫了一回,各人都散了。

到六月初二日,西門慶一頂大轎,四對紅紗燈籠,他小叔楊宗保頭上紮着髻兒,穿着青紗衣,撒騎在馬上,送他嫂子成親。西門慶答賀了他一疋錦段、一柄玉縧兒。蘭香、小鸞兩個丫頭,都跟了來鋪床疊被。小厮琴童方年十五歲,亦帶過來伏侍。到三日,楊姑娘家並婦人兩個嫂子孟大嫂、二嫂都來做生日。西門慶與他楊姑娘七十兩銀子、兩疋尺頭。自此親戚來往不絕。西門慶就把西廂房裡收拾三間,與他做房。排行第三,號玉樓,令家中大小都隨着叫三姨。到晚一連在他房中歇了三夜。正是:銷金帳裡,依然兩個新人;紅錦被中,現出兩般舊物。有詩為證:

\begin{myquote} 
怎覩多情風月標,教人無福也難消。\\風吹列子歸何處,夜夜嬋娟在柳梢。
\end{myquote} 
