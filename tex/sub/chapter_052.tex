\includepdf[pages={103,104},fitpaper=false]{tst.pdf}
\chapter*{第五十二回 應伯爵山洞戲春嬌 潘金蓮花園調愛婿}
\addcontentsline{toc}{chapter}{第五十二回 應伯爵山洞戲春嬌 潘金蓮花園調愛婿}
\markboth{{\titlename}卷之六}{第五十二回 應伯爵山洞戲春嬌 潘金蓮花園調愛婿}


詩曰:

\begin{myquote} 
春樓曉日珠簾映,紅粉春粧寶鏡催。\\已厭交歡憐舊枕,相將遊戲繞池臺。\\坐時衣帶縈纖草,行處裙裾掃落梅。\\更道明朝不當作,相期共鬬管絃來。
\end{myquote} 

話說那日西門慶在夏提刑家吃酒,見宋巡按送禮,他心中十分歡喜。夏提刑亦敬重不同往日,{\pangpi{勢利一時便起。}}攔門勸酒,吃至三更天氣纔放回家。潘金蓮又早向燈下除去冠兒,設放衾枕,薰香澡牝等候。西門慶進門,接着,見他酒帶半酣,連忙替他脫衣裳。春梅點茶吃了,打發上床歇息。見婦人脫得光赤條身子,坐在床沿,低垂着頭,將那白生生腿兒橫抱膝上纏脚,{\pangpi{那得不愛。}}換了雙大紅平底睡鞋兒。西門慶一見,淫心輒起,麈柄挺然而興。因問婦人要淫器包兒,婦人忙向褥子底下摸出來遞與他。西門慶把兩個托子都帶上,一手摟過婦人在懷裡,因說:「你達今日要和你幹個後庭花兒,你肯不肯?」那婦人瞅了一眼,說道:「好個沒廉恥冤家,你成日和書童兒小厮幹的不値了,又纏起我來了,你和那奴才幹去不是!」西門慶笑道:「怪小油嘴,罷麼!你若依了我,又稀罕小厮做甚麼?你不知你達心裡好的是這樁兒,管情放到裡頭去就過了。」{\pangpi{哄騙口角。}}婦人被他再三纏不不過,說道:「奴只怕挨不得你這大行貨。你把頭子上圈去了,我和你耍一遭試試。」西門慶眞個除去硫磺圈,根下只束着銀托子,令婦人馬爬在床上,屁股高蹶,將唾津塗抹在龜頭上,往來濡研頂入。龜頭昂健,半晌僅沒其稜。婦人在下蹙眉隱忍,口中咬汗巾子難捱,叫道:「達達慢着些。這個比不的前頭,撐得裡頭熱炙火燎的疼起來。」這西門慶叫道:「好心肝,你叫着達達,不妨事。到明日買一套好顏色粧花紗衣服與你穿。」婦人道:「那衣服倒也有在,我昨日見李桂姐穿的那玉色線掐羊皮挑的金油鵝黃銀條紗裙子,倒好看,說是裡邊買的。他每都有,只我沒這裙子。倒不知多少銀子,你倒買一條我穿罷了。」西門慶道:「不打緊,我到明日替你買。」一壁說着,在上頗作抽拽,只顧僅沒其稜,淺抽深送不已。婦人回首流眸叫道:「好達達,這裡緊着人疼的要不的,如何只顧這般動作起來了?我央及你,好歹快些丟了罷!」這西門慶不聽,且扶其股,玩其出入之勢。一面口中呼道:「潘五兒,小淫婦兒,你好生浪浪的叫着達達,哄出你達達㞞兒出來罷。」那婦人眞個在下星眼朦朧,鶯聲款掉,柳腰款擺,香肌半就,口中艷聲柔語,百般難述。良久,西門慶覺精來,兩手扳其股,極力而𢵞之,扣股之聲响之不絕。那婦人在下邊呻吟成一塊,不能禁止。臨過之時,西門慶把婦人屁股只一扳,麈柄盡沒至根,直抵於深異處,其美不可當。於是怡然感之,一泄如注。婦人承受其精,二體偎貼。良久拽出麈柄,但見猩紅染莖,蛙口流涎,婦人以帕抹之,方纔就寢。一宿晚景題過。

次日,西門慶早晨到衙門中回來,有安主事、黃主事那裡差人來下請書,二十二日在磚廠劉太監庄上設席,請早去。西門慶打發來人去了,從上房吃了粥,正出廳來,只見篦頭的小周兒扒倒地下磕頭。西門慶道:「你來的正好,我正要篦篦頭哩。」於是走到翡翠軒小捲棚內,坐在一張涼椅兒上,除了巾幘,開啟頭髮。小周兒鋪下梳篦家活,與他篦頭櫛髮。觀其泥垢,辨其風雪,跪下討賞錢,說:「老爹今歲必有大遷轉,髮上氣色甚旺。」西門慶大喜。篦了頭,又叫他取耳,掐捏身上。他有滾身上一弄兒家活,到處與西門慶滾捏過,又行導引之法,把西門慶弄的渾身通泰。賞了他五錢銀子,教他吃了飯,伺候着哥兒剃頭。西門慶就在書房內,倒在大理石床上就睡着了。

那日楊姑娘起身,王姑子與薛姑子要家去。吳月娘將他原來的盒子都裝了些蒸酥茶食,打發起身。兩個姑子,每人都是五錢銀子,兩個小姑子,與了他兩疋小布兒,管待出門。薛姑子又囑咐月娘:「到了壬子日把那藥吃了,管情就有喜事。」月娘道:「薛爺,你這一去,八月裡到我生日,好來走走,我這裡盼你哩。」薛姑子合掌問訊道:「打攪。菩薩這裡,我到那日已定來。」於是作辭。月娘衆人都送到大門首。月娘與大妗子回後邊去了。只有玉樓、金蓮、瓶兒、西門大姐、李桂姐抱着官哥兒,來到花園裡遊玩。李瓶兒道:「桂姐,你遞過來,等我抱罷。」桂姐道:「六娘,不妨事,我心裡要抱抱哥子。」玉樓道:「桂姐,你還沒到你爹新收拾書房裡瞧瞧哩。」到花園內,金蓮見紫薇花開得爛熳,摘了兩朵與桂姐戴。{\pangpi{偏有弄頭。}}於是順着松墻兒到翡翠軒,見裡面擺設的床帳屏幾、書畫琴棋,極其瀟灑。床上綃帳銀鉤,冰簟珊枕。西門慶倒在床上,睡思正濃。旁邊流金小篆,焚着一縷龍涎。綠窓半掩,窓外芭蕉低映。潘金蓮且在桌上掀弄他的香盒兒,玉樓和李瓶兒都坐在椅兒上,西門慶忽翻過身來,看剛見衆婦人都在屋裡,便道:「你每來做甚麼?」金蓮道:「桂姐要看看你的書房,俺每引他來瞧瞧。」那西門慶見他抱着官哥兒,又引逗了一回。忽見畫童來說:「應二爹來了。」衆婦人都亂走不迭,往李瓶兒那邊去了。

應伯爵走到松墻邊,看見桂姐抱着官哥兒,便道:「好呀!李桂姐在這裡。」故意問道:「你幾時來?」{\pangpi{映前,先說。}}那桂姐走了,說道:「罷麼,怪花子!又不關你事,問怎的?」伯爵道:「好小淫婦兒,不關我事也罷,你且與我個嘴着。」於是摟過來就要親嘴。被桂姐用手只一推,罵道:「賊不得人意怪攮刀子,若不是怕唬了哥子,我這一扇把子打的你……」西門慶走出來看見,說道:「怪狗才,看唬了孩兒!」因教書童:「你抱哥兒送與你六娘去。」那書童連忙接過來。奶子如意兒正在松墻拐角邊等候,接的去了。伯爵和桂姐兩個站着說話,問:「你的事怎樣了?」桂姐道:「多虧爹這裡可憐見,差保哥替我往東京說去了。」伯爵道:「好,好,也罷了。如此你放心些。」說畢,桂姐就往後邊去了。伯爵道:「怪小淫婦兒,你過來,我還和你說話。」桂姐道:「我走走就來。」於是也往李瓶兒這邊來了。伯爵與西門慶纔唱喏坐的。西門慶道:「昨日我在夏龍溪家吃酒,大巡宋道長那裡差人送禮,送了一口鮮豬。{\pangpi{以西門慶口腹,豈嗜一豬?而出之大巡,便覺視為上品異味。人情乎?勢利乎?吾所不解。}}我恐怕放不的,今早旋叫廚子來卸開,用椒料連豬頭燒了。你休去,如今請謝子純來,咱每打雙陸,同享了罷。」一面使琴童兒:「快請你謝爹去。你說應二爹在這裡。」琴童兒應諾去了。伯爵因問:「徐家銀子討來了不曾?」西門慶道:「賊沒行止的狗骨禿,明日纔先與二百五十兩。你教他兩個後日來,少的,我家裡湊與他罷。」伯爵道:「這等又好了。怕不得他今日也買些鮮物兒來孝順你。」西門慶道:「倒不消教他費心。」說了一回,西門慶問道:「老孫、祝麻子兩個都起身去了不曾?」伯爵道:「自從李桂兒家拏出來,在縣裡監了一夜,第二日,三個一條鐵索,都解上東京去了。到那裡,沒個清潔來家的!你只說成日圖飲酒吃肉,好容易吃的菓子兒!似這等苦兒,也是他受。路上這等大熱天,着鐵索扛着,又沒盤纏,有甚麼要緊。」{\meipi{數語蓋為此輩油手幫閑現身說法,不可作戲談閑話,草草看過。}}西門慶笑道:「怪狗才,充軍擺戰的不過!誰教他成日跟着王家小厮只胡撞來!他尋的苦兒他受。」伯爵道:「哥說的有理。蒼蠅不鑽沒縫的雞蛋,他怎的不尋我和謝子純?清的只是清,渾的只是渾。」

正說着,謝希大到了。唱畢喏坐下,只顧扇扇子。西門慶問道:「你怎的走恁一臉汗?」希大道:「哥別題起。今日平白惹了一肚子氣。大清早晨,老孫媽媽子走到我那裡,說我弄了他去。恁不合理的老淫婦!你家漢子成日摽着人在院裡大酒大肉吃,大把撾了銀子錢家去,你過陰去來?誰不知道!你討保頭錢,分與那個一分兒使也怎的?{\meipi{謝希大只同走一遭,便受一遭之累,擇交可不愼哉!}}交我扛了兩句走出來。不想哥這裡呼喚。」伯爵道:「我剛纔和哥不說,新酒放在兩下里,清自清,渾自渾。當初咱每怎麼說來?我說跟着王家小厮,到明日有一失。今日如何?撞到這網裡,怨悵不的人!」西門慶道:「王家那小厮,有甚大氣概?腦子還未變全,養老婆!還不勾俺每那咱撒下的,羞死鬼罷了!」伯爵道:「他曾見過甚麼大頭面目,比哥那咱的勾當,題起來把他唬殺罷了。」{\meipi{一自誇,一旁譽,眞相知。}}說畢,小厮拏茶上來吃了。西門慶道:「你兩個打雙陸。後邊做着水麵,等我叫小厮拏來咱每吃。」

不一時,琴童來放桌兒。畫童兒用方盒拏上四個小菜兒,又是三碟兒蒜汁、一大碗豬肉滷,一張銀湯匙、三雙牙筯。擺放停當,三人坐下,然後拏上三碗麵來,各人自取澆滷,傾上蒜醋。那應伯爵與謝希大拏起筯來,只三扒兩咽就是一碗。兩人登時狠了七碗。西門慶兩碗還吃不了,說道:「我的兒,你兩個吃這些!」伯爵道:「哥,今日這面是那位姐兒下的?又好吃又爽口。」謝希大道:「本等滷打的停當,我只是剛纔吃了飯了,不然我還禁一碗。」兩個吃的熱上來,把衣服脫了。見琴童兒收家活,便道:「大官兒,到後邊取些水來,俺每漱漱口。」謝希大道:「溫茶兒又好,熱的燙的死蒜臭。」少頃,畫童兒拏茶至。三人吃了茶,出來外邊松墻外各花臺邊走了一道。只見黃四家送了四盒子禮來。平安兒掇進來與西門慶瞧:一盒鮮烏菱、一盒鮮荸薺、四尾冰湃的大鰣魚、一盒枇杷果。伯爵看見說道:「好東西兒!他不知那裡剜的送來,我且嚐個兒着。」一手撾了好幾個,遞了兩個與謝希大,說道:「還有活到老死,還不知此是甚麼東西兒哩。」西門慶道:「怪狗才,還沒供養佛,就先撾了吃?」伯爵道:「甚麼沒供佛,我且入口無賍着。」西門慶分咐:「交到後邊收了。問你三娘討三錢銀子賞他。」伯爵問:「是李錦送來,是黃甯兒?」平安道:「是黃甯兒。」伯爵道:「今日造化了這狗骨禿了,又賞他三錢銀子。」{\meipi{此書只一味要打破世情,故不論事之大小冷熱,但世情所有,便一筆刺入。}}這裡西門慶看着他兩個打雙陸不題。

且說月娘和桂姐、李嬌兒、孟玉樓、潘金蓮、李瓶兒、大姐,都在後邊吃了飯,在穿廊下坐的。只見小周兒在影壁前探頭舒腦的,李瓶兒道:「小周兒,你來的好。且進來與小大官兒剃剃頭,他頭髮都長長了。」小周兒連忙向前都磕了頭,說:「剛纔老爹分咐,交小的進來與哥兒剃頭。」月娘道:「六姐,你拏曆頭看看,好日子,歹日子,就與孩子剃頭?」金蓮便交小玉取了曆頭來,揭開看了一回,說道:「今日是四月廿一日,是個庚戌日,金定婁金狗當直,宜祭祀、官帶、出行、裁衣、沐浴、剃頭、修造、動土,宜用午時。好日期。」月娘道:「既是好日子,叫丫頭熱水,你替孩兒洗頭,教小周兒慢慢哄着他剃。」{\meipi{看了好日子剃頭,卻幾乎將孩子剃殺,陰陽可信乎?不可信乎?微詞逗出。}}小玉在旁替他用汗巾兒接着頭髮,纔剃得幾刀,這官哥兒呱的怪哭起來。那小周連忙趕着他哭只顧剃,不想把孩子哭的那口氣憋下去,不做聲了,臉便脹的紅了。李瓶兒唬慌手脚,連忙說:「不剃罷,不剃罷!」那小周兒唬的收不迭家活,往外沒脚的跑。月娘道:「我說這孩予有些不長俊,護頭。自家替他剪剪罷。平白教進來剃,剃的好麼!」天假其便,那孩子憋了半日氣,纔放出聲來。李瓶兒方纔放心,只顧拍哄他,說道:「好小周兒,恁大膽!平白進來把哥哥頭來剃了去了。剃的恁半落不合的,欺負我的哥哥。還不拏回來,等我打與哥哥出氣。」{\meipi{開口便如天造地設,絕無一語杜撰,所以為妙。}}於是抱到月娘跟前。月娘道:「不長俊的小花子兒,剃頭耍了你了,這等哭?剩下這些,到明日做剪毛賊。」引逗了一回,李瓶兒交與奶子。月娘分咐:「且休與他奶吃,等他睡一回兒與他吃。」奶子抱的前邊去了。只見來安兒進來取小周兒的家活,說唬的小周兒臉焦黃的。月娘問道:「他吃了飯不曾?」來安道:「他吃了飯。爹賞他五錢銀子。」月娘教來安:「你拏一甌子酒出去與他。唬着人家,好容易討這幾個錢!」小玉連忙篩了一盞,拏了一碟臘肉,教來安與他吃了去了。

吳月娘因教金蓮:「你看看曆頭,幾時是壬子日?」金蓮看了,說道:「二十三日是壬子日,交芒種五月節。」便道:「姐姐你問他怎的?」月娘道:「我不怎的,問一聲兒。」李桂姐接過歷頭來看了,說道:「這二十四日,苦惱是俺娘的生日!{\meipi{月娘悠然接上,妙在個中;桂姐突然插入,趣在言外。讀而噴飯者,猶只解得此文一半。}}我不得在家。」月娘道:「前月初十日,是你姐姐生日,過了。這二十四日,可哥兒又是你媽的生日了。原來你院中人家一日害兩樣病,做三個生日:日裡害思錢病,黑夜思漢子的病。早晨是媽媽的生日,晌午是姐姐生日,晚夕是自家生日。怎的都擠在一塊兒?趁着姐夫有錢,攛掇着都生日了罷!」桂姐只是笑,不做聲。只見西門慶使了畫童兒來請,桂姐方向月娘房中粧點勻了臉,往花園中來。

捲棚內,又早放下八仙桌兒,桌上擺設兩大盤燒豬肉並許多餚饌。衆人吃了一回,桂姐在旁拏鍾兒遞酒,伯爵道:「你爹聽着說——不是我索落你,人情兒已是停當了。你爹又替你縣中說了,不尋你了。虧了誰?還虧了我再三央及你爹,他纔肯了。平白他肯替你說人情去?{\meipi{數語伯爵猶作戲說,若今人說來,便不以為戲矣。}}隨你心愛的甚麼曲兒,你唱個兒我下酒,也是拏勤勞准折。」桂姐笑罵道:「怪硶花子,你『虼𧒮包網兒——好大面皮』!爹他肯信你說話?」伯爵道:「你這賊小淫婦兒!你經還沒念,就先打和尚。要吃飯,休惡了火頭!你敢笑和尚投丈母,我就單丁擺佈不起你這小淫婦兒?你休笑譁,我半邊俏還動的。」被桂姐把手中扇靶子,侭力向他身上打了兩下。西門慶笑罵道:「你這狗才,到明日論個男盜女娼,還虧了原問處。」笑了一回,桂姐慢慢纔拏起琵琶,橫担膝上,啟朱唇,露皓齒,唱道:

\begin{myquote} 
{\markfont\small〔黃鶯兒〕}誰想有這一種。減香肌,憔瘦損。鏡鸞塵鎖無心整。脂粉倦勻,花枝又懶簪。空教黛眉蹙破春山恨。
\end{myquote} 

伯爵道:「你兩個當初好來,如今就為他耽些驚怕兒,也不該抱怨了。」桂姐道:「汗邪了你,怎的胡說!」

\begin{myquote} 
最難禁,樵樓上畫角,吹徹了斷腸聲。
\end{myquote} 

伯爵道:「腸子倒沒斷,這一回來提你的斷了線,你兩個休提了。」被桂姐盡力打了一下,罵道:「賊攘刀的,今日汗邪了你,只鬼混人的。」

\begin{myquote} 
{\markfont\small〔集資賓〕}幽窓靜悄月又明,恨獨倚幃屏。驀聽的孤鴻只在樓外鳴,把萬愁又還題醒。更長漏永,早不覺燈昏香燼眠未成。他那裡睡得安穩!
\end{myquote} 

伯爵道:「傻小淫婦兒,他怎的睡不安穩?又沒拏了他去。落的在家裡睡覺兒哩。你便在人家躲着,逐日懷着羊皮兒,直等東京人來,一塊石頭方落地。」桂姐被他說急了,便道:「爹,你看應花子,不知怎的,只發訕纏我。」伯爵道:「你這回纔認的爹了?」{\meipi{搶白得妙,奉承得巧,伯爵殊有竅。}}桂姐不理他,彈着琵琶又唱:

\begin{myquote} 
{\markfont\small〔雙聲疊韻〕}思量起,思量起,怎不上心?無人處,無人處,淚珠兒暗傾。
\end{myquote} 

伯爵道:「一個人慣溺尿。一日,他娘死了,守孝打鋪在靈前睡。晚了,不想又溺下了。人進來看見褥子濕,問怎的來,那人沒的回答,只說:『你不知,我夜間眼淚打肚裡流出來了。』就和你一般,為他聲說不的,只好背地哭罷了。」桂姐道:「沒羞的孩兒,你看見來?汗邪了你哩!」

\begin{myquote}
我怨他,我怨他,說他不盡,誰知道這裡先走滾。自恨我當初不合他認眞。
\end{myquote}

伯爵道:「傻小淫婦兒,如今年程,三歲小孩兒也哄不動,何況風月中子弟。你和他認眞?你且住了,等我唱個南曲兒你聽:『風月事,我說與你聽:如今年程,論不得假眞。個個人古怪精靈,個個人久慣牢成,倒將計活埋把瞎缸暗頂。老虔婆只要圖財,小淫婦兒少不得拽着脖子往前掙。苦似投河,愁如覓並。幾時得把業礶子塡完,就變驢變馬也不幹這營生。』」當下把桂姐說的哭起來了。{\meipi{伯爵戲搶桂姐,似乎沒趣,不知桂姐此事非西門慶所喜,特留情不言耳。西門慶不言而伯爵代言之,正是大湊趣。}}被西門慶向伯爵頭上打了一扇子,笑罵道:「你這搊斷腸子的狗才!生生兒吃你把人就歐殺了。」因叫桂姐:「你唱,不要理他。」謝希大道:「應二哥,你好沒趣!今日左來右去只欺負我這乾女兒。你再言語,口上生個大疔瘡。」那桂姐半日拏起琵琶,又唱:

\begin{myquote}
{\markfont\small〔簇御林〕}人都道他志誠。
\end{myquote}

伯爵纔待言語,被希大把口按了,{\meipi{又白描一曲,情景宛然。}}說道:「桂姐你唱,休理他!」桂姐又唱道:

\begin{myquote}
卻原來厮勾引。眼睜睜心口不相應。
\end{myquote}

希大放了手,伯爵又說:「相應倒好了。心口裡不相應,如今虎口裡倒相應。不多,也只三兩炷兒。」桂姐道:「白眉赤眼,你看見來?」伯爵道:「我沒看見,在樂星堂兒裡不是?」連西門慶衆人都笑起來了。桂姐又唱:

\begin{myquote}
山盟海誓,說假道眞,險些兒不為他錯害了相思病。負人心,看伊家做作,如何教我有前程?
\end{myquote}

伯爵道:「前程也不敢指望他,到明日,少不了他個招宣襲了罷。」桂姐又唱:

\begin{myquote}
{\markfont\small〔琥珀貓兒墜〕}日疎日遠,何日再相逢?枉了奴癡心寧耐等。想巫山雲雨夢難成。薄情,猛拚今生和你鳳拆鸞零。

{\markfont\small〔尾聲〕}冤家下得忒薄倖,割捨的將人孤另。那世裡的恩情翻成做話餅。
\end{myquote}

唱畢,謝希大道:「罷,罷。叫畫童兒接過琵琶去,等我酬勞桂姐一杯酒兒,消消氣罷。」{\meipi{桂姐自家理短,不敢十分認眞,若平日,不知如何拌嘴矣。}}伯爵道:「等我哺菜兒。我本領兒不濟事,拏勤勞准折罷了。」桂姐道:「花子過去,誰理你!你大拳打了人,這回拏手來摸挲。」當下,希大一連遞了桂姐三杯酒,拉伯爵道:「咱每還有那兩盤雙陸,打了罷。」於是二人又打雙陸。西門慶遞了個眼色與桂姐,就往外走。伯爵道:「哥,你往後邊去,稍些香茶兒出來。頭裡吃了些蒜,這回子倒反惡泛泛起來了。」西門慶道:「我那裡得香茶來!」伯爵道:「哥,你還哄我哩,杭州劉學官送了你好少兒,你獨吃也不好。」西門慶笑的後邊去了。桂姐也走出來,在太湖石畔推摘花兒戴,也不見了。伯爵與希大一連打了三盤雙陸,等西門慶白不見出來。問畫童兒:「你爹在後邊做甚麼哩?」畫童兒道:「爹在後邊,就出來了。」伯爵道:「就出來,有些古怪!」因交謝希大:「你這裡坐着,等我尋他尋去。」那謝希大且和書童兒兩個下象棋。

原來西門慶只走到李瓶兒房裡,吃了藥就出來了。在木香棚下看見李桂姐,就拉到藏春塢雪洞兒裡,把門兒掩着,坐在矮床兒上,把桂姐摟在懷中,腿上坐的,一徑露出那話來與他瞧,把桂姐唬了一跳。便問:「怎的就這般大?」西門慶悉把吃胡僧藥告訴了一遍。先交他低垂粉頸,款啟猩唇,品咂了一回。然後,輕輕搊起他兩隻小小金蓮來,跨在兩邊胳膊上,抱到一張椅兒上,兩個就幹起來。不想應伯爵到各亭兒上尋了一遭,尋不着,打滴翠巖小洞兒裡穿過去,到了木香棚,抹過葡萄架,到松竹深處,藏春塢邊,隱隱聽見有人笑聲,又不知在何處。這伯爵慢慢躡足潛蹤,掀開簾兒,見兩扇洞門兒虛掩,在外面只顧聽覷。聽見桂姐顫着聲兒,將身子只顧迎播着西門慶,叫:「達達,快些了事罷,只怕有人來。」被伯爵猛然大叫一聲,推開門進來,看見西門慶把桂姐扛着腿子正幹得好。說道:「快取水來,潑潑兩個摟心的,摟到一答裡了!」李桂姐道:「怪攘刀子,猛的進來,唬了我一跳!」伯爵道:「快些兒了事?好容易!也得値那些數兒是的。怕有人來看見,我就來了。且過來,等我抽個頭兒着。」{\meipi{情中着一痕屑子,便格格不化,西門慶與桂姐雖歡私如故,而實無心可談,故借伯爵一混,草草完事。}}西門慶便道:「怪狗才,快出去罷了,休鬼混!我只怕小厮來看見。」那應伯爵道:「小淫婦兒,你央及我央及兒。不然我就喓喝起來,連後邊嫂子每都嚷的知道。你既認做乾女兒了,好意教你躲住兩日兒,你又偸漢子。教你了不成!」桂姐道:「去罷,應怪花子!」伯爵道:「我去罷?我且親個嘴着。」於是按着桂姐親了一個嘴,纔走出來。西門慶道:「怪狗才,還不帶上門哩。」伯爵一面走來把門帶上,說道:「我兒,兩個盡着搗,盡着搗,搗弔底也不關我事。」纔走到那個松樹兒底下,又回來說道:{\meipi{又作餘波。}}「你頭裡許我的香茶在那裡?」西門慶道:「怪狗才,等住回我與你就是了,又來纏人!」那伯爵方纔一直笑的去了。桂姐道:「好個不得人意的攮刀子!」這西門慶和那桂姐兩個,在雪洞內足幹勾一個時辰,吃了一枚紅棗兒,纔得了事,雨散雲收。有詩為證:

\begin{myquote}
海棠枝上鶯梭急,綠竹陰中燕語頻。\\閑來付與丹青手,一段春嬌畫不成。
\end{myquote}

少頃,二人整衣出來。桂姐向他袖子內掏出好些香茶來袖了。西門慶使的滿身香汗,氣喘吁吁,走來馬纓花下溺尿。李桂姐腰裡摸出鏡子來,{\pangpi{妙。}}在月窓上擱着,整雲理髩,往後邊去了。

西門慶走到李瓶兒房裡,洗洗手出來。伯爵問他要香茶,西門慶道:「怪花子,你害了痞,如何只鬼混人!」每人掐了一撮與他。伯爵道:「只與我這兩個兒!繇他,繇他!等我問李家小淫婦兒要。」正說着,只見李銘走來磕頭。伯爵道:「李日新在那裡來?你沒曾打聽得他每的事怎麼樣兒了?」李銘道:「俺桂姐虧了爹這裡。這兩日,縣裡也沒人來催,只等京中示下哩。」伯爵道:「齊家那小老婆子出來了?」李銘道:「齊香兒還在王皇親宅內躲着哩。桂姐在爹這裡好,誰人敢來尋?」伯爵道:「要不然也費手,虧我和你謝爹再三央勸你爹:『你不替他處處兒,教他那裡尋頭腦去?』」李銘道:「爹這裡不管,就了不成。俺三嬸老人家,風風勢勢的,幹出甚麼事!」伯爵道:「我記的這幾時是他生日,{\meipi{偏他記得。}}俺每會了你爹,與他做做生日。」{\pangpi{就伸脚兒。}}李銘道:「爹每不消了。到明日事情畢了,三嬸和桂姐,愁不請爹每坐坐?」伯爵道:「到其間,俺每補生日就是了。」{\pangpi{便安根。}}因叫他近前:「你且替我吃了這鍾酒着。我吃了這一日,吃不的了。」那李銘接過銀把鍾來,跪着一飲而盡。謝希大交琴童又斟了一鍾與他。伯爵道:「你敢沒吃飯?」桌上還剩了一盤點心,謝希大又拏兩盤燒豬頭肉和鴨子遞與他。李銘雙手接的,下邊吃去了。伯爵用筯子又撥了半段鰣魚與他,說道:「我見你今年還沒食這個哩,且嘗新着。」西門慶道:「怪狗才,都拏與他吃罷了,又留下做甚麼?」伯爵道:「等住回吃的酒闌,上來餓了,我不會吃飯兒?你們那裡曉得,江南此魚一年只過一遭兒,吃到牙縫裡剔出來都是香的。好容易!公道說,就是朝廷還沒吃哩!不是哥這裡,誰家有?」{\meipi{要為李三表情,故有許多比論。}}正說着,只見畫童兒拏出四碟鮮物兒來:一碟烏菱、一碟荸薺、一碟雪藕、一碟枇杷。西門慶還沒曾放到口裡,被應伯爵連碟子都撾過去,倒的袖了。謝希大道:「你也留兩個兒我吃。」也將手撾一碟子烏菱來。只落下藕在桌子上。西門慶掐了一塊放在口內,別的與了李銘吃了。分付畫童後邊再取兩個枇杷來賞李銘。李銘接的袖了,纔上來拏箏彈唱。唱了一回,伯爵又出題目,叫他唱了一套《花葯欄》。

三個直吃到掌燈時候,還等後邊拏出綠荳白米水飯來吃了,纔起身。伯爵道:「哥,我曉得明日安主事請你,不得閑。李四、黃三那事,我後日會他來罷。」{\meipi{醉則醉,事在心頭。}}西門慶點頭兒,二人也不等送,就去了。西門慶教書童看收家伙,就歸後邊孟玉樓房中歇去了。一宿無話。

到次日早起,也沒往衙門中去,吃了粥,冠帶騎馬,書童、玳安兩個跟隨,出城南三十里,逕往劉太監庄上來赴席,不在話下。

潘金蓮趕西門慶不在家,與李瓶兒計較,將陳敬濟輸的那三錢銀子,又教李瓶兒添出七錢來,教來興兒買了一隻燒鴨、兩隻雞、一錢銀子下飯、一罈金華酒、一瓶白酒、一錢銀子裹餡涼糕,教來興兒媳婦整理端正。金蓮對着月娘說:「大姐那日鬬牌,贏了陳姐夫三錢銀子,李大姐又添了些,今治了東道兒,請姐姐在花園裡吃。」吳月娘就同孟玉樓、李嬌兒、孫雪娥、大姐、桂姐衆人,先在捲棚內吃了一回,然後拏酒菜兒,在山子上臥雲亭下棋,投壺,吃酒耍子。月娘想起問道:「今日主人,怎倒不來坐坐?」大姐道:「爹又使他往門外徐家催銀子去了,也好待來也。」

不一時,陳敬濟來到,向月娘衆人作了揖,就拉過大姐一處坐下。向月娘說:「徐家銀子討了來了,共五封二百五十兩,送到房裡,玉簫收了。」於是傳杯換盞,酒過數巡,各添春色。月娘與李嬌兒、桂姐三個下棋,玉樓衆人都起身向各處觀花玩草耍子。惟金蓮獨自手搖着白團紗扇兒,往山子後芭蕉深處納涼。{\meipi{獨自靜處走,未必無心。}}因見墻角草地下一朵野紫花兒可愛,便走去要摘。不想敬濟有心,一眼睃見,便悄悄跟來,{\meipi{柔情一牽,便不約而至。}}在背後說道:「五娘,你老人家尋甚麼?這草地上滑齏齏的,只怕跌了你,教兒子心疼。」{\pangpi{油嘴。}}那金蓮扭回粉頸,斜睨秋波,帶笑帶罵道:「好個賊短命的油嘴,跌了我,可是你就心疼哩?誰要你管!你又跟了我來做甚麼,也不怕人看着。」因問:「你買的汗巾兒怎了?」敬濟笑嘻嘻向袖於中取出,遞與他,說道:「六娘的都在這裡了。」又道:「汗巾兒買了來,你把甚來謝我?」於是把臉子挨的他身邊,被金蓮舉手只一推。不想李瓶兒抱着官哥兒,並奶子如意兒跟着,從松墻那邊走來。見金蓮手拏白團扇一動,不知是推敬濟,只認做撲蝴蝶,忙叫道:「五媽媽,撲的蝴蝶兒,把官哥兒一個耍子。」慌的敬濟趕眼不見,兩三步就鑽進山子裡邊去了。金蓮恐怕李瓶兒瞧見,故意問道:「陳姐夫與了汗巾不曾?」{\meipi{問得賊甚。瞧見不瞧見,都好轉嘴。}}李瓶兒道:「他還沒有與我哩。」金蓮道:「他剛纔袖着,對着大姐姐不好與咱的,悄悄遞與我了。」於是兩個坐在芭蕉叢下花臺石上,開啟分了。兩個坐了一回,李瓶兒說道:「這答兒裡到且是蔭涼。」因使如意兒:「你去叫迎春屋裡取孩子的小枕頭並涼蓆兒來,就帶了骨牌來,我和五娘在這裡抹回骨牌兒。你就在屋裡看罷。」如意兒去了。

不一時,迎春取了枕蓆併骨牌來。李瓶兒鋪下蓆,把官哥兒放在小枕頭兒上躺着,教他頑耍,他便和金蓮抹牌。抹了一回,交迎春往屋裡拏一壺好茶來。不想孟玉樓在臥雲亭上看見,點手兒叫李瓶兒說:「大姐姐叫你說句話兒。」李瓶兒撇下孩子,教金蓮看着:{\pangpi{瓶兒疎畧之甚。}}「我就來。」那金蓮記掛敬濟在洞兒裡,那裡又去顧那孩子,趕空兒兩三步走入洞門首,教敬濟,說:「沒人,你出來罷。」敬濟便叫婦人進去瞧蘑菇:「裡面長出這些大頭蘑菇來了。」哄的婦人入到洞裡,就摺疊腿跪着,要和婦人雲雨。兩個正接着親嘴。也是天假其便,李瓶兒走到亭子上,月娘說:「孟三姐和桂姐投壺輸了,你來替他投兩壺兒。」李瓶兒道:「底下沒人看孩子哩。」玉樓道:「左右有六姐在那裡,怕怎的。」月娘道:「孟三姐,你去替他看看罷。」{\pangpi{畢竟月娘深心。}}李瓶兒道:「三娘累你,亦發抱了他來罷。」{\meipi{瓶兒東便東,西便西,大沒主意。}}教小玉:「你去就抱他的蓆和小枕頭兒來。」那小玉和玉樓走到芭蕉叢下,孩子便躺在蓆上,蹬手蹬脚的怪哭,並不知金蓮在那裡。只見旁邊一個大黑貓,見人來,一溜烟跑了。玉樓道:「他五娘那裡去了?耶嚛,耶嚛!把孩子丟在這裡,吃貓唬了他了。」那金蓮連忙從雪洞兒裡鑽出來,說道:「我在這裡淨了淨手,誰往那裡去來!那裡有貓唬了他?白眉赤眼的!」那玉樓也更不往洞裡看,只顧抱了官哥兒,拍哄着他往臥雲亭兒上去了。小玉拏着枕蓆跟的去了。金蓮恐怕他學舌,隨屁股也跟了來。{\pangpi{偏有此賊智。}}月娘問:「孩子怎的哭?」玉樓道:「我去時,不知是那裡一個大黑貓蹲在孩子頭跟前。」月娘說:「乾淨唬着孩兒。」李瓶兒道,「他五娘看着他哩。」玉樓道:「六姐往洞兒裡淨手去來。」金蓮走上來說:「三姐,你怎的恁白眉赤眼兒的?那裡討個貓來!他想必餓了,要奶吃哭,就賴起人來。」{\pangpi{不得不賴。}}李瓶幾見迎春拏上茶來,就使他叫奶子來喂哥兒奶。陳敬濟見無人,從洞兒鑽出來,順着松墻兒轉過捲棚,一直往外去了。正是:

\begin{myquote}
兩手劈開生死路。一身跳出是非門。
\end{myquote}

月娘見孩子不吃奶,只是哭,分咐李瓶兒:「你抱他到屋裡,好好打發他睡罷。」於是也不吃酒,衆人都散了。原來陳敬濟也不曾與潘金蓮得手,事情不巧,歸到前邊廂房中,有些咄咄不樂。正是:

\begin{myquote}
無可奈何花落去,似曾相識燕歸來。
\end{myquote}

