\includepdf[pages={191,192},fitpaper=false]{tst.pdf}
\chapter*{第九十六回 春梅姐遊舊家池舘 楊光彥作當面豺狼}
\addcontentsline{toc}{chapter}{第九十六回 春梅姐遊舊家池舘 楊光彥作當面豺狼}
\markboth{{\titlename}卷之十}{第九十六回 春梅姐遊舊家池舘 楊光彥作當面豺狼}


詞曰:

\begin{myquote}
人生千古傷心事,還唱《後庭花》。舊時王謝,堂前燕子,飛向誰家?恍然一夢,仙肌勝雪,宮鬢堆鴉。江州司馬,青衫淚濕,想在天涯。

\raggedleft{——右調《青衫濕》\rightquadmargin}
\end{myquote}

話說光陰迅速,日月如梭,又早到正月二十一日。春梅和周守備說了,備一張祭桌,四樣羹菓,一罈南酒,差家人周義送與吳月娘。一者是西門慶三週年,二者是孝哥兒生日。月娘收了禮物,打發來人帕一方,銀三錢。這邊連忙就使玳安兒穿青衣,具請書兒請去。上寫着:

\begin{myquote}[\markfont]
重承厚禮,感感。即刻舍具菲酌,奉酬腆儀。仰希高軒俯臨,不外,幸甚。

\raggedleft{西門吳氏端肅拜請\rightquadmargin}

大德周老夫人粧次
\end{myquote}

春梅看了,到日中纔來。戴着滿頭珠翠金鳳頭面釵梳,胡珠環子。身穿大紅通袖、四獸朝麒麟袍兒,翠藍十樣錦百花裙,玉玎璫禁步,束着金帶。坐着四人大轎,青段銷金轎衣。軍牢執藤棍喝道,家人伴當跟隨,擡着衣匣。後邊兩頂家人媳婦小轎兒,緊緊跟隨。{\meipi{相如駟馬高車,不過如此。}}吳月娘這邊請了吳大妗子相陪,又叫了四箇唱的彈唱。聽見春梅來到,月娘亦盛粧縞素打扮,頭上五梁冠兒,戴着稀稀幾件金翠首飾,{\pangpi{正宜如此。}}上穿白綾襖,下邊翠藍段子裙,與大妗子迎接至前廳。春梅大轎子擡至儀門首,纔落下轎來。兩邊家人圍着,到於廳上叙禮,向月娘插燭也似拜下去。月娘連忙答禮相見,說道:「向日有累姐姐費心,粗尺頭又不肯受。今又重承厚禮祭桌,感激不盡。」春梅道:「惶恐。家官府沒甚麼,這些薄禮,表意而已。一向要請奶奶過去,家官府不時出巡,所以不曾請得。」月娘道:「姐姐,你是幾時好日子?我只到那日買禮看姐姐去罷。」{\meipi{行禮未畢,且忙問生日,似親熱而愈見往日之疎。}}春梅道:「奴賤日是四月廿五日。」月娘道:「奴到那日已定去。」兩箇叙禮畢,春梅務要把月娘讓起,受了兩禮。然後吳大妗子相見,亦還下禮去。春梅道:「你看大妗子,又沒正經。」一手扶起受禮。大妗子再三不肯,止受了半禮。一面讓上坐,月娘和大妗子主位相陪。然後家人、媳婦、丫鬟、養娘,都來參見。春梅見了奶子如意兒抱着孝哥兒,吳月娘道:「小大哥還不來與姐姐磕箇頭兒,謝謝姐姐。今日來與你做生日。」那孝哥兒眞箇下如意兒身來,與春梅唱喏。月娘道:「好小厮,不與姐姐磕頭,只唱喏。」那春梅連忙向袖中摸出一方錦手帕,一副金八吉祥兒,教替他㩟帽兒上。月娘道:「又教姐姐費心。」又拜謝了。落後小玉、奶子來見磕頭。春梅與了小玉一對金頭簪子,{\pangpi{不足報德。}}與了奶子兩枝銀簪兒。月娘道:「姐姐,你還不知,奶子與了來興兒做媳婦兒了。來興兒那媳婦害病沒了。」春梅道:「他一心要在咱家,倒也好。」{\pangpi{語着痛癢。}}一面丫鬟拿茶上來,吃了茶,月娘道:「請娘娘後邊明間內坐罷,這客位內冷。」春梅來後邊西門慶靈前,又早點起燈燭,擺下桌面祭禮。春梅燒了紙,落了幾點眼淚。{\meipi{春梅此時哭纔情深,不哭則情淺,落幾點眼淚,不深不淺,最得其情。}}然後周圍設放圍屏,火爐內生起炭火,安放八大仙桌席,擺茶上來。無非是細巧蒸酥,希奇菓品,絕品芽茶。月娘和大妗子陪着吃了茶,讓春梅進上房裡換衣裳。脫了上面袍兒,家人媳婦開衣匣,取出衣服,更換了一套綠遍地錦粧花襖兒,紫丁香色遍地金裙。在月娘房中坐着,說了一回,月娘因問道:「哥兒好麼?今日怎不帶他來這裡走走?」春梅道:「不是也帶他來與奶奶磕頭,他爺說天氣寒冷,怕風冒着他。他又不肯在房裡,只要那當直的抱出來廳上外邊走。{\pangpi{又映前。}}這兩日,不知怎的,只是哭。」月娘道:「他周爺也好大年紀,得你替他養下這點孩子也彀了,也是你裙帶上的福。說他孫二娘還有位姐兒,幾歲兒了?」春梅道:「他二娘養的叫玉姐,今年交生四歲。俺這箇叫金哥。」月娘道:「說他周爺身邊還有兩位房裡姐兒?」春梅道:「是兩箇學彈唱的丫頭子,都有十六七歲,成日淘氣在那裡。」月娘道:「他爺也常往他身邊去不去?」{\meipi{月娘只以西門慶行事作榜樣看天下人,所以語語滯呆。}}春梅道:「奶奶,他那裡得工夫在家?多在外,少在裡。如今四外好不盜賊生發,朝廷勑書上,又教他兼管許多事情:鎭守地方,巡理河道,提拿盜賊,操練人馬。常不時往外出巡幾遭,好不辛苦哩。」說畢,小玉又拿茶來吃了。春梅向月娘說:「奶奶,你引我往俺娘那邊花園山子下走走。」月娘道:「我的姐姐,還是那咱的山子花園哩!自從你爹下世,沒人收拾他,如今丟搭的破零零的。石頭也倒了,樹木也死了,俺等閑也不去了。」春梅道:「不妨,奴就往俺娘那邊看看去。」這月娘強不過,只得叫小玉拿花園門山子門鑰匙,開了門,月娘、大妗子陪春梅,到裡邊遊看了半日。但見:

\begin{myquote}
垣墻欹損,臺榭歪斜。兩邊畫壁長青笞,滿地花磚生碧草。山前怪石遭塌毀,不顯嵯峨;亭內涼床被滲漏,已無框檔。石洞口蛛絲結網,魚池內蝦䗫成羣。狐狸常睡臥雲亭,黃鼠往來藏春閣。料想經年無人到,也知盡日有雲來。
\end{myquote}

春梅看了一回,先走到李瓶兒那邊。見樓上丟着些折桌、壞櫈、破椅子,下邊房都空鎖着,地下草長的荒荒的。{\meipi{燕去巢空,一片荒涼情境,那能不傷心墮淚。}}方來到他娘這邊,樓上還堆着些生藥香料,下邊他娘房裡,止有兩座廚櫃,床也沒了。因問小玉:「俺娘那張床往那去了?怎的不見?」小玉道:「俺三娘嫁人,賠了俺三娘去了。」月娘走到跟前說:「因你爹在日,將他帶來那張八步床賠了大姐在陳家,落後他起身,卻把你娘這張床賠了他,嫁人去了。」春梅道:「我聽見大姐死了,說你老人家把床還擡的來家了。」月娘道:「那床沒錢使,只賣了八兩銀子,打發縣中皁隸,都使了。」春梅聽言,點了點頭兒。那星眼中繇不的酸酸的,口中不言,心內暗道:「想着俺娘那咱,爭強不伏弱的問爹要買了這張床。我實承望要回了這張床去,也做他老人家一念兒,不想又與了人去了。」繇不的心下慘切。{\meipi{春梅眷懷今昔,不減黍離之悲。}}又問月娘:「俺六娘那張螺甸床怎的不見?」月娘道:「一言難盡。自從你爹下世,日逐只有出去的,沒有進來的。常言家無營活計,不怕斗量金。也是家中沒盤纏,擡出去交人賣了。」春梅問:「賣了多少銀子?」月娘道:「止賣了三十五兩銀子。」春梅道:「可惜了,那張床,當初我聽見爹說,値六十兩多銀子,只賣這些兒。早知你老人家打發,我到與你老人家三四十兩銀子要了也罷。」月娘道:「好姐姐,人那有早知道的?」一面嘆息了半日。只見家人周仁走來接,說:「爺請奶奶早些家來,哥兒尋奶奶哭哩。」這春梅就抽身往後邊來。月娘叫小玉鎖了花園門,同來到後邊明間內。又早屏開孔雀,簾控鮫綃,擺下酒筵。兩箇妓女,銀箏琵琶,在旁彈唱。吳月娘遞酒安席,安春梅上座,{\meipi{昔年下婢,今日上賓,為正乎,為僭乎?所不辨也。}}春梅不肯,務必拉大妗子,同他一處坐的。月娘主位,筵前遞了酒,湯飯點心,割切上席。春梅叫家人周仁,賞了廚子三錢銀子。說不盡盤堆異品,酒泛金波。當下傳盃換盞,吃至晚色將落時分,只見宅內又差伴當,拿燈籠來接。月娘那裡肯放,教兩箇妓女在跟前跪着彈唱勸酒。分付:「你把好曲兒孝順你周奶奶一箇兒。」一面叫小玉斟上大鍾,放在跟前,說:「姐姐,你分付箇心愛的曲兒,叫他兩箇唱與你下酒。」春梅道:「奶奶,奴吃不得了,怕孩兒家中尋我。」月娘道:「哥兒尋,左右有奶子看着,天色也還早哩,我曉得你好小量兒!」春梅因問那兩箇妓女:「你叫甚名字?是誰家的?」兩箇跪下說:「小的一箇是韓金釧兒妹子韓玉釧兒,一箇是鄭愛香兒姪女鄭嬌兒。」春梅道:「你每會唱《懶畫眉》不會?」玉釧兒道:「奶奶分付,小的兩箇都會。」月娘道:「你兩箇既會唱,斟上酒你周奶奶吃,你每慢唱。」小玉在旁連忙斟上酒,兩箇妓女,一箇彈箏,一箇琵琶,唱道:

\begin{myquote}
冤家為你幾時休?捱過春來又到秋。誰人知道我心頭。天,害的我伶仃瘦,聽的音書兩淚流。從前已往訴緣繇,誰想你無情把我丟!
\end{myquote}

那春梅吃過,月娘又令鄭嬌兒遞上一盃酒與春梅。春梅道:「你老人家也陪我一盃。」兩家于是都齊斟上,兩箇妓女又唱道:

\begin{myquote}
冤家為你減風流,鵲噪簷前不肯休,死聲活氣沒來繇。天,倒惹的情拖逗,助的淒涼兩淚流。從他去後意無休,誰想你辜恩把我丟。
\end{myquote}

春梅說:「奶奶,你也教大妗子吃盃兒。」月娘道:「大妗子吃不的,教他拿小鍾兒陪你罷。」一面令小玉斟上大妗子一小鍾兒酒。兩箇妓女又唱道:

\begin{myquote}
冤家為你惹場憂,坐想行思日夜愁,香肌憔瘦減溫柔。天,要見你不能勾,悶的我傷心兩淚流。從前與你共綢繆,誰想你今番把我丟。
\end{myquote}

春梅見小玉在跟前,也斟了一大鍾教小玉吃。{\meipi{今日見春梅,惟小玉不愧。}}月娘道:「姐姐,他吃不的。」春梅道:「奶奶,他也吃兩三鍾兒,我那咱在家裡沒和他吃?」{\pangpi{明言之,愈見其高。}}于是斟上,教小玉也吃了一盃。妓女唱道:

\begin{myquote}
冤家為你惹閒愁,病枕着床無了休,滿腹憂悶鎖眉頭。天,忘了還依舊,助的我腮邊兩淚流。從前與你兩無休,誰想你經年把我丟。
\end{myquote}

看官聽說,當時春梅為甚教妓女唱此詞?一向心中牽掛陳敬濟,在外不得相會。情種心苗,故有所感,發於吟詠。又見他兩箇唱的口兒甜,乖覺,奶奶長、奶奶短奉承,心中歡喜。叫家人周仁近前來,拿出兩包兒賞賜來,每人二錢銀子。兩箇妓女放下樂器,磕頭謝了。不一時,春梅起身,月娘款留不住。伴當打燈籠,拜辭出門,坐上大轎。家人媳婦,都坐上小轎。前後打着四箇燈籠,軍牢喝道而去。正是:時來頑鐵有光輝,運去黃金無艷色。有詩為證:

\begin{myquote}
點絳唇紅弄玉嬌,鳳凰飛下品鸞簫。\\堂高閑把湘簾捲,燕子還來續舊巢。
\end{myquote}

且說春梅自從來吳月娘家赴席之後,因思想陳敬濟,不知流落在何處。{\meipi{感金蓮而思敬濟,情生情轉,默默自知。}}歸到府中,終日只是臥床不起,心下沒好氣。守備察知其意,{\pangpi{也只察得一半。}}說道:「只怕思念你兄弟,不得其所。」一面叫張勝、李安來,分付道:「我一向委你尋你奶奶兄弟,如何不用心找尋?」二人告道:「小的一向找尋來,一地裡尋不着下落,已回了奶奶話了。」守備道:「限你二人五日,若找尋不着,討分曉。」這張勝、李安領了鈞語下來,都帶了愁顏。沿街遶巷,各處留心找問,不題。

話分兩頭。單表陳敬濟自從守備府中打了出來,欲投宴公廟。又聽見人說師父任道士死了,就害怕不敢進廟來,又沒臉兒見杏庵王老,白日裡到處打油飛,夜晚間還鑽入冷鋪中存身。一日,也是合當有事,敬濟正在街上站立,只見鐵指甲楊大郎,頭戴新羅帽兒,身穿白綾襖子,騎着一匹驢兒,揀銀鞍轡,一箇小厮跟隨,正從街心走過來。敬濟認得是楊光彥,便向前一把手,把嚼環拉住,說道:「楊大哥,一向不見。自從清江浦把我半船貨物偷拐走了,我好意往你家問,反吃你兄弟楊二風拿瓦楔鑽破頭,趕着打上我家門來。今日弄的我一貧如洗,你是會搖擺受用。」那楊大郎見陳敬濟已自討吃,便佯佯而笑,說:「今日晦氣,出門撞見瘟死鬼!量你這餓不死賊花子,那裡討半船貨,我拐了你的?你不撒手,須吃我一頓馬鞭子。」敬濟便道:「我如今窮了,你有銀子,與我些盤纏。不然,咱到箇去處講講。」楊大郎見他不放,跳下驢來,向他身上抽了幾鞭子。喝令小厮:「與我撏了這少死的花子去!」那小厮使力把敬濟推了一交,楊大郎又向前踢了幾脚,踢打的敬濟怪叫。須臾,圍了許多人。旁邊閃過一箇人來,青高裝帽子,勒着手帕,倒披紫襖,白布褲子,精着兩條腿,靸着蒲鞋,生的阿兜眼,掃帚眉,料綽口,三鬚鬍子,面上紫肉橫生,手腕橫觔競起。吃的楞楞睜睜,提着拳頭,向楊大郎說道:「你此位哥好不近理,他年少這般貧寒,你只顧打他怎的?{\meipi{不意此等形象,卻風流而有情,觀人難哉!}}自古嗔拳不打笑面,他又不曾傷犯着你。你有錢,看平日相交,與他些;沒錢罷了,如何只顧打他?自古路見不平,也有向燈向火。」楊大郎說:「你不知,他賴我拐了他半船貨,量他恁窮樣,那有半船貨物?」那人道:「想必他當時也是有根基人家娃娃,天生就這般窮來?閣下就是這般有錢?老兄依我,你有銀子與他些盤纏罷。」那楊大郎見那人說了,袖內汗巾兒上拴着四五錢一塊銀子,解下來遞與敬濟,與那人舉一舉手兒,上驢子揚長去了。敬濟地下扒起來,擡頭看那人時,不是別人,卻是舊時同在冷鋪內,和他一鋪睡的土作頭兒飛天鬼侯林兒。近來領着五十名人,在城南水月寺曉月長老那裡做工,起蓋伽藍殿。因一隻手拉着敬濟說道:「兄弟,剛纔若不是我拿幾句言語譏犯他,他肯拿出這五錢銀子與你?那賊卻知見範,他若不知範時,好不好吃我一頓好拳頭。你跟着我,咱往酒店內吃酒去來。」到一箇食葷小酒店,案頭上坐下,叫量酒:「拿四賣嗄飯,兩大壺酒來。」不一時,量酒擺下小菜嗄飯,四盤四碟,兩大坐壺時興橄欖酒。不用小盃,拿大磁甌子,因問敬濟:「兄弟,你吃麵吃飯?」量酒道:「麵是溫淘,飯是白米飯。」敬濟道:「我吃麵。」須臾,掉上兩三碗溫麵上來。侯林兒只吃一碗,敬濟吃了兩碗。然後吃酒。侯林兒向敬濟說:「兄弟,你今日跟我往坊子裡睡一夜,{\pangpi{先講明,妙。}}明日我領你城南水月寺曉月長老那裡,修蓋伽藍殿,並兩廊僧房。你哥率領着五十名做工。你到那裡,不要你做重活,只擡幾筐土兒就是了,也算你一工,討四分銀子。我外邊賃着一間廈子,晚夕咱兩箇就在那裡歇,{\pangpi{口不放鬆。}}做些飯打發咱的人吃。把門你一把鎖了,家當都交與你,{\meipi{窮話富說,可發一笑。然敬濟當此飢寒切膚之時,有此遭際,雖眞謂之富貴可也。}}好不好?強如你在那冷鋪中,替花子搖鈴打梆,這箇還官樣些。」敬濟道:「若是哥哥這般下顧兄弟,可知好哩。不知這工程做的長遠不長遠?」侯林兒道:「纔做了一箇月。這工程做到十月裡,不知完不完。」兩箇說話之間,你一鍾,我一盞,把兩大壺酒都吃了。量酒算帳,該一錢三分半銀子。敬濟就要拿出銀子來秤,侯林兒推過一邊,說:「傻兄弟,莫不教你出錢?哥有銀子在此。」一面扯出包兒來,秤了一錢五分銀子與掌櫃的。還找了一分半錢袖了,搭伏着敬濟肩背,同到坊子裡,兩箇在一處歇臥。二人都醉了。這侯林兒晚夕幹敬濟後庭花,足幹了一夜。親哥、親達達、親漢子、親爺,口裡無般不叫將出來。

到天明,同往城南水月寺。果然寺外侯林兒賃下半間廈子,{\meipi{果然二字,寫出此輩言不足信是其常。}}裡面燒着炕柴,早也買下許多碗盞家活。早晨上工,叫了名字。衆人看見敬濟,不上二十四五歲,白臉子,生的眉目清俊,就知是侯林兒兄弟,都亂調戲他。先問道:「那小夥子兒,你叫甚名字?」陳敬濟道:「我叫陳敬濟。」那人道:「陳敬濟,可不繇着你就擠了。」又一人說:「你恁年小小的,怎幹的這營生?捱的這大扛頭子?」{\meipi{戲謔得俚言,方是俗人口中戲謔。}}侯林兒喝開衆人,罵:「怪花子,你只顧奚落他怎的?」一面散了鍬钁筐扛,派衆人擡土的擡土,和泥的和泥,打雜的打雜。

原來曉月長老,教一箇葉頭陀做火頭,造飯與各作匠人吃。這葉頭陀年約五十歲,一箇眼瞎,穿着皁直裰,精着脚,腰間束着爛絨縧,也不會看經,只會念佛,善會麻衣神相。衆人都叫他做葉道。一日做了工下來,衆人都吃畢飯,也有閑坐的,臥的,也有蹲着的。{\pangpi{傳神。}}只見敬濟走向前,問葉頭陀討茶吃。這葉頭陀只顧上上下下看他。內有一人說:「葉道,這箇小夥子兒是新來的,你相他一相。」又一人說:「你相他相,倒相箇兄弟。」一箇說:「倒相箇二尾子。」葉頭陀教他近前,端詳了一回,說道:「色怕嫩兮又怕嬌,聲嬌氣嫩不相饒。老年色嫩招辛苦,少年色嫩不堅牢。只吃了你面皮嫩的虧,一生多得陰人寵愛。八歲十八二十八,做作百般人可愛,縱然弄假又成眞。休怪我說,一生心伶機巧,常得陰人發跡。你今多大年紀?」敬濟道:「我二十四歲。」葉道道:「虧你前年怎麼過來,吃了你印堂太窄,子䘮妻亡,懸璧昏暗,人亡家破;唇不蓋齒,一生惹是招非;鼻若竈門,家私傾散。那一年遭官司口舌,傾家散業,見過不曾?」敬濟道:「都見過了。」葉頭陀道:「只一件,你這山根不宜斷絕。麻衣祖師說得兩句好:『山根斷兮早虛花,祖業飄零定破家。』早年父祖丟下家業,不拘多少,到你手裡,都了當了。你上停短兮下停長,主多成多敗,錢財使盡又還來。總然你久後營得家計,猶如烈日照冰霜。你如今往後,還有一步發跡,該有三妻之命。克過一箇妻宮不曾?」敬濟道:「已克過了。」葉頭陀道:「後來還有三妻之會,但恐美中不美。三十上,小人有些不足,花柳中少要行走。」一箇人說:「葉道,你相差了,他還與人家做老婆,那有三箇妻來?」衆人正笑做一團,只聽得曉月長老打梆了,各人都拿鍬钁筐扛,上工做活去了。如此者,敬濟在水月寺,也做了約一月光景。

一日,三月中旬天氣,敬濟正與衆人擡出土來,在山門墻下,倚着墻根,向日陽蹲踞着捉身上蝨蟣。只見一箇人,頭帶萬字頭巾,身穿青窄衫,紫裹肚,腰繫纏帶,脚穿扁靴,騎着一匹黃馬,手中提着一籃鮮花兒。{\pangpi{絕不枯澀。}}見了敬濟,猛然跳下馬來,向前深深的唱了諾,便叫:「陳舅,小人那裡沒尋,你老人家原來在這裡。」倒諕了敬濟一跳。連忙還禮不迭,問:「哥哥,你是那裡來的?」那人道:「小人是守備周爺府中親隨張勝,自從舅舅府中官事出來,奶奶不好直到如今,老爺使小人那裡不找尋舅舅,不知在這裡。今早不是俺奶奶使小人到外庄上,折取這幾雜芍藥花兒,打這裡過,怎得看見你老人家在這裡?一來也是你老人家際遇,二者小人有緣。不消猶豫,就騎上馬,我跟你老人家往府中去。」那衆做工的人看着,面面相覷,不敢做聲。{\meipi{彼此但不出一語,寫匆匆驚喜不定光景,妙甚。}}這陳敬濟把鑰匙遞與侯林兒,騎上馬,張勝緊緊跟隨,逕往守備府中來。正是:良人得意正年少,今夜月明何處樓?有詩為證:

\begin{myquote}
白玉隱於頑石裡,黃金埋在污泥中。\\今朝貴人提拔起,如立天梯上九重。
\end{myquote}

