\includepdf[pages={73,74},fitpaper=false]{tst.pdf}
\chapter*{第三十七回 馮媽媽說嫁韓愛姐 西門慶包占王六兒}
\addcontentsline{toc}{chapter}{第三十七回 馮媽媽說嫁韓愛姐 西門慶包占王六兒}
\markboth{{\titlename}卷之四}{第三十七回 馮媽媽說嫁韓愛姐 西門慶包占王六兒}


詞曰:

\begin{myquote}
淡粧多態,更的的頻回眄睞。便認得琴心先許,與綰合歡雙帶。記華堂風月逢迎,輕嚬淺笑嫣無奈。向睡鴨爐邊,翔鸞屏裡,暗把香羅偷解。

\raggedleft{——右調《薄倖前》\rightquadmargin}
\end{myquote}

話說西門慶打發蔡狀元、安進士去了。一日,騎馬帶眼紗在街上喝道而過,撞見馮媽媽,便叫小厮叫住,到面前問他:「你尋的那女子怎樣了?如何也不來回話?」婆子說道:「這幾日,雖是看了幾箇,都是賣肉的挑担兒的,怎好回你老人家話?不想天使其便,眼跟前一箇人家女兒,就想不起來。十分人材,屬馬的,交新年十五歲。若不是昨日打他門首過,他娘請我進去吃茶,我還不得看見他哩。纔弔起頭兒,戴着雲髻兒。好不筆管兒般直縷的身子兒,纏得兩隻脚兒一些些,搽的濃濃的臉兒,又一點小小嘴兒,鬼精靈兒是的。他娘說,他是五月端午日養的,小名叫做愛姐。休說俺們愛,就是你老人家見了,也愛的不知怎麼樣的哩!」西門慶道:「你看這風媽媽子,我平白要他做甚麼?家裡放着好少兒。實對你說了罷,此是東京蔡太師老爺府裡大管家翟爹,要做二房,圖生長,托我替他尋。你若與他成了,管情不虧你。」因問道:「是誰家女子?問他討箇庚帖兒來我瞧。」馮媽媽道:「誰家的?我教你老人家知道了罷,遠不一千,近只在一磚。不是別人,是你家開絨線韓夥計的女孩兒。你老人家要相看,等我和他老子說,討了帖兒來,約會下箇日子,你只顧去就是了,」

西門慶分咐道:「既如此這般,就和他說,他若肯了,討了帖兒,來宅內回我話。」那婆子應諾去了。過兩日,西門慶正在前廳坐的,忽見馮媽媽來回話,拿了帖兒與西門慶瞧,上寫着「韓氏,女命,年十五歲,五月初五日子時生」。便道:「我把你老人家的話對他老子說了,他說:『既是大爹可憐見,孩兒也是有造化的。但只是家寒,沒些備辦。』」西門慶道:「你對他說:不費他一絲兒東西,凡一應衣服首飾、粧奩箱櫃等件,都是我這裡替他辦備,還與他二十兩財禮。教他家止辦女孩兒的鞋脚就是了。臨期,還教他老子送他往東京去。比不的與他做房裡人,翟管家要圖他生長,做娘子。難得他女兒生下一男半女,也不愁箇大富貴。」馮媽媽道:「他那裡請問,你老人家幾時過去相看,好預備。」西門慶道:「既是他應允了,我明日就過去看看罷。他那裡要的急。就對他說,休要他預備什麼,我只吃鍾清茶就起身。」馮媽媽道:「爺嚛,你老人家上門兒怪人家,雖不稀罕他的,也畧坐坐兒。夥計家莫不空教你老人家來了!」西門慶道:「你就不是了。你不知我有事。」馮媽媽道:「既是恁的,等我和他說。」一面先到韓道國家,對他渾家王六兒,將西門慶的話一五一十說了一遍:「明日他衙門中散了,就過來相看。教你一些兒休預備,他只吃一鍾茶,看了就起身。」王六兒道:「眞箇?媽媽子休要說謊。」馮媽媽道:「你當家{\meipi{「你當家」三字,無意中已隱隱勾挑。}}不恁的說,我來哄你不成!他好少事兒,家中人來人去,通不斷頭的。」婦人聽言,安排了酒食與婆子吃了,打發去了,明日早來伺候。到晚,韓道國來家,婦人與他商議已定。早起往高井上叫了一担甜水,買了些好細菓仁,放在家中,還往鋪子裡做買賣去了。丟下老婆在家,{\meipi{觀「丟下」一語,則韓道國明放一着可知矣。}}艷粧濃抹,打扮的喬模喬樣,洗手剔甲,揩抹盃盞乾淨,剝下菓仁,頓下好茶等候,馮媽媽先來攛掇。西門慶衙門中散了,到家換了便衣靖巾,騎馬帶眼紗,玳安、琴童兩箇跟隨,逕來韓道國家,下馬進去。馮媽媽連忙請入裡面坐了,良久,王六兒引着女兒愛姐出來拜見。這西門慶且不看他女兒,不轉晴只看婦人。見他上穿着紫綾襖兒玄色段金比甲,玉色裙子下邊顯着趫趫的兩隻脚兒。生的長挑身材,紫膛色瓜子臉,描的水髩長長的。{\meipi{看得有次第,自是好色中明眼人。}}正是:未知就裡何如,先看他粧色油樣。但見:

\begin{myquote}
淹淹潤潤,不搽脂粉,自然體態妖嬈;嬝嬝娉娉,懶染鉛華,生定精神秀麗。兩彎眉畫遠山,一對眼如秋水。檀口輕開,勾引得蜂狂蝶亂;纖腰拘束,暗帶着月意風情。若非偷期崔氏女,定然聞瑟卓文君。
\end{myquote}

西門慶見了,心搖目蕩,不能定止,口中不說,心中暗道:「原來韓道國有這一箇婦人在家,怪不的前日那些人鬼混他。」{\meipi{想起從前作證,透甚,妙甚。}}又見他女孩兒生的一表人物,暗道:「他娘母兒生的這般人物,女兒有箇不好的?」婦人先拜見了,教他女兒愛姐轉過來,望上向西門慶花枝招颭也磕了四箇頭,起來侍立在旁。老媽連忙拿茶出來,婦人用手抹去盞上水漬,令他遞上。西門慶把眼上下觀看這箇女子:烏雲疊𩬆、粉黛盈腮,意態幽花秀麗,肌膚嫩玉生香。便令玳安氊包內取出錦帕二方、金戒指四箇、白銀二十兩,教老媽安放在茶盤內。他娘忙將戒指帶在女兒手上,朝上拜謝,回房去了。西門慶對婦人說:「遲兩日,接你女孩兒往宅裡去,與他裁衣服。這些銀子,你家中替他做些鞋脚兒。」婦人連忙又磕下頭去,謝道:「俺們頭頂脚踏都是大爹的,孩子的事又教大爹費心,俺兩口兒就殺身也難報大爹。{\meipi{口角甜甚,巧語撩人,豈能不惑!}}又多謝爹的插帶厚禮。」西門慶問道:「韓夥計不在家了?」婦人道:「他早晨說了話,就往鋪子裡走了。明日教他往宅裡與爹磕頭去。」西門慶見婦人說話乖覺,一口一聲只是爹長爹短,就把心來惑動了,臨出門上覆他:「我去罷。」婦人道:「再坐坐。」西門慶道:「不坐了。」{\meipi{「我去罷」、「不坐了」二語,不獨留戀不肯出門,且有許多追悔先回不坐之意在其中,下語微妙。}}于是出門。一直來家,把上項告吳月娘說了。月娘道:「也是千里姻緣着線牽。既是韓夥計這女孩兒好,也是俺們費心一場。」西門慶道:「明日接他來住兩日兒,好與他裁衣服。我如今先拿十兩銀子,替他打半副頭面簪環之類。」月娘道:「及緊儹做去,正好後日教他老子送去,咱這裡不着人去罷了。」西門慶道,「把鋪子關兩日也罷,還着來保同去,就府內問聲,前日差去節級送蔡駙馬的禮到也不曾?」話休饒舌。過了兩日,西門慶果然使小厮接韓家女兒。他娘王氏買了禮,親送他來,進門與月娘大小衆人磕頭拜見,說道:「蒙大爹、大娘並衆娘每擡舉孩兒,這等費心,俺兩口兒知感不盡。」先在月娘房擺茶,然後明間內管待。

李嬌兒、孟玉樓、潘金蓮、李瓶兒都陪坐。西門慶與他買了兩疋紅綠潞紬、兩疋綿紬,和他做裡衣兒。又叫了趙裁來,替他做兩套織金紗段衣服,一件大紅粧花段子袍兒。他娘王六兒安撫了女兒,晚夕回家去了。西門慶又替他買了半副嫁粧,描金箱籠、鑑粧、鏡架、盒礶、銅錫盆、淨桶、火架等件。非止一日,都治辦完備。寫了一封書信,擇定九月初十日起身。西門慶問縣裡討了四名快手,又撥了兩名排軍,執袋弓箭隨身。來保、韓道國顧了四乘頭口,緊緊保定車輛暖轎,送上東京去了,不題。丟的王六兒在家,前出後空,整哭了兩三日。

一日,西門慶無事,騎馬來獅子街房裡觀看。馮媽媽來遞茶,西門慶與了一兩銀子,說道:「前日韓夥什孩子的事累你,這一兩銀子,你買布穿。」婆子連忙磕頭謝了。西門慶又問:「你這兩日,沒到他那邊走走?」馮媽媽道:「老身那一日沒到他那裡做伴兒坐?他自從女兒去了,他家裡沒人,他娘母靠慣了他,整哭了兩三日,這兩日纔緩下些兒來了。他又說孩子事多累了爹,問我:『爹曾與你些辛苦錢兒沒有?』我便說:『他老人家事忙,我連日也沒曾去,隨他老人家多少與我些兒,我敢爭?』他也許我等他官兒回來,重重謝我哩!」西門慶道:「他老子回來已定有些東西,少不得謝你。」說了一回話,見左右無人,悄俏在婆子耳邊如此這般:「你閑了到他那裡,取巧兒和他說,就說我上覆他,閑中我要到他那裡坐半日,看他肯也不肯。我明日還來討回話。」

那婆子掩口冷冷笑道:「你老人家『坐家的女兒偷皮匠——逢着的就上』。一鍬撅了箇銀娃娃,還要尋他的娘母兒哩!夜晚些,等老身慢慢皮着臉對他說。爹,你還不知這婦人,他是咱后街宰牲口王屠的妹子,排行叫六姐,屬蛇的,二十九歲了,雖是打扮的喬樣,到沒見他輸身。{\meipi{便作身價。}}你老人家明日來,等我問他,討箇話兒回你。」西門慶道:「是了。」說畢,騎馬來家。婆子做飯吃了,鎖了房門,慢慢來到婦人家。婦人開門,便讓進房裡坐,道:「我昨日下了些面,等你來吃,就不來了。」婆子道:「我可要來哩,到人家就有許多事,掛住了腿,動不得身。」婦人造:「剛纔做的熱飯,炒麵觔兒,你吃些。」{\pangpi{逼眞。}}婆子道:「老身纔吃的飯來,呷些茶罷,」那婦人便濃濃點了一盞茶遞與他,看着婦人吃了飯,婦人道:「你看我恁苦!有我那冤家,靠定了他。自從他去了,弄的這屋裡空落落的,件件的都看了我。弄的我鼻兒烏,嘴兒黑,相箇人模樣?到不如他死了,扯斷腸子罷了。似這般遠離家鄉去了,你教我這心怎麼放的下來?急切要見他見,也不能勾。」{\meipi{似坐,似想,似托怨,口角宛然。}}說着,眼痠酸的哭了。婆子道:「說不得,自古養兒人家熱騰騰,養女人家冷清清,就是長一百歲,少不得也是人家的。你如今這等抱怨,到明日,你家姐姐到府裡脚硬,生下一男半女,你兩口子受用,就不說我老身了。」婦人道:「大人家的營生,三層大,兩層小,知道怎樣的?等他長進了,我們不知在那裡晒牙揸骨去了。」{\meipi{千古名言,可銷世人無限未來妾想。}}婆子道:「怎的恁般說!你們姐姐,比那箇不聰明伶俐,愁針指女工不會?各人裙帶衣食,你替他愁!」兩箇一遞一句說勾良久,看看說得入港,婆子道:「我每說箇傻話兒,你家官人不在,前後恁空落落的,你晚夕一箇人兒,不言怕麼?」婦人道:「你還說哩,都是你弄得我,肯晚夕來和我做做伴兒?」婆子道:「只怕我一時來不成,我舉保箇人兒來與你做伴兒,肯不肯?」婦人問:「是誰?」婆子掩口笑道:「一客不煩二主,宅裡大老爹昨日到那邊房子裡,如此這般對我說,見孩子去了,丟的你冷落,他要來和你坐半日兒,你怎麼說?這裡無人,你若與他凹上了,愁沒吃的、穿的、使的、用的!走熟了時,到明日房子也替你尋得一所,強如在這僻格剌子裡。」婦人聽了微笑說道:「他宅裡神道相似的幾房娘子,他肯要俺這醜貨兒?」{\meipi{數語是自謙,亦自喜出望外,所以一說便肯。}}婆子道:「你怎的這般說?自古道『情人眼內出西施』,一來也是你緣法湊巧,他好閑人兒,不留心在你時,他昨日巴巴的肯到我房子裡說?又與了一兩銀子,說前日孩子的事累我。落後沒人在跟前,就和我說,教我來對你說。你若肯時,他還等我回話去。典田賣地,你兩家願意,我莫非說謊不成!」婦人道:「既是下顧,明日請他過來,奴這裡等候。」這婆子見他吐了口兒,坐了一回去了。

到次日,西門慶來到,一五一十把婦人話告訴一遍。西門慶不勝歡喜,忙稱了一兩銀子與馮媽媽,拿去治辦酒菜。那婦人聽見西門慶來,收拾房中乾淨,薰香設帳,預備下好茶好水。不一時,婆子拿籃子買了許多嘎飯菜蔬菓品,來廚下替他安排。婦人洗手剔甲,又烙了一筯麵餅。明間內,揩抹桌椅光鮮。西門慶約下午時分,便衣小帽,帶着眼紗,玳安、棋童兩箇小厮跟隨,逕到門首,下馬進去。分咐把馬回到獅子街房子裡去,晚上來接,止留玳安一人答應。西門慶到明間內坐下。良久,婦人扮的齊齊整整,出來拜見,說道:「前日孩子累爹費心,一言難盡。」西門慶道:「一時不到處,你兩口兒休抱怨。」婦人道:「一家兒莫大之恩,豈有抱怨之理。」磕了四箇頭。馮媽媽拿上茶來,婦人選了茶。見馬回去了,玳安把大門關了。{\meipi{湊趣。}}婦人陪坐一回,讓進房裡坐。正面紙窻門兒廂的炕床,掛着四扇各樣顏色綾剪帖的張生遇鶯鶯蜂花香的弔屏兒,上桌鑑粧、鏡架、盒礶、錫器家活堆滿,地下插着棒兒香。{\pangpi{尤肖。}}上面設着一張東坡椅兒。{\meipi{寫景酷肖。}}西門慶坐下。婦人又濃濃點一盞胡桃夾鹽筍泡茶遞上去,西門慶吃了。婦人接了盞,在下邊炕沿兒上陪坐,問了回家中長短。西門慶見婦人自己拿托盤兒,說道:「你這裡還要箇孩子使纔好。」婦人道:「不瞞爹說,自從俺女兒去了,凡事不方便。少不的奴自己動手。」西門慶道:「這箇不打緊,明日教老馮替你看箇十三四歲的丫頭子,且胡亂替替手脚。」婦人道:「也得俺家的來,少不得東軿西輳的,{\pangpi{薰局,妙。}}央馮媽媽尋一箇孩子使。」西門慶道:「也不消,該多少銀子,等我與他。」那婦人道:「怎好又煩費你老人家,自恁累你老人家還少哩!」西門慶見他會說話,心中甚喜。一面馮媽媽進來安放桌兒,西門慶就對他說尋使女一節。馮媽媽道:「爹既是許了你,拜謝拜謝兒。南首趙嫂兒有箇十三歲的孩子,只要四兩銀子,教爹替你買下罷。」婦人連忙向前道了萬福。不一時,擺下案碟菜蔬,篩上酒來。婦人滿斟一盞,雙手遞與西門慶。纔待磕下頭去,西門慶連忙用手拉起,說:「頭裡已是見過,不消又下禮了,只拜拜便了。」婦人笑吟吟道了萬福,旁邊一箇小杌兒上坐下。廚下老媽將嘎飯菜菓,一一送上。又是兩筯軟餅,婦人用手揀肉絲細菜兒裹捲了,用小蝶兒托了,遞與西門慶吃。兩箇在房中,盃來盞去,做一處飲酒。玳安在廚房裡,老馮陪他另有坐處,打發他吃,不在話下。

彼此飲勾數巡,婦人把座兒挪近西門慶跟前,{\pangpi{甚在行。}}與他做一處說話,遞酒兒。然後西門慶與婦人一遞一口兒吃酒,見無人進來,摟過脖子來親嘴咂舌。婦人便舒手下邊,籠揝西門慶玉莖。彼此淫心蕩漾,把酒停住不吃了。掩上房門,褪去衣褲。婦人就在裡邊炕床上伸開被褥。那時已是日色平西時分。

西門慶乘着酒興,順袋內取出銀托子來使上。婦人用手打弄,見奢稜跳腦,紫強光鮮,沉甸甸甚是粗大。一壁坐在西門慶懷裡,一面在上,兩箇且摟着脖子親嘴。婦人乃蹺起一足,以手導那話入牝中,兩箇挺一回。西門慶摸見婦人肌膚柔膩,牝毛疎秀,先令婦人仰臥於床背,把雙手提其雙足,置之於腰眼間,肆行抽送。怎見得這場雲雨?但見:

\begin{myquote}
威風迷翠榻,殺氣瑣鴛衾。珊瑚枕上施雄,翡翠帳中鬬勇。男兒氣急,使鎗只去紮心窩;女帥心忙,開口要來吞腦袋。一箇使雙砲的,往來攻打內襠兵;一箇輪傍牌的,上下夾迎臍下將。一箇金雞獨立,高蹺玉腿弄精神;一箇枯樹盤根,倒入翎花來刺牝。戰良久朦朧星眼,但動些兒麻上來;鬬多時款擺纖腰,百戰百回挨不去。散毛洞主倒上橋,放水去淹軍;烏甲將軍虛點鎗,側身逃命走。臍膏落馬,須臾蹂踏肉為泥;溫緊粧呆,頃刻跌翻深澗底。大披掛七零八斷,猶如急雨打殘花;錦套頭力盡觔輸,恰似猛風飄敗葉。硫黃元帥,盔歪甲散走無門;銀甲將軍,守住老營還要命。
\end{myquote}

正是:

\begin{myquote}
愁雲托上九重天,一塊敗兵連地滾。
\end{myquote}

原來婦人有一件毛病,{\meipi{子平云:且病方為貴,端知王六兒之受用處,在有此毛病也。}}但凡交媾,只要教漢子幹他後庭花,在下邊揉着心子纔過。不然,隨問怎的,不得丟身子。就是韓道國與他相合,倒是後邊去的多,前邊一月走不的兩三遭兒。第二件,積年好咂𩫻䯲,把𩫻䯲常遠放在口裡一夜,他也無箇足處。隨問怎的出了毧,禁不的他吮舔挑弄,登時就起。自這兩樁兒,可在西門慶心坎上。當日和他纏到起更纔回家。婦人和西門慶說:「爹到明日再來早些,白日裡咱破工夫,脫了衣裳好生耍耍。」西門慶大喜。到次日,到了獅子街線鋪裡,就兌了四兩銀子與馮媽媽,討了丫頭使喚,改名叫做錦兒。

西門慶想着這箇甜頭兒,過了兩日,又騎馬來婦人家行走。原是棋童、玳安兩箇跟隨。到了門首,就分咐棋童把馬回到獅子街房裡去。那馮媽媽專一替他提壺打酒,街上買東西整理,通小殷勤兒,圖些油菜養口。西門慶來一遭,與婦人一二兩銀子盤纏。白日裡來,直到起更時分纔家去。瞞的家中鐵桶相似。馮媽媽每日在婦人這裡打勤勞兒,往宅裡也去的少了。李瓶兒使小厮叫了他兩三遍,只是不得閑,要便鎖着門去了一日。

一日,畫童兒撞見婆子,叫了來家。李瓶兒說道:「媽媽子成日影兒不見,幹的什麼貓兒頭差事?叫了一遍,只是不在,通不來這裡走走兒,忙的恁樣兒的!丟下好些衣裳帶孩子被褥,等你來幫着丫頭們拆洗拆洗,再不見來了。」婆子道:「我的奶奶,你到說得且是好,寫字的拿逃兵,我如今一身故事兒哩!賣鹽的做雕鑾匠,我是那鹹人兒?」{\meipi{瓶兒何等待老馮,老馮別有頭路,則一味虛混,此輩之無情不足取如此。}}李瓶兒道:「媽媽子請着你就是不閑,成日賺的錢,不知在那裡。」婆子道:「老身『大風颳了頰耳去——嘴也趕不上』。在這裡,賺甚麼錢?你惱我,可知心裡急急的要來,再轉不到這裡來,我也不知成日幹的什麼事兒哩。後邊大娘從那時與了銀子,教我門外頭替他稍箇拜佛的蒲甸兒來,我只要忘了。昨日甫能想起來,賣蒲甸的賊蠻奴才又去了,我怎的回他?」李瓶兒道:「你還敢說沒有他甸兒,你就信信拖拖跟了和尚去了罷了!他與了你銀子,這一向還不替他買將來,你這等粧憨打呆的。」婆子道,「等我也對大娘說去,就交與他這銀子去。昨日騎騾子,差些兒沒弔了他的。」李瓶兒道:「等你弔了他的,你死也。」

這媽媽一直來到後邊,未曾入月娘房,先走在廚下打探子兒。只見玉蕭和來興兒媳婦坐在一處,見了說道:「老馮來了!貴人,你在那裡來?你六娘要把你肉也嚼下來,說影邊兒就不來了。」那婆子走到跟前拜了兩拜,說道:「我纔到他前頭來,吃他咭咶了這一回來了。」玉蕭道:「娘問你替他稍的蒲甸兒怎樣的?」婆子道:「昨日拿銀子到門外,賣蒲甸的賣了家去了,直到明年三月裡纔來哩。銀子我還拿在這裡,姐你收了罷!」玉蕭笑道:「怪媽媽子,你爹還在屋裡兌銀子,等出去了,你還親交與他罷。」又道:「你且坐的。我問你,韓夥計送他女兒去了多少時了?也待回來,這一回來,你就造化了,他還謝你謝兒。」婆子道:「謝不謝,隨他了。他連今纔去了八日,也得盡頭纔得來家。」不一時,西門慶兌出銀子,與賁四拿了庄子上去,就出去了。婆子走在上房,見了月娘,也沒敢拿出銀子來,只說蠻子有幾箇粗甸子,都賣沒了,回家明年稍雙料好蒲甸來。月娘是誠實的人,說道:「也罷,銀子你還收着。到明年,我只問你要兩箇就是了。」與婆子幾箇茶食吃了。後又到李瓶兒房裡來,瓶兒因問:「你大娘沒罵你?」婆子道:「被我如此支吾,調的他喜歡了,倒與我些茶吃,賞了我兩箇餅定出來了。」李瓶兒道:「還是昨日他往喬大戶家吃滿月的餅定。媽媽子,不虧你這片嘴頭子,六月裡蚊子,也釘死了!」又道:「你今日與我洗衣服,不去罷了。」婆子道:「你收拾討下漿,我明日蚤來罷。後晌時分,還要到一箇熟主顧人家幹些勾當兒。」李瓶兒道:「你這老貨,偏有這些胡枝扯葉的。你明日不來,我和你答話!」那婆子說笑了一回,脫身走了。李瓶兒留他:「你吃了飯去。」婆子道:「還飽着哩,不吃罷。」恐怕西門慶往王六兒家去,兩步做一步。正是:

\begin{myquote}
媒人婆地裡小鬼,兩頭來回抹油嘴。\\一日走勾千千步,只是苦了兩隻腿。
\end{myquote}

