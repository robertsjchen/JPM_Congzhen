\part*{{\titlename}卷之六}
\addcontentsline{toc}{part}{{\titlename}卷之六}


\includepdf[pages={101,102},fitpaper=false]{tst.pdf}
\chapter*{第五十一回 打貓兒金蓮品玉 鬬葉子敬濟輸金}
\addcontentsline{toc}{chapter}{第五十一回 打貓兒金蓮品玉 鬬葉子敬濟輸金}
\markboth{{\titlename}卷之六}{第五十一回 打貓兒金蓮品玉 鬬葉子敬濟輸金}


詩曰:

\begin{myquote} 
羞看鸞鏡惜朱顏,手托香腮懶去眠。\\瘦損纖腰寬翠帶,淚流粉面落金鈿。\\薄倖惱人愁切切,芳心繚亂恨綿綿。\\何時借得東風便,刮得檀郎到枕邊。
\end{myquote} 

話說潘金蓮見西門慶拿了淫器包兒,與李瓶兒歇了,足惱了一夜沒睡,{\pangpi{此妬婦之苦。}}懷恨在心。到第二日,打聽西門慶往衙門裡去了,老早走到後邊對月娘說:「李瓶兒背地好不說姐姐哩!說姐姐會那等虔婆勢,喬坐衙,別人生日,又要來管。『你漢子吃醉了進我屋裡來,我又不曾在前邊,平白對着人羞我,望着我丟臉兒。交我惱了,走到前邊,把他爹趕到後邊來。落後他怎的也不在後邊,還到我房裡來了?我兩箇黑夜說了一夜梯己話兒,只有心腸五臟沒曾倒與我罷了。』」{\meipi{金蓮學瓶兒之言,妙在心思、口角仍是金蓮之言,若平心聽之,原不難辨,但恨聽言者觸於怒而不暇矣。}}這月娘聽了,如何不惱!因向大妗子、孟玉樓說:「你們昨日都在跟前看着,我又沒曾說他甚麼。小厮交燈籠進來,我只問了一聲:『你爹怎的不進來?』小厮倒說:『往六娘屋裡去了。』我便說:『你二娘這裡等着,恁沒槽道,卻不進來!』論起來也不傷他,怎的說我虔婆勢,喬坐衙?我還把他當好人看成,原來知人知面不知心,那裡看人去?乾淨是箇綿裡針、肉裡刺的貨,還不知背地在漢子跟前架甚麼舌兒哩!{\meipi{從認瓶兒為好人中,推勘其不好處,直寫出月娘信讒,一時之轉念,妙不容言。}}怪道他昨日決烈的就往前走了。傻姐姐,那怕漢子成日在你屋裡不出門,不想我這心動一動兒。{\pangpi{說不動正是動處。}}一箇漢子丟與你們,隨你們去,守寡的不過。想着一娶來之時,賊強人和我門裡門外不相逢,那等怎的過來?」{\pangpi{觸便想到,怨之難忘如此。}}大妗子在旁勸道:「姑娘罷麼,看孩兒的分上罷!自古宰相肚裡好行船。當家人是箇惡水缸兒,好的也放在心裡,歹的也放在心裡。」月娘道:「不拘幾時,我也要對這兩句話。等我問他,我怎麼虔婆勢,喬做衙?」金蓮慌的沒口子說道:「姐姐寬恕他罷。常言大人不責小人過,那箇小人沒罪過?{\meipi{語雖毒,未見太甚,轉生人之疑。}}他在背地挑唆漢子,俺們這幾箇誰沒吃他排說過?我和他緊隔着壁兒,要與他一般見識起來,倒了不成!行動只倚着孩兒降人,他還說的好話兒哩!說他的孩兒到明日長大了,有恩報恩,有仇報仇,俺們都是餓死的數兒。你還不知道哩!」吳大妗子道:「我的奶奶,那裡有此話說?」{\meipi{大妗子旁觀甚清。}}月娘一聲兒也沒言語。

常言路見不平,也有向燈向火。不想西門大姐平日與李瓶兒最好,常沒針線鞋面,李瓶兒不拘好綾羅段帛就與他,好汗巾手帕兩三方背地與大姐,銀錢不消說。當日聽了此話,如何不告訴他。李瓶兒正在屋裡與孩子做端午戴的絨線符牌,及各色紗小粽子,並解毒艾虎兒。{\pangpi{好點綴。}}只見大姐走來,李瓶兒讓他坐,又交迎春:「拿茶與你大姑娘吃。」大姐道:「頭裡請你吃茶,你怎的不來?」李瓶兒道:「打發他爹出門,我趕早涼與孩子做這戴的碎生活兒來。」大姐道:「有樁事兒,我也不是舌頭,敢來告你說:你沒曾惱着五娘?他對着俺娘,如此這般說了你一篇是非。說你說俺娘虔婆勢,喬做衙。如今俺娘要和你對話哩!你別要說我對你說,交他怪我。你須預備些話兒打發他。」

這李瓶兒不聽便罷,聽了此言,手中拿着那針兒通拿不起來,兩隻胳膊都軟了,半日說不出話來,對着大姐掉眼淚,說道:「大姑娘,我那裡有一字兒?昨晚我在後邊,聽見小厮說他爹往我這邊來了,我就來到前邊,催他往後邊去了。再誰說一句話兒來?你娘恁覷我一場,莫不我恁不識好歹,敢說這箇話?設使我就說,對着誰說來?也有箇下落。」大姐道:「他聽見俺娘說不拘幾時要對這話,他也就慌了。要是我,你兩箇當面鑼對面鼓的對不是!」李瓶兒道:「我對的過他那嘴頭子?只憑天罷了。他左右晝夜算計的只是俺娘兒兩箇,到明日終久吃他算計了一箇去,纔是了當。」{\meipi{人情皆惜瓶兒不能辨,不知瓶兒正妙在不能辨而西門慶始憐之也。若然,則瓶兒智出金蓮上矣,非也。瓶兒性實愚不能辨,非能辨而有不辨之妙,所以往往受金蓮之累也。}}說畢哭了。大姐坐着勸了一回,只見小玉來請六娘、大姑娘吃飯。李瓶兒丟下針指,同大姐到後邊,也不曾吃飯,回來房中,倒在床上就睡着了。西門慶衙門中來家,見他睡,問迎春。迎春道:「俺娘一日飯也還沒吃哩。」慌的西門慶向前問道:「你怎的不吃飯?你對我說。」又見他哭的眼紅紅的,只顧問:「你心裡怎麼的?對我說。」李瓶兒連忙起來,揉了揉眼說道:「我害眼疼,不怎的。今日心裡懶待吃飯。」並不題出一字兒來。正是:滿懷心腹事,盡在不言中。有詩為證:

\begin{myquote} 
莫道佳人總是癡,惺惺伶俐沒便宜。\\只因會盡人間事,惹得閒愁滿肚皮。
\end{myquote} 

大姐在後邊對月娘說:「纔五娘說的話,我問六娘來。他好不賭身發咒,望着我哭,說娘這般看顧他,他肯說此話!」吳大妗子道:「我就不信。李大姐好箇人兒,他怎肯說這等話!」月娘道:「想必兩箇有些小節不足,哄不動漢子,走來後邊,沒的拿我墊舌根。我這裡還多着箇影兒哩!」{\meipi{金蓮之讒,月娘此時已識破矣。猶曰兩箇,可見讒人者雖輸亦只平交,亦何憚而不讒人哉!}}大妗子道:「大姑娘,今後你也別要虧了人。不是我背地說,潘五姐一百箇不及他。為人心地兒又好,來了咱家恁二三年,要一些歪樣兒也沒有。」

正說着,只見琴童兒背進箇藍布大包袱來。月娘問是甚麼,琴童道:「是三萬鹽引。韓夥計和崔本纔從關上掛了號來,爹說打發飯與他二人吃,如今兌銀子打包。後日二十,是箇好日子,起身,打發他三箇往揚州去。」吳大妗子道:「只怕姐夫進來。我和二位師父往他二娘房裡坐去罷。」剛說未畢,只見西門慶掀簾子進來,慌的吳妗子和薛姑子、王姑子往李嬌兒房裡走不迭。早被西門慶看見,問月娘:「那箇是薛姑子?賊胖禿淫婦,來我這裡做甚麼!」月娘道:「你好恁枉口撥舌,不當家化化的,罵他怎的?他惹着你來?你怎的知道他姓薛?」西門慶道:「你還不知他弄的乾坤兒哩!他把陳參政的小姐吊在地藏庵兒裡和一箇小夥偷姦,他知情,受了三兩銀子。事發,拿到衙門裡,被我褪衣打了二十板,交他嫁漢子還俗。他怎的還不還俗?好不好,拿來衙門裡再與他幾拶子。」月娘道:「你有要沒緊,恁毀僧傍佛的。他一箇佛家弟子,想必善根還在,他平白還甚麼俗?你還不知他好不有道行!」{\pangpi{只聽先人之言。}}西門慶道:「你問他有道行一夜接幾箇漢子?」月娘道:「你就休汗邪!又討我那沒好口的罵你。」{\meipi{薛姑之醜,已和盤托出,月娘猶委曲迴護,婦人一種偏執之性,覺溺愛、佞佛俱說不着。}}因問:「幾時打發他三箇起身?」西門慶道:「我剛纔使來保會喬親家去了,他那裡出五百兩,我這裡出五百兩。二十是箇好日子,打發他每起身去罷了。」月娘道:「線鋪子卻交誰開?」西門慶道:「且交賁四替他開着罷。」說畢,月娘開箱子拿銀子,一面兌了出來,交付與三人,在捲棚內看着打包。每人又兌五兩銀子,交他家中收拾衣裝行李。

只見應伯爵走到捲棚裡,看見便問:「哥打包做甚麼?」西門慶因把二十日打發來保等往揚州支鹽去一節告訴一遍。伯爵舉手道:「哥,恭喜!此去回來必得大利。」西門慶一面讓坐,喚茶來吃。因問:「李三、黃四銀子幾時關?」應伯爵道:「也只在這箇月裡就關出來了。他昨日對我說,如今東平府又派下二萬香來了,還要問你挪五百兩銀子,接濟他這一時之急。如今關出這批銀子,一分也不動,都擡過這邊來。」西門慶道:「到是你看見,我打發揚州去還沒銀子,問喬親家借了五百兩在裡頭,那討銀子來?」伯爵道:「他再三央及我對你說,一客不煩二主,你不接濟他這一步兒,交他又問那裡借去?」西門慶道:「門外街東徐四鋪少我銀子,我那裡挪五百兩銀子與他罷。」伯爵道:「可知好哩。」正說着,只見平安兒拿進帖兒來,說:「夏老爹家差了夏壽,說請爹明日坐坐。」西門慶看了柬帖,道:「曉得了。」伯爵道:「我有樁事兒來報與哥:你知道李桂兒的勾當麼?他沒來?」西門慶道:「他從正月去了,再幾時來?我並不知道甚麼勾當。」伯爵因說道:「王招宣府裡第三的,原來是東京六黃太尉侄女兒女婿。從正月往東京拜年,老公公賞了一千兩銀子,與他兩口兒過節。你還不知六黃太尉這侄女兒生的怎麼標緻,上畫兒只畫半邊兒,也沒恁俊俏相的。你只守着你家裡的罷了,每日被老孫、祝麻子、小張閑三四箇摽着在院裡撞,把二條巷齊家那小丫頭子齊香兒梳籠了,又在李桂兒家走。把他娘子兒的頭面都拿出來當了。氣的他娘子兒家裡上吊。不想前日老公公生日,他娘子兒到東京只一說,老公公惱了,將這幾箇人的名字送與朱太尉,朱太尉批行東平府,着落本縣拿人。昨日把老孫、祝麻子與小張閑都從李桂兒家拿的去了。李桂兒便躲在隔壁朱毛頭家過了一夜。今日說來央及你來了。」西門慶道:「我說正月裡都摽着他走,這裡誰人家這銀子,那裡誰人家銀子。那祝麻子還對着我搗生鬼。」說畢,伯爵道:「我去罷。等住回只怕李桂兒來,你管他不管他,他又說我來串作你。」西門慶道:「我還和你說,李三,你且別要許他,等我門外討了銀子來,再和你說話。」伯爵道:「我曉的。」剛走出大門首,只見李桂姐轎子在門首,又早下轎進去了。伯爵去了。

西門慶正分咐陳敬濟,交他往門外徐四家催銀子去,只見琴童兒走來道:「大娘後邊請,李桂姨來了。」西門慶走到後邊,只見李桂姐身穿茶色衣裳,也不搽臉,用白挑線汗巾子搭着頭,雲鬟不整,花容淹淡,與西門慶磕着頭哭起來,說道:「爹可怎麼樣兒的,恁造化低的營生,正是關着門兒家裡坐,禍從天上來。一箇王三官兒,俺每又不認的他。平白的祝麻子、孫寡嘴領了來俺家討茶吃。俺姐姐又不在家,依着我說別要招惹他,那些兒不是,{\meipi{桂姐到此,只曰造化低,曰平白地,一字不肯認錯,轉滑強忍之極。}}俺這媽越發老的韶刀了。就是來宅裡與俺姑娘做生日的這一日,你上轎來了就是了,見祝麻子打旋磨兒跟着,從新又回去,對我說:『姐姐你不出去待他鍾茶兒,卻不難為囂了人?』他便往爹這裡來了。交我把門插了不出來,誰想從外邊撞了一夥人來,把他三箇不繇分說都拿的去了。王三官兒便奪門走了,我便走在隔壁人家躲了。家裡有箇人牙兒!纔使保兒來這裡接的他家去。到家把媽諕的魂都沒了,只要尋死。今日縣裡皁隸,又拿着票,喝羅了一清早起去了。如今坐名兒只要我往東京回話去。爹,你老人家不可憐見救救兒,卻怎麼樣兒的?娘也替我說說兒。」{\meipi{桂姐妙在不管人信不信,只一味強辨,全無慚色。既有說者,自有信者,然有良心人自說不出。}}西門慶笑道:「你起來。」因問票上還有誰的名字。桂姐道:「還有齊香兒的名字。他梳籠了齊香兒,在他家使錢,他便該當。{\meipi{只要洗自家清,便不顧推人落水,桂姐狠甚,惡甚,一毫無情。}}俺家若見了他一箇錢兒,就把眼睛珠子吊了;若是沾他沾身子兒,一箇毛孔兒裡生一箇天疱瘡。」月娘對西門慶道:「也罷,省的他恁說誓剌剌的,你替他說說罷。」西門慶道:「如今齊香兒拿了不曾?」桂姐道:「齊香兒他在王皇親宅裡躲着哩。」西門慶道:「既是恁的,你且在我這裡住兩日。我就差人往縣裡替你說去。」就叫書童兒:「你快寫箇帖兒,往縣裡見你李老爹,就說桂姐常在我這裡答應,看怎的擴音他罷。」書童應諾,穿青絹衣服去了。不一時,拿了李知縣回貼兒來。書童道:「李老爹說:『多上覆你老爹,別的事無不領命,這箇卻是東京上司行下來批文,委本縣拿人,縣裡只拘的人到。既是你老爹分上,我這裡且寬限他兩日。要擴音,還往東京上司說去。』」西門慶聽了,只顧沉吟,說道:「如今來保一兩日起身,東京沒人去。」月娘道:「也罷,你打發他兩箇先去,存下來保,替桂姐往東京說了這勾當,交他隨後邊趕了去罷。你看諕的他那腔兒。」那桂姐連忙與月娘、西門慶磕頭。{\pangpi{當機。}}西門慶隨使人叫將來保來,分咐:「二十日你且不去罷。教他兩箇先去。你明日且往東京替桂姐說說這勾當來。見你翟爹,如此這般,好歹差人往衛裡說說。」桂姐連忙就與來保下禮。慌的來保頂頭相還,說道:「桂姨,我就去。」{\meipi{籠絡得妙。不獨籠絡來保,並西門慶、月娘俱在其中矣。}}西門慶一面教書童兒寫就一封書,致謝翟管家前日曾巡按之事甚是費心,又封了二十兩折節禮銀子,連書交與來保。桂姐便歡喜了,拿出五兩銀子來與來保做盤纏,說道:「回來俺媽還重謝保哥。」西門慶不肯,還了桂姐,教月娘另拿五兩銀子與來保盤纏。桂姐道:「也沒這箇道理,我央及爹這裡說人情,又教爹出盤纏。」西門慶道:「你笑話我沒這五兩銀子盤纏了,要你的銀子!」{\meipi{怕人笑話,是大老官使錢撒漫之根。}}那桂姐方纔收了,向來保拜了又拜,說道:「累保哥,好歹明早起身罷,只怕遲了。」來保道:「我明日早五更就走道兒了。」于是領了書信,又走到獅子街韓道國家。

王六兒正在屋裡縫小衣兒哩,打窻眼看見是來保,忙道:「你有甚說話,請房裡坐。他不在家,往裁縫那裡討衣裳去了,便來也。」便叫錦兒:「還不往對過徐裁家叫你爹去!你說保大爺在這裡。」{\pangpi{好稱呼。}}來保道:「我來說聲,我明日還去不成,又有樁業障鑽出來,當家的留下,教我往東京替院裡李桂姐說人情去哩。他剛纔在爹跟前,再三磕頭禮拜央及我。{\pangpi{墜入桂姐術中矣。}}明早就起身了。且教韓夥計和崔大官兒先去,我回來就趕了來。」因問:「嫂子,你做的是甚麼?」王六兒道:「是他的小衣裳兒。」來保道:「你教他少帶衣裳。到那去處是出紗羅段絹的窩兒裡,愁沒衣裳穿!」{\pangpi{豈廉潔之言。}}正說着,韓道國來了。兩箇唱了喏,因把前事說了一遍,因說:「我到明日,揚州那裡尋你每?」韓道國道:「老爹分咐,教俺每馬頭上投經紀王伯儒店裡下。說過世老爹曾和他父親相交,他店內房屋寬廣,下的客商多,放財物不耽心。你只往那裡尋俺每就是了。」來保又說:「嫂子,我明日東京去,你沒甚鞋脚東西稍進府裡,與你大姐去?」王六兒道道:「沒甚麼,只有他爹替他打的兩對簪兒,並他兩雙鞋,起動保叔稍稍進去與他。」于是將手帕包袱停當,遞與來保。一面教春香看菜兒篩酒。婦人連忙丟下生活就放桌兒。來保道:「嫂子,你休費心,我不坐。{\pangpi{寫出老婆作主。}}我到家還要收拾褡褳,明日早起身。」王六兒笑嘻嘻道:「耶嚛,你怎的上門怪人家!夥計家,自恁與你餞行,也該吃鍾兒。」因說韓道國:「你好老實!桌兒不穩,你也撒撒兒,讓保叔坐。只相沒事的人兒一般。」{\meipi{此家常閑話,似無深意,然非老婆作主人家,決無此語。}}于是拿上菜兒來,斟酒遞與來保,王六兒也陪在旁邊,三人坐定吃酒。來保吃了幾鍾,說道:「我家去罷。晚了,只怕家裡關門早。」韓道國問道:「你頭口顧下了不曾?」來保道:「明日早顧罷了。鋪子裡鑰匙並帳簿都交與賁四罷了,省的你又上宿去。家裡歇息歇息,好走路兒。」韓道國道:「夥計說的是,我明日就交與他。」王六兒又斟了一甌子,說道:「保叔,你只吃這一鍾,我也不敢留你了。」來保道:「嫂子,你既要我吃,再篩熱着些。」{\meipi{彼此通家熟分,寫得宛然。}}那王六兒連忙歸到壺裡,教錦兒炮熱了,傾在盞內,雙手遞與來保,說道:「沒甚好菜兒與保叔下酒。」來保道:「嫂子好說,家無常禮。」拿起酒來與婦人對飲,一吸同幹,方纔作辭起身。王六兒便把女兒鞋脚遞與他,說道:「累保叔,好歹到府裡問聲孩子好不好,我放心些。」兩口兒齊送出門來。

不說來保到家收拾行李,第二日起身東京去了。單表這吳大舅前來對西門慶說:「有東平府行下文書來,派俺本衛兩所掌印千戶管工修理社倉,題準旨意,限六月工完,陞一級。違限,聽巡按御史查參。姐夫有銀子借得幾兩,工上使用。待關出工價來,一一奉還。」西門慶道:「大舅用多少,只顧拿去。」吳大舅道:「姐夫下顧,與二十兩罷。」一面同進後邊,見月娘說了話,教月娘拿二十兩出來,交與大舅,又吃了茶。因後邊有堂客,就出來了。月娘教西門慶留大舅大廳上吃酒。正飲酒中間,只見陳敬濟走來,與吳大舅作了揖,就回說:「門外徐四家,稟上爹,還要再讓兩日兒。」西門慶道:「胡說!我這裡等銀子使,照舊還去罵那狗弟子孩兒。」敬濟應諾。吳大舅就讓他打橫坐下,陪着吃酒不題。

且說後邊大妗子、楊姑娘、李嬌兒、孟玉樓、潘金蓮、李瓶兒、大姐,都伴桂姐在月娘房裡吃酒。先是郁大姐數了一回「張生遊寶塔」,放下琵琶。孟玉樓在旁斟酒遞菜兒與他吃,說道:「賊瞎轉磨的唱了這一日,又說我不疼你。」潘金蓮又大筯子夾塊肉放在他鼻子上,戲弄他頑耍。桂姐因叫玉簫姐:「你遞過郁大姐琵琶來,等我唱箇曲兒與姑奶奶和大妗子聽。」月娘道:「桂姐,你心裡熱剌剌的,不唱罷。」桂姐道:「不妨事。見爹娘替我說人情去了,我這回不焦了。」孟玉樓笑道:「李桂姐倒還是院中人家娃娃,做臉兒快。頭裡一來時,把眉頭忔㥮着,焦的茶兒也吃不下去。{\pangpi{補出愁容。}}這回說也有,笑也有。」當下桂姐輕舒玉指,頓撥冰弦,唱了一回。正唱着,只見琴童兒收進家活來。月娘便問道:「你大舅去了?」琴童兒道:「大舅去了。」吳大妗子道:「只怕姐夫進來,我每活變活變兒。」琴童道:「爹往五娘房裡去了。」這潘金蓮聽見,就坐不住,趨趄着脚兒只要走,又不好走的。{\meipi{若無心,竟走何妨,一有心便告難如此,可見身世之難,皆心所造。}}月娘也不等他動身,就說道:「他往你屋裡去了,你去罷。省的你欠肚兒親家是的。」{\pangpi{月娘嘴亦狠。}}那潘金蓮嚷可哥兒的起來,口兒裡硬着,那脚步兒且是去的快。來到房裡,西門慶已是吃了胡僧藥,教春梅脫了裳,在床上帳子裡坐着哩。金蓮看見笑道:「我的兒!今日好呀,不等你娘來就上床了。俺每在後邊吃酒,被李桂姐唱着,灌了我幾鍾好的。獨自一箇兒,黑影子裡,一步高一步低,不知怎的走來了。」{\meipi{自家寫出歸房急情。}}叫春梅:「你有茶倒甌子我吃。」那春梅眞箇點了茶來。金蓮吃了,努了箇嘴與春梅,那春梅就知其意。{\pangpi{二人相合在此。}}那邊屋裡早已替他熱下水,婦人抖些檀香白礬在裡面,{\pangpi{罪過。}}洗了牝。就燈下摘了頭,止撇着一根金簪子,拿過鏡子來,從新把嘴唇抹了胭脂,口中噙着香茶,走過這邊來。春梅床頭上取過睡鞋來與他換了,帶上房門出去。

這婦人便將燈臺挪近旁邊桌上放着,一手放下半邊紗帳子來,褪去紅褲,露出玉體。西門慶坐在枕頭上,那話帶着兩箇托子,一霎弄的大大的與他瞧。婦人燈下看見,諕了一跳,一手揝不過來,紫巍巍,沉甸甸。便昵瞅了西門慶一眼,說道:「我猜你沒別的話,已定吃了那和尚藥,{\meipi{他人俱問,只金蓮一猜便着,妙。}}弄聳的恁般大,一味要來奈何老娘。好酒好肉,王里長吃的去。你在誰人跟前試了新,這回剩了些殘軍敗將,{\meipi{妙語,聞所未聞。}}纔來我這屋裡來了。俺每是雌剩𩫻䯲㒲的,你還說不偏心哩!嗔道那一日我不在屋裡,三不知把那行貨包子偷的往他屋裡去了。原來晚夕和他幹這箇營生,他還對着人撇清搗鬼哩。你這行貨子,乾淨是箇沒挽回的三寸貨。想起來,一百年不理你纔好。」西門慶笑道:「小淫婦兒,你過來。你若有本事,把他咂過了,我輸一兩銀子與你。」婦人道:「汗邪了你了。你吃了甚麼行貨子,我禁的過他!」于是把身子斜軃在衽蓆之上,雙手執定那話,用朱唇吞裹。說道:「好大行貨子,把人的口也撐的生疼的。」說畢,出入嗚咂。或舌尖挑弄蛙口,舐其龜弦;或用口噙着,往來哺摔;或在粉臉上擂晃,百般搏弄,那話越發堅硬,𢳥掘起來。

西門慶垂首窺見婦人香肌掩映於紗帳之內,纖手捧定毛都魯那話,往口裡吞放,燈下一往一來。不想旁邊蹲着一箇白獅子貓兒,看見動彈,不知當做甚物件兒,撲向前,用爪兒來撾。{\meipi{此處人只知其善生情設色,作一回戲笑,不知已冷,冷伏雪獅子之脈矣。非細心人,不許讀此。}}這西門慶在上,又將手中拿的洒金老鴉扇兒,只顧引逗他耍子。被婦人奪過扇子來,把貓盡力打了一扇靶子,打出帳子外去了。昵向西門慶道:「怪發訕的冤家!緊着這紮紮的不得人意,又引逗他恁上頭上臉的,一時間撾了人臉卻怎的?好不好我就不幹這營生了。」西門慶道:「怪小淫婦兒,會張致死了!」婦人道:「你怎不叫李瓶兒替你咂來?我這屋裡盡着教你掇弄。不知吃了甚麼行貨子,咂了這一日,益發咂的沒些事兒。」西門慶于是向汗巾上小銀盒兒裡,用挑牙挑了些粉紅膏子藥兒,抹在馬口內,仰臥於上,教婦人騎在身上。婦人道:「等我𢵞着,你往裡放。」龜頭昂大,濡研半晌,僅沒龜稜。婦人在上,將身左右捱擦,似有不勝隱忍之態。因叫道:「親達達,裡邊緊澀住了,好不難捱。」一面用手摸之,窺見麈柄已被牝戶吞進半截,撐的兩邊皆滿。婦人用唾津塗抹牝戶兩邊,已而稍寬滑落,頗作往來,一舉一坐,漸沒至根。婦人因向西門慶說:「你每常使的顫聲嬌,在裡頭只是一味熱癢不可當,怎如和尚這藥,使進去,從子宮冷森森直掣到心上,這一回把渾身上下都酥麻了。我曉的今日死在你手裡了。好難捱忍也!」{\meipi{他人只蠢蠢然知快活而已,到金蓮便有許多賞鑑評品,妙人,妙人。}}西門慶笑道:「五兒,我有箇笑話兒說與你聽,是應二哥說的:一箇人死了,閻王就拿驢皮披在身上,教他變驢。落後判官查簿籍,還有他十三年陽壽,又放回來了。他老婆看見渾身都變過來了,只有陽物還是驢的,未變過來,那人道:『我往陰間換去。』他老婆慌了,說道:『我的哥哥,你這一去,只怕不放你回來怎了?等我慢慢兒的挨罷。』」婦人聽了,笑將扇把子打了一下子,說道:「怪不的應花子的老婆挨慣了驢的行貨。硶說嘴的賊,我不看世界,這一下打的你……」兩箇足纏了一箇更次,西門慶精還不過。他在下面合着眼,繇着婦人蹲踞在上極力抽提,提的龜頭刮答刮答怪响。提勾良久,又吊過身子去,朝向西門慶。西門慶雙手舉其股,僅沒其稜而提之,往來甚急。西門慶雖身接目視,而猶如無物。良久,婦人情急,轉過身子來,兩手摟定西門慶脖項,合伏在身上,舒舌頭在他口裡,那話直抵牝中,只顧揉搓,沒口子叫:「親達達,罷了,五兒的死了!」須臾,一陣昏迷,舌尖冰冷。泄訖一度,西門慶覺牝中一股熱氣直透丹田,心中翕翕然,美快不可言也。已而,淫津溢位,婦人以帕抹之。兩箇相摟相抱,交頭疊股,鳴咂其舌,那話通不拽出來。睡的沒半箇時辰,婦人淫情未定,爬上身去,兩箇又幹起來。婦人一連丟了兩遭身子,亦覺稍倦。西門慶只是佯佯不採,暗想胡僧藥神通。看看窻外雞鳴,東方漸白,婦人道:「我的心肝,你不過卻怎樣的?到晚夕你再來,等我好歹替你咂過了罷。」西門慶道:「就咂也不得過。管情只一樁事兒就過了。」婦人道:「告我說是那一樁兒?」西門慶道:「法不傳六耳,等我晚夕來對你說。」早晨起來梳洗,春梅打發穿上衣裳。韓道國、崔本又早外邊伺候。西門慶出來燒了紙,打發起身。交付二人兩封書:「一封到揚州馬頭上,投王伯儒店裡下;這一封就往揚州城內抓尋苗青,問他的事情下落,快來回報我。如銀子不勾,我後邊再教來保稍去。」崔本道:「還有蔡老爹書沒有?」西門慶道:「你蔡老爹書還不曾寫,教來保後邊稍了去罷。」二人拜辭,上頭口去了,不在話下。

西門慶冠帶了,就往衙門中來與夏提刑相會,道及昨承見招之意。夏提刑道:「今日奉屈長官一叙,再無他客。」發放已畢,各分散來家。只見一箇穿青衣皁隸,騎着快馬,夾着氊包,走的滿面汗流。到大門首,問平安:「此是提刑西門老爹家?」平安道:「你是那裡來的?」那人即便下馬作揖,說:「我是督催皇木的安老爹差來,送禮與老爹。俺老爹與管磚廠黃老爹,如今都往東平府胡老爹那裡吃酒,順便先來拜老爹,看老爹在家不在。」平安道:「有帖兒沒有?」那人向氊包內取出,連禮物都遞與平安。平安拿進去與西門慶看,見禮帖上寫着浙紬二端,湖綿四斤,香帶一束,古鏡一圓。分咐:「包五錢銀子,拿回帖打發來人,就說在家拱候老爹。」那人急急去了。

西門慶一面預備酒菜,等至日中,二位官員喝道而至,乘轎張蓋甚盛。先令人投拜帖,一箇是「侍生安忱拜」,一箇是「侍生黃葆光拜」。都是青雲白鷳補子,烏紗皁履,下轎揖讓而入。西門慶出大門迎接,至廳上叙禮,各道契闊之情,分賓主坐下:黃主事居左,安主事居右,西門慶主位相陪。先是黃主事舉手道:「久仰賢名芳譽,學生遲拜。」西門慶道:「不敢!辱承老先生先施枉駕,當容踵叩。敢問尊號?」安主事道:「黃年兄號泰宇,取『履泰定而發天光』之意。」黃主事道:「敢問尊號?」西門慶道:「學生賤號四泉,因小庄有四眼井之說。」{\meipi{寫西門慶市井口談,令人絕倒。}}安主事道:「昨日會見蔡年兄,說他與宋松原都在尊府打攪。」西門慶道:「因承雲峯尊命,又是敝邑公祖,敢不奉迎!小价在京已知鳳翁榮選,未得躬賀。」又問:「幾時起身府上來?」安主事道:「自去歲尊府別後,到家續了親,過了年,正月就來京了。選在工部,備員主事。欽差督運皇木,前往荊州,道經此處,敢不奉謁!」西門慶又說:「盛儀感謝不盡。」說畢,因請寬衣,令左右安放桌席。黃主事就要起身,安主事道:「實告:我與黃年兄,如今還往東平胡太府那裡赴席,因打尊府過,敢不奉謁。容日再來取擾。」西門慶道:「就是往胡公處,去路尚遠,縱二公不餓,其如從者何?學生敢不具酌,只備一飯在此,以犒從者。」

于是先打發轎上攢盤。廳上安放桌席。珍羞異品,極時之盛,就是湯飯點心、海鮮美味,一齊上來。西門慶將小金鍾,每人只奉了三盃,連桌兒擡下去,管待親隨家人吏典。少傾,兩位官人拜辭起身,安主事因向西門慶道:「生輩明日有一小東,奉屈賢公到我這黃年兄同僚劉老太監庄上一叙,未審肯命駕否?」西門慶道:「既蒙寵招,敢不趨命!」說畢,送出大門,上轎而去。只見夏提刑差人來邀。西門慶說道:「我就去。」一面分咐備馬,走到後邊換了冠帶衣服,出來上馬。玳安、琴童跟隨,排軍喝道,逕往夏提刑家來。到廳上叙禮,說道:「適有工部督催皇木安主政和磚廠黃主政來拜,留坐了半日,方纔去了。不然,也來的早。」說畢,讓至大廳,上面設放兩張桌席,讓西門慶居左,其次就是西賓倪秀才。{\pangpi{伏。}}座間因叙話問道:「老先生尊號?」倪秀才道:「學生賤名倪鵬,字時遠,號桂巖,見在府庠備數,在我這東主夏老先生門下,設館教習賢郎大先生舉業。友道之間,實有多愧。」說話間,兩箇小優兒上來磕頭,彈唱飲酒不題。

且說潘金蓮從打發西門慶出來,直睡到晌午纔爬起來。甫能起來,又懶待梳頭。恐怕後邊人說他,月娘請他吃飯也不吃,只推不好。大後晌纔出房門,{\meipi{一種風流睏倦情態,寫得懨懨在目。}}來到後邊。月娘因西門慶不在,要聽薛姑子講說佛法,演頌金剛科儀。在明間內安放一張經桌兒,焚下香。薛姑子與王姑子兩箇對坐,妙趣、妙鳳兩箇徒弟立在兩邊,接念佛號。大妗子、楊姑娘、吳月娘、李嬌兒、孟玉樓、潘金蓮、李瓶兒、孫雪娥和李桂姐衆人,一箇不少,都在跟前圍着他坐的,聽他演誦。先是,薛姑子道:

\begin{myquote}[\markfont]
「蓋聞電光易滅,石火難消。落花無返樹之期,逝水絕歸源之路。畫堂繡閣,命盡有若長空;極品高官,祿絕猶如作夢。黃金白玉,空為禍患之資;紅粉輕衣,總是塵勞之費。妻孥無百載之歡,黑暗有千重之苦。一朝枕上,命掩黃泉。青史揚虛假之名,黃土埋不堅之骨。田園百頃,其中被兒女爭奪;綾錦千箱,死後無寸絲之分。青春未半,而白髮來侵;賀者纔聞,而吊者隨至。{\meipi{讀數語,令人修行不及,為歡不及。奈何,奈何!}}苦,苦,苦!氣化清風塵歸土。點點輪迴喚不回,改頭換面無遍數。南無盡虛空遍法界,過去未來,佛法僧三寶。」

\kaishu{無上甚深微妙法,百千萬劫難遭遇。\\我今見聞得受持,願解如來眞實義。}
\end{myquote}

王姑子道:「當時釋迦牟尼佛,乃諸佛之祖,釋教之主,如何出家?願聽演說。」薛姑子便唱《五供養》:

\begin{myquote}[\markfont]
釋迦佛,梵王子,捨了江山雪山去,割肉喂鷹鵲巢頂。只修的九龍吐水混金身,纔成南無大乘大覺釋迦尊。
\end{myquote}

王姑子又道:「釋迦佛既聽演說,當日觀音菩薩如何修行,纔有莊嚴百化化身,有大道力?願聽其說。」薛姑子正待又唱,只見平安兒慌慌張張走來說道:「巡按宋爺差了兩箇快手、一箇門子送禮來。」月娘慌了,說道:「你爹往夏家吃酒去了,誰人打發他?」正說着,只見玳安兒回馬來家,放進氊包來,說道:「不打緊,等我拿帖兒對爹說去。教姐夫且請那門子進來,管待他些酒飯兒着。」{\meipi{玳安畢竟有正景,有主意。後之能為小員外者,非盡僥倖。}}這玳安交下氊包,拿着帖子,騎馬雲飛般走到夏提刑家,如此這般,說巡按宋老爺送禮來。西門慶看了帖子,上寫着「鮮豬一口,金酒二尊,公紙四刀,小書一部」,下書「侍生宋喬年拜」。連忙分咐:「到家交書童快拿我的官御雙摺手本回去,門子答賞他三兩銀子、兩方手帕,擡盒的每人與他五錢。」玳安來家,到處尋書童兒,那裡得來?急的只牛回磨轉。陳敬濟又不在,交傅夥計陪着人吃酒,玳安旋打後邊討了手帕、銀子出來,又沒人封,自家在櫃上彌封停當,叫傅夥計寫了,大小三包。因向平安兒道:「你就不知往那去了?」平安道:「頭裡姐夫在家時,他還在家來。落後姐夫往門外討銀子去了,他也不見了。」玳安道:「別要題,已定秫秫小厮在外邊胡行亂走的,養老婆去了。」正在急唣之間,只見陳敬濟與書童兩箇,疊騎騾子纔來,被玳安罵了幾句,教他寫了官御手本,打發送禮人去了。玳安道:「賊秫秫小厮,仰𢵞着掙了合蓬着去。{\pangpi{寫得恰好。}}爹不在,家裡不看,跟着人養老婆兒去了。爹又沒使你和姐夫門外討銀子,你平白跟了去做甚麼!看我對爹說不說!」書童道:「你說不是,我怕你?你不說就是我的兒。」{\pangpi{寫出大膽。}}玳安道:「賊狗攮的秫秫小厮,你賭幾箇眞箇?」走向前,一箇潑脚撇翻倒,兩箇就磆碌成一塊了。那玳安得手,吐了他一口唾沫纔罷了。說道:「我接爹去,等我來家和淫婦算帳。」騎馬一直去了。月娘在後邊,打發兩箇姑子吃了些茶食,又聽他唱佛曲兒,宣念偈子。那潘金蓮不住在旁先拉玉樓不動,又扯李瓶兒,又怕月娘說。月娘便道:「李大姐,他叫你,你和他去不是。省的急的他在這裡恁有㓦劃沒是處的。」那李瓶兒方纔同他出來。被月娘瞅了一眼,說道:「拔了蘿蔔地皮寬。交他去了,省的他在這裡跑兔子一般。原不是聽佛法的人。」{\meipi{金蓮之動,玉樓之靜,月娘之憎,瓶兒之隨,人各一心,心各一口,各說各是,都為寫出。}}這潘金蓮拉着李瓶兒走出儀門,因說道:「大姐姐好幹這營生,你家又不死人,平白交姑子家中宣起卷來了。都在那裡圍着他怎的?咱們出來走走,就看看大姐在屋裡做甚麼哩。」于是一直走出大廳來。只見廂房內點着燈,大姐和敬濟正在裡面絮聒,說不見了銀子。被金蓮向窻櫺上打了一下,說道:「後面不去聽佛曲兒,兩口子且在房裡拌的甚麼嘴兒?」陳敬濟出來,看見二人,說道:「早是我沒曾罵出來,原是五娘、六娘來了。請進來坐。」金蓮道:「你好膽子,罵不是!」進來見大姐正在燈下納鞋,說道:「這咱晚,熱剌剌的,還納鞋?」因問:「你兩口子嚷的是些甚麼?」陳敬濟道:「你問他。爹使我門外討銀子去,他與了我三錢銀子,就教我替他稍銷金汗巾子來。不想到那裡,袖子裡摸銀子沒了,不曾稍得來。來家他說我那裡養老婆,和我嚷罵了這一日,急的我賭身發咒。不想丫頭掃地,地下拾起來。他把銀子收了不與,還教我明日買汗巾子來。{\meipi{大姐既無容,又無情,徒以父母之勢降伏其夫,豈婦道哉!後之不得其死,有繇然矣。}}你二位老人家說,卻是誰的不是?」那大姐便罵道:「賊囚根子,別要說嘴。你不養老婆,平白帶了書童兒去做甚麼?剛纔教玳安甚麼不罵出來!想必兩箇打夥兒養老婆去來。去到這咱晚纔來,你討的銀子在那裡?」金蓮問道:「有了銀子不曾?」大姐道:「剛纔丫頭掃地,拾起來,我拿着哩。」金蓮道:「不打緊處。我與你些銀子,明日也替我帶兩方銷金汗巾子來。」李瓶兒便問:「姐夫,門外有,也稍幾方兒與我。」敬濟道:「門外手帕巷有名王家,專一發賣各色改樣銷金點翠手帕汗巾兒,隨你要多少也有。你老人家要甚麼顏色,銷甚花樣,早說與我,明日都替你一齊帶的來了。」李瓶兒道:「我要一方老黃銷金點翠穿花鳳的。」敬濟道:「六娘,老金黃銷上金不現。」李瓶兒道:「你別要管我。我還要一方銀紅綾銷江牙海水嵌八寶兒的,又是一方閃色芝蔴花銷金的。」敬濟便道:「五娘,你老人家要甚花樣?」金蓮道:「我沒銀子,只要兩方兒勾了。要一方玉色綾瑣子地兒銷金的。」敬濟道:「你又不是老人家,白剌剌的,要他做甚麼?」金蓮道:「你管他怎的!戴不的,等我往後有孝戴。」{\meipi{咒人。}}敬濟道:「那一方要甚顏色?」金蓮道:「那一方,我要嬌滴滴紫葡萄顏色四川綾汗巾兒。上銷金間點翠,十樣錦,同心結,方勝地兒。一箇方勝兒裡面一對兒喜相逢,兩邊欄子兒,都是纓絡珍珠碎八寶兒。」敬濟聽了,說道:「耶嚛,耶嚛!再沒了?『賣瓜子兒開啟箱子打嚏噴——瑣碎一大堆』。」金蓮道:「怪短命,有錢買了稱心貨,隨各人心裡所好,你管他怎的!」李瓶兒便向荷包裡拿出一塊銀子兒,遞與敬濟,說:「連你五娘的都在裡頭了。」金蓮搖着頭兒說道:「等我與他罷。」{\pangpi{又受了,又不服氣。}}李瓶兒道:「都一答交姐夫稍了來,那又起箇窖兒!」敬濟道:「就是連五娘的,這銀子還多着哩。」一面取等子稱稱,一兩九錢。李瓶兒道:「剩下的就與大姑娘稍兩方來。」大姐連忙道了萬福。金蓮道:「你六娘替大姐買了汗巾兒,把那三錢銀子拿出來,你兩口兒鬬葉兒,賭了東道罷。{\meipi{夫妻輸贏都要拿出來,何必賭?騙法妙甚。}}少便叫你六娘貼些兒出來,{\pangpi{還不饒他,惡人。}}明日等你爹不在,買燒鴨子、白酒咱每吃。」敬濟道:「既是五娘說,拿出來。」大姐遞與金蓮,金蓮交付與李瓶兒收着。拿出紙牌來,燈下大姐與敬濟鬬。金蓮又在旁替大姐指點,登時贏了敬濟三掉。忽聽前邊打門,西門慶來家,金蓮與李瓶兒纔回房去了。敬濟出來迎接西門慶回了話,說徐四家銀子,後日先送二百五十兩來,餘者出月交還。西門慶罵了幾句,酒帶半酣,也不到後邊,逕往金蓮房裡來。正是:

\begin{myquote}
自有內事迎郎意,何怕明朝花不開。
\end{myquote}

