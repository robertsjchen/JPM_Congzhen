\includepdf[pages={9,10},fitpaper=false]{tst.pdf}
\chapter*{第五回 捉姦情鄆哥定計 飲酖藥武大遭殃}
\addcontentsline{toc}{chapter}{第五回 捉姦情鄆哥定計 飲酖藥武大遭殃}
\markboth{{\titlename}卷之一}{第五回 捉姦情鄆哥定計 飲酖藥武大遭殃}


詩曰:

\begin{myquote} 
參透風流二字禪,好姻緣是惡姻緣。\\癡心做處人人愛,冷眼觀時箇箇嫌。\\野草閑花休採折,眞姿勁質自安然。\\山妻稚子家常飯,不害相思不損錢。
\end{myquote} 

話說當下鄆哥被王婆打了,心中正沒出氣處,提了雪梨籃兒,一徑奔來街上尋武大郎。轉了兩條街,只見武大挑着炊餅担兒,正從那條街過來。鄆哥見了,立住了脚,看着武大道:「這幾時不見你,吃得肥了!」武大歇下担兒道:「我只是這等模樣,有甚吃得肥處?」鄆哥道:「我前日要糴些麥稃,一地裡沒糴處,人都道你屋裡有。」武大道:「我屋裡並不養鵝鴨,那裡有這麥稃?」鄆哥道:「你說沒麥稃,怎的撰得你恁肥𦞂𦞂的,便顛倒提你起來也不妨,煮你在鍋裡也沒氣。」{\meipi{善說不如善激。}}武大道:「小囚兒,倒罵得我好!我的老婆又不偷漢子,{\pangpi{未必。}}我如何是鴨?」鄆哥道:「你老婆不偷漢子,只偷子漢。」武大扯住鄆哥道:「還我主兒來!」鄆哥道:「我笑你只會扯我,卻不道咬下他左邊的來。」武大道:「好兄弟,你對我說是誰,我把十箇炊餅送你。」鄆哥道:「炊餅不濟事。你只做箇東道,我吃三盃,便說與你。」武大道:「你會吃酒?跟我來。」武大挑了担兒,引着鄆哥,到箇小酒店裡,歇下担兒,拿幾箇炊餅,買了些肉,討了一鏇酒,請鄆哥吃着。武大道:「好兄弟,你說與我則箇。」鄆哥道:「且不要慌,等我一發吃完了,卻說與你。你卻不要氣苦,我自幫你打捉。」武大看那猴子吃了酒肉:「你如今卻說與我。」鄆哥道:「你要得知,把手來摸我頭上的肐𦞂。」武大道:「卻怎地來有這肐𦞂?」鄆哥道:「我對你說,我今日將這籃雪梨去尋西門大官,一地裡沒尋處。街上有人道:『他在王婆茶坊裡來,和武大娘子勾搭上了,每日只在那裡行走。』我指望見了他,撰他三五十文錢使。尀耐王婆那老豬狗,不放我去房裡尋他,大栗暴打出我來。我特地來尋你。我方纔把兩句話來激你,我不激你時,你須不來問我。」武大道:「眞箇有這等事?」鄆哥道:「又來了,我道你這般屁鳥人!那厮兩箇落得快活,只專等你出來,便在王婆房裡做一處。你問道眞箇也是假,難道我哄你不成?」武大聽罷,道:「兄弟,我實不瞞你說,我這婆娘每日去王婆家裡做衣服,做鞋脚,歸來便臉紅。我先妻丟下箇女孩兒,朝打暮罵,不與飯吃,這兩日有些精神錯亂,見了我,不做歡喜。我自也有些疑忌在心裡,這話正是了。我如今寄了担兒,便去捉姦如何?」鄆哥道:「你老大一條漢,元來沒些見識!那王婆老狗,什麼利害怕人的人!你如何出得他手?他二人也有箇暗號兒,見你入來拿他,把你老婆藏過了。那西門慶須了得!打你這般二十箇。若捉他不着,反吃他一頓好拳頭。他又有錢有勢,反告你一狀子,你須吃他一場官司,又沒人做主,乾結果了你性命!」{\meipi{此兒大有作用,然亦多事。}}武大道:「兄弟,你都說得是。我卻怎的出得這口氣?」鄆哥道:「我吃那王婆打了,也沒出氣處。我教你一着:今日歸去,都不要發作,也不要說,只自做每日一般。明朝便少做些炊餅出來賣,我自在巷口等你。若是見西門慶入去時,我便來叫你。你便挑着担兒只在左近等我。我先去惹那老狗,他必然來打我。我先把籃兒丟出街心來,你卻搶入。我便一頭頂住那婆子,你便奔入房裡去,叫起屈來。此計如何?」武大道:「既是如此,卻是虧了兄弟。我有兩貫錢,我把你去,你到明日早早來紫石街巷口等我。」鄆哥得了錢並幾箇炊餅,自去了。武大還了酒錢,挑了担兒,自去賣了一遭歸去。原來這婦人,往常時只是罵武大,百般的欺負他。近日來也自知無禮,只得窩盤他些箇。{\meipi{亦能自反。}}當晚武大挑了担兒歸來,也是和往日一般,並不題起別事。那婦人道:「大哥,買盞酒吃?」武大道:「卻纔和一般經紀人買了三盞吃了。」那婦人便安排晚飯與他吃了。當夜無話。次日飯後,武大只做三兩扇炊餅,安在担兒上。這婦人一心只想着西門慶,那裡來理會武大的做多做少。當日武大挑了担兒,自出去做買賣。這婦人巴不的他出去了,便踅過王婆茶坊裡來等西門慶。且說武大挑着担兒,出到紫石街巷口,迎見鄆哥提着籃兒在那裡張望。{\pangpi{有心哉。}}武大道:「如何?」鄆哥道:「還早些箇。你自去賣一遭來,那厮七八也將來也。你只在左近處伺候,不可遠去了。」武大雲飛也似去賣了一遭回來。鄆哥道:「你只看我籃兒丟擲來,你便飛奔入去。」武大把担兒寄下,不在話下。

卻說鄆哥提着籃兒,走入茶坊裡來,向王婆罵道:「老豬狗!你昨日為甚麼便打我?」那婆子舊性不改,便跳身起來喝道:「你這小猢猻!老娘與你無干,你如何又來罵我?」{\meipi{王婆老矣,戒之口得,且又全以血氣用事,宜乎其敗也。}}鄆哥道:「便罵你這馬伯六,做牽頭的老狗肉,直我𩫻䯲!」那婆子大怒,揪住鄆哥便打。鄆哥叫一聲:「你打我!」把那籃兒丟出當街上來。那婆子卻待揪他,被這小猴子叫一聲「你打」時,就打王婆腰裡帶箇住,看着婆子小肚上,只一頭撞將去,{\pangpi{仔細着,撞了進去。}}險些兒不跌倒,卻得壁子礙住不倒。那猴子死頂在壁上。{\pangpi{可觀}}只見武大從外裸起衣裳,大踏步直搶入茶坊裡來。那婆子見是武大,來得甚急,待要走去阻當,卻被這小猴子死力頂住,那裡肯放!婆子只叫得「武大來也!」那婦人正和西門慶在房裡,做手脚不迭,先奔來頂住了門。這西門慶便鑽入床下躲了。武大搶到房門首,用手推那房門時,那裡推得開!口裡只叫「做得好事!」那婦人頂着門,慌做一團,{\pangpi{可觀。}}口裡便說道:「你閑常時只好鳥嘴,賣弄殺好拳棒,臨時便沒些用兒!見了紙虎兒也嚇一交!」那婦人這幾句話,分明叫西門慶來打武大,奪路走。西門慶在床底下聽了婦人這些話,提醒他這箇念頭,便鑽出來說道:「不是我沒這本事,一時間沒這智量。」{\meipi{語云:能搏猛虎,不能不變色於蜂蠆,理固然也。}}便來拔開門,叫聲「不要來!」武大卻待揪他,被西門慶早飛起脚來。武大矮小,正踢中心窩,撲地望後便倒了。西門慶打鬧裡一直走了。鄆哥見勢頭不好,也撇了王婆,撒開跑了。街坊隣舍,都知道西門了得,誰敢來管事?王婆當時就地下扶起武大來,見他口裡吐血,面皮臘渣也似黃了,便叫那婦人出來,舀碗水來救得甦醒,兩箇上下肩攙着,便從後門歸到家中樓上去,安排他床上睡了。當夜無話。

次日,西門慶打聽得沒事,依前自來王婆家,和這婦人頑耍,只指望武大自死。武大一病五日不起,更兼要湯不見,要水不見,每日叫那婦人又不應。只見他濃粧艷抹了出去,歸來便臉紅。小女迎兒又吃婦人禁住,不得向前,嚇道:「小賤人,你不對我說,與了他水吃,都在你身上!」那迎兒見婦人這等說,怎敢與武大一點湯水吃!武大幾遍只是氣得發昏,又沒人來採問。一日,武大叫老婆過來,分付他道:「你做的勾當,我親手捉着你姦,你倒挑撥姦夫踢了我心。至今求生不生,求死不死,你們卻自去快活。我死自不妨,和你們爭執不得了。我兄弟武二,你須知他性格,倘或早晚歸來,他肯干休?你若肯可憐我,早早扶得我好了,他歸來時,我都不提起。你若不看顧我時,待他歸來,卻和你們說話。」這婦人聽了,也不回言,卻踅過王婆家來,一五一十都對王婆和西門慶說了。那西門慶聽了這話,似提在冷水盆內一般,說道:「苦也!我須知景陽岡上打死大蟲的武都頭。我如今卻和娘子眷戀日久,情孚意合,拆散不開。據此等說時,正是怎生得好?卻是苦也!」王婆冷笑道:「我倒不曾見,你是箇把舵的,我是箇撐船的,我倒不慌,你倒慌了手脚!」西門慶道:「我枉自做箇男子漢,到這般去處,卻擺佈不開。你有甚麼主見,遮藏我們則箇。」王婆道:「既要我遮藏你們,我有一條計。你們卻要長做夫妻,短做夫妻?」{\meipi{老奸可剮。}}西門慶道:「乾娘,你且說如何是長做夫妻、短做夫妻?」王婆道:「若是短做夫妻,你們就今日便分散。等武大將息好了起來,與他陪了話。武二歸來都沒言語,待他再差使出去,卻又來相會。這是短做夫妻。你們若要長做夫妻,每日同在一處,不耽驚受怕,我卻有這條妙計,只是難教你們!」西門慶道:「乾娘,周旋了我們則箇,只要長做夫妻。」王婆道:「這條計用着件東西,別人家裡都沒,天生天化,大官人家裡卻有。」西門慶道:「便是要我的眼睛,也剜來與你。卻是甚麼東西?」王婆道:「如今這搗子病得重,趁他狼狽,好下手。{\meipi{可殺。}}大官人家裡取些砒霜,卻交大娘子自去贖一帖心疼的藥來,卻把這砒霜下在裡面,把這矮子結果了,一把火燒得乾乾淨淨,沒了蹤跡。便是武二回來,他待怎的?自古道:『幼嫁從親,再嫁繇身。』小叔如何管得暗地裡事!半年一載,等待夫孝滿日,大官人娶到家去。這不是長遠夫妻,偕老同歡!此計如何?」西門慶道:「乾娘此計甚妙。

自古道:『欲救生快活,須下死功夫。』罷罷罷!一不做,二不休。」王婆道:「可知好哩!這是剪草除根,萌芽不發。{\pangpi{可殺。}}大官人往家裡去快取此物來,我自教娘子下手。事了時,卻要重重謝我。」西門慶道:「這箇自然,不消你說。」

\begin{myquote} 
雲情雨意兩綢繆,戀色迷花不肯休。\\畢竟人生如泡影,何須死下殺人謀?
\end{myquote} 

且說西門慶去不多時,包了一包砒霜,遞與王婆收了。這婆子看着那婦人道:「大娘子,我教你下藥的法兒。如今武大不對你說教你救活他?你便乘此把些小意兒貼戀他。他若問你討藥吃時,便把這砒霜調在心疼藥裡。待他一覺身動,你便把藥灌將下去。他若毒氣發時,必然腸胃迸斷,大叫一聲。你卻把被一蓋,不要使人聽見,緊緊的按住被角。{\meipi{劊子手無此毒腸。老奸百剁不足贖矣。}}預先燒下一鍋湯,煮着一條抹布。他那藥發之時,必然七竅內流血,口唇上有牙齒咬的痕跡。他若放了命,你便揭起被來,卻將煮的抹布只一揩,都揩沒了血跡,便入在材裡,扛出去燒了,有甚麼不了事!」那婦人道:「好卻是好,{\pangpi{有何好處?可恨,可恨。}}只是奴家手軟,臨時安排不得屍首。」婆子道:「這箇易得。你那邊只敲壁子,我自過來幫扶你。」西門慶道:「你們用心整理,明日五更,我來討話。」說罷,自歸家去了。王婆把這砒霜用手撚為細末,遞與婦人,將去藏了。

那婦人回到樓上,看着武大,一絲沒了兩氣,看看待死。那婦人坐在床邊假哭。武大道:「你做甚麼來哭?」婦人拭着眼淚道:「我的一時間不是,吃那西門慶局騙了。誰想脚踢中了你心。我問得一處有好藥,我要去贖來醫你,又怕你疑忌,不敢去取。」武大道:「你救我活,無事了,一筆都勾。武二來家,亦不提起。你快去贖藥來救我則箇!」那婦人拿了銅錢,逕來王婆家裡坐地,卻教王婆贖得藥來。把到樓上,交武大看了,說道:「這帖心疼藥,太醫交你半夜裡吃了,倒頭一睡,蓋一兩床被,發些汗,明日便起得來。」武大道:「卻是好也。生受大嫂,今夜醒睡些,半夜調來我吃。」{\pangpi{可憐。}}那婦人道:「你放心睡,我自扶持你。」{\meipi{讀此而不髮指心裂者,非情也。}}看看天色黑了,婦人在房裡點上燈,下面燒了大鍋湯,拿了一方抹布煮在鍋裡。聽那更鼓時,卻正好打三更。那婦人先把砒霜傾在盞內,卻舀一碗白湯,把到樓上,叫聲:「大哥,藥在那裡?」武大道:「在我蓆子底下枕頭邊,你快調來我吃!」{\pangpi{可憐。}}那婦人揭起蓆子,將那藥抖在盞子裡,將白湯冲在盞內,把頭上銀簪兒只一攪,調得勻了。左手扶起武大,右手把藥便灌。武大呷了一口,說道:「大嫂,這藥好難吃!」那婦人道:「只要他醫得病好,{\pangpi{當眞好否?}}管甚麼難吃!」武大再呷第二口時,被這婆娘就勢只一灌,一盞藥都灌下喉嚨去了。那婦人便放倒武大,慌忙跳下床來。武大哎了一聲,說道:「大嫂,吃下這藥去,肚裡倒疼起來。苦呀,苦呀!倒當不得了。」這婦人便去脚後扯過兩床被來,沒頭沒臉只顧蓋。武大叫道:「我也氣悶!」那婦人道:「太醫分付,教我與你發些汗,便好的快。」武大再要說時,這婦人怕他掙扎,便跳上床來,騎在武大身上,把手緊緊的按住被角,那裡肯放些鬆寬!正是:

\begin{myquote} 
油煎肺腑,火燎肝腸。心窩裡如霜刀相侵,滿腹中似鋼刀亂攪。渾身冰冷,七竅血流。牙關緊咬,三魂赴枉死城中;喉管枯乾,七魄投望鄉臺上。地獄新添食毒鬼,陽間沒了捉姦人。
\end{myquote} 

那武大當時哎了兩聲,喘息了一回,腸胃迸斷,嗚呼哀哉,身體動不得了。那婦人揭起被來,見了武大咬牙切齒,七竅流血,怕將起來,只得跳下床來,敲那壁子。王婆聽得,走過後門頭咳嗽。那婦人便下樓來,開了後門。王婆問道:「了也未?」那婦人道:「了便了了,只是我手脚軟了,安排不得。」王婆道:「有甚麼難處,我幫你便了。」那婆子便把衣袖捲起,舀了一桶湯,把抹布撇在裡面,掇上樓來。捲過了被,先把武大口邊唇上都抹了,卻把七竅淤血痕跡拭淨,便把衣裳蓋在身上。兩箇從樓上一步一掇扛將下來,就樓下尋扇舊門停了。與他梳了頭,戴上巾幘,穿了衣裳,取雙鞋襪與他穿了,將片白絹蓋了臉,揀床乾淨被蓋在死屍身上。卻上樓來,收拾得乾淨了,王婆自轉將歸去了。那婆娘卻號號地假哭起「養家人」來。看官聽說:原來但凡世上婦人哭有三樣:有淚有聲謂之哭,有淚無聲謂之泣,無淚有聲謂之號。當下那婦人乾號了半夜。

次早五更,天色未曉,西門慶奔來討信。王婆說了備細。西門慶取銀子把與王婆,教買棺材發送,就叫那婦人商議。這婆娘過來和西門慶說道:「我的武大今日已死,我只靠着你做主!不到後來『網巾圈兒打靠後』。」西門慶道:「這箇何須你費心!」婦人道:「你若負了心,怎的說?」西門慶道:「我若負了心,就是武大一般!」{\meipi{此誓非虛,要曉得金蓮手段原硬。}}王婆道:「大官人,如今只有一件事要緊:天明就要入殮,只怕被仵作看出破綻來怎了?團頭何九,他也是箇精細的人,只怕他不肯殮。」西門慶笑道:「這箇不妨事。何九我自分付他,他不敢違我的言語。」王婆道:「大官人快去分付他,不可遲了。」西門慶自去對何九說去了。正是:

\begin{myquote} 
三光有影誰能待,萬事無根只自生。\\雪隱鷺鷥飛始見,柳藏鸚鵡語方聞。
\end{myquote} 

