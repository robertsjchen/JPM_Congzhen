\includepdf[pages={93,94},fitpaper=false]{tst.pdf}
\chapter*{第四十七囘 苗青貪財害主 西門枉法受賍}
\addcontentsline{toc}{chapter}{第四十七囘 苗青貪財害主 西門枉法受賍}
\markboth{{\titlename}卷之五}{第四十七囘 苗青貪財害主 西門枉法受賍}


詩曰:

\begin{myquote}
懷璧身堪罪,償金跡未明。\\龍蛇一失路,虎豹屢相驚。\\蹔遣虞羅急,終知漢法平。\\須憑魯連箭,為汝謝聊成。
\end{myquote}

話說江南揚州廣陵城內,有一苗員外,名喚苗天秀。家有萬貫資財,頗好詩禮。年四十歲,身邊無子,止有一女尚未出嫁。其妻李氏,身染痼疾在床,家事盡托與寵妾刁氏,名喚刁七兒。原是娼妓出身,天秀用銀三百兩娶來家,納為側室,寵嬖無比。{\meipi{要娼妓便是死兆。}}忽一日,有一老僧在門首化緣,自稱是東京報恩寺僧,因為堂中缺少一尊鍍金銅羅漢,故雲遊在此,訪善紀錄。天秀聞之,不吝,即施銀五十兩與那僧人。僧人道:「不消許多,一半足矣。」天秀道:「吾師休嫌少,除完佛像,餘剩可作齋供。」那僧人問訊致謝,臨行向天秀說道:「員外左眼眶下有一道死氣,主不出此年當有大災。你有如此善緣與我,貧僧焉敢不預先說知。今後隨有甚事,切勿出境。戒之戒之。」言畢,作辭而去。

那消半月,天秀偶遊後園,見其家人苗青正與刁氏亭側私語,不意天秀卒至看見,不由分說,將苗青痛打一頓,誓欲逐之。苗青恐懼,轉央親隣再三勸留得免,終是切恨在心。不期有天秀表兄黃美,原是揚州人氏,乃舉人出身,在東京開封府做通判,亦是博學廣識之人。一日,寄一封書來與天秀,要請天秀上東京,一則遊玩,二者為謀其前程。苗天秀得書大喜,因向其妻妾說道:「東京乃輦轂之地,景物繁華,吾心久欲遊覽,無由得便。今不期表兄書來相招,實慰平生之意。」其妻李氏便說:「前日僧人相你面上有災厄,囑咐不可出門。此去京都甚遠,況你家私沉重,拋下幼女病妻在家,未審此去前程如何,不如勿往為善。」天秀不聽,反加怒叱,說道:「大丈夫生於天地之間,桑弧蓬矢,不能邀遊天下,觀國之光,徒老死牖下無益矣。況吾胸中有物,囊有餘資,何愁功名不到手?此去表兄必有美事於我,切勿多言!」於是分付家人苗青,收拾行李衣裝,多打點兩箱金銀,載一船貨物,帶了個安童並苗青,上東京。囑咐妻妾守家,擇日起行。正値秋末冬初之時,從揚州碼頭上船,行了數日,到徐州洪。但見一派水光,十分陰惡。但見:

\begin{myquote}
萬里長洪水似傾,東流海島若雷鳴。\\滔滔雪浪令人怕,客旅逢之誰不驚?
\end{myquote}

前過地名陝灣,苗員外看見天晚,命舟人泊住船隻。也是天數將盡,合當有事,不料搭的船隻卻是賊船。兩個艄子皆是不善之徒:一個名喚陳三,一個乃是翁八。常言道:不着家人,弄不得家鬼。這苗青深恨家主,日前被責之仇,一向要報無繇,口中不言,心內暗道:「不如我如此這般,與兩個艄子做一路,將家主害了性命,推在水內,盡分其財物。我囘去再把病婦謀死,這分家私連刁氏,都是我情受的。」正是:

\begin{myquote}
花枝葉下猶藏刺,人心怎保不懷毒。
\end{myquote}

這苗青於是與兩個艄子密密商量,說道:「我家主皮箱中還有一千兩金銀,二千兩段疋,衣服之類極廣。汝二人若能謀之,願將此物均分。」陳三、翁八笑道:「汝若不言,我等亦有此意久矣。」是夜天氣陰黑,苗天秀與安童在中艙裡睡,苗青在櫓後。將近三鼓時分,那苗青故意連叫有賊。苗天秀夢中驚醒,便探頭出艙外觀看,被陳三手持利刀,一下刺中脖下,推在洪波盪裡。那安童正要走時,吃翁八一悶棍打落水中。三人一面在船艙內開啟箱籠,取出一應財帛金銀,並其段貨衣服,點數均分。二艄便說:「我若留此貨物,必然有犯。你是他手下家人,載此貨物到於市店上發賣,沒人相疑。」因此二艄盡把皮箱中一千兩金銀,並苗員外衣服之類分訖,依前撐船囘去了。這苗青另搭了船隻,載至臨清碼頭上,鈔關上過了,裝到清河縣城外官店內卸下,見了揚州故舊商家,只說:「家主在後船,便來也。」這個苗青在店發賣貨物,不題。

常言:人便如此如此,天理未然未然。可憐苗員外平昔良善,一旦遭其僕人之害,不得好死,雖是不納忠言之勸,其亦大數難逃。不想安童被一棍打昏,雖落水中,幸得不死,浮沒蘆港。忽有一隻漁船撐將下來,船上坐着個老翁,頭頂箬笠,身披短蓑,聽得啼哭之聲。移船看時,卻是一個十七八歲小厮,慌忙救了。問其始末情繇,卻是揚州苗員外家安童,在洪上被劫之事。這漁翁帶下船,取衣服與他換了,給以飲食,因問他:「你要囘去,卻是同我在此過活?」安童哭道:「主人遭難,不見下落,如何囘得家去?願隨公公在此。」漁翁道:「也罷,你且隨我在此,等我慢慢替你訪此賊人是誰,再作理會。」安童拜謝公公,遂在此翁家過活。

一日,也是合當有事。年除歲末,漁翁忽帶安童正出河口賣魚,正撞見陳三、翁八在船上飲酒,穿着他主人衣服,上岸來買魚。安童認得,即密與漁翁說道:「主人之冤當雪矣。」漁翁道:「何不具狀官司處告理?」安童將情具告到巡河周守備府內。守備見沒賍證,不接狀子。又告到提刑院。夏提刑見是強盜劫殺人命等事,把狀批行了。從正月十四日差緝捕公人,押安童下來拏人。前至新河口,只把陳三、翁八獲住到案,責問了口詞。二艄見安童在旁執證,也沒得動刑,一一招了。供稱:「下手之時,還有他家人苗青,同謀殺其家主,分賍而去。」這裡把三人監下,又差人訪拏苗青,一起定罪。因節間放假,提刑官吏一連兩日沒來衙門中問事,早有衙門透信的人,悄悄把這件事兒報與苗青。

苗青慌了,把店門鎖了,暗暗躲在經紀樂三家。這樂三就住在獅子街韓道國家隔壁,他渾家樂三嫂,與王六兒所交極厚,常過王六兒這邊來做伴兒。王六兒無事,也常往他家行走,彼此打的熱鬧。這樂三見苗青面帶憂容,問其所以,說道:「不打緊,間壁韓家就是提刑西門老爹的外室,又是他家伙計,和俺家交往的甚好,幾事百依百隨,若要保得你無事,破多少東西,教俺家過去和他家說說。」這苗青聽了,連忙下跪,說道:「但得我身上沒事,恩有重報,不敢有忘。」於是寫了說帖,封下五十兩銀子,兩套粧花段子衣服,樂三教他老婆拏過去,如此這般對王六兒說。王六兒喜歡的要不的,把衣服銀子並說帖都收下,單等西門慶,不見來。

到十七日日西時分,只見玳安夾着氊包,騎着頭口,從街心裡來。王六兒在門首,叫下來問道:「你往那裡去來?」玳安道:「我跟爹走了個遠差,往東平府送禮去來。」王六兒道:「你爹如今來了不曾?」玳安道:「爹和賁四兩個先往家去了。」王六兒便叫進去,和他如此這般說話,拏帖兒與他瞧,玳安道:「韓大嬸,管他這事!休要把事輕看了,如今衙門裡監着那兩個船家,供着只要他哩。拏過幾兩銀子來,也不勾打發脚下人哩。我不管別的帳,韓大嬸和他說,只與我二十兩銀子罷。等我請將俺爹來,隨你老人家與俺爹說就是了。」王六兒笑道:「恠油嘴兒,要飯吃休要惡了火頭。事成了,你的事甚麼打緊?寧可我們不要,也少不得你的。」玳安道:「韓大嬸,不是這等說。常言:君子不羞當面。先斷過,後商量。」王六兒當下備幾樣菜,留玳安吃酒。玳安道:「吃的紅頭紅臉,怕家去爹問,卻怎的囘爹?」王六兒道:「怕怎的?你就說在我這裡來。」玳安只吃了一甌子,就走了。王六兒道:「好歹累你,說是我這裡等着哩。」

玳安一直來家,交進氊包。等的西門慶睡了一覺出來,在廂房中坐的。這玳安慢慢走到跟前,說:「小的囘來,韓大嬸叫住小的,要請爹快些過去,有句要緊話和爹說。」西門慶說:「甚麼話?我知道了。」說畢,正値劉學官來借銀子。{\pangpi{又映官吏債。}}打發劉學官去了,西門慶騎馬,帶着眼紗、小帽,便叫玳安、琴童兩個跟隨,來到王六兒家。下馬進去,到明間坐下,王六兒出來拜見了。那日,韓道國鋪子裡上宿,沒來家。老婆買了許多東西,叫老馮廚下整治。見西門慶來了,慌忙遞茶。西門慶分付琴童:「把馬送到對門房子裡去,把大門關上。」婦人且不敢就題此事,先只說:「爹家中連日擺酒辛苦。我聞得說哥兒定了親事,你老人家喜呀!」西門慶道:「只因舍親吳大妗那裡說起,和喬家做了這門親事。他家也只這一個女孩兒,論起來也還不般配,胡亂親上做親罷了。」王六兒道:「就是和他做親也好,只是爹如今居着恁大官,會在一處,不好意思的。」西門慶道:「說甚麼哩!」說了一囘,老婆道:「只怕爹寒冷,往房裡坐去罷。」一面讓至房中,一面安着一張椅兒,籠着火盆,西門慶坐下。婦人慢慢先把苗青揭帖拏與西門慶看,說:「他央了間壁經紀樂三娘子過來對我說:這苗青是他店裡客人,如此這般,被兩個船家拽扯,只望除豁了他這名字,擴音他。他備了些禮兒在此謝我。好歹望老爹怎的將就他罷。」

西門慶看了帖子,因問:「他拏了多少禮物謝你?」王六兒向箱中取出五十兩銀子來與西門慶瞧,說道:「明日事成,還許兩套衣裳。」西門慶看了,笑道:「這些東西兒,平白你要他做甚麼?你不知道,這苗青乃揚州苗員外家人,因為在船上與兩個船家殺害家主,攛在河裡,圖財謀命。如今見打撈不着屍首,他原跟來的一個小厮安童與兩個船家,當官三口執證着要他。這一拏去,穩定是個淩遲罪名。那兩個都是眞犯斬罪。兩個船家見供他有二千兩銀貨在身上。拏這些銀子來做甚麼?還不快送與他去!」這王六兒一面到廚下,使了丫頭錦兒把樂三娘子兒叫了來,將原禮交付與他,如此這般對他說了去。那苗青不聽便罷,聽他說了,猶如一桶水頂門上直灌到脚底下。正是:

\begin{myquote}
驚開六葉連肝肺,唬壞三魂七魄心。
\end{myquote}

即請樂三一處商議道:「寧可把二千貨銀都使了,只要救得性命家去。」樂三道:「如今老爹上邊既發此言,一些半些恆屬打不動。兩位官府,須得湊一千貨物與他。其餘節級、原解、緝捕,再得一半,纔得勾用。」苗青道:「況我貨物未賣,那討銀子來?」因使過樂三嫂來,和王六兒說:「老爹就要貨物,發一千兩銀子貨與老爹。如不要,伏望老爹再寬限兩三日,等我倒下價錢,將貨物賣了,親往老爹宅裡進禮去。」王六兒拏禮帖復到房裡與西門慶瞧。西門慶道:「既是恁般,我分付原解且寬限他幾日,教他即便進禮來。」當下樂三子得此口詞,囘報苗青,苗青滿心歡喜。西門慶見間壁有人,也不敢久坐,吃了幾鍾酒,與老婆坐了囘,見馬來接,就起身家去了。

次日,到衙門早發放,也不題問這件事。這苗青就托經紀樂三,連夜替他會了人,攛掇貨物出去。那消三日,都發盡了,共賣了一千七百兩銀子。把原與王六兒的不動,又另加上五十兩銀子、四套上色衣服。到十九日,苗青打點一千兩銀子,裝在四個酒罈內,又宰一口豬。約掌燈以後,擡送到西門慶門首。手下人都是知道的,玳安、平安、書童、琴童四個家人,與了十兩銀子纔罷。玳安在王六兒這邊,梯已又要十兩銀子。須臾,西門慶出來,捲棚內坐的,也不掌燈,月色朦朧纔上來,{\meipi{寫得暗暗昧昧,是個暮夜受金光景。}}擡至當面。苗青穿青衣,望西門慶只顧磕頭,說道:「小人蒙老爹超拔之恩,粉身碎骨難報。」西門慶道:「你這件事情,我也還沒好審問哩。那兩個船家甚是攀你,你若出官,也有老大一個罪名。既是人說,我饒了你一死。此禮我若不受你的,你也不放心。我還把一半送你掌刑夏老爹,同做分上。你不可久住,即便星夜囘去。」因問:「你在揚州那裡?」苗青磕頭道:「小的在揚州城內住。」西門慶分付後邊拏了茶來,那苗青在松樹下立着吃了,磕頭告辭囘去。又叫囘來問:「下邊原解的,你都與他說了不曾?」苗青道:「小的外邊已說停當了。」西門慶分付:「既是說了,你即囘家。」那苗青出門,走到樂三家收拾行李,還剩一百五十兩銀子。苗青拏出五十兩來,並餘下幾疋段子,都謝了樂三夫婦。五更替他顧長行牲口,起身往揚州去了。正是:

\begin{myquote}
忙忙如䘮家之狗,急急似漏網之魚。
\end{myquote}

不說苗青逃出性命去了。單表次日,西門慶、夏提刑從衙門中散了出來,並馬而行。走到大街口上,夏提刑要作辭分路,西門慶在馬上舉着馬鞭兒說道:「長官不棄,到舍下一叙。」把夏提刑邀到家來。進到廳上叙禮,請入捲棚裡,寬了衣服,左右拏茶吃了。書童、玳安就安放桌席。夏提刑道:「不當閑來打攪長官。」西門慶道:「豈有此理。」須臾,兩個小厮用方盒擺下各樣雞、蹄、鵝、鴨、鮮魚下飯。先吃了飯,收了家伙去,就是吃酒的各樣菜蔬出來。小金鐘兒,銀臺盤兒,慢慢斟勸。飲酒中間,西門慶方題起苗青的事來,道:「這厮昨日央及了個士夫,再三來對學生說,又餽送了些禮在此。學生不敢自專,今日請長官來,與長官計議。」於是,把禮帖遞與夏提刑。夏提刑看了,便道:「恁憑長官尊意裁處。」西門慶道:「依着學生,明日只把那個賊人、眞賍送過去罷,也不消要這苗青。那個原告小厮安童,便收領在外,待有了苗天秀屍首,歸結未遲。禮還送到長官處。」夏提刑道:「長官,這就不是了。長官見得極是,此是長官費心一番,何得見讓於我?決然使不得。」彼此推辭了半日,西門慶不得已,還把禮物兩家平分了,裝了五百兩在食盒內。夏提刑下席來,作揖謝道:「既是長官見愛,我學生再辭,顯的迂闊了。盛情感激不盡,實為多愧。」又領了幾盃酒,方纔告辭起身。西門慶隨即差玳安拏食盒,還當酒擡送到夏提刑家。夏提刑親在門上收了,拏囘帖,又賞了玳安二兩銀子,兩名排軍四錢,俱不在話下。

常言道:火到豬頭爛,錢到公事辦。西門慶、夏提刑已是會定了。次日到衙門裡陞廳,那提控、節級並緝捕、觀察,都被樂三上下打點停當。擺設下刑具,監中提出陳三、翁八審問情繇,只是供稱:「跟伊家人苗青同謀。」西門慶大怒,喝令左右:「與我用起刑來!你兩個賊人,專一積年在江河中,假以舟楫裝載為名,實是劫幫鑿漏,邀截客旅,圖財致命。見有這個小厮供稱,是你等持刀戮死苗天秀波中,又將棍打傷他落水,見有他主人衣服存證,你如何抵賴別人!」因把安童提上來,問道:「是誰刺死你主人?是誰推你在水中?」安童道:「某日三更時分,先是苗青叫有賊,小的主人出艙觀看,被陳三一刀戮死,推下水去。小的便被翁八一棍打落水中,纔得逃出性命。苗青並不知下落。」西門慶道:「據這小厮所言,就是實話,汝等如何輾轉得過?」於是每人兩夾棍,三十榔頭,打的脛骨皆碎,殺豬也似喊叫。一千兩賍貨已追出大半,餘者花費無存。這裡提刑做了文書,並賍貨申詳東平府。府尹胡師文又與西門慶相交,照原行文書疊成案卷,將陳三、翁八問成強盜殺人斬罪。安童保領在外聽候。有日走到東京,投到開封府黃通判衙內,具訴:「苗青奪了主人家事,使錢提刑衙門,除了他名字出來。主人冤仇,何時得報?」通判聽了,連夜修書,並他訴狀封在一處,與他盤費,就着他往巡按山東察院裡投下。這一來,管教苗青之禍從頭上起,西門慶往時做過事,今朝沒興一齊來。有詩為證:

\begin{myquote}
善惡從來報有因,吉兇禍福並肩行。\\平生不作虧心事,夜半敲門不吃驚。
\end{myquote}

