\includepdf[pages={149,150},fitpaper=false]{tst.pdf}
\chapter*{第七十五回 因抱恙玉姐含酸 為護短金蓮潑醋}
\addcontentsline{toc}{chapter}{第七十五回 因抱恙玉姐含酸 為護短金蓮潑醋}
\markboth{{\titlename}卷之八}{第七十五回 因抱恙玉姐含酸 為護短金蓮潑醋}


詩曰:

\begin{myquote} 
雙雙蛺蝶繞花溪,半是山南半水西。\\故院有情風月亂,美人多怨雨雲迷。\\頻開檀口言如織,漫托香腮醉似泥。\\莫道佳人太命薄,一鶯啼罷一鶯啼。
\end{myquote} 

話說月娘聽宣畢《黃氏寶卷》,各房宿歇不題。單表潘金蓮在角門邊,撞見西門慶,相攜到房中。見西門慶只顧坐在床上,{\meipi{寫出靦腆。}}因問:「你怎的不脫衣裳?」那西門慶摟定婦人,笑嘻嘻說道:「我特來對你說聲,我要過那邊歇一夜兒去。你拿那淫器包兒來與我。」婦人罵道:「賊牢,你在老娘手裡使巧兒,拿這面子話兒來哄我!{\meipi{老作家,自是瞞他不得。}}我剛纔不在角門首站着,你過去的不耐煩了,又肯來問我?這是你早晨和那𢱉剌骨商定了腔兒,嗔道頭裡使他來送皮襖兒,又與我磕了頭。小賊𢱉剌骨,把我當甚麼人兒?在我手內弄剌子。我還是李瓶兒時,教你活埋我!{\pangpi{可憐,可嘆。}}雀兒不在那窩兒裡,我不醋了!」{\meipi{瓶兒之死,金蓮快心滿志,卻從此處供出。}}西門慶笑道:「那裡有此勾當,他不來與你磕箇頭兒,你又說他的不是。」婦人沉吟良久,說道:「我放你去便去,{\meipi{一片妬心,卻是十分正理,心思周紮無比。}}不許你拿了這包子去,{\pangpi{欲留餘味。}}與那𢱉剌骨弄答的齷齷齪齪的,到明日還要來和我睡,好乾淨兒。」{\pangpi{毫不放鬆。}}西門慶道:「我使慣了,你不與我卻怎樣的!」纏了半日,婦人把銀托子掠與他,說道:「你要,拿了這箇行貨子去。」西門慶道:「與我這箇也罷。」一面接的袖了,趔趄着脚兒就往外走。婦人道:「你過來,我問你,莫非你與他一鋪兒長遠睡?惹得那兩箇丫頭也羞恥。無故只是睡那一回兒,還放他另睡去。」西門慶道:「誰和他長遠睡?」說畢就走。婦人又叫回來,說道:「你過來,我分付你,慌怎的?」{\meipi{一箇心忙意急,一箇慮此防比,情事徑庭,而熱中則一。}}西門慶道:「又說甚麼?」婦人道:「我許你和他睡便睡,不許你和他說甚閑話,教他在俺們跟前欺心大膽的。我到明日打聽出來,你就休要進我這屋裡來,我就把你下截咬下來。」{\pangpi{極,趣話。}}西門慶道:「怪小淫婦兒,瑣碎死了。」一直走過那邊去了。春梅便向婦人道:「繇他去,你管他怎的?婆婆口絮,媳婦耳頑,倒沒的教人與你為冤結仇,{\meipi{春梅局外,不解箇中一種熱癢私情,熱乎有此,然而心高氣硬,言下可定生平。}}誤了咱娘兒兩箇下棋。」一面叫秋菊關上角門,放卓兒擺下棋子。兩箇下棋不題。

且說西門慶走過李瓶兒房內,掀開簾子。如意兒正與迎春、綉春炕上吃飯,見了西門慶,慌的跳起身來。西門慶道:「你們吃飯。」於是走出明間李瓶兒影跟前一張交椅上坐下。不一時,如意兒笑嘻嘻走出來,說道:「爹,這裡冷,你往屋裡坐去罷。」這西門慶就一把手摟過來,就親了箇嘴。一面走到房中床正面坐了。火爐上頓着茶,迎春連忙點茶來吃了。如意兒在炕邊烤着火兒站立,問道:「爹,你今日沒酒,還有頭裡與娘供養的一桌菜兒,一素兒金華酒,留下預備篩來與爹吃。」西門慶道:「下飯你們吃了罷,只拿幾箇菓碟兒來,我不吃金華酒。」一面教綉春:「你打箇燈籠,往藏春塢書房內,還有一罈葡萄酒,你問王經要了來,篩與我吃。」綉春應諾,打着燈籠去了。迎春連忙放桌兒,拿菜兒。如意兒道:「姐,你揭開盒子,等我揀兩樣兒與爹下酒。」於是燈下揀了幾碟精味菓菜,擺在桌上。良久,綉春取了酒來,開啟篩熱了。如意兒斟在鍾內,遞上。西門慶嚐了嚐,十分精美。如意兒就挨近桌邊站立,侍奉斟酒,又親剝炒栗子兒與他下酒。迎春知局,就往後邊廚房內與綉春坐去了。西門慶見無人在跟前,就叫老婆坐在他膝蓋兒上,摟着與他一遞一口兒飲酒。{\meipi{人情,亦當情種。}}一面解開他對襟襖兒,露出他白馥馥酥胸,用手揣摸他乳頭,誇道:「我的兒,你達達不愛你別的,只愛你到好白淨皮肉兒,與你娘一般樣兒,我摟你就如同摟着他一般。」如意兒笑道:「爹,沒的說,還是娘的身上白。我見五娘雖好模樣兒,面板也中中兒的,紅白肉色兒,不如後邊大娘、三娘到白淨。三娘只是多幾箇麻兒。倒是他雪姑娘生得清秀,又白淨。」{\meipi{小人少得寸得地,便有一番批點推敲,金蓮所慮在是,比春梅未之或知也。}}又道:「我有句話對爹說,迎春姐有件正面戴仙子兒要與我,他要問爹討娘家常戴的金赤虎,正月裡戴,爹與了他罷。」西門慶道:「你沒正面戴的,等我叫銀匠拿金子另打一件與你,你娘的頭面箱兒,你大娘都拿的後邊去了,怎好問他要的。」老婆道:「也罷,你還另打一件赤虎與我罷。」一面走下來就磕頭謝了。兩箇吃了半日酒。如意兒道:「爹,你叫姐來也與他一盃酒吃,惹他不惱麼?」西門慶便叫迎春,不應。老婆親到走到廚房內,說道:「姐,爹叫你哩。」迎春一面到跟前。西門慶令如意兒斟了一甌酒與他,又揀了兩筯菜兒放在酒托兒上。那迎春站在旁邊,一面吃了。如意道:「你叫綉春姐來也吃些兒。」迎春去了,回來說道:「他不吃了。」就向炕上抱他鋪蓋,和綉春廚房炕上睡去了。

這老婆陪西門慶吃了一回酒,收拾家伙,又點茶與西門慶吃了。原來另預備着一床兒鋪蓋與西門慶睡,都是綾絹被褥,扣花枕頭,在薰籠內薰的煖烘烘的。老婆便問:「爹,你在炕上睡,床上睡?」西門慶道:「我在床上睡罷。」如意兒便將鋪蓋抱在床上鋪下,打發西門慶解衣上床。他又在明間內打水洗了牝,掩上房門,將燈移近床邊,方纔脫衣褲上床,與西門慶相摟相抱,並枕而臥。婦人用手捏弄他那話兒,上邊束着銀托子,猙獰跳腦,又喜又怕。兩箇口吐丁香,交摟在一處。西門慶見他仰臥在被窩內,脫的精赤條條,恐怕凍着他,又取過他的抹胸兒替他蓋着胸膛上。兩手執其兩足,極力抽提。老婆氣喘吁吁,被他㒲得面如火熱。又道:「這衽腰子還是娘在時與我的。」西門慶道:「我的心肝,不打緊處,到明日鋪子裡,拿半箇紅段子,做小衣兒穿在身上伏侍我。」老婆道:「可知好哩。」西門慶道:「我只要忘了,你今年多少年紀?你姓甚麼?排行幾姐?我只記你男子漢姓熊。」老婆道:「他便姓熊,叫熊旺兒。我娘家姓章,排行第四,今三十二歲。」西門慶道:「我原來還大你一歲。」一壁幹首,一面口中呼叫他:「章四兒,你用心伏侍我,等明日後邊大娘生了孩子,你好生看奶着。你若有造化,也生長一男半女,我就扶你起來,與我做一房小,就頂你娘的窩兒,你心下何如?」老婆道:「奴男子漢已是沒了,娘家又沒人,奴情願一心伏侍爹,就死也不出爹這門。若爹可憐見,可知好哩。」

西門慶見他言語兒投着機會,心中越發喜歡,揝着他雪白兩隻腿兒,{\meipi{只「雪白腿兒」四字,便足消魂,金蓮在其下風矣。}}只顧沒稜探腦,搦幹抽提,抽提的老婆在下,無不叫出來。嬌聲怯怯,星眼朦朦。良久,卻令他馬伏在下,自舒雙足,西門慶披着紅綾被,騎在他身上,那話插入牝中。燈光下,兩手按着他雪白的屁股,只顧搧打,口中叫:「章四兒,你好生叫着親達達,休要住了,我丟與你罷。」那婦人在下舉股相就,{\pangpi{妙。}}眞箇口中顫聲柔語,呼叫不絕,足頑了一箇時辰,西門慶方纔精泄。良久,拽出麈柄來,老婆取帕兒替他搽拭。摟着睡到五更雞叫時方醒,老婆又替他吮咂。西門慶告他說:「你五娘怎的替我咂半夜,怕我害冷,連尿也不教我下來溺,都替我嚥了。」{\pangpi{叫莫要說,又說出來。}}老婆道:「這不打緊,我也替爹吃了就是了。」{\pangpi{效尤的妙。}}這西門慶眞箇把胞尿都溺在老婆口內。{\meipi{合着寵利,丈夫吮癰呧痔者多矣,況婦人女子乎?大庭廣衆之中,寡廉喪恥者多矣,況閨榻房幃乎?莫訝,莫笑。}}當下兩箇旖旎溫存,萬千羅唣,㒲搗了一夜。

次日,老婆先起來,開了門,預備火盆,打發西門慶穿衣梳洗出門。到前邊分付玳安:「教兩名排軍把捲棚放的流金八仙鼎,寫帖兒擡送到宋御史老爹察院內,交付明白,討回貼來。」又叫陳敬濟,封了一疋金段,一疋色段,教琴童用氊包拿着,預備下馬,要早往清河口,拜蔡知府去。正在月娘房內吃粥,月娘問他:「應二那裡,俺們莫不都去,也留一箇兒看家?留下他姐在家,陪大妗子做伴兒罷。」西門慶道:「我已預備下五分人情,都去走走罷。左右有大姐在家陪大妗子,就是一般。我已許下應二了。」月娘聽了,一聲兒沒言語。李桂姐便拜辭說道:「娘,我今日家去罷。」月娘道:「慌去怎的,再住一日兒不是?」桂姐道:「不瞞娘說,俺媽心裡不自在,家中沒人,改日正月間來住兩回兒罷。」拜辭了西門慶。月娘裝了兩盤茶食,又與桂姐一兩銀子,吃了茶,打發出門。

西門慶纔穿上衣服,往前邊去,忽有平安兒來報:「荊都監老爹來拜。」西門慶即出迎接,至廳上叙禮。荊都監叩拜堂上道:「久違,欠禮,高轉失賀。」西門慶道:「多承厚貺,尚未奉賀。」叙畢契闊之情,分賓主坐下,左右獻上茶湯。荊都監便道:「良騎俟候何往?」西門慶道:「京中太師老爺第九公子九江蔡知府,昨日巡按宋公祖與工部安鳳山、錢雲野、黃泰宇,都借學生這裡作東,請他一飯。蒙他具拜貼與我,我豈可不回拜他拜去?誠恐他一時起身去了。」{\meipi{逐日所難。}}荊都監道:「正是。小弟有一事特來奉瀆。巡按宋公正月間差滿,只怕年終舉劾地方官員,望乞四泉借重與他一說。聞知昨日在宅上吃酒,故此斗膽恃愛。倘得寸進,不敢有忘。」西門慶道:「此是好事,你我相厚,敢不領命?你寫箇說貼來,幸得他後日還有一席酒在我這裡,等我抵面和他說又好說些。」荊都監連忙下位來,又與西門慶打一躬道:「多承盛情,啣結難忘。」便道:「小弟已具了履歷手本在此。」一面叫寫字的取出,荊都監親手遞上,與西門慶觀看。上面寫着:「山東等處兵馬都監清河左衛指揮僉事荊忠,年三十二歲。系山後檀州人。繇祖後軍功累陞本衛正千戶。從某年由武舉中式,歷陞今職,管理濟州兵馬。」一一開載明白。西門慶看畢,荊都監又向袖中取出禮貼來,遞上說道:「薄儀望乞笑留。」{\meipi{逐日所難。}}西門慶見上面寫着「白米二百石」,說道:「豈有此理,這箇學生斷不敢領,以此視人,相交何在?」荊都監道:「不然。總然四泉不受,轉送宋公也是一般,何見拒之深耶?倘不納,小弟亦不敢奉瀆。」推讓再三,西門慶只得收了,說道:「學生暫且收下。」一面接了,說道:「學生明日與他說了,就差人回報。」茶湯兩換,荊都監拜謝起身去了。西門慶上馬,琴童跟隨,拜蔡知府去了。

卻說玉簫打發西門慶出門,就走到金蓮房中,說:「五娘,昨日怎的不往後邊去坐?俺娘好不說五娘哩。說五娘聽見爹前邊散了,往屋裡走不迭。昨日三娘生日,就不放往他屋裡去,把攔的爹恁緊。{\meipi{鬬氣之根坐此。}}三娘道:『沒的羞人子剌剌的,誰耐煩爭他。左右是這幾房裡,隨他串去。』」{\meipi{若非西門慶靦腆,未免口硬不得。}}金蓮道:「我待說,就沒好口,㒲瞎了他的眼來!昨日你道他在我屋裡睡來麼?」玉簫道:「前邊老到只娘屋裡。六娘又死了,爹卻往誰屋裡去?」金蓮道:「雞兒不撒尿——各自有去處。死了一箇,還有一箇頂窩兒的。」玉簫又說:「俺娘又惱五娘問爹討皮襖不對他說。落後爹送鑰匙到房裡,娘說了爹幾句好的,說:『早是李大姐死了,便指望他的,他不死只好看一眼兒罷了。』」{\pangpi{果然。}}金蓮道:「沒的扯那𣭈淡!有一箇漢子做主兒罷了,你是我婆婆?你管着我。我把攔他,我拿繩子拴着他腿兒不成?偏有那些𣭈聲浪氣的!」玉簫道:「我來對娘說,娘只放在心裡,休要說出我來。今日桂姐也家去了,俺娘收拾戴頭面哩,五娘也快些收拾了罷。」說畢,玉簫後邊去了。

這金蓮向鏡臺前搽胭抹粉,插花戴翠,又使春梅後邊問玉樓,今日穿甚顏色衣裳。玉樓道:「你爹嗔換孝,都教穿淺色衣服。」五箇婦人會定了,都是白鬏髻,珠子箍兒,淺色衣服。惟吳月娘戴着白縐紗金梁冠兒,上穿着沉香遍地金粧花補子襖兒,紗綠遍地金裙。一頂大轎,四頂小轎,排軍喝路,棋童、來安三箇跟隨,拜辭了吳大妗子、三位師父、潘姥姥,逕往應伯爵家吃滿月酒去了。不題。

卻說如意兒和迎春,有西門慶晚夕來吃的一桌菜,安排停當,還有一壺金華酒,{\pangpi{映前。}}向罈內又打出一壺葡萄酒來,午間請了潘姥姥、春梅,郁大姐彈唱着,在房內做一處吃。吃到中間,也是合當有事,春梅道:「只說申二姐會唱的好《掛眞兒》,沒箇人往後邊去叫他來,好歹教他唱箇咱們聽。」迎春纔待使綉春叫去,只見春鴻走來烘火。春梅道:「賊小蠻囚兒,你原來今日沒跟轎子去。」春鴻道:「爹派下教王經去了,留我看家。」春梅道:「賊小蠻囚兒,你不是凍的那腔兒,還不尋到這屋裡來烘火。」因叫迎春:「你釃半甌子酒與他吃。」分付:「你吃了,替我後邊叫將申二姐來。就說我要他唱曲兒與姥姥聽。」春鴻把酒吃了,一直走到後邊,不想申二姐伴着大妗子、大姐、三箇姑子、玉簫都在上房裡坐的,正吃茶哩。忽見春鴻掀簾子進來,叫道:「申二姐,你來,俺大姑娘前邊叫你唱箇曲兒與他聽去哩。」這申二姐道:「你大姑娘在這裡,又有箇大姑娘出來了?」春鴻道:「是俺前邊春梅姑娘叫你。」申二姐道:「你春梅姑娘他稀罕怎的,也來叫我?有郁大姐在那裡,也是一般。我這裡唱與大妗奶奶聽哩。」大妗子道:「也罷,申二姐,你去走走再來。」那申二姐坐住了,不動身。{\meipi{有此自負,宜其不平愈甚。}}春鴻一直走到前邊,對春梅說:「我叫他,他不來哩。」春梅道:「你說我叫他,他就來了。」春鴻道:「我說前邊大姑娘叫你,他意思不動,說這是大姑娘,那裡又鑽出箇大姑娘來了?我說是春梅姑娘,他說你春梅姑娘便怎的,有郁大姐罷了,他從幾時來也來叫我,我不得閑,在這裡唱與大妗奶奶聽哩。大妗奶奶到說你去走走再來,他不肯來哩。」這春梅不聽便罷,聽了三屍神暴跳,五臟氣冲天,一點紅從耳畔起,須臾紫遍了雙腮。衆人攔阻不住,一陣風走到上房裡,指着申二姐一頓大罵道:「你怎麼對着小厮說我『那裡又鑽出箇大姑娘來了』,『稀罕他,也來叫我』?你是甚麼總兵官娘子,不敢叫你!俺們在那毛裡夾着,是你擡舉起來,如今從新鑽出來了?你無非是箇走千家門、萬家戶,賊狗攮的瞎淫婦!你來俺家纔走了多少時兒,就敢恁量視人家?你會曉的甚麼好成樣的套數兒,左右是那幾句東溝籬,西溝壩,油嘴狗舌,不上紙筆的那胡歌野詞,就拿班做勢起來!俺家本司三院唱的老婆,不知見過多少,稀罕你。韓道國那淫婦家興你,俺這裡不興你。{\meipi{扯得奇。}}你就學與那淫婦,我也不怕。你好不好趁早兒去,賈媽媽與我離門離戶。」那大妗子攔阻說道:「快休要破口。」把申二姐罵的睜睜的,敢怒而不敢言,說道:「耶嚛嚛,這位大姐,怎的恁般粗魯性兒,就是剛纔對着大官兒,我也沒曾說甚歹話,怎就這般言語,潑口罵出來!此處不留人,更有留人處。」春梅越發惱了,罵道:「賊㒲遍街搗遍巷的瞎淫婦,你家有恁好大姐!比是你有恁性氣,不該出來往人家求衣食,唱與人家聽。趁早兒與我走,再也不要來了。」申二娘道:「我沒的賴在你家!」春梅道:「賴在我家,叫小厮把髩毛都撏光了你的。」{\meipi{春梅大膽,口氣自是不凡。}}大妗子道:「你這孩兒,今日怎的恁樣兒的,還不往前邊去罷。」那春梅只顧不動身。這申二姐一面哭哭啼啼下炕來,拜辭了大妗子,收拾衣裳包子,也等不的轎子來,央及大妗子使平安對過叫將畫童兒來,領他往韓道國家去了。春梅罵了一頓,往前邊去了。大妗子看着大姐和玉簫說道:「他敢前邊吃了酒進來,不然如何恁冲言冲語的!罵的我也不好看的了。你叫他慢慢收拾了去就是了,立逼着攆他去了,又不叫小厮領他,十分水深人不過。」玉簫道:「他們敢在前頭吃酒來?」

卻說春梅走到前邊,還氣狠狠的向衆人說道:「方纔把賊瞎淫婦兩箇耳刮子纔好。他還不知道我是誰哩!叫着他張兒致兒,拿班做勢兒的。」迎春道:「你砍一枝損百枝,{\pangpi{妙。}}忌口些,郁大姐在這裡。」春梅道:「不是這等說。像郁大姐在俺家這幾年,大大小小,他惡訕了那箇來?教他唱箇兒,他就唱。那里像這賊瞎淫婦大膽。他記得甚麼成樣的套數,左來右去,只是那幾句《山坡羊》、《瑣南枝》,油裏滑言語,上箇甚麼擡盤兒也怎的?我纔乍聽這箇曲兒也怎的?我見他心裡就要把郁大姐掙下來一般。」郁大姐道:「可不怎的。昨日晚夕,大娘教我唱小曲兒,他就連忙把琵琶奪過去,他要唱。大姑娘你也休怪,他怎知道咱家裡深淺?他還不知把你當誰人看成。」春梅道:「我剛纔不罵的:你上覆韓道國老婆那賊淫婦,你就學與他,我也不怕他。」潘姥姥道:「我的姐姐,你沒要緊氣的恁樣兒的。」{\meipi{老人家口角,然亦見其實貧婆常語。}}如意兒道:「我傾盃兒酒,與大姐姐消消兒惱。」迎春道:「我這女兒着惱就是氣。」便道:「郁大姐,你揀套好曲兒唱箇伏侍他。」這郁大姐拿過琵琶來,說道:「等我唱箇「鶯鶯鬧臥房」《山坡羊》兒。與姥姥和大姑娘聽罷。」如意兒道:「你用心唱,等我斟上酒。」那迎春拿起盃兒酒來,望着春梅道:「罷罷,我的姐姐,你也不要惱了,胡亂且吃你媽媽這鐘酒兒罷。」那春梅忍不住笑罵道:「怪小淫婦兒,你又做起我媽媽來了!」{\meipi{春梅熱鬧,迎春冷韻,自此雙鬟可稱不俗。}}又說道:「郁大姐,休唱《山坡羊》,你唱箇《江兒水》俺們聽罷。」這郁大姐在旁彈着琵琶,慢慢唱「花嬌月艷」,與衆人吃酒不題。

且說西門慶從新河口拜了蔡九知府,回來下馬,平安就稟:「今日有衙門裡何老爹差答應的來,請爹明日早進衙門中,拿了一起賊情審問。又本府胡老爹送了一百本新曆日。荊都監老爹差人送了一口鮮豬,一罈荳酒,又是四封銀子。姐夫收下,交到後邊去了,沒敢與他回貼兒。晚上,他家人還來見爹說話哩。只胡老爹家與了回貼,賞了來人一錢銀子。又是喬親家爹送貼兒,明日請爹吃酒。」玳安兒又拿宋御史回貼兒來回話:「小的送到察院內,宋老爹說,明日還奉價過來。賞了小的並擡盒人五錢銀子,一百本歷日。」西門慶走到廳上,春鴻連忙報與春梅衆人,說道:「爹來家了,還吃酒哩。」春梅道:「怪小蠻囚兒,爹來家隨他來去,管俺們腿事!沒娘在家,他也不往俺這邊來。」

衆人打夥兒吃酒頑笑,只顧不動身。西門慶到上房,大妗子和三箇姑子,都往那邊屋裡去了。玉簫向前與他接了衣裳,坐下,放桌兒打發他吃飯。教來興兒定桌席:三十日與宋巡按擺酒;初一日劉、薛二內相,帥府周爺衆位,吃慶官酒。分付去了。玉簫在旁請問:「爹吃酒,篩甚麼酒吃?」西門慶道:「有剛纔荊都監送來的那荳酒取來,開啟我嚐嚐,看好不好。」只見來安兒進來,稟問接月娘去。玉簫便使他提酒來,打破泥頭,傾在鍾內,遞與西門慶呷了一呷,碧靛般清,其味深長。西門慶令:「斟來我吃。」須臾,擺上菜來,西門慶在房中吃酒。

卻說來安同排軍拿燈籠,晚夕接了月娘衆人來家。都穿着皮襖,都到上房來拜西門慶。惟雪娥與西門慶磕頭,起來又與月娘磕頭。拜完了,又都過那邊屋裡,去拜大妗子與三箇姑子。月娘便坐着與西門慶說話:「應二嫂見俺們都去,好不喜歡!酒席上有隔壁馬家娘子和應大嫂、杜二娘,也有十來位娘子。叫了兩箇女兒彈唱。養了好箇平頭大臉的小厮兒。原來他房裡春花兒,比舊時黑瘦了好些,只剩下箇大驢臉一般的,也不自在哩。今日亂的他家裡大小不安,本等沒人手。臨來時,應二歌與俺們磕頭,謝了又謝,多多上覆你,多謝重禮。」西門慶道:「春花兒那成精奴才,也打扮出來見人?」{\meipi{罵得好笑,具見輕薄。}}月娘道:「他比那箇沒鼻子?沒眼兒?是鬼兒?出來見不的?」西門慶道:「那奴才,撒把黑荳只好教豬拱罷。」月娘道:「我就聽不上你恁說嘴。只你家的好,拿掇的,出來見的人!」那王經在旁立着,說道:「應二爹見娘們去,先頭不敢出來見,躲在下邊房裡,打窻戶眼兒望前瞧。被小的看見了,說道:『你老人家沒廉恥,平日瞧甚麼!」他趕着小的打。」西門慶笑的沒眼縫兒,{\meipi{摹寫笑處,便見其胸中有一種賣弄諸姬人物之意,春花之貶,應為此地。}}說道:「你看這賊花子,等明日他來,着老寔抹他一臉粉。」王經笑道:「小的知道了。」月娘喝道:「這小厮別要胡說。他幾時瞧來?平白枉口拔舌的。一日誰見他箇影兒?只臨來時,纔與俺們磕頭。」王經站了一回出來了。

月娘也起身過這邊屋裡,拜大妗子並三箇師父。大姐與玉簫衆丫頭媳婦都來磕頭。月娘便問:「怎的不見申二姐?」衆人都不作聲。玉簫說:「申二姐家去了。」月娘道:「他怎的不等我來就去?」大妗子隱瞞不住,把春梅罵他之事,說了一遍。月娘就有幾分惱,{\meipi{該惱。}}說道:「他不唱便罷了,這丫頭恁慣的沒張倒置的,平白罵他怎麼的?怪不的俺家主子也沒那正主了,奴才也沒箇規矩,成甚麼道理!」{\meipi{語出愛憎人口,便是非顛倒。}}望着金蓮道:「你也管他管兒,慣的他通沒些摺兒。」金蓮在旁笑着說道:「也沒見這箇瞎曳麼的,風不搖,樹不動。你走千家門,萬家戶,在人家無非只是唱。人叫你唱箇兒,也不失了和氣,誰教他拿班兒做勢的,他不罵他嫌腥。」月娘道:「你到且是會說話兒的。都像這等,好人歹人都吃他罵了去?也休要管他一管兒了!」金蓮道:「莫不為瞎淫婦打他幾棍兒?」{\meipi{金蓮出語狠辣,似少平日機變。然非王士有以中之,當不至是,所以成心不可使有。}}月娘聽了他這句話,氣的他臉通紅了,說道:「慣着他,明日把六隣親戚都教他罵遍了罷!」於是起身,走過西門慶這邊來。西門慶便問:「怎麼的?」月娘道:「情知是誰,你家使的有好規矩的大姐,如此這般,把申二姐罵的去了。」西門慶笑道:「誰教他不唱與他聽來。{\pangpi{好混。}}也不打緊處,到明日使小厮送他一兩銀子,補伏他,也是一般。」玉簫道:「申二姐盒子還在這裡,沒拿去哩。」月娘見西門慶笑,便說道:「不說教將來嗔喝他兩句,{\pangpi{大是。}}虧你還雌着嘴兒,不知笑的是甚麼?」玉樓、李嬌兒見月娘惱起來,就都先歸房去了。西門慶只顧吃酒。良久,月娘進裡間內,脫衣裳摘頭,便問玉簫:「這箱上四包銀子是那裡的?」西門慶說:「是荊都監的二百兩銀子,要央宋巡按,圖幹陞轉。」玉簫道:「頭裡姐夫送進來,我就忘了對娘說。」月娘道:「人家的,還不收進櫃裡去哩。」玉簫一面安放在廚櫃中。金蓮在那邊屋裡只顧坐的,要等西門慶一答兒往前邊去,今日晚夕要吃薛姑子符藥,與他交媾,圖壬子日好生子。見西門慶不動身,走來掀簾子兒叫他說:「你不往前邊去,我等不得你,我先去也。」{\meipi{有此至情,不宜硬氣。急態也急心。}}西門慶道:「我兒,你先走一步兒,我吃了這些酒來。」那金蓮一直往前去了。月娘道:「我偏不要你去,{\pangpi{妙。}}我還和你說話哩。你兩箇合穿着一條褲子也怎的?強汗世界,巴巴走來我屋裡,硬來叫你。沒廉恥的貨,只你是他的老婆,別人不是他的老婆?你這賊皮搭行貨子,怪不的人說你。一視同仁,都是你的老婆,休要顯出來便好。{\meipi{雖月娘一時憤激之言,然一段宜家道理。金蓮則小不忍而亂大謀,可惜,可戒。}}就吃他在前邊把攔住了,從東京來,通影邊兒不進後邊歇一夜兒,教人怎麼不惱?你冷竈着一把兒,熱竈着一把兒纔好,通教他把攔住了,我便罷了,不和你一般見識,別人他肯讓的過?{\pangpi{要見此道原妙。}}口兒內雖故不言語,好殺他心兒裡也有幾分惱。{\pangpi{夫子自道。}}今日孟三姐在應二嫂那裡,通一日沒吃甚麼兒,不知掉了口冷氣,只害心悽噁心。來家,應二嫂遞了兩鍾酒,都吐了。你還不往屋裡瞧他瞧去?」{\meipi{金蓮十分熱急,玉樓一味酸柔,熱使人愛,酸使人憐。}}西門慶聽了,說道:「眞箇?分付收了家伙罷,我不吃酒了。」於是走到玉樓房中。只見婦人已脫了衣裳,摘去首飾,渾衣兒𢱉在炕上,正倒着身子嘔吐。西門慶見他呻吟不止,慌問道:「我的兒,你心裡怎麼的來?對我說,明日請人來看你。」{\meipi{試看西門慶光景,多少乏趣,故處家無僻。}}婦人一聲不言語,只顧嘔吐。被西門慶一面抱起他來,與他坐的,見他兩隻手只揉胸前,便問:「我的心肝,心裡怎麼?告訴我。」婦人道:「我害心悽的慌,你問他怎的?你幹你那營生去。」西門慶道:「我不知道,剛纔上房對我說,我纔曉的。」婦人道:「可知你不曉的。{\pangpi{雋。}}俺每不是你老婆,你疼你那心愛的去罷。」西門慶於是摟過粉項來親箇嘴,說道:「怪油嘴,就奚落我起來。」便叫蘭香:「快頓好苦艷茶兒來,與你娘吃。」蘭香道:「有茶伺候着哩。」一面捧茶上來。

西門慶親手拿在他口兒邊吃。婦人道:「拿來,等我自吃。會那等喬劬勞,旋蒸熱賣兒的,{\pangpi{雋。}}誰這裡爭你哩!{\pangpi{雋。}}{\meipi{口說不爭,卻語冷情悽,猶深於爭。讀之一回心酸,一回心癢。}}今日日頭打西出來,稀罕往俺這屋裡來走一走兒。也有這大娘,平白說怎的,爭出來が包氣。」西門慶道:「你不知,我這兩日七事八事,心不得箇閑。」{\pangpi{扯淡得妙。}}婦人道:「可知你心不得閑,自有那心愛的扯落着你哩。把俺們這僻時的貨兒,都打到贅字型大小聽題去了,後十年掛在你那心裡。」{\meipi{玉樓、金蓮素稱莫逆,一到此際,含酸帶刺,有無限低徊,可見利害一切於己,交情知愛,又落第二義矣。}}見西門慶嘴搵着他那香腮,便道:「吃的那酒氣,還不與我過一邊去。人一日黃湯辣水兒誰嘗着來,那裡有甚麼神思和你兩箇纏!」西門慶道:「你沒吃甚麼兒?叫丫頭拿飯來咱們吃,我也還沒吃飯哩。」婦人道:「你沒的說,人這裡悽疼的了不得,且吃飯!你要吃,你自家吃去!」西門慶道:「你不吃,我敢也不吃了,咱兩箇收拾睡了罷。明日早,使小厮請任醫官來看你。」婦人道:「繇他去,請甚麼任醫官、李醫官,教劉婆子來,吃他服藥也好了。」西門慶道:「你睡下,等我替你心口內撲撒撲撒,管情就好了。你不知道,我專一會揣骨捏病。」{\pangpi{扯淡得趣。}}西門慶忽然想起道:「昨日劉學官送了十圓廣東牛黃蠟丸,那藥,酒兒吃下極好。」即使蘭香:「問你大娘要去,在上房磁礶兒內盛着哩。就拿素兒帶些酒來。吃了管情手到病除。」婦人道:「我不好罵出來,你會揣甚麼病?要酒,俺這屋裡有酒。」

不一時,蘭香到上房要了兩丸來。西門慶看篩熱了酒,剝去臘,裡面露出金丸來,拿與玉樓吃下去。西門慶因令蘭香:「趁着酒,你篩一鍾兒來,我也吃了藥罷。」被玉樓瞅了一眼,{\pangpi{嬌絕。}}說道:「就休要汗邪,你要吃藥,往別人房裡去吃。{\pangpi{此人是誰?}}你這裡且做甚麼哩,卻這等胡作做。你見我不死,來攛掇上路兒來了。緊要教人疼的魂也沒了,還要那等掇弄人,虧你也下般的,誰耐煩和你兩箇只顧涎纏。」西門慶笑道:「罷罷,我的兒,我不吃藥了,咱兩箇睡罷。」那婦人一面吃畢藥,與西門慶兩箇解衣上床同寢。西門慶在被窩內,替他手撒撲着酥胸,揣摸香乳,一手摟其粉項,問道:「我的親親,你心口這回吃下藥覺好些?」{\pangpi{此兒亦善修飾。}}婦人道:「疼便止了,還有些嘈雜。」西門慶道:「不打緊,消一回也好了。」因說道:「你不在家,我今日兌了五十兩銀子與來興兒,後日宋御史擺酒,初一日燒紙還願心,到初三日,再破兩日工夫,把人都請了罷。受了人家許多人情禮物,只顧挨着,也不是事。」婦人道:「你請也不在我,不請也不在我。明日三十日,我教小厮來攢帳,交與你,隨你交付與六姐,教他管去。也該教他管管兒,卻是他昨日說的:『甚麼打緊處,雕佛眼兒便難,等我管。』」西門慶道:「你聽那小淫婦兒,他勉強,着緊處他就慌了。亦發擺過這几席酒兒,你交與他就是了。」{\meipi{金蓮別有所長,無事勉強,西門慶固善於因才任使。}}玉樓道:「我的哥哥,誰養的你恁乖!還說你不護他,這些事兒就見出你那心兒來了。擺過酒兒交與他,俺們是合死的?像這清早晨,得梳箇頭兒?小厮你來我去,稱銀換錢,氣也掏乾了。饒費了心,那箇道箇是也怎的!」{\meipi{道破持家之難,不差一黍。}}西門慶道:「我的兒,常言道:『當家三年狗也嫌。』」說着,一面慢慢搊起一隻腿兒,{\pangpi{妙有措置。}}跨在胳膊上,摟抱在懷裡,揝着他白生生的小腿兒,穿着大紅綾子的綉鞋兒,說道:「我的兒,你達不愛你別,只愛你這兩隻白腿兒,就是普天下婦人選遍了,也沒你這等柔嫩可愛。」{\pangpi{諛處帶幾分自愧意,深想乃見。}}婦人道:「好箇說嘴的貨,誰信那棉花嘴兒,可哥兒的就是普天下婦人選遍了沒有來!不說俺們皮肉兒粗糙,你拿左話兒右說着哩。」西門慶道:「我的心肝,我有句謊就死了我。」婦人道:「行貨子,沒要緊賭什麼誓。」{\pangpi{好人好心。}}這西門慶說着就把那話帶上了銀托子,插放入他牝中。婦人道:「我說你行行就下道兒來了。」因摸見銀托子,說道:「從多咱三不知就帶上這行貨子了,還不趁早除下來哩。」

那西門慶那裡肯依,抱定他一隻腿在懷裡,只顧僅沒其稜,淺抽深送。須臾淫水浸出,往來有聲,如狗茶鏹子一般,婦人一面用絹抹盡了去,口裡內不住作柔顫聲,叫他:「達達,你省可往裡邊去,奴這兩日好不腰痠,下邊流白漿子出來。」西門慶道:「我到明日問任醫官討服煖藥來,你吃就好了。」

不說兩箇在床上歡娛頑耍,單表吳月娘在上房陪着大妗子、三位師父,晚夕坐的說話。因說起春梅怎的罵申二姐,罵的哭涕,又不容他坐轎子去,旋央及大妗子,對過叫畫童兒送他往韓道國家去。大妗子道:「本等春梅出來的言語粗魯,饒我那等說着,還刀截的言語罵出來,他怎的不急了!他平昔不曉的恁口潑罵人,我只說他吃了酒。」小玉道:「他們五箇在前頭吃酒來。」月娘道:「恁不合理的行貨子,生生把丫頭慣的恁沒大沒小的,還嗔人說哩。到明日不管好歹,人都吃他罵了去罷,要俺們在屋裡做甚麼?{\meipi{大是。}}一箇女兒,他走千家門,萬家戶,教他傳出去好聽?敢說西門慶家那大老婆,也不知怎麼出來的。亂世不知那箇是主子,那箇是奴才。不說你們這等慣的沒些規矩,恰似俺們不長俊一般,成箇甚麼道理!」大妗子道:「隨他去罷,他姑夫不言語,怎好惹氣?」當夜無辭,同歸到房中歇了。

次日,西門慶早起往衙門中去了。潘金蓮見月娘攔了西門慶不放來,又誤了壬子日期,{\meipi{天下事原有此等不湊巧、不知趣者為可恨耳。}}心中甚是不悅。{\pangpi{自然。}}次日,老早就使來安叫了一頂轎子,把潘姥姥打發往家去了。{\pangpi{不知日去了。}}吳月娘早晨起來,三箇姑子要告辭家去,月娘每箇一盒茶食,五錢銀子,又許下薛姑子正月裡庵裡打齋,先與他一兩銀子,請香燭紙馬,到臘月還送香油、白麵、細米素食與他齋僧供佛。因擺下茶,在上房內管待,同大妗子一處吃。先請了李嬌兒、孟玉樓、大姐,都坐下。問玉樓:「你吃了那蠟丸,心口內不疼了?」玉樓道:「今早吐了兩口酸水,纔好了。」叫小玉往前邊:「請潘姥姥和五娘來吃點心。」玉簫道:「小玉在後邊蒸點心哩。我去請罷。」{\pangpi{禍根。}}於是一直走了前邊金蓮房中,便問他:「姥姥怎的不見?後邊請姥姥和五娘吃茶哩。」金蓮道:「他今日早晨,我打發他家去了。」玉簫說:「怎的不說聲,三不知就去了?」金蓮道:「住的人心淡,只顧住着怎的!」玉簫道:「我拿了塊臘肉兒,四箇甜醬瓜茄子,與他老人家,誰知他就去了。五娘你替老人家收着罷。」於是遞與秋菊,放在抽替內。這玉簫便向金蓮說道:「昨日晚夕五娘來了,俺娘如此這般對着爹好不說五娘強汗世界,與爹兩箇合穿着一條褲子,沒廉恥,怎的把攔老爹在前邊,不往後邊來。落後把爹打發三娘房裡歇了一夜,又對着大妗子、三位師父,怎的說五娘慣的春梅沒規矩,毀罵申二姐。爹到明日還要送一兩銀子與申二姐遮羞。」一五一十說了一時。這金蓮聽記在心。玉簫先來回月娘說:「姥姥起早往家去了,五娘便來也。」月娘便望着大妗子道:「你看,昨日說了他兩句兒,今日就使性子,也不進來說聲兒,老早打發他娘去了。我猜姐姐又不知心裡安排着要起甚麼水頭兒哩。」{\meipi{兩下蓄心已久,一觸便來,眞如雷轟電掣,不假思議,不用安排,可觀可聽,妙,妙!}}當下月娘自知屋裡說話,不防金蓮暗走到明間簾下,聽覷多時了,猛可開言說道:「可是大娘說的,我打發了他家去,我好把攔漢子?」月娘道:「是我說來,你如今怎麼我?本等一箇漢子,從東京來了,成日只把攔在你那前頭,通不來後邊傍箇影兒。原來只你是他的老婆,別人不是他的老婆?行動題起來,別人不知道,我知道。就是昨日李桂姐家去了,大妗子問了聲:『李桂姐住了一日兒,如何就家去了?他姑夫因為甚麼惱他?』我還說:『誰知為甚麼惱他?』你便就撐着頭兒說:『別人不知道,只我曉的。』{\pangpi{扯着處,頭頭是道,可見蓄心之久。}}你成日守着他,怎麼不曉的!」金蓮道:「他不往我那屋裡去,我莫不拿豬毛繩子套了他去不成!那箇浪的慌了也怎的?」{\meipi{所以構爭者,非止此一節,然所以構爭,何莫非此一節?鬬氣在此,婆心在此。}}月娘道:「你不浪的慌,他昨日在我屋裡好好兒坐的,你怎的掀着簾子硬入來叫他前邊去,是怎麼說?漢子頂天立地,吃辛受苦,犯了甚麼罪來,你拿豬毛繩子套他?{\pangpi{撮着一句,便可入罪,女流慣用此法。}}賤不識高低的貨,俺每倒不言語了,你倒只顧趕人。一箇皮襖兒,你悄悄就問漢子討了,穿在身上,掛口兒也不來後邊題一聲兒。{\meipi{月娘大是。}}都是這等起來,俺每在這屋裡放小鴨兒?就是孤老院裡也有箇甲頭。一箇使的丫頭,和他貓鼠同眠,慣的有些摺兒!不管好歹就罵人。說着你,嘴頭子不伏箇燒埋。」金蓮道:「是我的丫頭也怎的?你每打不是!我也在這裡,還多着箇影兒哩。皮襖是我問他要來。莫不只為我要皮襖,開門來也拿了幾件衣裳與人,那箇你怎的就不說了?{\meipi{只為如意一宿,冤及金蓮,故氣苦不平乃爾。}}丫頭便是我慣了他,是我浪了圖漢子喜歡。像這等的卻是誰浪?」吳月娘吃他這兩句,觸在心上,便紫漒了雙腮,說道:「這箇是我浪了,隨你怎的說。我當初是女兒塡房嫁他,不是趁來的老婆。那沒廉恥趁漢精便浪,俺每眞材寔料,不浪。」{\pangpi{月娘亦屬牽強。相罵到此,不得不搬出矣。故凡人脚跟要硬。}}吳大妗子便在跟前攔說:「三姑娘,你怎的,快休舒口。」饒勸着,那月娘口裡的話紛紛發出來,說道:「你害殺了一箇,只多我了。」{\meipi{善哉,善哉!大為瓶兒吐氣,即我胸中鬱結,亦為一開。}}孟玉樓道:「耶嚛,耶嚛,大娘,你今日怎的這等惱的大發了,連累俺每,一棒打着好幾箇。也沒見這六姐,你讓大娘一句兒也罷了,只顧拌起嘴來了。」大妗子道:「常言道,要打沒好手,厮罵沒好口。不爭你姊妹每嚷鬬,俺每親戚在這裡住着也羞。姑娘,你不依我,想是嗔我在這裡,叫轎子來我家去罷!」被李嬌兒一面拉住大妗子,那潘金蓮見月娘罵他這等言語,坐在地下就打滾撒潑。{\pangpi{有趣,有趣,好看,妙,妙!}}自家打幾箇嘴巴,{\pangpi{妙,妙!}}頭上鬏髻都撞落一邊,放聲大哭,叫起來說道:「我死了罷,要這命做什麼,你家漢子說條念款說將來,我趁將你家來了!這也不難的勾當,等他來家,與了我休書,我去就是了。你趕人不得趕上。」月娘道:「你看就是了,潑脚子貨。別人一句兒還沒說出來,你看他嘴頭子,就相淮洪一般。他還打滾兒賴人,莫不等的漢子來家,把我別變了!你放恁箇刁兒,那箇怕你麼?」金蓮道:「你是眞材寔料的,誰敢辯別你?」月娘越發大怒,說道:「我不眞材寔料,我敢在這家裡養下漢來?」{\pangpi{有得他說。}}金蓮道:「你不養下漢,誰養下漢來?你就拿主兒來與我!」{\pangpi{難道陳敬濟現在。}}玉樓見兩箇拌的越發不好起來,一面拉金蓮往前邊去,說道:「你恁怪剌剌的,大家都省口些罷了。只顧亂起來,左右是兩句話,教三位師父笑話。你起來,我送你前邊去罷。」那金蓮只顧不肯起來,被玉樓和玉簫一齊扯起來,送他前邊去了。大妗子便勸住月娘,說道:「姑娘,你身上又不方便,好惹氣,分明沒要緊。你姐妹們歡歡喜喜,俺每在這裡住着有光。似這等合氣起來,又不依箇勸,卻怎樣兒的?」那三箇姑子見嚷鬧起來,打發小姑兒吃了點心,包了盒子,告辭月娘衆人,月娘道:「三位師父,休要笑話。」薛姑子道:「我的佛菩薩,沒的說,誰家竈內無烟?心頭一點無明火,些兒觸着便生烟。大家儘讓些就罷了。{\pangpi{賊禿,你與王姑爭攬經錢,為何不盡讓?佛法何在?}}佛法上不說的好:『冷心不動一孤舟,淨掃靈臺正好修。』若還繩頭鬆鬆,就是萬箇金剛也降不住。為人只把這心猿意馬牢拴住了,成佛作祖,都打這上頭起。{\meipi{有此密口,不信敬者幾人?}}貧僧去也,多有打攪菩薩。好好兒的。」一面打了兩箇問訊。月娘連忙還萬福,說道:「空過師父,多多有慢。另日着人送齋襯去。」即叫大姐:「你和二娘送送三位師父出去,看狗。」於是打發三箇姑子出門去了。月娘陪大妗子坐着,說道:「你看這回氣的我,兩隻胳膊都軟了,手冰冷的。從早晨吃了口清茶,還汪在心裡。」大妗子道:「姑娘,我這等勸你少攬氣,你不依我。你又是臨月的身子,有甚要緊。」月娘道:「嫂子,早是你在這裡住看着,又是我和他合氣?如今犯夜的倒拿住巡更的。我倒容了人,人倒不肯容我。{\meipi{天下人有終身不白而徐俟論定如瓶兒者,猶不足數,故處世接物,要具兩隻明眼,不可當面錯過。}}一箇漢子,你就通身把攔住了,和那丫頭通同作弊,在前頭幹的那無所不為的事,人幹不出來的,你幹出來。女婦人家,通把箇廉恥也不顧。他燈臺不照自己,還張着嘴兒說人浪。想着有那一箇在,成日和那一箇合氣,對着俺每,千也說那一箇的不是,他就是清淨姑姑兒了。單管兩頭和番,曲心矯肚,人面獸心。行說的話兒,就不承認了。賭的那誓諕人子。我洗着眼兒看着他,到明日還不知怎麼樣兒死哩。剛纔擺着茶兒,我還好意等他娘來吃,誰知他三不知的就打發去了。就安排要嚷的心兒,悄悄兒走來這裡聽。聽怎的?那箇怕你不成!待等漢子來,輕學重告,把我休了就是了。」小玉道:「俺每都在屋裡守着爐臺站着,不知五娘幾時走來,也不聽見他脚步兒響。」孫雪娥道:「他單會行鬼路兒,脚上只穿氊底鞋,你可知聽不見。想着起頭兒一來時,該和我合了多少氣!背地打夥兒嚼說我,教爹打我那兩頓,娘還說我和他偏生好鬬的。」{\meipi{一提起便着自己,並及來旺,仇口固無譽言,然而虛心處良知終是不昧。}}月娘道:「他活埋慣了人,今日還要活埋我哩。你剛纔不見他那等撞頭打滾兒,一徑使你爹來家知道,管就把我翻倒底下。」李嬌兒笑道:「大娘沒的說,反了世界!」月娘道:「你不知道,他是那九條尾的狐狸精,把好的吃他弄死了,且稀罕我能多少骨頭肉兒!你在俺家這幾年,雖是箇院中人,不像他久慣牢頭。你看他昨日那等氣勢,硬來我屋裡叫漢子:『你不往前邊去,我等不的你,先去。』恰似只他一箇人的漢子一般,就佔住了。不是我心中不惱,他從東京來家,就不放一夜兒進後邊來。一箇人的生日,也不往他屋裡走走兒去。十箇指頭,都放在你口內纔罷了。」{\pangpi{十箇指頭,不抵一箇此物。}}大妗子道:「姑娘,你耐煩,你又常病兒痛兒的,不貪此事,隨他去罷。不爭你為衆好,與人為怨結仇。」勸了一回,玉簫安排上飯來,也不吃,說道:「我這回好頭疼,心口內有些惡沒沒的上來。」教玉簫:「那邊炕上,放下枕頭,我且躺躺去。」分付李嬌兒:「你們陪大妗子吃飯。」那日,郁大姐也要家去,月娘分付:「裝一盒子點心,與他五錢銀子。」打發去了。

卻說西門慶衙門中審問賊情,到午牌時分纔來家。正値荊都監家人討回帖,西門慶道:「多謝你老爹重禮。如何這等計較?你還把那禮扛將回去,等我明日說成了取家來。」家人道:「家老爹沒分付,小的怎敢將回去,放在老爹這裡也是一般。」西門慶道:「既恁說,你多上覆,我知道了。」拿回貼,又賞家人一兩銀子。因進上房,見月娘睡在炕上,叫了半日,白不答應。問丫鬟,都不敢說。走到前邊金蓮房裡,見婦人蓬頭撒腦,拿着箇枕頭睡,問着又不言語,更不知怎的。一面封銀子,打發荊都監家人去了,走到孟玉樓房中問。玉樓隱瞞不住,只得把月娘和金蓮早晨嚷鬧合氣之事,備說一遍。這西門慶慌了,走到上房,一把手把月娘拉起來,說道:「你甚要緊,自身上不方便,理那小淫婦兒做甚麼?平白和他合甚麼氣?」月娘道:「我和他合氣,是我偏生好鬬尋趁他來?他來尋趁將我來!你問衆人不是?早晨好意擺下茶兒,請他娘來吃。他使性子把他娘打發去了,便走來後邊撐着頭兒和我嚷,自家打滾撞頭,鬟髻都踩扁了,皇帝上位的叫,只是沒打在我臉上罷了。若不是衆人拉勸着,是也打成一塊。他平白欺負慣了人,他心裡也要把我降伏下來。行動就說:『你家漢子說條念款將我來了,打發了我罷,我不在你家了。』一句話兒出來,他就是十句說不下來,嘴一似淮洪一般,我拿甚麼骨禿肉兒拌的他過?專會那潑皮賴肉的,氣的我身子軟癱兒熱化,甚麼孩子李子,就是太子也成不的。如今倒弄的不死不活,心口內只是發脹,肚子往下鱉墜着疼,頭又疼,兩隻胳膊都麻了。剛纔桶子上坐了這一回,又不下來。若下來也乾淨了,省的死了做帶累肚子鬼。到半夜尋一條繩子,等我弔死了,隨你和他過去。往後沒的又像李瓶兒,吃他害死了。{\meipi{人之將死,其言也善。只為瓶兒臨終一言,刻入心肺。}}我曉的你三年不死老婆,也是大悔氣。」西門慶不聽便罷,聽的說,越發慌了,一面把月娘摟抱在懷裡,說道:「我的好姐姐,你別和那小淫婦兒一般見識,他識什麼高低香臭?沒的氣了你,倒値了多的。我往前邊罵這賊小淫婦兒去。」月娘道:「你還敢罵他,他還要拿豬毛繩子套你哩。」西門慶道:「你教他說,惱了我,吃我一頓好脚。」因問月娘:「你如今心內怎麼的?吃了些甚麼兒沒有?」月娘道:「誰嘗着些甚麼兒?大清早晨纔拿起茶,等着他娘來吃,他就走來和我嚷起來。如今心內只發脹,肚子往下鱉墜着疼,腦袋又疼,兩隻胳膊都麻了。你不信,摸我這手,恁半日還沒握過來。」西門慶聽了,只顧跌脚,說道:「可怎樣兒的,快着小厮去請任醫官來看看。」月娘道:「請什麼任醫官?隨他去,有命活,沒命教他死,纔趁了人的心。什麼好的老婆?是墻上土坯,去了一層又一層。我就死了,把他扶了正就是了。恁箇聰明的人兒,當不的家?」西門慶道:「你也耐煩,把那小淫婦兒只當臭屎一般丟着他去便罷了。你如今不請任後溪來看你看,一時氣裹住了這胎氣,弄的上不上,下不下,怎麼了?」月娘道:「這等,叫劉婆子來瞧瞧,吃他服藥,再不,頭上剁兩針,繇他自好了。」西門慶道:「你沒的說,那劉婆子老淫婦,他會看甚胎產?叫小厮騎馬快請任醫官來看。」月娘道:「你敢去請!你就請了來,我也不出去。」西門慶不依他,走到前邊,即叫琴童:「快騎馬往門外請任老爹,緊等着,一答兒就來。」琴童應諾,騎上馬雲飛一般去了。西門慶只在屋裡厮守着月娘,分付丫頭,連忙熬粥兒拿上來,勸他吃,月娘又不吃。等到後晌時分,琴童空回來說:「任老爹在府裡上班,未回來。他家知道咱這裡請,說明日任老爹絕早就來了。」

月娘見喬大戶一替兩替來請,便道:「太醫已是明日來了,你往喬親家那裡去罷。天晚了,你不去,惹的喬親家怪。」西門慶道:「我去了,誰看你?」月娘笑道:「傻行貨子,誰要你做恁箇腔兒。你去,我不妨事。等我消一回兒,慢慢䦛䦟着起來,與大妗子坐的吃飯。你慌的是些甚麼?」西門慶令玉簫:「快請你大妗子來,和你娘坐的。」又問:「郁大姐在那裡?叫他唱與娘聽。」玉簫道:「郁大姐往家去,不耐煩了。」西門慶道:「誰教他去來?留他兩住兩日兒也罷了。」趕着玉簫踢了兩脚。月娘道:「他見你家反宅亂,要去,管他腿事?」玉簫道:「正經罵申二姐的倒不踢。」{\pangpi{妙。}}那西門慶只做不聽見,一面穿了衣裳,往喬大戶家吃酒去了。未到起更時分,就來家,到了上房。月娘正和大妗子、玉樓、李嬌兒四箇坐的。大妗子見西門慶進來,忙往後邊去了。

西門慶便問月娘道:「你這咱好些了麼?」月娘道:「大妗子陪我吃了兩口粥兒,心口內不大十分脹了,還只有些頭疼腰痠。」西門慶道:「不打緊,明日任後溪來看,吃他兩服藥,解散散氣,安安胎就好了。」月娘道:「我那等樣教你休請他,你又請他。白眉赤眼,教人家漢子來做甚麼?你明日看我出去不出去!」因問:「喬親家請你做甚麼?」西門慶道:「他說我從東京來了,與我坐坐。今日他也費心,整治許多菜蔬,叫兩箇唱的,落後又邀過朱臺官來陪我。我熱着你,心裡不自在,吃了幾鍾酒,老早就來了。」月娘道:「好箇說嘴的貨!我聽不上你這巧言花語,可哥兒就是熱着我來?我是那活佛出現,也不放在你那心上。就死了也不値箇破沙鍋片子。」又問:「喬親家再沒和你說什麼話?」西門慶方告說:「喬親家如今要趁着新例,上三十兩銀子納箇義官。銀子也封下了,教我對胡府尹說。我說不打緊,胡府尹昨日送了我一百本歷日,我還沒曾回他禮。等我送禮時,稍了貼子與他,問他討一張義官箚付來與你就是了。他不肯,他說納些銀子是正理。如今央這裡分上討討兒,免上下使用,也省十來兩銀子。」月娘道:「既是他央及你,替他討討兒罷。你沒拿他銀子來?」西門慶道:「他銀子明日送過來。還要買分禮來,我止住他了。到明日,咱僉一口豬,一罈酒,送胡府尹就是了。」說畢,西門慶晚夕就在上房睡了一夜。

到次日,宋巡按擺酒,後廳筵席治酒,裝定菓品。大清早晨,本府出票撥了兩院三十名官身樂人,兩名伶官、四名排長領着,來西門慶宅中答應。只見任醫官從早晨就騎馬來了,西門慶忙迎到廳上陪坐,道連日闊懷之事。任醫官道:「昨日盛使到,學生該班,至晚纔來家,見尊刺,今日不俟駕而來。敢問何人欠安?」西門慶道:「大賤內偶然有些失調,請後溪一診。」須臾茶至。吃了茶,任醫官道:「昨日聞得明川說,老先生恭喜,容當奉賀。」西門慶道:「菲才備員而已,何賀之有。」一面西門慶分付:「後邊對你大娘說,任老爹來了,明間內收拾。」琴童應諾,到後邊。大妗子、李嬌兒、孟玉樓都在房內,只見琴童來說:「任醫官來了,爹分付教收拾明間裡坐的。」月娘只不動身,說道:「我說不要請他,平白教人家漢子,睜着活眼,把手捏腕的,不知做甚麼!叫劉媽媽子來,吃兩服藥,繇他好了。好這等搖鈴打鼓的,好與人家漢子喂眼。」玉樓道:「大娘,已是請人來了,你不出去卻怎樣的,莫不回了人去不成?」大妗子又在旁邊勸着說:「姑娘,他是箇太醫,你教他看看你這脈息,還知道你這病源,不知你為甚起氣惱,傷犯了那一經。吃了他藥,替你分理理氣血,安安胎氣也好。劉婆子他曉得甚麼病源脈理?一時耽誤怎了。」月娘方動身梳頭,戴上冠兒,玉簫拿鏡子,孟玉樓跳上炕去,替他拿抿子掠後𩬆。李嬌兒替他勒鈿兒。孫雪娥預備拿衣裳。{\meipi{有妾者何等便當,凡為大娘子者,何苦不容。}}不一時,打扮的粉粧玉琢,正是:

\begin{myquote} 
羅浮仙子臨凡世,月殿嬋娟出畫堂。
\end{myquote} 

