\includepdf[pages={23,24},fitpaper=false]{tst.pdf}
\chapter*{第十二回 潘金蓮私僕受辱 劉理星魘勝求財}
\addcontentsline{toc}{chapter}{第十二回 潘金蓮私僕受辱 劉理星魘勝求財}
\markboth{{\titlename}卷之二}{第十二回 潘金蓮私僕受辱 劉理星魘勝求財}


詩曰:

\begin{myquote}
可憐獨立樹,枝輕根亦搖。\\雖為露所浥,復為風所飄。\\錦衾襞不開,端坐夜及朝。\\是妾愁成瘦,非君重細腰。
\end{myquote}

話說西門慶在院中貪戀桂姐姿色,{\pangpi{元不及情。}}約半月不曾來家。吳月娘使小厮拿馬接了數次,李家把西門慶衣帽都藏過,不放他起身。丟的家中這些婦人都閑靜了。別人猶可,惟有潘金蓮這婦人,青春未及三十歲,慾火難禁一丈高。每日打扮的粉粧玉琢,皓齒硃唇,無日不在大門首倚門而望,只等到黃昏。到晚來歸入房中,粲枕孤幃,鳳台無伴,睡不着,走來花園中,款步花苔。看見那月漾水底,便疑西門慶情性難拿;偶遇着玳瑁貓兒交歡,越引逗的他芳心迷亂。當時玉樓帶來一箇小厮,名喚琴童,年約十六歲,纔留起頭髮,生的眉目清秀,乖滑伶俐。{\meipi{此何物,豈可置之閨人左右?西門慶元自疎畧。}}西門慶教他看管花園,晚夕就在花園門首一間小耳房內安歇。金蓮和玉樓白日裡常在花園亭子上一處做針指或下棋。這小厮專一獻小殷勤,{\pangpi{便非蠢人。}}常觀見西門慶來,就先來告報。以此婦人喜他,常叫他入房,賞酒與他吃。兩箇朝朝暮暮,眉來眼去,都有意了。

不想到了七月,西門慶生日將近。吳月娘見西門慶留戀烟花,因使玳安拿馬去接。這潘金蓮暗暗修了一柬帖,交付玳安,教:「悄悄遞與你爹,說五娘請爹早些家去罷。」這玳安兒一直騎馬到李家,只見應伯爵、謝希大、祝實念、孫寡嘴、常峙節衆人,正在那裡伴着西門慶,摟着粉頭歡樂飲酒。西門慶看見玳安來到,便問:「你來怎麼?家中沒事?」玳安道:「家中沒事。」西門慶道:「前邊各項銀子,叫傅二叔討討,等我到家算帳。」{\pangpi{財主口角。}}玳安道:「這兩日傅二叔討了許多,等爹到家上帳。」{\meipi{沒要沒緊,俱文人玩世心思所寄。}}西門慶道:「你桂姨那一套衣服,稍來不曾?」{\pangpi{大老官口聲。}}玳安道:「已稍在此。」便向氊包內取出一套紅衫藍裙,遞與桂姐。桂姐道了萬福,收了,連忙分付下邊,管待玳安酒飯。那小厮吃了酒飯,復走來上邊伺候。悄悄向西門慶耳邊說道:「五娘使我稍了箇帖兒在此。請爹早些家去。」西門慶纔待用手去接,早被李桂姐看見,只道是西門慶那箇表子寄來的情書,一手撾過來,拆開觀看,卻是一幅迴文錦箋,上寫着幾行墨蹟。桂姐遞與祝實念,教念與他聽。這祝實念見上面寫詞一首,名《落梅風》,念道:

\begin{myquote}[\markfont]
黃昏想,白日思;盼殺人多情不至。因他為他憔悴死,可憐也,綉衾獨自!\\
燈將殘,人睡也,空留得半窻明月。眠心硬,渾似鐵,這淒涼怎捱今夜?

\raggedleft{下書:愛妾潘六兒拜。\rightquadmargin}
\end{myquote}

那桂姐聽畢,撇了酒席,走入房中,倒在床上,面朝裡邊睡了。{\pangpi{好做作。}}西門慶見桂姐惱了,把帖子扯的稀爛,衆人前把玳安踢了兩脚。{\pangpi{着鬼。}}請桂姐兩遍不來,慌的西門慶親自進房,抱出他來,說道:「分付帶馬回去,家中那箇淫婦使你來,我這一到家,都打箇臭死!」玳安只得含淚回家。西門慶道:「桂姐,你休惱,這帖子不是別人的,乃是我第五箇小妾寄來,請我到家有些事兒計較,再無別故。」祝實念在旁戲道:「桂姐,你休聽他哄你哩!這箇潘六兒乃是那邊院裡新叙的一箇表子,生的一表人物。你休放他去。」西門慶笑趕着打,說道:「你這賊天殺的,單管弄死了人,緊着他恁麻犯人,你又胡說。」{\meipi{佯嗔故惱,冷幫熱襯,鬨然一堂之上,彷彿如覩。}}李桂卿道:「姐夫差了,既然家中有人拘管,就不消梳籠人家粉頭,自守着家裡的便了。{\meipi{桂卿又老着臉兒說正經話,妙甚!}}纔相伴了多少時,便就要拋離了去。」{\pangpi{好寬皮散。}}應伯爵插口道:「說的有理。你兩人都依我,大官人也不消家去,桂姐也不必惱。今日說過,那箇再恁,每人罰二兩銀子,買酒咱大家吃。」於是西門慶把桂姐摟在懷中陪笑,一遞一口兒飲酒。少傾,拿了七鍾細茶來,馨香可掬,每人面前一盞。應伯爵道:「我有箇曲兒,單道這茶好處:

\begin{myquote}
{\markfont\small〔朝天子〕}這細茶的嫩芽,生長在春風下。不揪不採葉兒楂,但煮着顏色大。絕品清奇,難描難畫。口裡兒常時呷,醉了時想他,醒來時愛他。{\meipi{雙關得妙。}}原來一簍兒千金價。」
\end{myquote}

謝希大笑道:「大官人使錢費物,不圖這『一摟兒』,{\pangpi{收科。}}卻圖些甚的?如今每人有詞的唱詞,不會詞,每人說箇笑話兒,與桂姐下酒。」就該謝希大先說,因說道:「有一箇泥水匠,在院中墁地。老媽兒怠慢了他,他暗把陰溝內堵上塊磚。落後天下雨,積的滿院子都是水。老媽慌了,尋的他來,多與他酒飯,還秤了一錢銀子,央他打水準。那泥水匠吃了酒飯,悄悄去陰溝內把那塊磚拿出,那水登時出的罄盡。老媽便問作頭:『此是那裡的病?』泥水匠回道:『這病與你老人家的病一樣,有錢便流,無錢不流。』」桂姐見把他家來傷了,便道:「我也有箇笑話,回奉列位。有一孫眞人,擺着筵席請人,卻教座下老虎去請。那老虎把客人都路上一箇箇吃了。眞人等至天晚,不見一客到。不一時老虎來,眞人便問:『你請的客人都那裡去了?』老虎口吐人言:{\pangpi{俚得妙。}}『告師父得知,我從來不曉得請人,只會白嚼人。』」當下把衆人都傷了。應伯爵道:「可見的俺們只是白嚼你家孤老,就還不起箇東道?」於是向頭上撥下一根鬧銀耳斡兒來,重一錢;謝希大一對鍍金網巾圈,秤了秤重九分半;祝實念袖中掏出一方舊汗巾兒,算二百文長錢;{\meipi{妙在件件皆清客之物,與珠玉等項自別。}}孫寡嘴腰間解下一條白布裙,{\pangpi{此物太醜。}}當兩壺半酒;常峙節無以為敬,問西門慶借了一錢銀子。都遞與桂卿,置辦東道,請西門慶和桂姐。那桂卿將銀錢都付與保兒,買了一錢豬肉,又宰了一隻雞,自家又賠些小菜兒,{\pangpi{偏周到。}}安排停當。大盤小碗拿上來,衆人坐下,說了一聲「動筯吃」時,說時遲,那時快,但見:

\begin{myquote}
人人動嘴,箇箇低頭。遮天映日,猶如蝗蚋一齊來;擠眼掇肩,好似餓牢纔打出。這箇搶風膀臂,如經年未見酒和餚;那箇連三筷子,成歲不逢筵與席。一箇汗流滿面,卻似與雞骨禿有冤仇;一箇油抹唇邊,把豬毛皮連唾嚈。吃片時,盃盤狼藉;啖頃刻,筯子縱橫。這箇稱為食王元帥,那箇號作淨盤將軍。酒壺番晒又重斟,盤饌已無還去探。
\end{myquote}

正是:

\begin{myquote}
珍馐百味片時休,果然都送入五臟廟。{\meipi{寫得盡情痛快,此風雖文人不免,何況伯爵一輩。}}
\end{myquote}

當下衆人,吃得箇淨光王佛。西門慶與桂姐吃不上兩鍾酒,揀了些菜蔬,又被這夥人吃去了。那日把席上椅子坐折了兩張,前邊跟馬的小厮,不得上來掉嘴吃,把門前供養的土地翻倒來,便剌了一泡稒谷都的熱屎。{\meipi{院中實有此景,非點綴也。}}臨出門來,孫寡嘴把李家明間內供養的鍍金銅佛,㩙在褲腰裡;應伯爵推鬬桂姐親嘴,把頭上金琢針兒戲了;謝希大把西門慶川扇兒藏了;祝實念走到桂卿房裡照面,溜了他一面水銀鏡子。常峙節借的西門慶一錢銀子,竟是寫在闝賬上了。原來這起人,只伴着西門慶頑耍,好不快活。有詩為證:

\begin{myquote}
工妍掩袖媚如猱,乘興閒來可暫留。\\若要死貪無厭足,家中金鑰教誰收?
\end{myquote}

按下衆人簇擁着西門慶飲酒不題。單表玳安回馬到家,吳月娘和孟玉樓、潘金蓮正在房坐的,見了便問玳安:「你去接爹來了不曾?」玳安哭的兩眼紅紅的,說道:「被爹踢罵了小的來了。爹說那箇再使人接,來家都要罵。」月娘便道:「你看恁不合理,不來便了,如何又罵小厮?」孟玉樓道:「你踢將小厮便罷了,如何連俺們都罵將來?」潘金蓮道:「十箇九箇院中淫婦,和你有甚情實!常言說的好:『船載的金銀,塡不滿烟花塞。』」金蓮只知說出來,不防李嬌兒見玳安自院中來家,便走來窻下潛聽。見金蓮罵他家千淫婦萬淫婦,暗暗懷恨在心。從此二人結仇,不在話下。正是:

\begin{myquote}
甜言美語三冬煖,惡語傷人六月寒。
\end{myquote}

不說李嬌兒與潘金蓮結仇。單表金蓮歸到房中,捱一刻似三秋,盼一時如半夏。知道西門慶不來家,把兩箇丫頭打發睡了,推往花園中游翫,將琴童叫進房與他酒吃。把小厮灌醉了,{\meipi{琴童何脩而得此?為之不平。}}掩上房門,褪衣解帶,兩箇就幹做一處。但見:

\begin{myquote}
一箇不顧綱常貴賤,一箇那分上下高低。一箇色膽歪邪,管甚丈夫利害;一箇淫心蕩漾,從他律法明條。百花園內,翻為快活排場;主母房中,變作行樂世界。霎時一滴驢精髓,{\pangpi{不忿語。}}傾在金蓮玉體中。
\end{myquote}

自此為始,每夜婦人便叫琴童進房如此。未到天明,就打發出來。背地把金裹頭簪子兩三根帶在頭上,又把裙邊帶的錦香囊葫蘆兒也與了他。豈知這小厮不守本分,常常和同行小厮街上吃酒耍錢,頗露機關。常言:「若要不知,除非莫為。」有一日,風聲吹到孫雪娥、李嬌兒耳朵內,說道:「賊淫婦,往常假撇清,如何今日也做出來了?」齊來告月娘。月娘再三不信,{\meipi{月娘非不信,只一味解紛息爭耳。}}說道:「不爭你們和他合氣,惹的孟三姐不怪?只說你們擠撮他的小厮。」說的二人無言而退。落後婦人夜間和小厮在房中行事,忘記關廚房門,不想被丫頭秋菊出來淨手,看見了。次日傳與後邊小玉,小玉對雪娥說。雪娥同李嬌兒又來告訴月娘如此這般:「他屋裡丫頭親口說出來,又不是俺們葬送他。大娘不說,俺們對他爹說。若是饒了這箇淫婦,非除饒了蠍子!」此時正値七月二十七日,西門慶從院中來家上壽。月娘道:「他纔來家,又是他好日子,你們不依我,只顧說去!等他反亂將起來,我不管你。」二人不聽月娘,約的西門慶進入房中,齊來告訴金蓮在家怎的養小厮一節。這西門慶不聽萬事皆休,聽了怒從心上起,惡向膽邊生。走到前邊坐下,一片聲叫琴童兒。早有人報與潘金蓮。金蓮慌了手脚,使春梅忙叫小厮到房中,囑咐千萬不要說出來,把頭上簪子都拿過來收了。着了慌,就忘解了香囊葫蘆下來。{\pangpi{有波瀾。}}被西門慶叫到前廳跪下,分付三四箇小厮,選大板子伺候。西門慶道:「賊奴才,你知罪麼?」那琴童半日不敢言語。西門慶令左右:「拔下他簪子來我瞧!」見沒了簪子,因問:「你戴的金裹頭銀簪子,往那裡去了?」琴童道:「小的並沒甚銀簪子。」西門慶道:「奴才還搗鬼!與我旋剝了衣服,拿板子打!」當下兩三箇小厮扶侍一箇,剝去他衣服,扯了褲子。見他身底下穿着玉色絹𧜽兒,𧜽兒帶上露出錦香囊葫蘆兒。西門慶一眼看見,便叫:「拿上來我瞧!」認的是潘金蓮裙邊帶的物件,{\meipi{偏看見,偏認得,絕有情景。}}不覺心中大怒,就問他:「此物從那裡得來?你寔說是誰與你的?」諕的小厮半日開口不得,說道:「這是小的某日打掃花園,在花園內拾的。並不曾有人與我。」西門慶越怒,切齒喝令:「與我綑起來着實打!」當下把琴童繃子繃着,打了三十大棍,打得皮開肉綻,鮮血順腿淋漓。又叫來保:「把奴才兩箇鬢毛與我撏了!趕將出去,再不許進門!」{\meipi{不待審問的確,竟日打逐,似暴燥,又似隱忍,妙得其情。}}那琴童磕了頭,哭哭啼啼出門去了。

潘金蓮在房中聽見,如提在冷水盆內一般。不一時,西門慶進房來,嚇的戰戰兢兢,渾身無了脈息,小心在旁扶侍接衣服,被西門慶兜臉一箇耳刮子,把婦人打了一交。分付春梅:「把前後角門頂了,不放一箇人進來!」拿張小椅兒,坐在院內花架兒底下,取了一根馬鞭子,拿在手裡,喝令:「淫婦,脫了衣裳跪着!」那婦人自知理虧,不敢不跪,眞箇脫去了上下衣服,跪在面前,低垂粉面,不敢出一聲兒。{\pangpi{便自可憐。}}西門慶便問:「賊淫婦,你休推夢裡睡裡,奴才我已審問明白,他一一都供出來了。你寔說,我不在家,你與他偷了幾遭?」婦人便哭道:「天那,天那!可不冤屈殺了我罷了!自從你不在家半箇來月,奴白日裡只和孟三兒一處做針指,到晚夕早關了房門就睡了。沒勾當,不敢出這角門邊兒來。你不信,只問春梅便了。{\pangpi{好證見。}}有甚和鹽和醋,他有箇不知道的?」因叫春梅:「姐姐你過來,親對你爹說。」西門慶罵道:「賊淫婦!有人說你把頭上金裹頭簪子兩三根都偷與了小厮,你如何不認?」{\pangpi{淺甚。}}婦人道:「就屈殺了奴罷了!是那箇不逢好死的嚼舌根的淫婦,嚼他那旺跳身子。見你常時進奴這屋裡來歇,無非都氣不憤,拿這有天沒日頭的事壓枉奴。就是你與的簪子,都有數兒,一五一十都在,你查不是!我平白想起甚麼來與那奴才?好成材的奴才,也不枉說的,恁一箇尿不出來的毛奴才,{\pangpi{精卻出來。}}平空把我篡一篇舌頭!」西門慶道:「簪子有沒罷了。」因向袖中取出那香囊來,說道:「這箇是你的物件兒,如何打小厮身底下捏出來?你還口強甚麼?」說着,紛紛的惱了,向他白馥馥香肌上,『颼』的一馬鞭子來,打的婦人疼痛難忍,眼噙粉淚,沒口子叫道:「好爹爹,你饒了奴罷!你容奴說便說,不容奴說,你就打死了奴,也只臭爛了這塊地。這箇香囊葫蘆兒,你不在家,奴那日同孟三姐在花園裡做生活,因從木香棚下過,帶兒系不牢,就抓落在地,{\pangpi{可哥合着,妙。若先知會,便無味矣。}}我那裡沒尋,誰知這奴才拾了。奴並不曾與他。」{\meipi{先作萬分不可解之勢,忽一語解之,令讀者怒喜無定。}}只這一句,就合着琴童供稱一樣的話,又見婦人脫的光赤條條,花朵兒般身子,嬌啼嫩語,跪在地下,那怒氣早已鑽入爪窪國去了,把心已回動了八九分,因叫過春梅,摟在懷中,問他:{\pangpi{自尋出路。}}「淫婦果然與小厮有首尾沒有?你說饒了淫婦,我就饒了罷。」那春梅撒嬌撒癡,坐在西門慶懷裡,說道:「這箇,爹你好沒的說!我和娘成日唇不離腮,娘肯與那奴才?這箇都是人氣不憤俺娘兒們,做作出這樣事來。爹,你也要箇主張,好把醜名兒頂在頭上,傳出外邊去好聽?」{\pangpi{反把一堆泥,堆在西門慶頭上,巧甚!}}幾句把西門慶說的一聲兒沒言語,丟了馬鞭子,一面叫金蓮起來,穿上衣服,{\meipi{畢竟愛心勝,稍有一絲出脫之路,便出脫之矣。}}分付秋菊看菜兒,放桌兒吃酒。這婦人滿斟了一盃酒,雙手遞上去,跪在地下,等他鍾兒。西門慶分付道:「我今日饒了你。我若但凡不在家,要你洗心改正,早關了門戶,不許你胡思亂想。我若知道,並不饒你!」婦人道:「你分付,奴知道了。」{\meipi{大家都含糊罷了,妙。}}又與西門慶磕了四箇頭,方纔安坐兒,在旁陪坐飲酒。潘金蓮平日被西門慶寵的狂了,今日討這場羞辱在身上。正是:

\begin{myquote}
為人莫作婦人身,百年苦樂由他人。
\end{myquote}

當下西門慶正在金蓮房中飲酒,忽小厮打門,說:「前邊有吳大舅、吳二舅、傅夥計、女兒、女婿,衆親戚送禮來祝壽。」方纔撇了金蓮,出前邊陪待賓客。那時應伯爵、謝希大衆人都有人情,院中李桂姐家亦使保兒送禮來。{\pangpi{伏。}}西門慶前邊亂着收人家禮物,發柬請人,不在話下。

且說孟玉樓打聽金蓮受辱,約的西門慶不在房裡,瞞着李嬌兒、孫雪娥,走來看望。見金蓮睡在床上,因問道:「六姐,你端的怎麼緣故?告我說則箇。」

那金蓮滿眼流淚哭道:「三姐,你看小淫婦,今日在背地裡白唆調漢子,打了我恁一頓。我到明日,和這兩箇淫婦冤仇結得有海深。」{\meipi{不怨自家差錯,只記恨別人,婦人腸肚,大率類此。}}玉樓道:「你便與他有瑕玷,如何做作着把我的小厮弄出去了?六姐,你休煩惱,莫不漢子就不聽俺們說句話兒?若明日他不進我房裡來便罷,但到我房裡來,等我慢慢勸他。」金蓮道:「多謝姐姐費心。」一面叫春梅看茶來吃。坐着說了回話,玉樓告回房去了。至晚,西門慶因上房吳大妗子來了,走到玉樓房中宿歇。玉樓因說道:「你休枉了六姐心,六姐並無此事,都是日前和李嬌兒、孫雪娥兩箇有言語,平白把我的小厮紮罰了。你不問箇青紅皁白,就把他屈了,卻不難為他了!我就替他賭箇大誓,若果有此事,大姐姐有箇不先說的?」{\pangpi{解亦巧。}}西門慶道:「我問春梅,他也是這般說。」{\pangpi{呆人可笑。}}玉樓道:「他今在房中不好哩,你不去看他看去?」西門慶道:「我知道,明日到他房中去。」當晚無話。

到第二日,西門慶正生日。有周守備、夏提刑、張團練、吳大舅許多官客飲酒,拿轎子接了李桂姐並兩箇唱的,唱了一日。李嬌兒見他姪女兒來,引着拜見月娘衆人,{\pangpi{宛然。}}在上房裡坐吃茶。請潘金蓮見,連使丫頭請了兩遍,金蓮不出來,只說心中不好。到晚夕,桂姐臨家去,拜辭月娘。月娘與他一件雲絹比甲兒、汗巾花翠之類,同李嬌兒送出門首。桂姐又親自到金蓮花園角門首:「好歹見見五娘。」{\pangpi{定要見,促恰之甚。}}那金蓮聽見他來,使春梅把角門關得鐵桶相似,說道:「娘分付,我不敢開。」這花娘遂羞訕滿面而回,{\pangpi{亦自取。}}不題。

單表西門慶至晚進入金蓮房內來,那金蓮把雲鬟不整,花容倦淡,{\pangpi{四字可憐。}}迎接進房,替他脫衣解帶,伺候茶湯脚水,百般殷勤扶侍。到夜裡枕蓆歡娛,屈身忍辱,無所不至,說道:「我的哥哥,這一家誰是疼你的?都是露水夫妻,再醮貨兒。惟有奴知道你的心,你知道奴的意。旁人見你這般疼奴,在奴身邊的多,都氣不憤,背地裡駕舌頭,在你跟前唆調。我的傻冤家!你想起甚麼來,中人的拖刀之計,把你心愛的人兒這等下無情的折挫!常言道:『家雞打的團團轉,野雞打的貼天飛。』你就把奴打死了,也只在這屋裡。{\meipi{百分小心,只不放倒架子。然而思悲語苦。}}就是前日你在院裡踢罵了小厮來,早是有大姐姐、孟三姐在跟前,我自不是說了一聲,恐怕他家粉頭掏淥壞了你身子,院中唱的一味愛錢,有甚情節?誰人疼你?誰知被有心的人聽見,兩箇背地做成一幫兒算計我。自古人害人不死,天害人才害死了。往後久而自明,只要你與奴做箇主兒便了。」幾句把西門慶窩盤住了。是夜與他淫慾無度。

過了幾日,西門慶備馬,玳安、平安兩箇跟隨,往院中來。卻說李桂姐正打扮着陪人坐的,聽見他來,連忙走進房去,洗了濃粧,除了簪環,倒在床上裹衾而臥。{\meipi{娼家假態,曲曲寫出。}}西門慶走到,坐了半日,老媽纔出來,道了萬福,讓西門慶坐下,問道:「怎的姐夫連日不進來走走?」西門慶道:「正是因賤日窮冗,家中無人。」虔婆道:「姐兒那日打攪。」西門慶道:「怎的那日桂卿不來走走?」虔婆道:「桂卿不在家,被客人接去店裡。這幾日還不放了來。」說了半日話,纔拿茶來陪着吃了。西門慶便問:「怎的不見桂姐?」虔婆道:「姐夫還不知哩,小孩兒家,不知怎的,那日着了惱,來家就不好起來,睡倒了。房門兒也不出,直到如今。姐夫好狠心,也不來看看姐兒。」{\meipi{先冷冷落落,推他開口,方婉婉說入,的是虔婆伎倆。}}西門慶道:「眞箇?我通不知。」{\meipi{難道是假不成?}}因問:「在那邊房裡?我看看去。」虔婆道:「在他後邊臥房裡睡。」慌忙令丫鬟掀簾子。

西門慶走到他房中,只見粉頭烏雲散亂,粉面慵粧,裹被坐在床上,面朝裡,見了西門慶,不動一動兒。{\pangpi{畫。}}西門慶道:「你那日來家,怎的不好?」也不答應。{\pangpi{畫。}}又問:「你着了誰人惱,你告我說。」問了半日,那桂姐方開言說道:「左右是你家五娘子。你家中既有恁好的迎歡賣俏,{\pangpi{深譏。}}又來稀罕俺們這樣淫婦做甚麼?俺們雖是門戶中出身,蹺起脚兒,比外邊良人家不成的貨色兒高好些!{\pangpi{深譏。}}我前日又不是供唱,我也送人情去。大娘到見我甚是親熱,又與我許多花翠衣服。待要不請他見,又說俺院中沒禮法。聞說你家有五娘子,當即請他拜見,又不出來。家來同俺姑娘又辭他去,他使丫頭把房門關了。端的好不識人敬重!」西門慶道:「你到休怪他。他那日本等心中不自在,他若好時,有箇不出來見你的?這箇淫婦,我幾次因他咬羣兒,口嘴傷人,也要打他哩!」桂姐反手向西門慶臉上一掃,說道:「沒羞的哥兒,你就打他?」{\pangpi{激得妙。}}西門慶道:「你還不知我手段,除了俺家房下,家中這幾箇老婆丫頭,但打起來也不善,着緊二三十馬鞭子還打不下來。好不好還把頭髮都剪了。」桂姐道:「我見砍頭的,沒見吹嘴的,你打三箇官兒,唱兩箇喏,誰見來?你若有本事,到家裡只剪下一柳子頭髮,拿來我瞧,我方信你是本司三院有名的子弟。」{\meipi{既激之以怒,又歆之以名,桂姐亦是辣手。}}西門慶道:「你敢與我排手?」{\pangpi{呆甚。}}那桂姐道:「我和你排一百箇手。」當日西門慶在院中歇了一夜,到次日黃昏時分,辭了桂姐,上馬回家。桂姐道:「哥兒,你這一去,沒有這物件兒,看你拿甚嘴臉見我!」

這西門慶吃他激怒了幾句話,歸家已是酒酣,不往別房裡去,逕到潘金蓮房內來。婦人見他有酒了,加意用心伏侍。問他酒飯都不吃。分付春梅把床上枕蓆拭抹乾淨,帶上門出去。他便坐在床上,令婦人脫靴。那婦人不敢不脫。須臾,脫了靴,打發他上床。西門慶且不睡,坐在一隻枕頭上,令婦人褪了衣服,地下跪着。{\meipi{先尋事,起水頭,寫得肺肝如見。}}那婦人嚇的捏兩把汗,又不知因為甚麼,於是跪在地下,柔聲痛哭道:「我的爹爹!你透與奴箇伶俐說話,奴死也甘心。饒奴終日恁提心弔膽,陪着一千箇小心,還投不着你的機會,只拿鈍刀子鋸處我,教奴怎生吃受?」西門慶罵道:「賤淫婦,你眞箇不脫衣裳,我就沒好意了!」因叫春梅:「門背後有馬鞭子,與我取了來!」{\meipi{割所愛以奉所愛,似乎近愚,然亦前氣未消盡故耳。}}那春梅只顧不進房來,叫了半日,纔慢條厮理{\pangpi{一味恃寵。}}推開房門進來。看見婦人跪在床地平上,向燈前倒着桌兒下,由西門慶使他,只不動身。婦人叫道:「春梅,我的姐姐,你救我救兒,他如今要打我。」西門慶道:「小油嘴兒,你不要管他。你只遞馬鞭子與我打這淫婦。」{\pangpi{勢緊語鬆。}}春梅道:「爹,你怎的恁沒羞!娘幹壞了你甚麼事兒?你信淫婦言語,平地裡起風波,要便搜尋娘?還教人和你一心一計哩!你教人有那眼兒看得上你!倒是我不依你。」拽上房門,走在前邊去了。那西門慶無法可處,倒呵呵笑了,{\pangpi{過下無痕。}}向金蓮道:「我且不打你。你上來,我問你要樁物兒,你與我不與我?」{\meipi{到此方入題,西門慶亦費許多曲折矣。}}婦人道:「好親親,奴一身骨朵肉兒都屬了你,{\pangpi{情急語。}}隨要甚麼,奴無有不依隨的。不知你心裡要甚麼兒?」西門慶道:「我要你頂上一柳兒好頭髮。」婦人道:「好心肝!奴身上隨你怎的揀着燒遍了也依,這箇剪頭髮卻依不的,可不嚇死了我罷了。奴出娘胞兒,活了二十六歲,從沒幹這營生。打緊我頂上這頭髮近來又脫了好些,{\meipi{金蓮此時情亦苦矣。}}只當可憐見我罷。」西門慶道:「你只怪我惱,我說的你就不依。」婦人道:「我不依你,再依誰?」因問:「你實對奴說,要奴這頭髮做甚麼?」西門慶道:「我要做網巾。」婦人道:「你要做網巾,奴就與你做,休要拿與淫婦,教他好壓鎭我。」西門慶道:「我不與人便了,要你髮兒做頂線兒。」婦人道:「你既要做頂線,待奴剪與你。」當下婦人分開頭髮,西門慶拿剪刀,按婦人頂上,齊臻臻剪下一大柳來,用紙包放在順袋內。{\meipi{燒琴煮鶴且不可,況剪美人之髮乎!剪而相贈猶不可,況因氣而相逼乎!為之痛惜!}}婦人便倒在西門慶懷中,嬌聲哭道:「奴凡事依你,只願你休忘了心腸,隨你前邊和人好,只休拋閃了奴家!」是夜與他歡會異常。

到次日,西門慶起身,婦人打發他吃了飯,出門騎馬,逕到院裡。桂姐便問:「你剪的他頭髮在那裡?」西門慶道:「有,在此。」便向茄袋內取出,遞與桂姐。開啟看,果然黑油也一般好頭髮,{\pangpi{襯出。}}就收在袖中。西門慶道:「你看了還與我,他昨日為剪這頭髮,好不煩難,吃我變了臉惱了,他纔容我剪下這一柳子來。我哄他,只說要做網巾頂線兒,逕拿進來與你瞧。可見我不失信。」桂姐道:「甚麼稀罕貨,慌的恁箇腔兒!等你家去,我還與你。比是你恁怕他,就不消剪他的來了。」{\meipi{拿來火熱,卻又搶白得冰冷,桂姐利嘴可畏。}}西門慶笑道:「那裡是怕他!恁說我言語不的了。」桂姐一面叫桂卿陪着他吃酒,走到背地裡,把婦人頭髮早絮在鞋底下,每日踹踏,{\pangpi{寫出噁心。}}不在話下。卻把西門慶纏住,連過了數日,不放來家。

金蓮自從頭髮剪下之後,覺道心中不快,每日房門不出,茶飯慵餐。吳月娘使小厮請了家中常走看的劉婆子來看視,說:「娘子着了些暗氣,惱在心中,不能迴轉,頭疼噁心,飲食不進。」一面開啟藥包來,留了兩服黑丸子藥兒:「晚上用薑湯吃。」又說:「我明日叫我老公來,替你老人家看看今歲流年,有災沒災。」金蓮道:「原來你家老公也會算命?」劉婆道:「他雖是箇瞽目人,到會兩三樁本事:第一善陰陽算命,與人家禳保;第二會針灸收瘡;第三樁兒不可說,{\pangpi{作聲價。}}——單管與人家回背。」婦人問道:「怎麼是回背?」劉婆子道:「比如有父子不和,兄弟不睦,{\meipi{明明要說夫妻,卻從父子、兄弟開科。小人小術何嘗無次第!}}大妻小妻爭鬬,教了俺老公去說了,替他用鎭物安鎭,畫些符水與他吃了,不消三日,教他父子親熱,兄弟和睦,妻妾不爭。若人家買賣不順溜,田宅不興旺者,{\pangpi{又補出數事,若不為夫妻發者。妙甚!}}常與人開財門發利市。治病灑掃,禳星告鬬都會。因此人都叫他做劉理星。也是一家子,新娶箇媳婦兒是小人家女兒,有些手脚兒不穩,常偷盜婆婆家東西往娘家去。丈夫知道,常被責打。俺老公與他回背,畫了一道符,燒灰放在水缸下埋着,合家大小吃了缸內水,眼看媳婦偷盜,只相沒看見一般。{\meipi{引不端事作證,諧甚!}}又放一件鎭物在枕頭內,男子漢睡了那枕頭,好似手封住了的,再不打他了。」那金蓮聽見遂留心,便呼丫頭,打發茶湯點心與劉婆吃。臨去,包了三錢藥錢,另外又秤了五錢,要買紙紮信物。明日早飯時叫劉瞎來燒神紙。那婆子作辭回家。

到次日,果然大清早晨,領賊瞎逕進大門往裡走。那日西門慶還在院中,看門小厮便問:「瞎子往那裡走?」劉婆道:「今日與裡邊五娘燒紙。」小厮道:「既是與五娘燒紙,老劉你領進去。仔細看狗。」這婆子領定,逕到潘金蓮臥房明間內,等了半日,婦人纔出來。瞎子見了禮,坐下。婦人說與他八字,賊瞎用手捏了捏,說道:「娘子庚辰年,庚寅月,乙亥日,己丑時。初八日立春,已交正月算命。依子平正論,娘子這八字,雖故清奇,一生不得夫星濟,子上有些防礙。乙木生在正月間,亦作身旺論,不尅當自焚。又兩重庚金,羊刃太重,夫星難為,尅過兩箇纔好。」婦人道:「已尅過了。」賊瞎子道:「娘子這命中,休怪小人說,子平雖取煞印格,只吃了亥中有癸水,醜中又有癸水,水太多了,冲動了,只一重巳土,官煞混雜。論來,男人煞重掌威權,女子煞重必刑夫。所以主為人聰明機變,得人之寵。只有一件,今歲流年甲辰,歲運並臨,災殃立至。命中又犯小耗勾絞,兩位星辰打攪,雖不能傷,卻主有比肩不和,小人嘴舌,常沾些啾唧不寧之狀。」婦人聽了,說道:「累先生仔細用心,與我回背回背。我這裡一兩銀子相謝先生,買一盞茶吃。奴不求別的,只願得小人離退,夫主愛敬便了。」一面轉入房中,拔了兩件首飾遞與賊瞎。賊瞎收入袖中,說道:「既要小人回背,用柳木一塊,刻兩箇男女人形,書着娘子與夫主生辰八字,用七七四十九根紅線紮在一處。上用紅紗一片,蒙在男子眼中,用艾塞其心,用針釘其手,下用膠粘其足,暗暗埋在睡的枕頭內。又硃砂書符一道燒灰,暗暗攪茶內。若得夫主吃了茶,到晚夕睡了枕頭,不過三日,自然有驗。」婦人道:「請問先生,這四樁兒是怎的說?」賊瞎道:「好教娘子得知:用紗矇眼,使夫主見你一似西施嬌艷;用艾塞心,使他心愛到你;用針釘手,隨你怎的不是,使他再不敢動手打你;用膠粘足者,使他再不往那裡胡行。」{\meipi{事事俱打到婦人心坎上,賊瞎狡甚。}}婦人聽言,滿心歡喜。當下備了香燭紙馬,替婦人燒了紙。到次日,使劉婆送了符水鎭物與婦人,如法安頓停當,將符燒灰,頓下好茶,待的西門慶家來,婦人叫春梅遞茶與他吃。到晚夕,與他共枕同床,過了一日兩,兩日三,似水如魚,歡會異常。看觀聽說:但凡大小人家,師尼僧道,乳母牙婆,切記休招惹他,背地什麼事不幹出來?古人有四句格言說得好:

\begin{myquote}
堂前切莫走三婆,後門常鎖莫通和。\\院內有井防小口,便是禍少福星多。
\end{myquote}

