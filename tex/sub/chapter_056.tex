\includepdf[pages={111,112},fitpaper=false]{tst.pdf}
\chapter*{第五十六回 西門慶捐金助朋友 常峙節得鈔傲妻兒}
\addcontentsline{toc}{chapter}{第五十六回 西門慶捐金助朋友 常峙節得鈔傲妻兒}
\markboth{{\titlename}卷之六}{第五十六回 西門慶捐金助朋友 常峙節得鈔傲妻兒}


詩曰:

\begin{myquote}
清河豪士天下奇,意氣相投山可移。\\濟人不惜千金諾,狂飲寧辭百夜期。\\雕盤綺食會衆客,吳歌趙舞香風吹。\\堂中亦有三千士,他日酬恩知是誰?
\end{myquote}

話說西門慶留下兩箇歌童,隨即打發苗家人回書禮物,又賞了些銀錢。苗實領書,磕頭謝了出門。後來不多些時,春燕死了,止春鴻一人,正是:

\begin{myquote}
千金散盡教歌舞,留與他人樂少年。
\end{myquote}

卻說常峙節自那日求了西門慶的事情,還不得到手,房主又日夜催逼。恰遇西門慶從東京回家,今日也接風,明日也接風,一連過了十來日,只不得箇會面。常言道,見面情難盡。一箇不見,卻告訴誰?每日央了應伯爵,只走到大官人門首問聲,說不在,就空回了。{\pangpi{一求人,便有此苦。}}回家又被渾家埋怨道:「你也是男子漢大丈夫,房子沒間住,吃這般懊惱氣。你平日只認的西門大官人,今日求些賙濟,也做了瓶落水。」說的常峙節有口無言,呆瞪瞪不敢做聲。{\meipi{貧賤與富貴交,往往有虛名無實一惠,數口掃盡。}}到了明日,早起身尋了應伯爵,來到一箇酒店內,便請伯爵吃三盃。{\pangpi{亦所不免。}}伯爵道:「這卻不當生受。」常峙節拉了坐下,量酒打上酒來,擺下一盤燻肉、一盤鮮魚。酒過兩巡,常峙節道:「小弟向求哥和西門大官人說的事情,這幾日通不能會面,房子又催逼的緊,昨晚被房下聒絮了一夜,耐不的。五更抽身,專求哥趁着大官人還沒出門時,慢慢的候他。{\pangpi{苦語。}}不知哥意下如何?」應伯爵道:「受人之托,必當終人之事。我今日好歹要大官人助你些就是了。」兩箇又吃過幾盃,應伯爵便推早酒不吃了。常峙節又勸一盃,算還酒錢,一同出門,徑奔西門慶家裡來。

那時,正是新秋時候,金風薦爽。西門慶連醉了幾日,覺精神減了幾分。正遇周內相請酒,便推事故不去,自在花園藏春塢,和吳月娘、孟玉樓、潘金蓮、李瓶兒五箇尋花問柳頑耍,好不快活。{\meipi{窮鬼已自可憐,而複寫一段富貴飽暖受用,與之相形,惡甚。}}常峙節和應伯爵來到廳上,問知大官人在屋裡,滿心歡喜。坐着等了好半日,卻不見出來。只見門外書童和畫童兩箇擡着一隻箱子,都是綾絹衣服,氣吁吁走進門來,亂嚷道:「等了這半日,還只得一半。」就廳上歇下。應伯爵便問:「你爹在那裡?」書童道:「爹在園裡頑耍哩。」伯爵道:「勞你說聲。」兩箇依舊擡着進去了。不一時,書童出來道:「爹請應二爹、常二叔少待,便來也。」兩人又等了一回,西門慶纔走出來。二人作了揖,便請坐的。伯爵道:「連日哥吃酒忙,不得些空,今日卻怎的在家裡?」西門慶道:「自從那日別後,整日被人家請去飲酒,醉的了不的,通沒些精神。今日又有人請酒,我只推有事不去。」伯爵道:「方纔那一箱衣服,是那裡擡來的?」西門慶道:「目下交了秋,大家都要添些秋衣。方纔一箱,是你大嫂子的。還做不完,纔勾一半哩。」常峙節伸着舌道:「六房嫂子,就六箱了,好不費事!小戶人家,一疋布也難得。哥果是財主哩。」{\meipi{孟子曰:「勿視其巍巍然」,正欲開豁此等眼孔。}}西門慶和應伯爵都笑起來。伯爵道:「這兩日,杭州貨船怎的還不見到?不知買賣貨物何如。這幾日,不知李三、黃四的銀子,曾在府裡頭開了些送來與哥麼?」西門慶道:「貨船不知在那裡耽擱着,書也沒稍封寄來,好生放不下。李三、黃四的,又說在出月纔關。」應伯爵捱到身邊坐下,乘閑便說:{\meipi{開口告人之難如此。}}「常二哥那一日在哥席上求的事情,一向哥又沒的空,不曾說的。常二哥被房主催逼慌了,每日被嫂子埋怨,二哥只麻作一團,沒箇理會。如今又是秋涼了,身上皮襖兒又當在典鋪裡。哥若有好心,常言道:救人須救急時無,省的他嫂子日夜在屋裡絮絮叨叨。況且尋的房子住着,也是哥的體面。因此,常二哥央小弟特地來求哥,早些賙濟他罷。」西門慶道:「我曾許下他來,因為東京去,費的銀子多了,本待等韓夥計到家,和他理會。如今又恁的要緊?」伯爵道:「不是常二哥要緊,當不的他嫂子聒絮,只得求哥早些便好。」

西門慶躊躇了半晌道:「既這等,也不難。且問你,要多少房子纔勾住?」伯爵道:「他兩口兒,也得一間門面、一間客坐、一間床房、一間廚竈,四間房子,是少不得的。論着價銀,也得三四箇多銀子。哥只早晚湊些,教他成就了這樁事罷。」西門慶道:「今日先把幾兩碎銀與他拏去,買件衣服,辦些家活,盤攪過來,待尋下房子,我自兌銀與你成交,可好麼?」兩箇一齊謝道:「難得哥好心。」西門慶便叫書童:「去對你大娘說,皮匣內一包碎銀取了出來。」書童應諾。不一時,取了一包銀子出來,遞與西門慶。西門慶對常峙節道:「這一包碎銀子,是那日東京太師府賞封剩下的十二兩,你拏去好襍用。」開啟與常峙節看,都是三五錢一塊的零碎紋銀。常峙節接過放在衣袖裡,就作揖謝了。西門慶道:「我這幾日不是要遲你的,你又沒曾尋的。只等你尋下,待我有銀,一起兌去便了。」常峙節又稱謝不迭。{\meipi{此一番稱頌不可少。}}三箇依舊坐下,伯爵便道:「多少古人輕財好施,到後來子孫高大門閭,把祖宗基業一發增的多了。慳吝的,積下許多金寶,後來子孫不好,連祖宗墳土也不保。可知天道好還哩!」西門慶道:「兀那東西,是好動不喜靜的,怎肯埋沒在一處!也是天生應人用的,一箇人堆積,就有一箇人缺少了。因此積下財寶,極有罪的。」{\meipi{不以施予為功,而反以積財為罪,雖不可為敗子藉口,然自是千古名言至理。西門慶始終用財,不出此意。}}正說着,只見書童托出飯來。三人吃畢,常峙節作謝起身,袖着銀子歡喜走到家來。剛剛進門,只見渾家鬧吵吵嚷將出來,罵道:「『梧桐葉落——滿身光棍的行貨子!』出去一日,把老婆餓在家裡,尚兀自千歡萬喜到家來,可不害羞哩!房子沒的住,受別人許多酸嘔氣,只教老婆耳朵裡受用。」那常二只是不開口,{\meipi{袖中有物,便覺舉止安祥。}}任老婆罵的完了,輕輕把袖裡銀子摸將出來,放在桌兒上,開啟瞧着道:「孔方兄,孔方兄!我瞧你光閃閃、响噹噹無價之寶,滿身通麻了,恨沒口水咽你下去。你早些來時,不受這淫婦幾場氣了。」{\pangpi{數語又是一錢神小贊。}}那婦人明明看見包裡十二三兩銀子一堆,喜的搶近前來,就想要在老公手裡奪去。{\pangpi{急情饞眼,摹寫殆盡。}}常二道:「你生世要罵漢子,見了銀子,就來親近哩。我明日把銀子買些衣服穿,自去別處過活,再不和你鬼混了。」那婦人陪着笑臉道:「我的哥!端的此是那裡來的這些銀子?」常二也不做聲。婦人又問道:「我的哥,難道你便怨了我?我也只是要你成家。今番有了銀子,和你商量停當,買房子安身卻不好?倒恁地喬張致!我做老婆的,不曾有失花兒,憑你怨我,也是枉了。」常二也不開口。那婦人只顧饒舌,又見常二不揪不採,自家也有幾分慚愧,禁不得掉下淚來。常二看了,嘆口氣道:「婦人家,不耕不織,把老公恁地發作!」那婦人一發掉下淚來。{\meipi{止此一物,其未得也,婦人怨之罵之而啞口不能對;其既得也,則冷譏熱訕,使之陪笑,陪笑不已,使之下淚。寫貧家一種有柴米而無恩愛夫妻情景,眞令人慾哭。}}兩箇人都閉着口,又沒箇人勸解,悶悶的坐着。{\pangpi{聲臭俱無處,偏能摹寫。}}常二尋思道:「婦人家也是難做。受了辛苦,埋怨人,也怪他不的。我今日有了銀子不採他,人就道我薄情。{\meipi{轉念方想到情義,更可悲。}}便大官人知道,也須斷我不是。」就對那婦人笑道:「我自耍你,誰怪你來!只你時常聒噪,我只得忍着出門去了,卻誰怨你來?我明白和你說:這銀子,原是早上耐你不的,特地請了應二哥在酒店裡吃了三盃,一同往大官人宅裡等候。恰好大官人正在家,沒曾去吃酒,虧了應二哥許多婉轉,纔得這些銀子到手。還許我尋下房子,兌銀與我成交哩!這十二兩,是先教我盤攪過日子的。」那婦人道:「原來正是大官人與你的,如今不要花費開了,尋件衣服過冬,省的耐冷。」常二道:「我正要和你商量,十二兩紋銀,買幾件衣服,辦幾件家活在家裡。等有了新房子,搬進去也好看些。只是感不盡大官人恁好情,{\meipi{西門慶施予借貸多矣,背地感恩止博此一語。}}後日搬了房子,也索請他坐坐是。」婦人道:「且到那時再作理會。」正是:

\begin{myquote}
惟有感恩並積恨,萬年千載不生塵。
\end{myquote}

常二與婦人說了一回,婦人道:「你吃飯來沒有?」常二道:「也是大官人屋裡吃來的。你沒曾吃飯,就拏銀子買了米來。」婦人道:「仔細拴着銀子,我等你就來。」{\meipi{寫窮則一團寒酸之氣逼人。}}常二取栲栳望街上買了米,栲栳上又放着一大塊羊肉,拏進門來。婦人迎門接住道:「這塊羊肉,又買他做甚?」常二笑道:「剛纔說了許多辛苦,不爭這一些羊肉,就牛也該宰幾箇請你。」婦人笑指着常二罵道:「狠心的賊!今日便懷恨在心,看你怎的奈何了我!」常二道:「只怕有一日,叫我一萬聲:『親哥,饒我小淫婦罷!』我也只不饒你哩。試試手段看!」{\meipi{纔數語,便近於戲,富貴易淫可想。}}那婦人聽說,笑的往井邊打水去了。當下婦人做了飯,切了一碗羊肉,擺在桌兒上,便叫:「哥,吃飯。」常二道:「我纔吃的飯,不要吃了。你餓的慌,自吃些罷。」那婦人便一箇自吃了。收了家活,打發常二去買衣服。常二袖着銀子,一直奔到大街上來。看了幾家,都不中意。只買了一件青杭絹女襖、一條綠紬裙子、一件月白雲紬衫兒、一件紅綾襖子、一件白紬裙兒,共五件。自家也對身買了一件鵝黃綾襖子、一件丁香色紬直身,又買幾件布草衣服。共用去六兩五錢銀子。打做一包,背到家中,叫婦人開啟看看。婦人看了,便問:「多少銀子買的?」常二道:「六兩五錢銀子。」婦人道:「雖沒便宜,卻値這些銀子。」一面收拾箱籠放好,明日去買家活。當日婦人歡天喜地過了一日,埋怨的話都掉在東洋大海里去了,不在話下。

再表應伯爵和西門慶兩箇,自打發常峙節出門,依舊在廳上坐的。西門慶因說起:「我雖是箇武職,恁的一箇門面,京城內外也交結許多官員,近日又拜在太師門下,那些通問的書柬,流水也似往來,我又不得細工夫料理。我一心要尋箇先生在屋裡,教他替寫寫,省些力氣也好,只沒箇有才學的人。你看有時,便對我說。」伯爵道:「哥,你若要別樣卻有,要這箇倒難。第一要才學,第二就要人品了。又要好相處,沒些說是說非,翻唇弄舌,這就好了。若是平平才學,又做慣搗鬼的,怎用的他!小弟只有一箇朋友,他現是本州秀才,應舉過幾次,只不得中。他胸中才學,果然班馬之上,就是人品,也孔孟之流。他和小弟,通家兄弟,極有情分。曾記他十年前,應舉兩道策,那一科試官極口贊好。不想又有一箇賽過他的,便不中了。後來連走了幾科,禁不的髮白髩斑。如今雖是飄零書劍,家裡也還有一百畝田,三四帶房子住着。」西門慶道:「他家幾口兒也勾用了,卻怎的肯來人家坐館?」應伯爵道:「當先有的田房,都被那些大戶人家買去了,如今只剩得雙手皮哩。」{\meipi{此後薦水秀才數段,皆以戲謔取笑而已。}}西門慶道:「原來是賣過的田,算什麼數!」伯爵道:「這果是算不的數了。只他一箇渾家,年紀只好二十左右,生的十分美貌,又有兩箇孩子,纔三四歲。」西門慶道:「他家有了美貌渾家,那肯出來?」伯爵道:「喜的是兩年前,渾家專要偷漢,跟了箇人,走上東京去了,兩箇孩子又出痘死了,如今只存他一口,定然肯出來。」西門慶笑道:「恁他說的他好,都是鬼混。你且說他姓甚麼?」伯爵道:「姓水,他才學果然無比,哥若用他時,管情書柬詩詞,一件件增上哥的光輝。人看了時,都道西門大官人恁地才學哩!」西門慶道:「你都是吊慌,我卻不信。你記的他些書柬兒,念來我聽,看好時,我就請他來家,撥間房子住下。只一口兒,也好看承的。」伯爵道:「曾記得他稍書來,要我替他尋箇主兒。這一封書,畧記的幾句,念與哥聽:

\begin{myquote}
{\markfont\small〔黃鶯兒〕}書寄應哥前,別來思,不待言。滿門兒托賴都康健。舍字在邊,傍立着官,有時已定求方便。羨如椽,往來言疏,落筆起雲烟。
\end{myquote}

西門慶聽畢,便大笑將起來,道:「他既要你替他尋箇好主子,卻怎的不稍書來,到寫一隻曲兒來?又做的不好。可知道他才學荒疎,人品散蕩哩。」伯爵道:「這到不要作準他。只為他與我是三世之交,自小同上學堂。先生曾道:『應家學生子和水學生子一般的聰明伶俐,後來已定長進。」落後做文字,一樣同做,再沒些妒忌,極好兄弟。故此不拘形跡,{\meipi{人有欲譽妻妾美而難於發言者,乃譽姨之美與妻相似,此正師其意而反用之。}}便隨意寫箇曲兒。況且那隻曲兒,也倒做的有趣。」西門慶道:「別的罷了,只第五句是甚麼說話?」白爵道:「哥不知道,這正是拆白道字,尤人所難。『舍』字在邊,旁立着『官』字,不是箇『館』字?若有館時,千萬要舉薦。因此說『有時定要求方便』。哥你看他詞裡,有一箇字兒是閑話麼?只這幾句,穩穩把心窩裡事都寫在紙上,可不好哩!」西門慶被伯爵說的他恁地好處,到沒的說了。只得對伯爵道:「到不知他人品如何?」伯爵道:」他人品比才學又高。前年,他在一箇李侍郎府裡坐館,那李家有幾十箇丫頭,一箇箇都是美貌俊俏的。又有幾箇伏侍的小厮,也一箇箇都標緻龍陽的。那水秀才連住了四五年,再不起一些邪念。後來不想被幾箇壞事的丫頭小厮,見他似聖人一般,反去日夜括他。那水秀才又極好慈悲的人,便口軟勾搭上了。{\meipi{今人實有類此而大言不慚者。}}因此,被主人逐出門來,鬨動街坊,人人都說他無行。其實,水秀才原是坐懷不亂的。若哥請他來家,憑你許多丫頭、小厮,同眠同宿,你看水秀才亂麼?再不亂的。」西門慶笑罵道:「你這狗才,單管說慌吊皮鬼混人。前月敝同僚夏龍溪請的先生倪桂巖,曾說他有箇姓溫的秀才。且待他來時再處。」正是:

\begin{myquote}
將軍不好武,稚子總能文。
\end{myquote}


