\includepdf[pages={153,154},fitpaper=false]{tst.pdf}
\chapter*{第七十七回 西門慶踏雪訪愛月 賁四嫂帶水戰情郎}
\addcontentsline{toc}{chapter}{第七十七回 西門慶踏雪訪愛月 賁四嫂帶水戰情郎}
\markboth{{\titlename}卷之八}{第七十七回 西門慶踏雪訪愛月 賁四嫂帶水戰情郎}


詞曰:

\begin{myquote} 
梅共雪,歲暮鬬新粧。月底素華同弄色,風前輕片半含香,不比柳花狂。雙雀影,堪比雪衣娘。六出光中曾結伴,百花頭上解尋芳,爭似兩鴛鴦。

\raggedleft{——右調《望江南》\rightquadmargin}
\end{myquote} 

話說溫秀才求見西門慶不得,自知慚愧,隨移家小,搬過舊家去了。西門慶收拾書院,做了客坐,不在話下。

一日,尚舉人來拜辭,上京會試,問西門慶借皮箱氊衫。西門慶陪坐待茶,因說起喬大戶、雲理守:「兩位舍親,一受義官,一受祖職,見任管事,欲求兩篇軸文奉賀。不知老翁可有相知否?借重一言,學生具幣禮相求。」尚舉人笑道:「老翁何用禮,學生敝同窻聶兩湖,見在武庫肄業,與小兒為師,本領雜作極富。學生就與他說,老翁差盛使持軸來就是了。」西門慶連忙致謝。茶畢起身。西門慶隨即封了兩方手帕、五錢白金,差琴童送軸子並氊衫、皮箱,到尚舉人處放下。那消兩日,寫成軸文差人送來。西門慶掛在壁上,但見金字輝煌,文不加點,心中大喜。只見應伯爵來問:「喬大戶與雲二哥的事,幾時舉行?軸文做了不曾?溫老先兒怎的連日不見?」西門慶道:「又題什麼溫老先兒,通是個狗類之人!」如此這般,告訴一遍。伯爵道:「哥,我說此人言過其實,虛浮之甚,早時你有後眼,不然,教他調壞了咱家小兒每了。」又問他:「二公賀軸,何人寫了?」西門慶道:「昨日尚小塘來拜我,說他朋友聶兩湖善於詞藻,央求聶兩湖作了。文章已寫了來,你瞧!」於是引伯爵到廳上觀看,喝采不已,又說道:「人情都全了,哥,你早送與人家,好預備。」西門慶道:「明日好日期,早差人送去。」

正說着,忽報:「夏老爹兒來拜辭,說初六日起身去。小的回爹不在家。他說教對何老爹那裡說聲,差人那邊看守去。」西門慶看見貼兒上寫着「寅家晚生夏承恩頓首拜,謝辭」。西門慶道:「連尚舉人搭他家,就是兩分程儀香絹。」分付琴童:「連忙買了,教你姐夫封了,寫貼子送去。」正在書房中留伯爵吃飯,忽見平安兒慌慌張張拏進三個貼兒來報:「參議汪老爹、兵備雷老爹、郎中安老爹來拜。」西門慶看貼兒:「汪伯彥、雷啟元、安忱拜。」連忙穿衣繫帶。伯爵道:「哥,你有事,我去罷。」西門慶道:「我明日會你哩。」一面整衣出迎。三官員皆相讓而入。進入大廳,叙禮,道及向日叨擾之事。少頃茶罷,坐話間,安郎中便道:「雷東谷、汪少華並學生,又來干瀆:有浙江本府趙大尹,新陞大理寺正,學生三人借尊府奉請,已發柬,定初九日。主家共五席。戲子學生那裡叫來。未知肯允諾否?」西門慶道:「老先生分付,學生掃門拱候。」{\meipi{照出。}}安郎中令吏取分資三兩遞上,西門慶令左右收了,相送出門。雷東谷向西門慶道:「前日錢雲野書到,說那孫文相乃是貴夥計,學生已並他除開了,曾來相告不曾?」西門慶道:「正是,多承老先生費心,容當叩拜。」雷兵備道:「你我相愛間,何為多數。」言畢,相揖上轎而去。

原來潘金蓮自從當家管理銀錢,另定了一把新等子。每日小厮買進菜蔬來,拏到跟前與他瞧過,方數錢與他。{\meipi{雖算小,卻是當家人要着。}}他又不數,只教春梅數錢,提等子。小厮被春鴻罵的狗血淋頭,行動就說落,教西門慶打。以此衆小厮互相抱怨,都說在三娘手兒裡使錢好。

卻說次日,西門慶衙門中散了,對何千戶說:「夏龍溪家小已是起身去了,長官可曾委人那裡看守門戶去?」何千戶道:「正是,昨日那邊着人來說,學生已令小价去了。」西門慶道:「今日同長官那邊看看去。」於是出衙門,並馬到了夏家宅內。家小已是去盡了,伴當在門首伺候。兩位官府下馬,進到廳上。西門慶引着何千戶前後觀看了,又到前邊花亭上,見一片空地,無甚花草。西門慶道:「長官到明日還收拾個耍子所在,栽些花柳,把這座亭子修理修理。」何千戶道:「這個已定。學生開春從新修整修整,蓋三間捲棚,早晚請長官來消閑散悶。」看了一回,分付家人收拾打掃,關閉門戶。不日寫書往東京回老公公話,趕年裡搬取家眷。西門慶作別回家。何千戶還歸衙門去了。到次日纔搬行李來住,不在言表。

西門慶剛到家下馬,見何九買了一疋尺頭、四樣下飯、一罈酒來謝。又是劉內相差人送了一食盒蠟燭,二十張桌圍,八十股官香,一盒沉速料香,一罈自造內酒,一口鮮豬。西門慶進門,劉公公家人就磕頭,說道:「家公多多上履,這些微禮,與老爹賞人。」西門慶道:「前日空過老公公,怎又送這厚禮來?」便令左右:「快收了,請管家等等兒。」少頃,畫童兒拏出一鍾茶來,打發吃了。西門慶封了五錢銀子賞錢,拏回貼,打發去了。一面請何九進去。西門慶見何九,一把手扯在廳上來。何九連忙倒身磕下頭去,道:「多蒙老爹天心,超生小人兄弟,感恩不淺。」請西門慶受禮,西門慶不肯受磕頭,拉起來,說道:「老九,你我舊人,快休如此。」就讓他坐。何九說道:「小人微末之人,豈敢僭坐。」只站立在旁邊。西門慶也站着,陪吃了一盞茶,說道:「老九,你如何又費心送禮來?我斷然不受,若有甚麼人欺負你,只顧來說,我替你出氣。倘縣中派你甚差事,我拏貼兒與你李老爹說。」何九道:「蒙老爹恩典,小人知道。小人如今也老了,差事已告與小兒何欽頂替了。」西門慶道:「也罷,也罷,你清閑些好。」又說道:「既你不肯,我把這酒禮收了,那尺頭你還拏去,我也不留你坐了。」那何九千恩萬謝,拜辭去了。

西門慶就坐在廳上,看看打點禮物菓盒、花紅羊酒、軸文並各人分資。先差玳安送往喬大戶家去,後叫王經送往雲理守家去。玳安回來,喬家與了五錢銀子。王經到雲理守家,管待了茶食,與了一疋眞青大布、一雙琴鞋,回「門下辱愛生」雙貼兒:「多上覆老爹,改日奉請。」西門慶滿心歡喜,到後邊月娘房中擺飯吃,因向月娘說:「賁四去了,吳二舅在獅子街賣貨,我今日倒閑,往那裡看看去。」月娘道:「你去不是,若是要酒菜兒,蚤使小厮來家說。」西門慶道:「我知道。」一面分付備馬,就戴着氊忠靖巾,貂鼠暖耳,綠絨補子氅褶,粉底皁靴,琴童、玳安跟隨,徑往獅子街來。到房子內,吳二舅與來昭正掛着花拷拷兒,發買紬絹、絨線、絲綿,擠一鋪子人做買賣,打發不開。西門慶下馬,看了看,走到後邊暖房內坐下。吳二舅走來作揖,因說:「一日也攢銀二三十兩。」西門慶又分付來昭妻一丈青:「二舅每日茶飯休要誤了。」來昭妻道:「逐日伺候酒飯,不敢有誤。」

西門慶見天色陰晦,彤雲密佈,冷氣侵人,將有作雪的模樣。忽然想起要往鄭月兒家去,即令琴童:「騎馬家中取我的皮襖來,問你大娘,有酒菜兒稍一盒與你二舅吃。」琴童應諾。到家,不一時,取了貂鼠皮襖,並一盒酒菜來。西門慶陪二舅在房中吃了三盃,分付:「二舅,你晚夕在此上宿,慢慢再用。我家去罷。」於是帶上眼紗,騎馬,玳安、琴童跟隨,逕進構欄,往鄭愛月兒家來。轉過東街口,只見天上紛紛揚揚,飄起一天瑞雪來。但見:

\begin{myquote} 
漠漠嚴寒匝地,這雪兒下得正好。扯絮撏綿,裁成片片,大如拷拷。見林間竹筍茆茨,爭些被他壓倒。富豪俠卻言:消災障猶嫌少。圍向那紅爐獸炭,穿的是貂裘繡襖。手拈梅花,唱道是國家祥瑞,不念貧民些小。高臥有幽人,吟詠多詩草。
\end{myquote} 

西門慶踏着那亂瓊碎玉,進入構欄,到於鄭愛月兒家門首下馬。只見丫鬟飛報進來,說:「老爹來了。」鄭媽媽看見,出來,至於中堂見禮,說道:「前日多謝老爹重禮,姐兒又在宅內打攪,又教他大娘、三娘賞他花翠汗巾。」西門慶道:「那日空了他來。」一面坐下。西門慶令玳安:「把馬牽進來,後邊院落安放。」老媽道:「請爹後邊明間坐罷。月姐纔起來梳頭,只說老爹昨日來,到伺候了一日,今日他心中有些不快,起來的遲些。」這西門慶一面進入他後邊明間內,但見綠牕半啟、氊幙低張,地平上黃銅大盆生着炭火。西門慶坐在正面椅上。先是鄭愛香兒出來相見了,遞了茶。然後愛月兒纔出來,頭挽一窩絲杭州纘,翠梅花鈕兒,金趿釵梳,海獺臥兔兒。打扮的霧靄雲鬟,粉粧玉琢。笑嘻嘻向西門慶道了萬福,說道:「爹,我那一日來晚了。緊自前邊散的遲,到後邊,大娘又只顧不放俺每,留着吃飯,來家有三更天了。」西門慶笑道:「小油嘴兒,你倒和李桂姐兩個把應花子打的好響瓜兒。」鄭愛月兒道:「誰教他怪叨嘮,在酒席上屎口兒傷俺每來!那一日祝麻子也醉了,哄我,要送俺每來。我便說:『沒爹這裡燈籠送俺每,蔣胖子弔在陰溝裡——缺臭了你了。』」西門慶道:「我昨日聽見洪四兒說,祝麻子又會着王三官兒,大街上請了榮嬌兒。」鄭月兒道:「只在榮嬌兒家歇了一夜,燒了一炷香,不去了。如今還在秦玉芝兒走着哩。」說了一回話,道:「爹,只怕你冷,往房裡坐。」這西門慶到於房中,脫去貂裘,和粉頭圍爐共坐,房中香氣襲人。須臾,丫頭拏了三甌兒黃芽韭菜肉包、一寸大的水角兒來。姊妹二人陪西門慶,每人吃了一甌兒。愛月兒又撥上半甌兒,添與西門慶。西門慶道:「我勾了,纔吃了兩個點心來了。心裡要來你這裡走走,不想恰好天氣又落下雪來了。」愛月兒道:「爹前日不會下我?我昨日等了一日不見爹,不想爹今日纔來。」西門慶道:「昨日家中有兩位士夫來望,亂着就不曾來得。」愛月兒道:「我要問爹,有貂鼠買個兒與我,我要做了圍脖兒戴。」西門慶道:「不打緊,昨日韓夥計打遼東來,送了我幾個好貂鼠。你娘們都沒圍脖兒,到明日一總做了,送一個來與你。」愛香兒道:「爹只認的月姐,就不送與我一個兒。」西門慶道:「你姊妹兩個一家一個。」{\meipi{問着的就送,才算子弟,知今人不掛炭,都饒稱摧漫矣。}}於是愛香、愛月兒連忙起身道了萬福。西門慶分付:「休見了桂姐、銀姐說。」鄭月兒道:「我知道。」因說:「前日李桂姐見吳銀兒在那裡過夜,問我他幾時來的,我沒瞞他,教我說:『昨日請周爺,俺每四個都在這裡唱了一日。爹說有王三官兒在這裡,不好請你的。今日是親朋會中人吃酒,纔請你來唱。』他一聲兒也沒言語。」西門慶道:「你這個回的他好。前日李銘,我也不要他唱來,再三央及你應二爹來說。落後你三娘生日,桂姐買了一分禮來,再一與我陪不是。你娘們說着,我不理他。昨日我竟留下銀姐,使他知道。」愛月兒道:「不知三娘生日,我失誤了人情。」西門慶道:「明日你雲老爹擺酒,你再和銀姐來唱一日。」愛月兒道:「爹分付,我去。」說了回話,粉頭取出三十二扇象牙牌來,和西門慶在炕氊條上抹牌頑耍。愛香兒也坐在旁邊同抹。三人抹了回牌,須臾,擺上酒來,愛香與愛月兒一邊一個捧酒,不免箏排雁柱,款跨鮫綃,姊妹兩個彈唱。唱了一套,姐妹兩個又拏上骰盆兒來,和西門慶搶紅頑笑。盃來盞去,各添春色。西門慶忽看見鄭愛月兒房中,床旁側錦屏風上,掛着一軸「愛月美人圖」,題詩一首:

\begin{myquote} 
有美人兮迥出羣,輕風斜拂石榴裙。\\花開金谷春三月,月轉花陰夜十分。\\玉雪精神聯仲琰,瓊林才貌過文君。\\少年情思應須慕,莫使無心托白雲。{\meipi{王三泉此詩,較蔡狀元尚通。}}

\raggedleft{三泉主人醉筆\rightquadmargin}
\end{myquote} 

西門慶看了,便問:「三泉主人是王三官兒的號?」慌的鄭愛月兒連忙摭說道:「這還是他舊時寫下的。他如今不號三泉了,號小軒了。{\meipi{今之號軒、亭、橋、泉者,熟讀此書者也。}}他告人說,學爹說:『我號四泉,他怎的號三泉?』他恐怕爹惱,因此改了號小軒。」一面走向前,取筆過來,把那「三」字就塗抹了。西門慶滿心歡喜,{\pangpi{歡喜是何主意?}}說道:「我並不知他改號一節。」粉頭道:「我聽見他對一個人說來,我纔曉的。說他去世的父親號逸軒,他故此改號小軒。」

說畢,鄭愛香兒往下邊去了,獨有愛月兒陪西門慶在房內。兩個並肩疊股,搶紅飲酒,因說起林太太來,怎的大量,好風月:「我在他家吃酒,那日王三官請我到後邊拜見。還是他主意,教三官拜認我做義父,教我受他禮,委托我指教他成人。」{\pangpi{好說。}}粉頭拍手大笑道:「還虧我指與爹這條路兒,到明日,連三官兒娘子不怕不屬了爹。」{\pangpi{可惡。}}西門慶道:「我到明日,我先燒與他一炷香。到正月裡,請他和三官娘子往我家看燈吃酒,看他去不去。」粉頭道:「爹,你還不知三官娘子生的怎樣標緻,就是個燈人兒也沒他那一段風流妖艷。今年十九歲兒,只在家中守寡,王三官兒通不着家。{\pangpi{可恨。}}爹,你肯用些工夫兒,不愁不是你的人。」兩個說話之間,相挨相湊。只見丫鬟又拏上許多細菓碟兒來,粉頭親手奉與西門慶下酒。又用舌頭噙鳳香蜜餅送入他口中,又用纖手解開西門慶褲帶,露出那話來,教他弄。那話猙獰跳腦,紫強光鮮,西門慶令他品之。這粉頭眞個低垂粉項,輕啟朱唇,半吞半吐,或進或出,嗚咂有聲,品弄了一回。靈犀已透,淫心似火,便欲交歡。粉頭便往後邊去了。西門慶出房更衣,見雪越下得甚緊。回到房中,丫鬟向前打發脫靴解帶,先上牙床。粉頭澡牝回來,掩上雙扉,共入鴛帳。正是:得多少動人春色嬌還媚,惹蝶芳心軟欲濃。有詩為證:

\begin{myquote} 
聚散無憑在夢中,起來殘燭映紗紅。\\鍾情自古多神合,誰道陽臺路不通。
\end{myquote} 

兩個雲雨歡娛,到一更時分起來。整衣理鬢,丫鬟復釃美酒,重整佳餚,又飲勾幾盃。問玳安:「有燈籠、傘沒有?」玳安道:「琴童家去取燈籠、傘來了。」這西門慶方纔作別,鴇子、粉頭相送出門,看着上馬。鄭月兒揚聲叫道:「爹若叫我,蚤些來說。」西門慶道:「我知道。」一面上馬,打着傘出院門,一路踏雪到家中。對着吳月娘,只說在獅子街和吳二舅飲酒,不在話下。

一宿晚景題過。到次日,卻是初八日,打聽何千戶行李,都搬過夏家房子內去了,西門慶送了四盒細茶食、五錢折帕賀儀過去。只見應伯爵驀地走來。西門慶見雪晴,風色甚冷,留他前邊書房中向火,叫小厮拏菜兒,留他吃粥,因說道:「昨日喬親家、雲二哥禮並折帕,都送去了。你的人情,我也替你封了二錢出上了。你不消與他罷,只等發柬請吃酒。」應伯爵舉手謝了,因問:「昨日安大人三位來做甚麼?那兩位是何人?」西門慶道:「那兩個,一個是雷兵備,一個是汪參議,都是浙江人。要在我這裡擺酒。明日請杭州趙霆知府,新陞京堂大理寺丞,是他每本府父母官,相處分上,又不可回他的。通身只三兩分資。」伯爵道:「大凡文職好細,三兩銀子勾做甚麼!哥少不得賠些兒。」西門慶道:「這雷兵備,就是問黃四小舅子孫文相的,昨日還對我題起開除他罪名哩。」伯爵道:「你說他不仔細,如今還記着,折準擺這席酒纔罷了。」{\meipi{肯准折的還是清廉官。}}

說話之間,伯爵叫:「應寶,你叫那個人來見你大爹。」西門慶便問:「是何人?」伯爵道:「一個小後生,倒也是舊人家出身。父母都沒了,自幼在王皇親宅內答應。已有了媳婦兒,因在庄子上和一般家人不和,出來了。如今閑着,做不的甚麼。他與應寶是朋友,央及應寶要投個人家。今早應寶對我說:『爹倒好舉薦與大爹宅內答應。』我便說:『不知你大爹用不用?』」因問應寶:「他叫甚麼名字?你叫他進來。」應寶道:「他姓來,叫來友兒。」只見那來友兒,扒在地上磕了個頭起來,簾外站立。伯爵道:「若論他這身材膂力盡有,掇輕負重卻去的。」因問:「你多少年紀了?」來友兒道:「小的二十歲了。」又問:「你媳婦沒子女?」那人道:「只光兩口兒。」應寶道:「不瞞爹說,他媳婦纔十九歲兒,廚竈針線,大小衣裳都會做。」西門慶見那人低頭並足,為人樸實,便道:「既是你應二爹來說,用心在我這裡答應。」分付:「揀個好日期,寫紙文書,兩口兒搬進來罷。」那來友兒磕了個頭。西門慶就叫琴童兒領到後邊,見月娘衆人磕頭去。月娘就把來旺兒原住的那一間房與他居住。伯爵坐了回,家去了。應寶同他寫了一紙投身文書,交與西門慶收了,改名來爵,不在話下。

卻說賁四娘子,自從他家長兒與了夏家,每日買東買西,只央及平安兒和來安、畫童兒。西門慶家中這些大官兒,常在他屋裡打平和兒吃酒。賁四娘子和氣,就定出菜兒來,或要茶水,應手而至。就是賁四一時鋪中歸來撞見,亦不見怪。以此今日他不在家,使着那個不替他動?玳安兒與平安兒,在他屋裡坐的更多。初九日,西門慶與安郎中、汪參議、雷兵備擺酒,請趙知府,俱不必細說。

那日蚤辰,來爵兩口兒就搬進來。他媳婦兒後邊見月娘衆人磕頭。月娘見他穿着紫紬襖,青布披襖,綠布裙子,生的五短身材,瓜子面皮兒,搽脂抹粉,纏的兩隻脚翹翹的,問起來,諸般針指都會做。取了他個名字,叫做惠元,與惠秀、惠祥一遞三日上竈,不題。

一日,門外楊姑娘沒了。安童兒來報䘮。西門慶整治了一張插桌,三牲湯飯,又封了五兩香儀。吳月娘、李嬌兒、孟玉樓、潘金蓮四頂轎子,都往北邊與他燒紙弔孝,琴童兒、棋童兒、來爵兒、來安兒四個,都跟轎子,不在家。

西門慶在對過段鋪子書房內,看着毛襖匠與月娘做貂鼠圍脖,先攢出一個圍脖兒,使玳安送與院中鄭月兒去,封了十兩銀子與他過節。鄭家管待酒饌,與了他三錢銀子。玳安走來,回西門慶話,說:「月姨多上覆,多謝了,前日空過了爹來。與了小的三錢銀子。」西門慶道:「你收了罷。」因問他:「賁四不在家,你頭裡從他屋裡出來做甚麼?」玳安道:「賁四娘子從他女孩兒嫁了,沒人使,常央及小的每替他買買甚麼兒。」西門慶道:「他既沒人使,你每替他勤勤兒也罷。」又悄悄向玳安道:「你慢慢和他說,如此這般,爹要來看你看兒,你心下如何?看他怎的說。他若肯了,你問他討個汗巾兒來與我。」玳安道:「小的知道了。」領了西門慶言語,應諾下去。西門慶就走到家中來。

只見王經向顧銀鋪內取了金赤虎,並四對金頭銀簪兒,交與西門慶。西門慶留下兩對在書房內,餘者袖進李瓶兒房內,與了如意兒那赤虎,又是一對簪兒。把那一對簪兒就與了迎春。二人接了,連忙磕頭。西門慶就令迎春取飯去。須臾,拏飯來吃了,出來又到書房內坐下。只見玳安慢慢走到跟前,見王經在旁,不言語。西門慶使王經後邊取茶去。那玳安方說:「小的將爹言語對他說了,他笑了。約會晚上些伺候,等爹進去。叫小的拏了這汗巾兒來。」西門慶見紅綿紙兒,包着一方紅綾織錦廻紋汗巾兒,聞了聞噴鼻香,滿心歡喜,連忙袖了。只見王經拏茶來,吃了,又走過對門,看匠人做生活去。

忽報:「花大舅來了。」西門慶道:「請過來這邊坐。」花子繇走到書房暖閣兒裡,作揖坐下。致謝外日相擾。叙話間,畫童兒拏過茶來吃了。花子繇道:「門外一個客人,有五百包無錫米,凍了河,緊等要賣了回家去。我想着姐夫,倒好買下等價錢。」西門慶道:「我平白要他做甚麼?凍河還沒人要,到開河船來了,越發價錢跌了。如今家中也沒銀子。」即分付玳安:「收拾放桌兒,家中說,看菜兒來。」一面使畫童兒:「請你應二爹來,陪你花爹坐。」

不一時,伯爵來到。三人共在一處,圍爐飲酒。又叫烙了兩炷餅吃,良久,只見吳道官徒弟應春,送節禮疏誥來。西門慶請來同坐吃酒。就攬李瓶兒百日經,與他銀子去。吃至日落時分,花子繇和應春二人先起身去了。次後甘夥計收了鋪子,又請來坐,與伯爵擲骰猜枚談話,不覺到掌燈已後。吳月娘衆人轎子到了,來安走來回話。伯爵道:「嫂子們今日都往那裡去來?」西門慶道:「楊姑娘沒了,今日三日念經,我這裡備了張祭卓,又封了香儀兒,都去弔問。」伯爵道:「他老人家也高壽了。」西門慶道:「敢也有七十五六。男花女花都沒有,只靠侄兒那裡養活,材兒也是我替他備下這幾年了。」伯爵道:「好好,老人家有了黃金入櫃,就是一場事了,哥的大陰騭。」說畢,酒過數巡,伯爵與甘夥計作辭去了。西門慶就起身走過來,分付後生王顯:「仔細火燭。」王顯道:「小的知道。」看着把門關上了。這西門慶見沒人,兩三步就走入賁四家來。只見賁四娘子兒在門首獨自站立已久,見對門關的門響,西門慶從黑影中走至跟前。這婦人連忙把封門一開,西門慶鑽入裡面。婦人還扯上封門,說道:「爹請裡邊紙門內坐罷。」原來裡間槅扇廂着後半間,紙門內又有個小炕兒,籠着旺旺的火。桌上點着燈,兩邊護炕糊的雪白。婦人勒着翠藍銷金箍兒,上穿紫紬襖,青綃絲披襖,玉色綃裙子,向前與西門慶道了萬福,連忙遞了一盞茶與西門慶吃,因悄悄說:「只怕隔壁韓嫂兒知道。」西門慶道:「不妨事。黑影子裡他那裡曉的。」{\meipi{有本者如是。}}於是不繇分說,把婦人摟到懷中就親嘴。拉過枕頭來,解衣按在炕沿子上,扛起腿來就聳。那話上已束着托子,剛插入牝中,就拽了幾拽,婦人下邊淫水直流,把一條藍布褲子都濕了。西門慶拽出那話來,向順袋內取出包兒顫聲嬌來,蘸了些在龜頭上,攮進去,方纔澀住淫津,肆行抽拽。婦人雙手扳着西門慶肩膊,兩廂迎湊,在下揚聲顫語,呻吟不絕。這西門慶乘着酒興,架起兩腿在胳膊上,只顧僅沒其稜,銳進長驅,肆行搧磞,何止二三百度。須臾,弄的婦人雲髻鬅鬆,舌尖冰冷,口不能言。西門慶則氣喘吁吁,靈龜暢美,一泄如注。良久,拽出那話來,淫水隨出,用帕搽之。兩個整衣繫帶,復理殘粧。西門慶向袖中掏出五六兩一包碎銀子,又是兩對金頭簪兒,遞與婦人節間買花翠帶。婦人拜謝了,悄悄打發出來。那邊玳安在鋪子裡,專心只聽這邊門環兒響,便開大門,放西門慶進來。自知更無一人曉的。後次朝來暮往,也入港一二次。正是:若要人不知,除非己莫為。不想被韓嫂兒冷眼睃見,傳的後邊金蓮知道了。這金蓮亦不說破他。

一日,臘月十五日,喬大戶家請吃酒。西門慶會同應伯爵、吳大舅一齊起身。那日有許多親朋看戲飲酒,至二更方散。第二日,每家一張卓面,俱不必細說。

單表崔本治了二千兩湖州紬絹貨物,臘月初旬起身,顧船裝載,趕至臨清馬頭。教後生榮海看守貨物,便顧頭口來家,取車稅銀兩,到門首下頭口。琴童道:「崔大哥來了,請廳上坐。爹在對門房子裡,等我請去。」一面走到對門,不見西門慶,因問平安兒,平安兒道:「爹敢進後邊去了。」這琴童走到上房問月娘,月娘道:「見鬼的,你爹從蚤辰出去,再幾時進來?」又到各房裡,並花園、書房都瞧遍了,沒有。琴童在大門首揚聲道:「省恐殺人,不知爹往那裡去了,白尋不着!大白日裡把爹來不見了。崔大哥來了這一日,只顧教他坐着。」那玳安分明知道,只不做聲。{\meipi{絕不說出在那裡,妙甚。}}不想西門慶忽從前邊進來,把衆人唬了一驚。原來西門慶在賁四屋裡入港,纔出來。那平安打發西門慶進去了,望着琴童兒吐舌頭,都替他捏兩把汗道:「管情崔大哥去了,有幾下子打。」不想西門慶走到廳上,崔本見了,磕頭畢,交了書帳,說:「船到馬頭,少車稅銀兩。我從臘月初一日起身,在揚州與他兩個分路。

他每往杭州去了,俺每都到苗青家住了兩日。」因說:「苗青替老爹使了十兩銀子,擡了揚州衛一個千戶家女子,十六歲了,名喚楚雲。說不盡生的花如臉,玉如肌,星如眼,月如眉,腰如柳,襪如鉤,兩隻脚兒,恰剛三寸。端的有沉魚落雁之容,閉月羞花之貌。腹中有三千小曲,八百大麯。苗青如此還養在家,替他打粧奩,治衣服。待開春,韓夥計、保官兒船上帶來,伏侍老爹,消愁解悶。」西門慶聽了,滿心歡喜,說道:「你船上稍了來也罷。又費煩他治甚衣服,打甚粧奩,愁我家沒有?」於是恨不的騰雲展翅,飛上揚州,搬取嬌姿,賞心樂事。正是:鹿分鄭相應難辨,蝶化莊周未可知。有詩為證:

\begin{myquote} 
聞道揚州一楚雲,偶憑青鳥語來眞。\\不知好物都離隔,試把梅花問主人。
\end{myquote} 

西門慶陪崔本吃了飯,兌了五十兩銀子做車稅錢,又寫書與錢主事,煩他青目。崔本言訖,作辭,往喬大戶家回話去了。平安見西門慶不尋琴童兒,都說:「我兒,你不知有多少造化。爹今日不知有甚事喜歡,若不是,綁着鬼有幾下打。」琴童笑道:「只你知爹性兒。」

比及起了貨,來到獅子街卸下,就是下旬時分。西門慶正在家打發送節禮,忽見荊都監差人拏貼兒來,問:「宋大巡題本已上京數日,{\meipi{絕妙書劄。}}未知旨意下來不曾?伏惟老翁差人察院衙門一打聽為妙。」西門慶即差答應節級,拏了五錢銀子,往巡按公衙打聽。果然昨日東京邸報下來,寫抄得一紙,全報來與西門慶觀看。上面寫着:

\begin{myquote}[\markfont]
山東巡按監察御史宋喬年一本:循例舉劾地方文武官員,以勵人心,以隆聖治事。竊惟吏以撫民,武以御亂,所以保障地方,以司民命者也。苟非其人,則處置乖方,民受其害,國何賴焉!臣奉命按臨山東等處,吏政民瘼,監司守禦,無不留心諮訪。覆命按撫大臣,詳加鑑別,各官賢否,頗得其實。茲當差滿之期,敢不一一陳之。訪得山東左布政陳四箴操履忠貞,撫民有方;廉使趙訥,綱紀肅清,士民服習;兵備副使雷啟元,軍民咸服其恩威,僚幕悉推其練達;濟南府知府張叔夜,經濟可觀,才堪司牧;東平府知府胡師文,居任清愼,視民如傷。此數臣者,皆當薦獎而優擢者也。又訪得左參議馮廷鵠,傴僂之形,桑榆之景,形若木偶,尚肆貪婪;東昌府知府徐松,縱父妾而通賄,譭謗騰於公堂,慕羨餘而誅求,詈言遍於間裡。此二臣者,所當亟賜置斥者也。再訪得左軍院僉書守備周秀,器宇恢弘,操持老練,軍心允服,賊盜潛消;濟州兵馬都監荊忠,年力精強,才猶練達,冠武科而稱為儒將,勝算可以臨戎,號令而極其嚴明,長策卒能禦侮。此二臣者,所當亟賜遷擢者也。清河縣千戶吳鎧,以練達之才,得衛守之法,驅兵以搗中堅,靡攻不克;儲食以資糧餉,無人不飽。推心置腹,人思效命。實一方之保障,為國家之屏藩。宜特加超擢,鼓舞臣寮。陛下如以臣言可採,舉而行之,庶幾官爵不濫而人思奮,守牧得人而聖治有賴矣。等因。奉欽依:該部知道。

續該吏、兵二部題前事:看得御史宋喬年所奏內,劾舉地方文武官員,無非體國之忠,出於公論,詢訪事實,以裨聖治之事。優乞聖明俯賜施行,天下幸甚,生民幸甚。奉欽依:擬行。
\end{myquote} 

西門慶一見,滿心歡喜。拏着邸報,走到後邊,對月娘說:「宋道長本下來了。已是保舉你哥陞指揮僉事,見任管屯。周守備與荊大人都有獎勵,轉副參、統制之任。如今快使小厮請他來,對他說聲。」月娘道:「你使人請去,我交丫鬟看下酒菜兒。我愁他這一上任,也要銀子使。」西門慶道:「不打緊,我借與他幾兩銀子也罷了。」不一時,請得吳大舅到了。西門慶送那題奏旨意與他瞧。吳大舅連忙拜謝西門慶與月娘,說道:「多累姐夫、姐姐扶持,恩當重報,不敢有忘。」西門慶道:「大舅,你若上任擺酒沒銀子,我這裡兌些去使。」那大舅又作揖謝了。於是就在月娘房中,安排上酒來吃酒。月娘也在旁邊陪坐。西門慶即令陳敬濟把全抄寫了一本,與大舅拏着。即差玳安拏貼送邸報往荊都監、周守禦兩家報喜去。正是:

\begin{myquote}
勸君不費鐫研石,路上行人口似碑。
\end{myquote}

