\includepdf[pages={49,50},fitpaper=false]{tst.pdf}
\chapter*{第二十五回 吳月娘春晝鞦韆 來旺兒醉中謗訕}
\addcontentsline{toc}{chapter}{第二十五回 吳月娘春晝鞦韆 來旺兒醉中謗訕}
\markboth{{\titlename}卷之三}{第二十五回 吳月娘春晝鞦韆 來旺兒醉中謗訕}


詞曰:

\begin{myquote} 
蹴罷鞦韆,起來整頓纖纖手。露濃花瘦,薄汗輕衣透。見客入來,襪剗金釵溜。和羞走,倚門回首,卻把青梅嗅。

\raggedleft{——右調《點絳唇》\rightquadmargin}
\end{myquote} 

話說燈節已過,又早清明將至。西門慶有應伯爵早來邀請,說孫寡嘴作東,邀了郊外耍子去了。先是吳月娘花園中紮了一架鞦韆。這日見西門慶不在家,閑中率衆姊妹遊戲,以消春困。

先是月娘與孟玉樓打了一回下來,教李嬌兒和潘金蓮打。李嬌兒辭說身體沉重,打不的,卻教李瓶兒和金蓮打。打了一回,玉樓便叫:「六姐過來,我和你兩箇打箇立秋千。」分咐:「休要笑。」當下兩箇玉手挽定彩繩,將身立於畫板之上。月娘卻教蕙蓮、春梅兩箇相送。正是:

\begin{myquote}
紅粉面對紅粉面,玉酥肩並玉酥肩。\\兩雙玉腕挽復挽,四隻金蓮顛倒顛。
\end{myquote}

那金蓮在上面笑成一塊。{\pangpi{笑得妙。}}月娘道:「六姐你在上頭笑不打緊,只怕一時滑倒,不是耍處。」說着,不想那畫板滑,又是高底鞋,跐不牢,只聽得『滑浪』一聲,把金蓮擦下來,{\pangpi{跌得尤妙。}}早是扶住架子不曾跌着,險些沒把玉樓也拖下來。月娘道:「我說六姐笑的不好,只當跌下來。」{\meipi{不笑不跌,有何趣味。}}因望李嬌兒衆人說道:「這打鞦韆,最不該笑。笑多了,已定腿軟了,跌下來。咱在家做女兒時,隔壁周臺官家花園中紮着一座鞦韆。也是三月佳節,一日他家周小姐和俺一般三四箇女孩兒,都打鞦韆耍子,也是這等笑的不了,把周小姐滑下來,騎在畫板上,把身子喜抓去了。落後嫁與人家,被人家說不是女兒,休逐來家,今後打鞦韆,先要忌笑。」金蓮道:「孟三兒不濟,{\meipi{金蓮自跌了,轉說孟三兒不濟,妙甚。}}等我和李大姐打箇立秋千。」月娘道:「你兩箇仔細打。」卻教玉簫、春梅在旁推送。纔待打時,只見陳敬濟自外來,說道:「你每在這裡打鞦韆哩。」月娘道:「姐夫來的正好,且來替你二位娘送送兒。丫頭每氣力少。」這敬濟「老和尚不撞鐘——得不的一聲」,於是撥步撩衣,向前說:「等我送二位娘。」先把金蓮裙子帶住,說道:「五娘站牢,兒子送也。」那鞦韆飛在半空中,猶若飛仙相似。李瓶兒見鞦韆起去了,唬的上面怪叫道:「不好了,姐夫你也來送我送兒。」敬濟道:「你老人家到且性急,也等我慢慢兒的打發將來。這裡叫,那裡叫,把兒子手脚都弄慌了。」於是把李瓶兒裙子掀起,露着他大紅底衣,推了一把。李瓶兒道:「姐夫,慢慢着些!我腿軟了!」敬濟道:「你老人家原來吃不得緊酒。」金蓮又說:「李大姐,把我裙子又兜住了。」兩箇打到半中腰裡,都下來了。卻是春梅和西門大姐兩箇打了一回。然後,教玉簫和蕙蓮兩箇打立秋千。這蕙蓮手挽彩繩,身子站的直屢屢的,脚跐定下邊畫板,也不用人推送,那鞦韆飛在半天雲裡,然後忽地飛將下來,端的卻是飛仙一般,甚可人愛。{\pangpi{殊亦可人。}}月娘看見,對玉樓、李瓶兒說:「你看媳婦子,他倒會打。」這裡月娘衆人打鞦韆不題。

話分兩頭。卻表來旺兒往杭州,織造蔡太師生辰衣服回來,押着許多馱垜箱籠船上,先走來家。到門首,下了頭口,收卸了行李,進到後邊。只見雪娥正在堂屋門首,作了揖。那雪娥滿面微笑,{\meipi{一「揖」便有可疑。一「微」字便有要搬嘴之意。}}說道:「好呀,你來家了。{\pangpi{喜之之辭。}}路上風霜,多有辛苦!幾時沒見,吃得黑胖了。」來旺因問:「爹娘在那裡?」雪娥道:「你爹今日被應二衆人,邀去門外耍子去了。你大娘和大姐,都在花園中打鞦韆哩。」來旺兒道:「啊呀,打他則甚?」雪娥便倒了一盞茶與他吃,因問:「媳婦子在竈上,怎的不見?」那雪娥冷笑了一聲,說道:「你的媳婦子,如今還是那時的媳婦兒哩?好不大了!他每日只跟着他娘每夥兒裡下棋,撾子兒,抹牌頑耍。他肯在竈上做活哩!」正說着,小玉走到花園中,報與月娘。月娘自前邊走來,來旺兒向前磕了頭,立在旁邊。問了些路上往回的話,月娘賞了兩瓶酒。吃一回,他媳婦宋蕙蓮來到。月娘道:「也罷,你辛苦了,且往房裡洗洗頭面,歇宿歇宿去。等你爹來,好見你爹回話。」那來旺兒便歸房裡。蕙蓮先付鑰匙開了門,又舀些水與他洗臉攤塵,收拾褡褳去,說道:「賊黑囚,幾時沒見,便吃得這等肥肥的。」又替他換了衣裳,安排飯食與他吃。睡了一覺起來,已是日西時分。

西門慶來家,來旺兒走到跟前參見,說道:「杭州織造蔡太師生辰的尺頭並家中衣服,俱已完備,打成包裹,裝了四箱,搭在官船上來家,只少顧夫過稅。」西門慶滿心歡喜,與了他趕脚銀兩,明日早裝載進城。又賞銀五兩,房中盤纏;又教他管買辦東西。{\pangpi{伏。}}這來旺兒私已帶了些人事,悄悄送了孫雪娥兩方綾汗巾,兩隻裝花膝褲,四匣杭州粉,二十箇胭脂。{\meipi{雪娥與來旺私情,絕不露一語,只脈脈畫箇影子,有意到筆不到之妙。}}雪娥背地告訴來旺兒說:「自從你去了四箇月,你媳婦怎的和西門慶勾搭,玉簫怎的做牽頭,金蓮屋裡怎的做窩窠。先在山子底下,落後在屋裡,成日明睡到夜,夜睡到明。與他的衣服、首飾、花翠、銀錢,大包帶在身邊。使小厮在門首買東西,見一日也使二三錢銀子。」來旺道:「怪道箱子裡放着衣服、首飾!我問他,他說娘與他的。」雪娥道:「那娘與他?到是爺與他的哩!」{\pangpi{妙語。}}這來旺兒遂聽記在心。

到晚夕,吃了幾鍾酒,歸到房中。常言「酒發頓腹之言」,因開箱子,看見一疋藍段子,甚是花樣奇異,便問老婆:「是那裡的段子?誰人與你的?趁早寔說。」老婆不知就裡,故意笑着,回道:「怪賊囚,問怎的?此是後邊見我沒箇襖兒,與了這疋段子,放在箱中,沒工夫做。端的誰肯與我?」來旺兒罵道:「賊淫婦!還搗鬼哩!端的是那箇與你的?」又問:「這些首飾是那裡的?」婦人道:「呸!怪囚根子,那箇沒箇娘老子,就是石頭罅剌兒裡迸出來,也有箇窩巢兒,為人就沒箇親戚六眷?此是我姨娘家借來的釵梳。是誰與我的!」被來旺兒一拳,險不打了一交,{\pangpi{何其狠也。}}說:「賊淫婦,還說嘴哩!有人親看見,你和那沒人倫的豬狗有首尾!玉簫丫頭怎的牽頭,送段子與你,在前邊花園內兩箇幹,落後弔在潘家那淫婦屋裡明幹,成日㒲的不値了。賊淫婦,你還要我手裡弔子曰兒。」那婦人便大哭起來,說道:「賊不逢好死的囚根子!你做甚麼來家打我?我幹壞了你甚麼事來?你恁是言不是語,丟塊磚瓦兒也要箇下落。是那箇嚼舌根的,沒空生有,調唆你來欺負老娘?{\pangpi{自家沒得說,反劈空罵人,妙絕。}}我老娘不是那沒根基的貨!教人就欺負死,也揀箇乾淨地方。你問聲兒,宋家的丫頭,若把脚畧趄兒,把『宋』字兒倒過來!你這賊囚根子,得不箇風兒就雨兒。萬物也要箇實。人教你殺那箇人,你就殺那箇人?」幾句說的來旺兒不言語了。婦人又道:「這疋藍段子,越發我和你說了罷,也是去年十一月裡三娘生日,娘見我上穿着紫襖,下邊借了玉簫的裙子穿着,說道:『媳婦子怪剌剌的,甚麼樣子?』纔與了我這疋段子。誰得閑做他?那箇是不知道!就纂我恁一遍舌頭。你錯認了老娘,老娘不是箇饒人的。明日我咒罵箇樣兒與他聽。破着我一條性命,自恁尋不着主兒哩。」{\meipi{以死嚇人,是淫婦伎倆。}}來旺兒道:「你既沒此事,{\pangpi{虎頭蛇尾,可笑。}}平白和人合甚氣?{\pangpi{怕纏出雪娥來。}}快些打鋪我睡。」這婦人一面把鋪伸下,說道:「怪倒路的囚根子,𠳹了那黃湯,挺你那覺!平白惹老娘罵。」把來旺掠翻在炕上,鼾聲如雷。

看官聽說:但凡世上養漢的婆娘,饒他男子漢十八分精細,吃他幾句左話兒右說,十箇九箇都着了道兒。正是:「東淨裡磚兒——又臭又硬。」這宋蕙蓮窩盤住來旺兒,過了一宿。到次日,往後邊問玉簫,誰人透露此事,終莫知其所繇,只顧海罵。一日,月娘使小玉叫雪娥,一地裡尋不着。走到前邊,只見雪娥從來旺兒房裡出來,只猜和他媳婦說話,不想走到廚下,蕙蓮又在裡面切肉,良久,西門慶前邊陪着喬大戶說話,只為揚州鹽商王四峯,被按撫使送監在獄中,許銀二千兩,央西門慶對蔡太師討人情釋放。剛打發大戶去了,西門慶叫來旺,來旺從他屋裡跑出來。正是:

\begin{myquote}
雪隱鷺鷥飛始見,柳藏鸚鵡語方知。
\end{myquote}

以此都知雪娥與來旺兒有尾首。一日,來旺兒吃醉了,和一般家人小厮在前邊恨罵西門慶,說:「怎的我不在家,使玉簫丫頭拏一疋藍段子,在房裡哄我老婆。把他弔在花園奸耍,後來潘金蓮怎的做窩主。繇他,只休要撞到我手裡。我教他白刀子進去,紅刀子出來。好不好,把潘家那淫婦也殺了,也只是箇死。你看我說出來做的出來。潘家那淫婦,想着他在家擺死了他漢子武大,他小叔武松來告狀,多虧了誰替他上東京打點,把武松墊發充軍去了?今日兩脚踏住平川路,落得他受用,還挑撥我的老婆養漢。我的仇恨,與他結的有天來大。常言道『一不做,二不休』,到跟前再說話。『破着一命剮,便把皇帝打』!」{\meipi{此等事雖不得不恨,不得不罵,然雪娥事卻又如何?古今自非紀臣,而往往謗訕朝廷以賈禍者,率此類也。}}這來旺兒自知路上說話,不知草裡有人,不想被同行家人來興兒聽見。這來興兒在家,西門慶原派他買辦食用撰錢過日,只因與來旺媳婦勾搭,把買辦奪了,卻教來旺兒管領。來興兒就與來旺不睦,聽見發此言語,就悄悄走來潘金蓮房裡告訴。

金蓮正和孟玉樓一處坐的,只見來興兒掀簾子進來,金蓮便問來興兒:「你來有甚事?你爹今日往誰家吃酒去了?」來興道:「今日俺爹和應二爹往門外送殯去了。適有一件事,告訴老人家,只放在心裡,休說是小的來說。」金蓮道:「你有甚事,只顧說,不妨事!」來興兒道:「別無甚事,叵耐來旺兒,昨日不知那裡吃的醉稀稀的,在前邊大喓小喝,指豬罵狗,罵了一日。又邏着小的厮打,小的走來一邊,不理他。對着家中大小,又罵爹和五娘。」潘金蓮就問:「賊囚根子,罵我怎的?」來興說:「小的不敢說。三娘在這裡,也不是別人。那厮說爹怎的打發他不在家,耍了他的老婆,說五娘怎的做窩主,撰他老婆在房裡和爹兩箇明睡到夜,夜睡到明。他打下刀子,要殺爹和五娘,白刀子進去,紅刀子出來。又說,五娘那咱在家,毒藥擺殺了親夫,多虧了他上東京去打點,救了五娘一命。說五娘恩將仇報,挑撥他老婆養漢。小的穿青衣抱黑住,先來告訴五娘說聲,早晚休吃那厮暗算。」玉樓聽了,如提在冷水盆內一般,吃了一驚。這金蓮不聽便罷,聽了,粉面通紅,銀牙咬碎,罵道:「這犯死的奴才!我與他往日無冤近日無仇,他主子要了他的老婆,他怎的纏我?我若教這奴才在西門慶家,永不算老婆!怎的我虧他救活了性命?」因分咐來興兒:「你且去,等你爹來家問你時,你也只照恁般說。」來興兒說:「五娘說那裡話!小的又不賴他,有一句說一句。隨爹怎的問,也只是這等說。」{\pangpi{說得斬釘截鐵。}}說畢,往前邊去了。

玉樓便問金蓮:「眞箇他爹和這媳婦子有?」{\pangpi{寫出玉樓無心。}}金蓮道:「你問那沒廉恥的貨!甚的好老婆,也不枉了教奴才這般挾制了。在人家使過了的奴才淫婦,當初在蔡通判家,和大婆作弊養漢,壞了事,纔打發出來,嫁了蔣聰。{\pangpi{到此又補出。}}豈止見過一箇漢子兒?有一拏小米數兒,甚麼事兒不知道!賊強人瞞神嚇鬼,使玉簫送段子兒與他做襖兒穿。一冬裡,我要告訴你,沒告訴你。那一日,大姐姐往喬大戶家吃酒,咱每都不在前邊下棋?只見丫頭說他爹來家,咱每不散了?落後我走到後邊儀門首,見小玉立在穿廊下,我問他,小玉望着我搖手兒。我剛走到花園前,只見玉簫那狗肉在角門首站立,原來替他觀風。我還不知,教我徑往花園裡走。玉簫攔着我,不教我進去,說爹在裡面。教我罵了兩句。我到疑影和他有些甚麼查子帳,不想走到裡面,他和媳婦子在山洞裡幹營生。{\meipi{叙往事,覺眉目宛如對面。}}媳婦子見我進去,把臉飛紅的走出來了。他爹見了我,訕訕的,吃我罵了兩句『沒廉恥』。落後媳婦子走到屋裡,打旋磨跪着我,教我休對他娘說。落後正月裡,他爹要把淫婦安托在我屋裡過一夜兒,吃我和春梅折了兩句,再幾時容他傍箇影兒!賊萬殺的奴才,沒的把我扯在裡頭。好嬌態的奴才淫婦,我肯容他在那屋裡頭弄硶兒?就是我罷了,俺春梅那小肉兒,他也不肯容他。」玉樓道:「嗔道賊臭肉在那裡坐着,見了俺每意意似似,待起不起的,誰知原來背地有這本帳!{\pangpi{想起從前有致。}}論起來,他爹也不該要他。那裡尋不出老婆來,教奴才在外邊倡揚,甚麼樣子?」金蓮道:「左右的皮靴兒沒番正,你要奴才老婆,奴才暗地裡偷你的小娘子,彼此換着做!賊小婦奴才,千也嘴頭子嚼說人,萬也嚼說,今日打了嘴,也不說的!」{\meipi{說到雪娥又罵一頓,映出恨心不忘。}}玉樓向金蓮道:「這樁事,咱對他爹說好,不說好?大姐姐又不管。倘忽那厮眞箇安心,咱每不言語,他爹又不知道,一時遭了他手怎了?六姐,你還該說說。」金蓮道:「我若是饒了這奴才,除非是他㒲出我來。」正是:

\begin{myquote}
平生不作皺眉事,世上應無切齒人。
\end{myquote}

西門慶至晚來家,只見金蓮在房中雲鬟不整,睡搵香腮,哭的眼壞壞的。{\meipi{長技。}}問其所以,遂把來旺兒醉酒發言,要殺主之事訴說一遍:「見有來興兒親自聽見,思想起來,你背地圖他老婆,他便背地要你家小娘子。你的皮靴兒沒番正。那厮殺你便該當,與我何干?連我一例也要殺!趁早不為之計,夜頭早晚,人無後眼,只怕暗遭他毒手。」西門慶因問:「誰和那厮有首尾?」金蓮道:「你休來問我,只問小玉便知。」又說:「這奴才欺負我,不是一遭兒了。說我當初怎的用藥擺殺漢子,你娶了我來,虧他尋人情搭救我性命來。在外邊對人揭條。早是奴沒生下兒沒長下女,若是生下兒女,教賊奴才揭條着好聽?{\meipi{偏有許多設想,妙舌。}}敢說『你家娘當初在家不得地時,也虧我尋人情救了他性命。』恁說在你臉上也無光了!你便沒羞恥,我卻成不的,要這命做甚麼?」西門慶聽了婦人之言,走到前邊,叫將來興兒到無人處,問他始末緣繇。這小厮一五一十說了一遍。又走到後邊,摘問了小玉口詞,與金蓮所說無差:委的某日,親眼看見雪娥從來旺兒屋裡出來,他媳婦兒不在屋裡,的有此事。這西門慶心中大怒,把孫雪娥打了一頓,被月娘再三勸了,拘了他頭面衣服,只教他伴着家人媳婦上竈,不許他見人。此事表過不題。

西門慶在後邊,因使玉簫叫了宋蕙蓮,背地親自問他。{\pangpi{呆甚。}}這婆娘便道:「啊呀,爹,你老人家沒的說,他是沒有這箇話。我就替他賭了大誓。他酒便吃兩鍾,敢恁七箇頭八箇膽,背地裡罵爹?又吃紂王水土,又說紂王無道!他靠那裡過日子?爹,你不要聽人言語。我且問爹,聽見誰說這箇話來?」{\pangpi{反問他要人,妙。}}那西門慶被婆娘一席話兒,閉口無言。問的急了,說:「是來興兒告訴我說的。」蕙蓮道:「來興兒因爹叫俺這一箇買辦,說俺每奪了他的,不得撰些錢使,結下這仇恨兒,平空拏這血口噴他,爹就信了。他有這箇欺心的事,我也不饒他。爹你依我,不要教他在家裡,與他幾兩銀子本錢,教他信信脫脫,遠離他鄉做買賣去。{\pangpi{處法亦善。}}他出去了,早晚爹和我說句話兒也方便些。」西門慶聽了滿心歡喜,說道:「我的兒,說的是。我有心要叫他上東京,與鹽商王四峯央蔡太師人情,回來,還要押送生辰担去,只因他纔從杭州來家,不好又使他的,打帳叫來保去。既你這樣說,我明日打發他去便了。回來,我教他領一千兩銀子,同主管往杭州販買紬絹絲線做買賣。你意下如何?」老婆心中大喜,說道:「爹若這等纔好。」正說着,西門慶見無人,就摟他過來親嘴。婆娘忙遞舌頭在他口裡,兩箇咂做一處。婦人道:「爹,你許我編鬏髻,怎的還不替我編?{\pangpi{不放鬆,妙。}}恁時候不戴到幾時戴?只教我成日戴這頭髮殼子兒?」西門慶道:「不打緊,到明日將八兩銀子,往銀匠家替你拔絲去。」西門慶又道:「怕你大娘問,怎生回答?」婦人道:「不打緊,我自有話打發他,只說問我姨娘家借來戴戴,怕怎的?」當下二人說了一回話,各自分散了。

到了次日,西門慶在廳上坐着,叫過來旺兒來:「你收拾衣服行李,趕明日三月二十八日起身,往東京央蔡太師人情。回來,我還打發你杭州做買賣去。」這來旺心中大喜,{\meipi{此何足喜,已微有拐銀棄妻之意。}}應諾下來,回房收拾行李,在外買人事。來興兒打聽得知,就來告報金蓮知道。金蓮打聽西門慶在花園捲棚內,走到那裡,不見西門慶,只見陳敬濟在那裡封禮物。金蓮便道:「你爹在那裡?你封的是甚麼?」敬濟道:「爹剛纔在這裡,往大娘那邊兌鹽商王四峯銀子去了。我封的是往東京央蔡太師的禮。」金蓮問:「打發誰去?」敬濟道:「我聽見昨日爹分咐來旺兒去。」這金蓮纔待下臺基,往花園那條路上走,正撞見西門慶拏了銀子來。叫到屋裡,問他:「明日打發誰往東京去?」西門慶道:「來旺兒和吳主管二人同去。因有鹽商王四峯一千幹事的銀兩,以此多着兩箇去。」婦人道:「隨你心下,我說的話兒你不依,到聽那奴才淫婦一面兒言語。他隨問怎的,只護他的漢子。那奴才有話在先,不是一日兒了。左右破着老婆丟與你,坑了你這銀子,拐的往那頭裡停停脫脫去了,看哥哥兩眼兒空哩。你的白丟了罷了,難為人家一千兩銀子,不怕你不賠他。{\meipi{雖挑撥,然亦有理。}}我說在你心裡,也隨你。老婆無故只是為他。不爭你貪他這老婆,你留他在家裡也不好,你就打發他出去做買賣也不好。你留他在家裡,早晚沒這些眼防範他。你打發他外邊去,他使了你本錢,頭一件你先說不得他。你若要他這奴才老婆,不如先把奴才打發他離門離戶。常言道:『剪草不除根,萌芽依舊生;剪草若除根,萌芽再不生』。就是你也不耽心,老婆他也死心塌地。」{\pangpi{二語動人。}}一席話兒,說得西門慶如醉方醒。正是:

\begin{myquote}
數語撥開君子路,片言提醒夢中人。
\end{myquote}

