\includepdf[pages={65,66},fitpaper=false]{tst.pdf}
\chapter*{第三十三回 陳敬濟失鑰罰唱 韓道國縱婦爭鋒}
\addcontentsline{toc}{chapter}{第三十三回 陳敬濟失鑰罰唱 韓道國縱婦爭鋒}
\markboth{{\titlename}卷之四}{第三十三回 陳敬濟失鑰罰唱 韓道國縱婦爭鋒}


詞曰:

\begin{myquote} 
衣染鶯黃,愛停板駐拍,勸酒持觴。低鬟蟬影動,私語口脂香。簷滴露、竹風涼,拚劇飲琳琅。夜漸深,籠燈就月,仔細端相。

\raggedleft{——右調《意難忘前》\rightquadmargin}
\end{myquote} 

話說西門慶衙門中來家,進門就問月娘:「哥兒好些?使小厮請太醫去。」月娘道:「我已叫劉婆子來了。吃了他藥,孩子如今不洋奶,穩穩睡了這半日,覺好些了。」西門慶道:「信那老淫婦胡針亂灸,還請小兒科太醫看纔好。既好些了,罷。若不好,拿到衙門裡去拶與老淫婦一拶子。」月娘道:「你恁的枉口拔舌罵人。你家孩兒現吃了他藥好了,還恁舒着嘴子罵人!」說畢,丫鬟擺上飯來。

西門慶剛纔吃了飯,只見玳安兒來報:「應二爹來了。」西門慶教小厮:「拿茶出去,請應二爹捲棚內坐。」向月娘道:「把剛纔我吃飯的菜蔬休動,教小厮拿飯出去,教姐夫陪他吃,說我就來。」月娘便問:「你昨日早晨使他往那裡去?那咱纔來。」西門慶便告說:「應二哥認的一箇湖州客人何官兒,門外店裡堆着五百兩絲線,急等着要起身家去,來對我說要折些發脫。我只許他四百五十兩銀子。昨日使他同來保拿了兩錠大銀子作樣銀,已是成了來了,約下今日兌銀子去。我想來,獅子街房子空閑,開啟門面兩間,倒好收拾開箇絨線鋪子,搭箇夥計。況來保已是鄆王府認納官錢,教他與夥計在那裡,又看了房兒,又做了買賣。」月娘道:「少不得又尋夥計。」西門慶道:「應二哥說他有一相識,姓韓,原是絨線行,如今沒本錢,閑在家裡,說寫算皆精,行止端正,再三保舉。改日領他來見我,寫立合同。」

說畢,西門慶在房中兌了四百五十兩銀子,教來保拿出來。陳敬濟已陪應伯爵在捲棚內吃完飯,等的心裡火發。見銀子出來,心中歡喜,與西門慶唱了喏,說道:「昨日打攪哥,到家晚了,今日再扒不起來。」西門慶道:「這銀子我兌了四百五十兩,教來保取搭連眼同裝了。今日好日子,便顧車輛搬了貨來,鎖在那邊房子裡就是了。」伯爵道:「哥主張的有理。只怕蠻子停留長智,推進貨來就完了帳。」于是同來保騎頭口,打着銀子,逕到門外店中成交易去。誰知伯爵背地裡與何官兒砸殺了,只四百二十兩銀子,打了三十兩背工。對着來保,當面只拿出九兩用銀來,二人均分了。顧了車脚,即日推貨進城,堆在獅子街空房內,鎖了門,來回西門慶話。西門慶教應伯爵,擇吉日領韓夥計來見。其人五短身材,三十年紀,言談滾滾,滿面春風。西門慶即日與他寫立合同。同來保領本錢顧人染絲,在獅子街開張鋪面,發賣各色絨絲。一日也賣數十兩銀子,不在話下。

光陰迅速,日月如梭,不覺八月十五日,月娘生辰來到,請堂客擺酒。留下吳大妗子、潘姥姥、楊姑娘並兩箇姑子住兩日,晚夕宣唱佛曲兒,常坐到二三更纔歇。那日,西門慶因上房有吳大妗子在這裡,不方便,走到前邊李瓶兒房中看官哥兒,心裡要在李瓶兒房裡睡。李瓶兒道:「孩子纔好些兒,我心裡不耐煩,往他五媽媽房裡睡一夜罷。」{\meipi{善哉,善哉。}}西門慶笑道:「我不惹你。」于是走過金蓮這邊來。那金蓮聽見漢子進他房來,如同拾了金寶一般,連忙打發他潘姥姥過李瓶兒這邊宿歇。他便房中高點銀燈,款伸錦被,薰香澡牝,夜間陪西門慶同寢。枕畔之情,百般難述,無非只要牢寵漢子心,使他不往別人房裡去。正是:鼓鬣遊蜂,嫩蕋半勻春盪漾;餐香粉蝶,花房深宿夜風流。

李瓶兒見潘姥姥過來,連忙讓在炕上坐的。教迎春安排酒菜菓餅,晚夕說話,坐半夜纔睡。到次日,與了潘姥姥一件蔥白綾襖兒,兩雙段子鞋面,二百文錢。把婆子歡喜的眉歡眼笑,過這邊來,拿與金蓮瞧,說:「這是那邊姐姐與我的。」金蓮見了,反說他娘:「好恁小眼薄皮的,什麼好的,拿了他的來!」潘姥姥道:「好姐姐,人倒可憐見與我,你卻說這箇話。你肯與我一件兒穿?」金蓮道:「我比不得他有錢的姐姐。我穿的還沒有哩,拿什麼與你!你平白吃了人家的來,等住回可整理幾碟子來,篩上壺酒,拿過去還了他就是了。到明日少不的教人わ言試語,我是聽不上。」{\meipi{以己意度人,是因有君子小人之別。}}一面分咐春梅,定八碟菜蔬,四盒菓子,一錫瓶酒。打聽西門慶不在家,教秋菊用方盒拿到李瓶兒房裡,說:「娘和姥姥過來,無事和六娘吃盃酒。」李瓶兒道:「又教你娘費心。」少頃,金蓮和潘姥姥來,三人坐定,把酒來斟。春梅侍立斟酒。娘兒每說話間,只見秋菊來叫春梅,說:「姐夫在那邊尋衣裳,教你去開外邊樓門哩。」金蓮分咐:「叫你姐夫尋了衣裳來這裡喝甌子酒去。」不一時,敬濟尋了幾家衣服,就往外走。春梅進來回說:「他不來。」金蓮道:「好歹拉了他來。」又使出綉春去把敬濟請來。潘姥姥在炕上坐,小桌兒擺着菓盒兒,金蓮、李瓶兒陪着吃酒。連忙唱了喏。金蓮說:「我好意教你來吃酒兒,你怎的張致不來?就弔了造化了?」努了箇嘴兒,教春梅:「拿寬盃兒來,篩與你姐夫吃。」敬濟把尋的衣服放在炕上,坐下。春梅做定科範,取了箇茶甌子,流沿邊斟上,遞與他。慌的敬濟說道:「五娘賜我,寧可吃兩小鍾兒罷。外邊鋪子裡許多人等着要衣裳。」金蓮道:「教他等着去,我偏教你吃這一大鍾,那小鍾子刁刁的不耐煩。」潘姥姥道:「只教哥哥吃這一鍾罷,只怕他買賣事忙。」金蓮道:「你信他!有什麼忙!吃好少酒兒,金漆桶子吃到第二道箍上。」那敬濟笑着拿酒來,剛呷了兩口。潘姥姥叫春梅:「姐姐,你拿筯兒與哥哥。教他吃寡酒?」春梅也不拿筯,故意毆他,向攢盒內取了兩箇核桃遞與他。{\meipi{春梅、金蓮此唱彼和,的眞肝膽相知。}}那敬濟接過來道:「你敢笑話我就禁不開他?」于是放在牙上只一磕,咬碎了下酒。潘姥姥道:「還是小後生家,好口牙。相老身,東西兒硬些就吃不得。」敬濟道:「兒子世上有兩樁兒:鵝卵石、牛犄角吃不得罷了。」金蓮見他吃了那鍾酒,教春梅再斟上一鍾兒,說:「頭一鍾是我的了。你姥姥和六娘不是人麼?也不教你吃多,只吃三甌子,饒了你罷。」敬濟道:「五娘可憐見兒子來,眞吃不得了。此這一鍾,恐怕臉紅,惹爹見怪。」金蓮道:「你也怕你爹?我說你不怕他。{\meipi{只一語,隱出戲狎心腸。}}你爹今日往那裡吃酒去了?」敬濟道:「後晌往吳驛丞家吃酒,如今在對門喬大戶房子裡看收拾哩。」金蓮問:「喬大戶家昨日搬了去,咱今日怎不與他送茶?」敬濟道:「今早送茶去了。」李瓶兒問:「他家搬到那裡住去了?」敬濟道:「他在東大街上使了一千二百銀子,買了所好不大的房子,與咱家房子差不多兒,門面七間,到底五層。」說話之間,敬濟捏着鼻子又捱了一鍾,趁金蓮眼錯,得手拿着衣服往外一溜烟跑了。迎春道:「娘你看,姐夫忘記鑰匙去了。」那金蓮取過來坐在身底下,向李瓶兒道:「等他來尋,你每且不要說,等我奈何他一回兒纔與他。」潘姥姥道:「姐姐與他罷了,又奈何他怎的。」

那敬濟走到鋪子裡,袖內摸摸,不見鑰匙,一直走到李瓶兒房裡尋。金蓮道:「誰見你什麼鑰匙,你管着什麼來?放在那裡,就不知道?」春梅道:「只怕你鎖在樓上了。」敬濟道:「我記的帶出來。」金蓮道:「小孩兒家屁股大,敢弔了心!又不知家裡外頭什麼人扯落的你恁有魂沒識,心不在肝上。」{\meipi{隱然自居,妙。}}敬濟道:「有人來贖衣裳,可怎的樣?趁爹不過來,免不得叫箇小爐匠來開樓門,纔知有沒。」那李瓶兒忍不住,只顧笑。敬濟道:「六娘拾了,與了我罷。」金蓮道:「也沒見這李大姐,不知和他笑什麼,恰似我每拿了他的一般。」急得敬濟只是牛回磨轉,轉眼看見金蓮身底下露出鑰匙帶兒來,說道:「這不是鑰匙!」纔待用手去取,被金蓮褪在袖內,不與他,說道:「你的鑰匙兒,怎落在我手裡?」急得那小夥兒只是殺雞扯膝。金蓮道:「只說你會唱的好曲兒,倒在外邊鋪子裡唱與小厮聽,怎的不唱箇兒我聽?今日趁着你姥姥和六娘在這裡,只揀眼生好的唱箇兒,我就與你這鑰匙。不然,隨你就跳上白塔,我也沒有。」敬濟道:「這五娘,就勒掯出人痞來。誰對你老人家說我會唱?」金蓮道:「你還搗鬼?『南京沈萬三,北京枯樹彎——人的名兒,樹的影兒』。」那小夥兒吃他奈何不過,說道:「死不了人,等我唱。我肚子裡撐心柱肝,要一百箇也有!」金蓮罵道:「說嘴的短命!」自把各人面前酒斟上。金蓮道:「你再吃一盃,蓋着臉兒好唱。」敬濟道:「我唱了慢慢吃。我唱箇菓子名《山坡羊》你聽:

\begin{myquote} 
初相交,在桃園兒裡結義。相交下來,把你當玉黃李子兒擡舉。人人說你在青翠花家飲酒,氣的我把頻波臉兒撾的粉粉的碎。我把你賊,你學了虎刺賓了,外實裡虛,氣的我李子眼兒珠淚垂。我使的一對桃奴兒尋你,見你在軟棗兒樹下就和我別離了去。氣的我鶴頂紅剪一柳青絲兒來呵,你海東紅反說我理虧。罵了句生心紅的強賊,逼的我急了,我在弔枝幹兒上尋箇無常,到三秋,我看你倚靠着誰?」
\end{myquote} 

唱畢,就問金蓮要鑰匙,說道:「五娘快與了我罷!夥計鋪子裡不知怎的等着我哩。只怕一時爹過來。」金蓮道:「你倒自在性兒,說的且是輕巧。等你爹問,我就說你不知在那裡吃了酒,把鑰匙不見了,走來俺屋裡尋。」敬濟道:「爺嚛!五娘就是弄人的劊子手。」{\meipi{人固會弄。}}李瓶兒和潘姥姥再三旁邊說道:「姐姐與他去罷。」金蓮道:「若不是姥姥和你六娘勸我,定罰教你唱到天晚。頭裡騙嘴說一百箇,纔唱一箇曲兒就要騰翅子?我手裡放你不過。」敬濟道:「我還有一箇兒看家的,是銀名《山坡羊》,亦發孝順你老人家罷。」于是頓開喉音唱道:

\begin{myquote} 
冤家你不來,白悶我一月,閃的人反拍着外膛兒細絲諒不徹。我使獅子頭定兒小厮拏着黃票兒請你,你在兵部窪兒裡元寶兒家歡娛過夜。我陪銅磬兒家私為焦心一旦兒棄捨,我把如同印箝兒印在心裡愁無救解。叫着你把那挺臉兒高揚着不理,空教我撥着雙火筒兒頓着礶子等到你更深半夜。氣的奴花銀竹葉臉兒咬定銀牙來呵,喚官銀頂上了我房門,隨那潑臉兒冤家輕敲兒不理。罵了句煎徹了的三傾兒搗槽斜賊,空把奴一腔子煖汁兒眞心倒與你,只當做熱血。
\end{myquote} 

敬濟唱畢,金蓮纔待叫春梅斟酒與他,忽有月娘從後邊來,見奶子如意兒抱着官哥兒在房門首石基上坐,便說道:「孩子纔好些,你這狗肉又抱他在風裡,還不抱進去!」金蓮問:「是誰說話?」綉春回道:「大娘來了。」敬濟慌的拿鑰匙往外走不迭。衆人都下來迎接月娘。月娘便問:「陳姐夫在這裡做什麼來?」金蓮道:「李大姐整治些菜,請俺娘坐坐。{\meipi{開口推及瓶兒,以敬濟之故。}}陳姐夫尋衣服,叫他進來吃一盃。姐姐,你請坐,好甜酒兒,你吃一盃。」月娘道:「我不吃。後邊他大妗子和楊姑娘要家去,我又記掛着這孩子,逕來看看。李大姐,你也不管,又教奶子抱他在風裡坐的。前日劉婆子說他是驚寒,你還不好生看他!」李瓶兒道:「俺陪着姥姥吃酒,誰知賊臭肉三不知抱他出去了。」月娘坐了半歇,回後邊去了。一回,使小玉來,請姥姥和五娘、六娘後邊坐。那潘金蓮和李瓶兒勻了臉,同潘姥姥往後邊來,陪大妗子、楊姑娘吃酒。到日落時分,與月娘送出大門,上轎去了。都在門裡站立,先是孟玉樓說道:「大姐姐,今日他爹不在,往吳驛丞家吃酒去了,咱到好往對門喬大戶家房裡瞧瞧。」月娘問看門的平安兒:「誰拿着那邊鑰匙哩?」平安道:「娘每要過去瞧,開着門哩。來興哥看着兩箇坌工的在那裡做活。」月娘分咐:「你教他躲開,等俺每瞧瞧去。」平安兒道:「娘每只顧瞧,不妨事。他每都在第四層大空房撥灰篩土,叫出來就是了。」

當下月娘、李嬌兒、孟玉樓、潘金蓮、李瓶兒,都用轎子短搬擡過房子內。進了儀門,就是三間廳。第二層是樓。月娘要上樓去,可是作怪,剛上到樓梯中間,不料梯磴陡趄,只聞月娘哎了一聲,滑下一隻脚來,早是月娘攀住樓梯兩邊欄杆。慌了玉樓,便道:「姐姐怎的?」連忙搊住他一隻胳膊,不曾跌下來。月娘吃了一驚,就不上去。衆人扶了下來,諕的臉蠟查兒黃了。玉樓便問:「姐姐,怎麼上來滑了脚,不曾扭着那裡?」月娘道:「跌倒不曾跌着,只是扭了腰子,諕的我心跳在口裡。樓梯子趄,我只當咱家裡樓上來,滑了脚。早是攀住欄杆,不然怎了!」李嬌兒道:「你又身上不方便,早知不上樓也罷了。」于是衆姊妹相伴月娘回家。剛到家,叫的應就肚中疼痛。月娘忍不過,趁西門慶不在家,使小厮叫了劉婆子來看。婆子道:「你已是去經事來着傷,多是成不的了。」月娘道:「便了五箇多月了,上樓着了扭。」婆子道:「你吃了我這藥,安不住,下來罷了。」{\meipi{劉婆可恨。不得安胎而得摧生者,醫家妙訣。}}月娘道:「下來罷!」婆子于是留了兩服大黑丸子藥,教月娘用艾酒吃。那消半夜,弔下來了,在馬桶裡。點燈撥看,原來是箇男胎,已成形了。正是:

\begin{myquote} 
胚胎未能成性命,眞靈先到杳冥天。
\end{myquote} 

幸得那日西門慶在玉樓房中歇了。到次日,玉樓早晨到上房,問月娘:「身子如何?」月娘告訴:「半夜果然疼不住,落下來了,倒是小厮兒。」玉樓道:「可惜了!他爹不知道?」月娘道:「他爹吃酒來家,到我屋裡纔待脫衣裳,我說你往他們屋裡去罷,我心裡不自在。他纔往你這邊來了。我沒對他說。我如今肚裡還有些隱隱的疼。」玉樓道:「只怕還有些餘血未盡,篩酒吃些鍋臍灰兒就好了。」又道:「姐姐,你還計較兩日兒,且在屋裡不可出去。小產比大產還難調理,只怕掉了風寒,難為你的身子。」月娘道:「你沒的說,倒沒的倡揚的一地裡知道,平白噪剌剌的抱什麼空窩,惹的人動那唇齒。」{\meipi{用語隱然有指。}}以此就沒教西門慶知道。此事表過不題。

且說西門慶新搭的開絨線鋪夥計,也不是守本分的人,姓韓名道國,字希堯,乃是破落戶韓光頭的兒子。如今跌落下來,替了大爺的差使,亦在鄆王府做校尉,見在縣東街牛皮小巷居住。其人性本虛飄,言過其實,巧於詞色,善於言談。許人錢,如捉影捕風;騙人財,如探囊取物。自從西門慶家做了買賣,手裡財帛從容,新做了幾件虼蚤皮,在街上掇着肩膊兒就搖擺起來。人見了不叫他箇韓希堯,只叫他做「韓一搖」。他渾家乃是宰牲口王屠妹子,排行六兒,生的長跳身材,瓜子面皮,紫膛色,約二十八九年紀。身邊有箇女孩兒,嫡親三口兒度日。他兄弟韓二,名二搗鬼,是箇耍錢的搗子,在外邊另住。舊與這婦人有姦,趕韓道國不在家,鋪中上宿,他便時常走來與婦人吃酒,到晚夕刮涎就不去了。不想街坊有幾箇浮浪子弟,見婦人搽脂抹粉,打扮的喬模喬樣,常在門首站立睃人,人畧鬬他鬬兒,又臭又硬,就張致罵人。因此街坊這些小夥子兒,心中有幾分不憤,暗暗三兩成羣,背地講論,看他背地與什麼人有首尾。那消半箇月,打聽出與他小叔韓二這件事來。原來韓道國這間屋門面三間,房裡兩邊都是隣舍,後門逆水塘。這夥人,單看韓二進去,或夜晚扒在墻上看覷,或白日裡暗使小猴子在後塘推道捉蛾兒,單等捉姦。不想那日二搗鬼打聽他哥不在,大白日裝酒和婦人吃醉了,倒插了門,在房裡幹事。不防衆人睃見蹤跡,小猴子扒過來,把後門開了,衆人一齊進去,掇開房門。韓二奪門就走,被一少年一拳打倒拿住。老婆還在炕上,慌穿衣不迭。一人進去,先把褲子撾在手裡,都一條繩子拴出來。{\meipi{此輩捉姦,手脚敏捷可喜。}}須臾,圍了一門首人,跟到牛皮街廂鋪裡,就鬨動了那一條街巷。這一箇來問,那一箇來瞧,內中一老者見男婦二人拴做一處,便問左右看的人:「此是為什麼事的?」旁邊有多口的道:「你老人家不知,此是小叔姦嫂子的。」那老都點了點頭兒說道:「可傷,原來小叔兒要嫂子的,到官,叔嫂通姦,兩箇都是絞罪。」那旁邊多口的,認的他有名叫做陶扒灰,一連娶三箇媳婦,都吃他扒了,因此插口說道:「你老人家深通條律,相這小叔養嫂子的便是絞罪,若是公公養媳婦的卻論什麼罪?」那老者見不是話,低着頭一聲兒沒言語走了。{\meipi{心虛人,可公道話都難說。}}正是:各人自掃簷前雪,莫管他人屋上霜。這裡二搗鬼與婦人被捉不題。

單表那日,韓道國鋪子裡不該上宿,來家早,八月中旬天氣,身上穿着一套兒輕紗軟絹衣服,新盔的一頂帽兒,在街上闊行大步搖擺。但遇着人,或坐或立,口惹懸河,滔滔不絕。就是一回,內中遇着他兩箇相熟的人,一箇是開紙鋪的張二哥,一箇是開銀鋪的白四哥,慌作揖舉手。張好問便道:「韓老兄,連日少見,聞得恭喜在西門大官府上,開寶鋪做買賣,我等缺禮失賀,休怪休怪!」一面讓他坐下。那韓道國坐在櫈上,把臉兒揚着,手中搖着扇兒,說道:「學生不才,仗賴列位餘光,與我恩主西門大官人做夥計,三七分錢。掌鉅萬之財,督數處之鋪,甚蒙敬重,比他人不同。」白汝晃道:「聞老兄在他門下只做線鋪生意。」韓道國笑道:「二兄不知,線鋪生意只是名目而已。他府上大小買賣,出入資本,那些兒不是學生算帳!言聽計從,禍福共知,通沒我一時兒也成不得。大官人每日衙門中來家擺飯,常請去陪侍,沒我便吃不下飯去。俺兩箇在他小書房裡,閑中吃菓子說話兒,常坐半夜他方進後邊去。昨日他家大夫人生日,房下坐轎子行人情,他夫人留飲至二更方回。彼此通家,再無忌憚。不可對兄說,就是背地他房中話兒,也常和學生計較。學生先一箇行止端莊,立心不苟,與財主興利除害,拯溺救焚。凡百財上分明,取之有道。就是傅自新也怕我幾分。不是我自己誇獎,大官人正喜我這一件兒。」剛說在熱鬧處,忽見一人慌慌張張走向前叫道:「韓大哥,你還在這裡說什麼,教我鋪子裡尋你不着。」{\meipi{湊巧事,天下不少。}}拉到僻靜處告他說:「你家中如此這般,大嫂和二哥被街坊衆人撮弄了,拴到鋪裡,明早要解縣見官去。你還不早尋人情理會此事?」這韓道國聽了,大驚失色。口中只咂嘴,下邊頓足,就要翅趫走。被張好問叫道:「韓老兄,你話還未盡,如何就去了?」這韓道國舉手道:「大官人有要緊事,尋我商議,不及奉陪。」慌忙而去。正是:

\begin{myquote} 
誰人挽得西江水,難洗今朝一面羞。
\end{myquote} 

