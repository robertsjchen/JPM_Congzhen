\includepdf[pages={29,30},fitpaper=false]{tst.pdf}
\chapter*{第十五回 佳人笑賞翫燈樓 狎客幫嫖麗春院}
\addcontentsline{toc}{chapter}{第十五回 佳人笑賞翫燈樓 狎客幫嫖麗春院}
\markboth{{\titlename}卷之二}{第十五回 佳人笑賞翫燈樓 狎客幫嫖麗春院}


詩曰:

\begin{myquote}
樓上多嬌艷,當窻並三五。\\爭弄遊春陌,相邀開綉戶。\\轉態結紅裾,含嬌入翠羽。\\留賓乍拂絃,托意時移柱。
\end{myquote}

話說光陰迅速,又早到正月十五日。西門慶先一日差玳安送了四盤羹菜、一罈酒、一盤壽桃、一盤壽麵、一套織金重絹衣服,寫吳月娘名字,送與李瓶兒做生日。李瓶兒纔起來梳粧,叫了玳安兒到臥房裡,說道:「前日打攪你大娘,今日又教你大娘費心送禮來。」玳安道:「娘多上覆,爹也上覆二娘,{\meipi{下語絕有弄頭,若只曰「爹娘上覆」,便文心死矣。}}不多些微禮,送二娘賞人。」李瓶兒一面分付迎春罷四盤茶食管待玳安。臨出門與二錢銀子、一方閃色手帕:「到家多上覆你家列位娘,我這裡就使老馮拿帖兒來請。好歹明日都要光降走走。」

玳安磕頭出門,兩箇擡盒子的與一百文錢。李瓶兒隨即使老馮拿着五箇柬帖兒,十五日請月娘和李嬌兒、孟玉樓、孫雪娥、潘金蓮,又稍了一箇帖兒,暗暗請西門慶那日晚夕赴席。

月娘到次日,留下孫雪娥看家,同李嬌兒、孟玉樓、潘金蓮四頂轎子出門,都穿着粧花錦綉衣服,來興、來安、玳安、畫童四箇小厮跟隨着,竟到獅子街燈市李瓶兒新買的房子裡來。這房子門面四間,到底三層:臨街是樓;儀門內兩邊廂房,三間客坐,一間梢間;過道穿進去,第三層三間臥房,一間廚房。後邊落地緊靠着喬皇親花園。{\pangpi{伏。}}李瓶兒知月娘衆人來看燈,臨街樓上設放圍屏桌席,懸掛許多花燈。先迎接到客位內,見畢禮數,次讓入後邊明間內待茶,不必細說。到午間,客位內設四張桌席,叫了兩箇唱的——董嬌兒、韓金釧兒——彈唱飲酒。前邊樓上設着細巧添換酒席,又請月娘衆人登樓看燈玩耍。樓簷前掛着湘簾,懸着燈彩。吳月娘穿着大紅粧花通袖襖兒,嬌綠段裙,貂鼠皮襖。李嬌兒、孟玉樓、潘金蓮都是白綾襖兒,藍段裙。李嬌兒是沉香色遍地金比甲,孟玉樓是綠遍地金比甲,潘金蓮是大紅遍地金比甲,頭上珠翠堆盈,鳳釵半卸。俱搭伏定樓窻觀看。那燈市中人烟湊集,十分熱鬧。當街搭數十座燈架,四下圍列諸般買賣,玩燈男女,花紅柳綠,車馬轟雷。但見:

\begin{myquote}
山石穿雙龍戲水,雲霞映獨鶴朝天。金屏燈、玉樓燈見一片珠璣;荷花燈、芙蓉燈散千圍錦綉。綉毬燈,皎皎潔潔,雪花燈,拂拂紛紛。秀才燈,揖讓進止,存孔孟之遺風;{\pangpi{以下俱俚得妙!}}媳婦燈,容德溫柔,效孟姜之節操。和尚燈,月明與柳翠相連;判官燈,鍾馗與小妹並坐。師婆燈,揮羽扇假降邪神,劉海燈,背金蟾戲吞至寶。駱駝燈、青獅燈,馱無價之奇珍;猿猴燈、白象燈,進連城之秘寶。七手八脚螃蠏燈,倒戲清波,巨大口髯鮎魚燈,平吞綠藻。銀蛾鬬彩,雪柳爭輝。魚龍沙戲,七眞五老獻丹書;吊掛流蘇,九夷八蠻來進寶。村裡社鼓,隊隊喧闐;百戲貨郎,樁樁鬬巧。轉燈兒一來一往,吊燈兒或仰或垂。瑠璃瓶映美女奇花,雲母障並瀛州閬苑。王孫爭看小欄下,蹴踘齊雲;仕女相攜高樓上,妖嬈衒色。卦肆雲集,相幙星羅:講新春造化如何,定一世榮枯有準。又有那站高坡打談的,詞曲楊恭;到看這搧響鈸遊脚僧,演說三藏。賣元宵的高堆菓餡,粘梅花的齊插枯枝。剪春娥,鬢邊斜插鬧東風;禱涼釵,頭上飛金光耀日。圍屏畫石崇之錦帳,珠簾繪梅月之雙清。雖然覽不盡鰲山景,也應豐登快活年。
\end{myquote}

月娘看了一回,見樓下人亂,就和李嬌兒各歸席上吃酒去了。惟有潘金蓮、孟玉樓同兩箇唱的,只顧搭伏着樓窻子望下觀看。那潘金蓮一徑把白綾襖袖子兒摟着,顯他那遍地金掏袖兒,露出那十指春蔥來,帶着六箇金馬鐙戒指兒,探着半截身子,口中磕瓜子兒,把磕的瓜子皮兒都吐落在人身上,{\pangpi{奇想。}}和玉樓兩箇嘻笑不止。{\meipi{金蓮輕佻處,曲曲摹盡。}}一回指道:「大姐姐,你來看,那家房簷下掛的兩盞繡毬燈,一來一往,滾上滾下,倒好看。」一回又道:「二姐姐,你來看,這對門架子上,挑着一盞大魚燈,下面還有許多小魚鱉蠏兒,跟着他倒好耍子。」一回又叫:「三姐姐,你看,這首裡這箇婆兒燈,那箇老兒燈。」正看着,忽然一陣風來,把箇婆兒燈下半截割了一箇大窟窿。{\pangpi{趣甚。}}婦人看見,笑箇不了,引惹的那樓下看燈的人,挨肩擦背,仰望上瞧,通擠匝不開,都壓𨇽𨇽兒。內中有幾箇浮浪子弟,直指着談論。一箇說道:「已定是那公侯府裡出來的宅眷。」一箇又猜:「是貴戚王孫家艷妾,來此看燈。不然如何內家粧束?」又一箇說道:「莫不是院中小娘兒?是那大人家叫來這裡看燈彈唱。」又一箇走過來說道:「只我認的,你們都猜不着。這兩箇婦人,也不是小可人家的,他是閻羅大王的妻,五道將軍的妾,是咱縣門前開生藥鋪、放官吏債西門大官人的婦女。你惹他怎的?想必跟他大娘來這裡看燈。這箇穿綠遍地金比甲的,我不認的。{\pangpi{補得妙。}}那穿大紅遍地金比甲兒,上戴着箇翠面花兒的,倒好似賣炊餅武大郎的娘子。大郎因為在王婆茶坊內捉姦,被大官人踢死了。把他娶在家裡做妾。後次他小叔武松告狀,誤打死了皁隸李外傅,被大官人墊發充軍去了。{\meipi{金蓮往事,無意中又閑提一過,前後之脈落俱靈。}}如今一二年不見出來,落的這等標緻了。」正說着,吳月娘見樓下圍的人多了,叫了金蓮、玉樓席坐下,聽着兩箇粉頭彈唱燈詞,飲酒。坐了一回,月娘要起身,說道:「酒勾了,我和二娘先行一步,留下他姊妹兩箇再坐一回兒,以盡二娘之情。今日他爹不在家,家裡無人,光丟着些丫頭們,{\pangpi{「光」字連雪娥在內。}}我不放心。」這李瓶兒那裡肯放,說道:「好大娘,奴沒盡心也是的。今日大節間,燈兒也沒點,飯兒也沒上,就要家去,就是西門爹不在家中,還有他姑娘們哩,怕怎的?待月色上來,奴送四位娘去。」月娘道:「二娘,不是這等說。我又不大十分用酒,留下他姊妹兩箇,就同我一般。」李瓶兒道:「大娘不用,二娘也不吃一鍾,也沒這箇道理。想奴前日在大娘府上,那等鍾鍾不辭,衆位娘竟不肯饒我。今日來到奴這湫窄之處,雖無甚物供獻,也盡奴一點勞心。」於是拿大銀鍾遞與李嬌兒,說道:「二娘好歹吃一盃兒。大娘,奴不敢奉大盃,只奉小盃兒罷。」於是滿斟遞與月娘。兩箇唱的,月娘每人與他二錢銀子。待的李嬌兒吃過酒,月娘就起身,又囑咐玉樓、金蓮道:「我兩箇先去,就使小厮拿燈籠來接你們,也就來罷。家裡沒人。」玉樓應諾。李瓶兒送月娘、李嬌兒到門首,上轎去了。歸到樓上,陪玉樓、金蓮飲酒,看看天晚,樓上點起燈來,兩箇唱的彈唱飲酒,不在話下。

卻說西門慶那日同應伯爵、謝希大兩箇,{\meipi{不說到金蓮席散,便叙西門慶,此文家勾搭之妙。}}家中吃了飯,同往燈市裡遊玩。到了獅子街東口,西門慶因為月娘衆人都在李瓶兒家吃酒,恐怕他兩箇看見,就不往西街去看大燈,只到賣紗燈的跟前就回了。不想轉過灣來,撞遇孫寡嘴、祝實念,唱喏說道:「連日不會哥,心中渴想。」見了應伯爵、謝希大罵道:「你兩箇天殺的好人兒,你來和哥遊玩,就不說叫俺一聲兒!」西門慶道:「祝兄弟,你錯怪了他兩箇,剛纔也是路上相遇。」祝實念道:「如今看了燈往那裡去?」西門慶道:「同衆位兄弟到大酒樓上吃三盃兒,不是也請衆兄弟家去,今日房下們都往人家吃酒去了。」祝實念道:「比是哥請俺每到酒樓上,何不往裡邊望望李桂姐去?只當大節間拜拜年,去混他混。前日俺兩箇在他家,他望着俺們好不哭哩!說他從臘裡不好到如今,大官人通影邊兒不進去看他看。哥今日倒閑,俺們情願相伴哥進去走走。」西門慶因記掛晚夕李瓶兒有約,故推辭道:「今日我還有小事,明日去罷。」怎禁這夥人死拖活拽,於是同進院中去。正是:

\begin{myquote}
柳底花陰壓路塵,一回遊賞一回新。\\不知買盡長安笑,活得蒼生幾戶貧?
\end{myquote}

西門慶同衆人到了李家,桂卿正打扮着在門首站立,一面迎接入中堂相見了。祝實念就高叫道:{\pangpi{酷肖。}}「快請三媽出來!還虧俺衆人,今日請的大官人來了。」少頃,老虔婆扶拐而出,與西門慶見禮畢,說道:「老身又不曾怠慢了姐夫,如何一向不進來看看姐兒?想必別處另叙了新表子來。」祝實念插口道:「你老人家會猜算,俺大官人近日相了箇絕色的表子,每日只在那裡走,不想你家桂姐兒。剛纔不是俺二人在燈市裡撞見,拉他來,他還不來哩!媽不信,問孫伯脩就是了。」因指着應伯爵、謝希大說道:「這兩箇天殺的,和他都是一路神只。」老虔婆聽了,哈哈笑道:「好應二哥,俺家沒惱着你,如何不在姐夫面前美言一句兒?雖故姐夫裡邊頭絮兒多,常言道:『好子弟不嫖一箇粉頭』,『天下錢眼兒都一樣』。{\pangpi{忽說出心事,妙。}}不是老身誇口說,我家桂姐也不醜,姐夫自有眼,今也不消人說。」孫寡嘴道:「我是老寔說,哥如今新叙的這箇表子,不是裡面的,是外面的表子。」{\pangpi{微詞,妙。}}西門慶聽了,趕着孫寡嘴只顧打,說道:「老媽,你休聽這天災人禍的老油嘴,老殺才!」孫寡嘴和衆人笑成一塊。

西門慶向袖中掏出三兩銀子來,遞與桂卿:「大節間,我請衆朋友。」桂卿不肯接,遞與老媽。老媽說道:「怎麼的?姐夫就笑話我家,大節下拿不出酒菜兒管待列位老爹?又教姐夫壞鈔,拿出銀子。顯的俺們院裡人家只是愛錢了。」應伯爵走過來說道:「老媽,你依我收了,快安排酒來俺們吃。」那虔婆說道:「這箇理上卻使不得。」一壁推辭,一壁把銀子接來袖了,深深道了箇萬福,說道:「謝姐夫的布施。」{\pangpi{好名色。}}應伯爵道:「媽,你且住。我說箇笑話兒你聽:{\pangpi{又頓一頓,妙甚。}}一箇子弟在院中嫖小娘兒。那一日做耍,裝做貧子進去。老媽見他衣服襤褸,不理他。坐了半日,茶也不拿出來。子弟說:『媽,我肚飢,有飯尋些來吃。』老媽道:『米囤也晒,那討飯來?』子弟又道:『既沒飯,有水拿些來,我洗臉。』老媽道:『少挑水錢,連日沒送水來。』這子弟向袖中取出十兩一錠銀子,放在桌上,教買米顧水去。慌的老媽沒口子道:『姐夫吃了臉洗飯,洗了飯吃臉!』」把衆人都笑了。虔婆道:「你還是這等快取笑,可哥兒的來,自古有恁說沒這事。」{\pangpi{只不認泛,妙。}}應伯爵道:「你拿耳朵來,我對你說:大官人新近請了花二哥表子——{\pangpi{雙關語,驚人,妙。}}後巷的吳銀兒了,不要你家桂姐哩!」虔婆笑道:「我不信,俺桂姐今日不是強口,比吳銀兒還比得過。我家與姐夫是快刀兒割不斷的親戚。{\meipi{又映李嬌兒,文情深冷之至。}}姐夫是何等人兒?他眼裡見得多,着緊處,金子也估出箇成色來!」說畢,入去收拾酒菜去了。

少頃,李桂姐出來,家常挽着一窩絲杭州攢,金縷絲釵,翠梅花鈿兒,珠子箍兒,金籠墜子,上穿白綾對襟襖兒,下着紅羅裙子,打扮的粉粧玉琢,望下道了萬福,與桂卿一邊一箇打橫坐下。須臾,泡出茶來,桂卿、桂姐每人遞了一盞,陪着吃畢。保兒就來打抹春臺,纔待收拾擺放案酒,忽見簾子外探頭舒腦,有幾箇穿襤褸衣者——謂之架兒,{\pangpi{絕,有生髮。}}進來跪下,手裡拿着三四升瓜子兒:「大節間,孝順大老爹。」西門慶只認頭一箇叫于春兒,問:「你們那幾箇在這裡?」於春道:「還有段綿紗、青聶鉞,在外邊伺候。」段綿紗進來,看見應伯爵在裡,說道:「應爹也在這裡。」連忙磕了頭。西門慶分付收了他瓜子兒,開啟銀包兒,捏一兩一塊銀子掠在地下。于春兒接了,和衆人扒在地下磕了箇頭,說道:「謝爹賞賜。」往外飛跑。有《朝天子》單道架兒行藏:

\begin{myquote}
這家子打和,那家子撮合。他的本分少,虛頭大,一些兒不巧又騰挪,遶院裡都踅過。席面上幫閒,把牙兒閑磕。攘一回纔散火,{\pangpi{妙。}}賺錢又不多。歪厮纏怎麼?他在虎口裡求津唾。
\end{myquote}

西門慶打發架兒出門,安排酒上來吃。桂姐滿泛金盃,雙垂紅袖,餚烹異品,菓獻時新,倚翠偎紅,花濃酒艷。酒過兩巡,桂卿、桂姐一箇彈箏,一箇琵琶,兩箇彈着唱了一套「霽景融和」。正唱在熱鬧處,見三箇穿青衣黃板鞭者——謂之圓社,手裡捧着一隻燒鵝,提着兩瓶老酒,大節間來孝順大官人,向前打了半跪。西門慶平昔認的,一箇喚白禿子,一箇喚小張閑,一箇是羅回子,{\meipi{命名妙甚,宛然二人在目。}}因說道:「你們且外邊候候,待俺們吃過酒,踢三跑。」於是向桌子上拾了四盤嗄飯、一大壺酒、一碟點心,打發衆圓社吃了,整理氣毬伺候。西門慶吃了一回酒,出來外面院子裡,先踢了一跑。次教桂姐上來,與兩箇圓社踢。一箇揸頭,一箇對障,勾踢拐打之間,無不假喝彩奉承。就有些不到處,都快取過去了。反來向西門慶面前討賞錢,說:「桂姐的行頭,就數一數二的,強如二條巷董官女兒數十倍。」當下桂姐踢了兩跑下來,使的塵生眉畔,汗濕腮邊,氣喘吁吁,腰肢睏乏。袖中取出春扇兒搖涼,與西門慶攜手,看桂卿與謝希大、張小閑踢行頭。白禿子、羅回子在旁虛撮脚兒等漏,往來拾毛。亦有《朝天子》一詞,單表這踢圓的始末:

\begin{myquote}
在家中也閑,到處刮涎,生理全不幹,{\meipi{只此便是生理。}}氣毬兒不離在身邊,每日街頭站。窮的又不趨,富貴他偏羨。從早晨只到晚,不得甚飽餐。轉不得大錢,他老婆常被人包占。
\end{myquote}

西門慶正看着衆人在院內打雙陸、踢氣毬,飲酒,只見玳安騎馬來接,悄悄附耳低言道:「大娘、二娘家去了。花二娘叫小的請爹早些過去哩!」這西門慶聽了,暗暗叫玳安:「把馬弔在後門邊,等着我。」於是酒也不吃,拉桂姐到房中,只坐了一回兒,就出來推淨手,於後門上馬,一溜烟走了。應伯爵使保兒去拉扯,西門慶只說:「我家裡有事。」那裡肯轉來!教玳安兒拿了一兩五錢銀子打發三箇圓社。李家恐怕他又往後巷吳銀兒家去,使丫鬟直跟至院門首方回。應伯爵等衆人,還吃到二更纔散。正是:

\begin{myquote}
笑罵由他笑罵,歡娛我且歡娛。
\end{myquote}

