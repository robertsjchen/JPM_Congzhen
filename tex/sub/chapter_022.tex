\includepdf[pages={43,44},fitpaper=false]{tst.pdf}
\chapter*{第二十二回 蕙蓮兒偷期蒙愛 春梅姐正色閑邪}
\addcontentsline{toc}{chapter}{第二十二回 蕙蓮兒偷期蒙愛 春梅姐正色閑邪}
\markboth{{\titlename}卷之三}{第二十二回 蕙蓮兒偷期蒙愛 春梅姐正色閑邪}


詞曰:

\begin{myquote} 
今宵何夕?月痕初照。等閒間一見猶難,平白地兩邊湊巧。向燈前見他,向燈前見他,一似夢中來到。何曾心料,他怕人瞧。驚臉兒紅還白,熱心兒火樣燒。

\raggedleft{——右調《桂枝香》\rightquadmargin}
\end{myquote} 

話說次日,有吳大妗子、楊姑娘、潘姥姥衆堂客,因來與孟玉樓做生日,月娘都留在後廳飲酒,其中惹出一件事兒。那來旺兒,因他媳婦癆病死了,月娘新又與他娶了一房媳婦,乃是賣棺材宋仁的女兒,也名喚金蓮。{\pangpi{便妙。}}當先賣在蔡通判家房裡使喚,後因壞了事出來,嫁與廚役蔣聰為妻。這蔣聰常在西門慶家答應,來旺兒早晚到蔣聰家叫他去,看見這箇老婆,兩箇吃酒刮言,就把這箇老婆刮上了。一日,不想這蔣聰因和一般廚役分財不均,酒醉厮打,動起刀杖來,把蔣聰戳死在地,那人便越墻逃走了。老婆央來旺兒對西門慶說了,替他拿帖兒縣裡和縣丞說,差人捉住正犯,問成死罪,抵了蔣聰命。{\meipi{蕙蓮肯為蔣聰報仇,雖淫亦當正論。}}後來,來旺兒哄月娘,只說是小人家媳婦兒,會做針指。月娘使了五兩銀子,兩套衣服,四疋青紅布,並簪環之類,娶與他為妻。月娘因他叫金蓮,不好稱呼,遂改名為蕙蓮。這箇婦人小金蓮兩歲,今年二十四歲,生的白淨,身子兒不肥不瘦,模樣兒不短不長,比金蓮脚還小些兒。性明敏,善機變,會粧飾,{\pangpi{要緊。}}就是嘲漢子的班頭,壞家風的領袖。若說他底的本事,他也曾:

\begin{myquote} 
斜倚門兒立,人來側目隨。\\托腮並咬指,無故整衣裳。\\坐立頻搖腿,無人曲唱低。\\開窻推戶牖,停針不語時。{\pangpi{恐不至此。}}\\未言先欲笑,必定與人私。
\end{myquote} 

初來時,同衆媳婦上竈,還沒甚麼粧飾。後過了箇月有餘,因看見玉樓、金蓮打扮,他便把鬏髻墊的高高的,頭髮梳的虛籠籠的,水𩬆描的長長的,{\meipi{雖非婢學夫人,卻亦漸入佳境。}}在上邊遞茶遞水,被西門慶睃在眼裡。{\pangpi{不放過。}}一日,設了條計策,教來旺兒押了五百兩銀子,往杭州替蔡太師製造慶賀生辰錦綉蟒衣,並家中穿的四季衣服,往回也有半年期程。從十一月半頭,搭在旱路車上起身去了。西門慶安心早晚要調戲他這老婆。不期到此正値孟玉樓生日,月娘和衆堂客在後廳吃酒,西門慶那日沒往那去。月娘分咐玉簫:「房中另放桌兒,打發酒菜你爹吃。」西門慶因打簾內看見蕙蓮身上穿着紅紬對襟襖、紫絹裙子,在席上斟酒,問玉簫道:「那箇是新娶的來旺兒的媳婦子蕙蓮?怎的紅襖配着紫裙子,怪模怪樣。{\meipi{見怪不怪方妙,一見怪,則着鬼矣。}}到明日對你娘說,另與他一條別的顏色裙子配着穿。」玉簫道:「這紫裙子,還是問我借的。」說着就罷了。須臾,過了玉樓生日。一日,月娘往對門喬大戶家吃酒去了。約後晌時分,西門慶從外來家,已有酒了,走到儀門首,這蕙蓮正往外走,兩箇撞箇滿懷。西門慶便一手摟過脖子來,就親了箇嘴,口中喃喃吶吶說道:「我的兒,你若依了我,頭面衣服,隨你揀着用。」{\meipi{純以利動之,已落第二義。}}那婦人一聲兒沒言語,推開西門慶手,一直往前走了。西門慶歸到上房,叫玉簫送了一疋藍段子到他屋裡,如此這般對他說:「爹昨日見你穿着紅襖,配着紫裙子,怪模怪樣的不好看,纔拿了這疋段子,使我送與你,教你做裙子穿。」這蕙蓮開看,卻是一疋翠藍兼四季團花喜相逢段子。說道:「我做出來,娘見了問,怎了?」玉簫道:「爹到明日還對娘說,你放心。爹說來,你若依了這件事,隨你要甚麼,爹與你買。今日趕娘不在家,要和你會會兒,你心下如何?」那婦人聽了,微笑不言,因問:「爹多咱時分來?我好在屋裡伺候。」玉簫道:「爹說小厮們看着,不好進你屋裡來的。教你悄悄往山子底下洞兒裡,那裡無人,堪可一會。」老婆道:「只怕五娘、六娘知道了,不好意思的。」玉簫道:「三娘和五娘都在六娘屋裡下棋,你去不妨事。」當下約會已定,玉簫走來回西門慶說話。兩箇都往山子底下成事,玉簫在門首與他觀風。正是:

\begin{myquote} 
解帶色已戰,{\pangpi{未必。}}觸手心愈忙。\\
那識羅裙內,銷魂別有香。
\end{myquote} 

不想金蓮、玉樓都在李瓶兒房裡下棋,只見小鸞來請玉樓,說:「爹來家了。」三人就散了,玉樓回後邊去了。金蓮走到房中,勻了臉,{\pangpi{不漏。}}亦往後邊來。走入儀門,只見小玉立在上房門首。金蓮問:「你爹在屋裡?」小玉搖手兒,往前指。{\pangpi{畫。}}金蓮就知其意,走到前邊山子角門首,只見玉簫攔着門。金蓮只猜玉簫和西門慶在此私狎,便頂進去。玉簫慌了,說道:「五娘休進去,爹在裡頭有勾當哩!」金蓮罵道:「怪狗肉,我又怕你爹了?」不繇分說,進入花園裡來,各處尋了一遍。走到藏春塢山子洞兒裡,只見他兩箇人在裡面纔了事。婦人聽見有人來,連忙繫上裙子往外走,看見金蓮,把臉通紅了。金蓮問道:「賊臭肉,你在這裡做甚麼?」蕙蓮道:「我來叫畫童兒。」

說着,一溜烟走了。金蓮進來,看見西門慶在裡邊繫褲子,罵道:「賊沒廉恥的貨,你和奴才淫婦大白日裡在這裡,端的幹這勾當兒,剛纔我打與淫婦兩箇耳刮子纔好,不想他往外走了。原來你就是畫童兒,他來尋你!你與我寔說,和這淫婦偷了幾遭?若不寔說,等住回大姐姐來家,看我說不說。我若不把奴才淫婦臉打的脹豬,也不算。俺們閑的聲喚在這裡,你也來插上一把子。{\meipi{氣妒語,妙在說得帶幾分無恥,以見為淫也,非為情也。}}老娘眼裡卻放不過!」西門慶笑道:「怪小淫婦兒,悄悄兒罷,休要嚷的人知道。我實對你說,如此這般,連今日纔第一遭。」金蓮道:「一遭二遭,我不信。你既要這奴才淫婦,兩箇瞞神謊鬼弄刺子兒,我打聽出來,休怪了我卻和你們答話!」那西門慶笑的出去了。金蓮到後邊,聽見衆丫頭們說:「爹來家,使玉簫手巾裹着一疋藍段子往前邊去,不知與誰。」金蓮就知是與蕙蓮的,對玉樓也不題起此事。

這婦人每日在那邊,或替他造湯飯,或替他做針指鞋脚,或跟着李瓶兒下棋,常賊乖趨附金蓮。被西門慶撞在一處,無人,教他兩箇苟合,圖漢子喜歡。蕙蓮自從和西門慶私通之後,背地與他衣服、首飾、香茶之類不算,只銀子成兩家帶在身邊,在門首買花翠胭脂,漸漸顯露,打扮的比往日不同。西門慶又對月娘說,他做的好湯水,不教他上大竈,只教他和玉簫兩箇,在月娘房裡後邊小竈上,專頓茶水,整理菜蔬,打發月娘房裡吃飯,與月娘做針指,不必細說。看官聽說:凡家主,切不可與奴僕並家人之婦苟且私狎,久後必紊亂上下,竊弄奸欺,敗壞風俗,殆不可制。

一日,臘月初八日,西門慶早起,約下應伯爵,與大街坊尚推官家送殯。叫小厮馬也備下兩匹,等伯爵,白不見到,一面李銘來了。西門慶就在大廳上圍爐坐的,教春梅、玉簫、蘭香、迎春一般兒四箇,都打扮出來,看着李銘指撥、教演他彈唱。女婿陳敬濟,在傍陪着說話。正唱《三弄梅花》,還未了,只見伯爵來,應保夾着氊包進門。那春梅等四箇就要往後走,被西門慶喝住,說道:「左右只是你應二爹,都來見見罷,躲怎的!」與伯爵兩箇相見作揖,纔待坐下,西門慶令四箇過來:「與應二爹磕頭。」那春梅等朝上磕頭下去,慌的伯爵還喏不迭,誇道:「誰似哥有福,出落的恁四箇好姐姐,水蔥兒的一般,一箇賽一箇。卻怎生好?你應二爹今日素手,促忙促急,沒曾帶的甚麼在身邊,改日送胭脂錢來罷。」春梅等四人,見了禮去了。陳敬濟向前作揖,一同坐下。西門慶道:「你如何今日這咱纔來?」應伯爵道:「不好告訴你的。大小女病了一向,近日纔好些。房下記掛着,今日接了他家來散心住兩日。亂着,旋叫應保叫了轎子,買了些東西在家,我纔來了。」{\meipi{如在,是伯爵家事。}}西門慶道:「教我只顧等着你。咱吃了粥,好去了。」隨即分付後邊看粥來吃。只見李銘見伯爵打了半跪。伯爵道:「李日新,一向不見你。」李銘道:「小的有。連日小的在北邊徐公公那裡答應來。」說着,小厮放桌兒,拿粥來吃。西門慶陪應伯爵、陳敬濟吃了。就拿小銀鍾篩金華酒,每人吃了三盃。壺裡還剩下上半壺酒,分付畫童兒:「連桌兒擡去廂房內,與李銘吃。」{\meipi{偏在絕沒要緊弄巧,一味文心細冷。}}就穿衣服起身,同伯爵並馬而行,與尚推官送殯去了。只落下李銘在西廂房,吃畢酒飯。玉簫和蘭香衆人,打發西門慶出了門,在廂房內厮亂,頑成一塊。{\pangpi{必至之情。}}一回,都往對過東廂房西門大姐房裡摑混去了,止落下春梅一箇,和李銘在這邊教演琵琶。李銘也有酒了。春梅袖口子寬,把手兜住了。李銘把他手拿起,畧按重了些。被春梅怪叫起來,{\meipi{寫得似有意,似無意,以見半是春梅之性燥也。}}罵道:「好賊忘八!你怎的撚我的手調戲我?賊少死的忘八,你還不知道我是誰哩!{\pangpi{自負不卑。}}一日好酒好肉,越發養活的你這忘八靈聖兒出來了,{\pangpi{罵得妙。}}平白撚我的手來了。賊忘八,你錯下這箇鍬撅了。你問聲兒去,我手裡你來弄鬼!爹來家等我說了,把你這賊忘八,一條棍攆的離門離戶!沒你這忘八,學不成唱了?愁本司三院尋不出忘八來?撅臭了你這忘八了!」被他千忘八,萬忘八,罵的李銘拿着衣服,往外走不迭。正是:

\begin{myquote} 
兩手劈開生死路,翻身跳出是非門。
\end{myquote} 

當下春梅氣狠狠,直罵進後邊來。金蓮正和孟玉樓、李瓶兒並宋蕙蓮在房裡下棋,只聽見春梅從外罵將來。金蓮便問道:「賊小肉兒,你罵誰哩,誰惹你來?」春梅道:「情知是誰,叵耐李銘那忘八!爹臨去,好意分付小厮,留下一桌菜並粳米粥兒與他吃。也有玉簫他們,你推我,我打你,頑成一塊,對着忘八,呲牙露嘴的,狂的有些褶兒也怎的。頑了一回,都往大姐那邊去了。忘八見無人,盡力把我手上撚一下。吃的醉醉的,看着我嗤嗤待笑。{\pangpi{寫得出。}}那忘八見我喓喝罵起來,他就夾着衣裳往外走了。剛纔打與賊忘八兩箇耳刮子纔好!賊忘八,你也看箇人兒行事,我不是那不三不四的邪皮行貨,教你這箇忘八在我手裡弄鬼。我把忘八臉打綠了!」金蓮道:「怪小肉兒,學不學沒要緊,把臉氣的黃黃的,等爹來家說了,把賊忘八攆了去就是了。那裡緊等着供唱賺錢哩,{\meipi{金蓮愛惜春梅至矣,故後感之不忘。}}怎的教忘八調戲我這丫頭!我知道賊忘八業礶子滿了。」春梅道:「他就倒運,着量二娘的兄弟。{\meipi{偏照映得到。}}那怕他!二娘莫不挾仇打我五棍兒?」宋蕙蓮道:「論起來,你是樂工,在人家教唱,也不該調戲良人家女子!照顧你一箇錢,也是養身父母,休說一日三茶六飯兒扶侍着。」金蓮道:「扶侍着?臨了還要錢兒去了。按月兒,一箇月與他五兩銀子。賊忘八,錯上了墳。你問聲家裡這些小厮們,那箇敢望着他呲牙笑一笑兒,弔箇嘴兒?遇喜歡罵兩句;若不歡喜,拉倒他主子跟前就是打。賊忘八,造化低,你惹他生薑,你還沒曾經着他辣手!」{\meipi{一味為春梅作聲價。}}因向春梅道:「沒見你,你爹去了,你進來便罷了,平白只顧和他那房裡做甚麼?卻教那忘八調戲你!」春梅道:「都是玉簫和他們,只顧還笑成一塊,不肯進來。」玉樓道:「他三箇如今還在那屋裡?」春梅道:「都往大姐房裡去了。」玉樓道:「等我瞧瞧去。」那玉樓起身去了。良久,李瓶兒亦回房,使綉春叫迎春去。至晚,西門慶來家,金蓮一五一十告訴西門慶。西門慶分付來興兒,今後休放進李銘來走動。自此斷了路兒,不敢上門。正是:

\begin{myquote} 
習教歌妓逞家豪,每日閒庭弄錦槽。\\不是朱顏容易變,何繇聲價競天高。
\end{myquote} 

