\part*{{\titlename}卷之八}
\addcontentsline{toc}{part}{{\titlename}卷之八}


\includepdf[pages={141,142},fitpaper=false]{tst.pdf}
\chapter*{第七十一回 李瓶兒何家托夢 提刑官引奏朝儀}
\addcontentsline{toc}{chapter}{第七十一回 李瓶兒何家托夢 提刑官引奏朝儀}
\markboth{{\titlename}卷之八}{第七十一回 李瓶兒何家托夢 提刑官引奏朝儀}


詞曰:

\begin{myquote}
花事闌珊芳草歇,客裡風光,又過些時節。小院黃昏人憶別,淚痕點點成紅血。咫尺江山分楚越,目斷神驚,只道芳魂絕。夢破五更心欲折,角聲吹落梅花月。

\raggedleft{——右調《蝶戀花》\rightquadmargin}
\end{myquote}

話說西門慶同何千戶回來,走到大街,何千戶就邀請西門慶到家一飯。西門慶再三固辭。何千戶令手下把馬環拉住,說道:「學生還有一事與長官商議。」於是並轡同到宅前下馬。賁四同擡盒逕往崔中書家去了。原來何千戶盛陳酒筵在家等候。進入廳上,但見獸炭焚燒,金爐香靄。正中獨設一席,下邊一席相陪,旁邊東首又設一席。皆盤堆異菓,花插金瓶。西門慶問道:「長官今日筵何客?」何千戶道:「家公公今日下班,敢屈長官一飯。」西門慶道:「長官這等費心,就不是同僚之情。」何千戶道:「家公公粗酌屈尊,長官休怪。」一面看茶吃了。西門慶請老公公拜見,何千戶道:「家公公便出來。」不一時,何太監從後邊出來,穿着綠絨蟒衣,冠帽皁鞋,寶石縧環。西門慶展拜四拜:「請公公受禮。」何太監不肯,說道:「使不的。」{\pangpi{禮應如此,但有托與他,焉得受禮。}}西門慶道:「學生與天泉同寅晚輩,老公公齒德俱尊,又系中貴,自然該受禮。」講了半日,何太監受了半禮,讓西門慶上坐,他主席相陪,何千戶旁坐。西門慶道:「老公公,這箇斷然使不得。同僚之間,豈可旁坐!老公公叔侄便罷了,學生使不的。」何太監大喜道:「大人甚是知禮,{\meipi{內臣心性口角,如聞如睹。}}罷罷,我閣老位兒旁坐罷,教做官的陪大人就是了。」西門慶道:「這等,學生坐的也安。」於是各照位坐下。何太監道:「小的兒們,{\pangpi{酷肖。}}再燒了炭來。今日天氣甚是寒冷。」須臾,左右火池火叉,拿上一包水磨細炭,向火盆內只一倒。廳前放下油紙煖簾來,日光掩映,十分明亮。何太監道:「大人請寬了盛服罷。」西門慶道:「學生裡邊沒穿甚麼衣服,使小价下處取來。」何太監道:「不消取去。」令左右接了衣服,「拿我穿的飛魚綠絨氅衣來,與大人披上。」西門慶笑道:「老先生職事之服,學生何以穿得?」{\meipi{淡淡一語,寫出名分之爛。}}何太監道:「大人只顧穿,怕怎的!昨日萬歲賜了我蟒衣,我也不穿他了,就送了大人遮衣服兒罷。」不一時,左右取上來,西門慶令玳安接去員領,披上氅衣,作揖謝了。又請何千戶也寬去上蓋陪坐。又拿上一道茶來吃了,何太監道:「叫小厮們來。」{\pangpi{此內相家所必有。}}{\meipi{的是內相口中話,一字挪移不動。}}原來家中教了十二名吹打的小厮,兩箇師範領着上來磕頭。何太監就分付動起樂來,然後遞酒上坐。何太監親自把盞,西門慶慌道:「老公公請尊便。有長官代勞,只安放鍾筯兒就是一般。」何太監道:「我與大人遞一鍾兒。我家做官的初入蘆葦,不知深淺,望乞大人凡事扶持一二,就是情了。」{\pangpi{映後同往一事。}}西門慶道:「老公公說那裡話!常言:同僚三世親。學生亦托賴老公公餘光,豈不同力相助!」{\meipi{西門慶處世情,亦頗在行。}}何太監道:「好說,好說。共同王事,彼此扶持。」西門慶也沒等他遞酒,只接了盃兒,領到席上,隨即回奉一盃,安在何千戶並何太監席上,彼此告揖過,坐下。吹打畢,三箇小厮連師範,在筵前銀箏象板,三絃琵琶,唱了一套《正宮•端正好》「雪夜訪趙普」、「水晶宮鮫綃帳」。唱畢下去。

酒過數巡,食割兩道,看看天晚,秉上燈來。西門慶喚玳安拿賞賜與廚役並吹打各色人役,就起身,說道:「學生厚擾一日了,就此告回。」那公公那裡肯放,說道:「我今日正下班,要與大人請教。有甚大酒席,只是清坐而已,教大人受飢。」西門慶道:「承老公公賜這等美饌,如何反言受飢!學生回去歇息歇息,明早還要與天泉參謁參謁兵科,好領箚付掛號。」何太監道:「既是大人要與我家做官的同幹事,何不令人把行李搬過來我家住兩日?我這後園兒裡有幾間小房兒,甚是僻靜,就早晚和做官的理會些公事兒也方便些,強如在別人家。」西門慶道:「在這裡最好,只是使夏公見怪,相學生疎他一般。」何太監道:「沒的說。如今時年,早晨不做官,晚夕不唱喏,衙門是恁偶戲衙門。{\pangpi{見道語。}}雖故當初與他同僚,今日前官已去,後官接管承行,與他就無干。他若這等說,他就是箇不知道理的人了。{\meipi{世情即是道理,信口說破,覺翟公書門、孟嘗唾面,俱見之晚也。}}今日我定要和大人坐一夜,不放大人去。」喚左右:「下邊房裡快放桌兒,管待你西門老爹大官兒飯酒。我家差幾箇人,跟他即時把行李都搬了來。」又分付:「打掃後花園西院乾淨,預備鋪陳,炕中籠下炭火。」堂上一呼,堦下百諾,答應下去了。西門慶道:「老公公盛情,只是學生得罪夏公了。」何太監道:「他既出了衙門,不在其位,不謀其政。他管他那鑾駕庫的事,{\pangpi{愈淺愈眞。}}管不的咱提刑所的事了。難怪於你。」不繇分說,就打發玳安並馬上人吃了酒飯,差了幾名軍牢,各拿繩扛,逕往崔中書家搬取行李去了。何太監道:「又一件相煩大人:我家做官的到任所,還望大人替他看所宅舍兒,好搬取家小。今先教他同大人去,待尋下宅子,然後打發家小起身。也不多,連幾房家人也只有二三十口。」西門慶道:「老公公分付,要看多少銀子宅舍?」何太監道:「也得千金外房兒纔勾住。」西門慶道:「夏龍溪他京任不去了,他一所房子倒要打發,老公公何不要了與天泉住,一舉兩得其便。此宅門面七間,到底五層,儀門進去大廳,兩邊廂房,鹿角頂,後邊住房、花亭,周圍羣房也有許多,街道又寬闊,正好天泉住。」何太監道:「他要許多價値兒?」西門慶道:「他對我說原是一千三百兩,又後邊添蓋了一層平房,收拾了一處花亭。老公公若要,隨公公與他多少罷了。」何太監道:「我托大人,隨大人主張就是了。趁今日我在家,差箇人和他說去,討他那原文書我瞧瞧。難得尋下這房舍兒,我家做官的去到那裡,就有箇歸着了。」

不一時,只見玳安同衆人搬了行李來回話。西門慶問:「賁四、王經來了不曾?」玳安道:「王經同押了衣箱行李先來了。還有轎子,叫賁四在那裡看守着哩。」西門慶因附耳低言:「如此這般上覆夏老爹,借過那裡房子的原契來,何公公要瞧瞧。就同賁四一答兒來。」這玳安應的去了。不一時,賁四青衣小帽,同玳安拿文書回西門慶說:「夏老爹多多上覆:既是何公公要,怎好說價錢!原文書都拿的來了。又收拾添蓋,使費了許多,隨爹主張了罷。」西門慶把原契遞與何太監親看了一遍,見上面寫着一千二百兩,說道:「這房兒想必也住了幾年,未免有些糟爛,也別要說收拾,大人面上還與他原價。」那賁四連忙跪下說:「何爺說的是。自古道:使的憨錢,治的庄田。千年房舍換百主,一番拆洗一番新。」何太監聽了喜歡道:{\meipi{此數語何足喜?而何太監喜之,所謂內臣心性也。}}「你是那裡人?倒會說話兒。常言成大事者不惜小費,其寔說的是。他教甚麼名字?」西門慶道:「他名喚賁四。」何太監道:「也罷,沒箇中人兒,你就做箇中人兒,替我討了文書來。今日是箇好日期,就把銀子兌與他罷。」{\meipi{夫下事皆有如此做,何患叢挫?}}西門慶道:「如今晚了,待的明日也罷了。」何太監道:「到五更我早進去,明日大朝。今日不如先交與他銀子,就了事。」西門慶問道:「明日甚時駕出?」何太監道:「子時駕出到壇,三更鼓祭了,寅正一刻就回宮。擺了膳,就出來設朝,陞大殿,朝賀天下,諸司都上表拜冬。{\meipi{只就時刻寥寥數語,而皇家氣象宛然。}}次日,文武百官吃慶成宴。你每是外任官,大朝引奏過就沒事了。」說畢,何太監分付何千戶進後邊,打點出二十四錠大元寶來,用食盒擡着,差了兩箇家人,同賁四、玳安押送到崔中書家交割。夏公見擡了銀子來,滿心歡喜,隨即親手寫了文契,付與賁四等,拿來遞上。何太監不勝歡喜,賞了賁四十兩銀子,玳安、王經每人三兩。西門慶道:「小孩子家,不當賞他。」何太監道:「胡亂與他買嘴兒吃。」三人磕頭謝了。何太監分付管待酒飯,又向西門慶唱了兩箇喏:「全仗大人餘光。」西門慶道:「還是看老公公金面。」何太監道:「還望大人對他說說,早把房兒騰出來,就好打發家小起身。」西門慶道:「學生已定與他說,教他早騰。長官這一去,且在衙門公廨中權住幾日。待他家小搬到京,收拾了,長官寶眷起身不遲。」何太監道:「收拾直待過年罷了,先打發家小去纔好。十分在衙門中也不方便。」說話之間,已有一更天氣,西門慶說道:「老公公請安置罷!學生亦不勝酒力了。」何太監方作辭歸後邊歇息去了。何千戶教家樂彈唱,還與西門慶吃了一回,方纔起身,送至後園。三間書院,臺榭湖山,盆景花木,房內絳燭高燒,篆內香焚麝餅,十分幽雅。何千戶陪西門慶叙話,又看茶吃了,方道安置,歸後邊去了。

西門慶摘去冠帶,解衣就寢。王經、玳安打發了,就往下邊煖炕上歇去了。西門慶有酒的人,睡在枕畔,見滿窻月色,翻來覆去。良久只聞夜漏沉沉,花陰寂寂,寒風吹得那窻紙有聲,況離家已久。正要呼王經進來陪他睡,忽聽得窻外有婦人語聲甚低,即披衣下牀,靸着鞋襪,悄悄啟戶視之。只見李瓶兒霧𩬆雲鬟,淡粧麗雅,素白舊衫籠雪體,淡黃軟襪襯弓鞋,輕移蓮步,立於月下。{\meipi{以瓶兒之事,死見子虛於地下,方且慚愧,謝罪改過不遑,乃猶眷西門慶,與子虛為仇如此,可見淫婦人一種癡情,雖鬼神亦無如之何矣。}}西門慶一見,挽之入室,相抱而哭,說道:「冤家,你如何在這裡?」李瓶兒道:「奴尋訪至此。對你說,我已尋了房兒了,今特來見你一面,早晚便搬去了。」西門慶忙問道:「你房兒在於何處?」李瓶兒道:「咫尺不遠。出此大街迤東造釜巷中間便是。」言訖,西門慶共他相偎相抱,上床雲雨,不勝美快之極。已而整衣扶髻,徘徊不捨。李瓶兒叮嚀囑咐西門慶道:「我的哥哥,切記休貪夜飲,早早回家。那厮不時伺害於你,千萬勿忘!」言訖,挽西門慶相送。走出大街上,見月色如晝,果然往東轉過牌坊,到一小巷,見一座雙扇白板門,指道:「此奴之家也。」言畢,頓袖而入。西門慶急向前拉之,恍然驚覺,乃是南柯一夢。但見月影橫窻,花枝倒影矣。{\meipi{寫夢境可謂幽冷有致,卻又帶夢遺,發一笑。文心遊戲處,決不為筆墨縛束。}}西門慶向褥底摸了摸,見精流滿席,餘香在被,殘唾猶甜。追悼莫及,悲不自勝。正是:

\begin{myquote}
玉宇微茫霜滿襟,疎窻淡月夢魂驚。\\淒涼睡到無聊處,恨殺寒雞不肯鳴。
\end{myquote}

西門慶夢醒睡不着,巴不得天亮。比及天亮,又睡着了。次日早,何千戶家童僕起來伺候,打發西門慶梳洗畢,何千戶又早出來陪侍,吃了薑茶,放桌兒請吃粥。西門慶問:「老公公怎的不見?」何千戶道:「家公公從五更就進內去了。」須臾拿上粥來。吃了粥,又拿上一盞肉圓子餛飩雞蛋頭腦湯。一面吃着,就分付備馬。何千戶與西門慶冠冕,僕從跟隨,早進內參見兵科。出來,何千戶便分路來家,西門慶又到相國寺拜智雲長老。{\pangpi{忽插一閑人,妙。}}長老又留擺齋。西門慶只吃了一箇點心,餘者收與手下人吃了,就起身從東街穿過來,要往崔中書家拜夏龍溪去。因從造釜巷所過,中間果見有雙扇白板門,與夢中所見一般。悄悄使玳安問隔壁賣荳腐老姬:「此家姓甚名誰?」老姬答道:「此袁指揮家也。」西門慶於是不勝嘆異。到了崔中書家,夏公纔待出門拜人,見西門慶到,忙令左右把馬牽過,迎至廳上,拜揖叙禮。西門慶令玳安拿上賀禮:青織金綾紵一端、色段一端。夏公道:「學生還不曾拜賀長官,到承長官先施。昨日小房又煩費心,感謝不盡。」西門慶道:「昨日何太監說起看房,我因堂尊分上,就說此房來。何公討了房契去看了,一口就還原價。果是內臣性兒,立馬蓋橋就成了。還是堂尊大福!」說畢,二人笑了。夏公道:「何天泉,我也還未回拜他。」因問:「他此去與長官同行罷了。」西門慶道:「他已會定同學生一路去,家小且待後。昨日他老公公多致意,煩堂尊早些把房兒騰出來,搬取家眷。他如今權在衙門裡住幾日罷了。」夏公道:「學生也不肯久稽,待這裡尋了房兒,就使人搬取家小。也只待出月罷了。」說畢,西門慶起身,又留了箇拜帖與崔中書,夏公送出上馬,歸至何千戶家。何千戶又早有午飯等候。

西門慶悉把拜夏公之事說了一遍:「騰房已在出月。」何千戶大喜,謝道:「足見長官盛情。」吃畢飯,二人正在廳上着棋,忽左右來報:「府裡翟爹差人送下程來了。抓尋到崔老爹那裡,崔老爹使他這裡來了。」{\pangpi{周密。}}於是拿帖看,上寫着:「謹具金段一端、雲紵一端、鮮豬一口、北羊一腔、內酒一罈、點心二盒。眷生翟謙頓首拜。」西門慶見來人,說道:「又蒙你翟爹費心。」一面收了禮物,寫回帖,賞來人二兩銀子,擡盒人五錢,說道:「客中不便,有褻管家。」那人磕頭收了。王經在旁悄悄說:「小的姐姐說,教我府裡去看看愛姐,有物事稍與他。」西門慶問:「甚物事?」王經道:「是家中做的兩雙鞋脚手。」西門慶道:「單單兒怎好拿去?」分付玳安:「我皮箱內有帶的玫瑰花餅,取兩礶兒。」{\pangpi{西門慶做事心頗細。}}就把口帖付與王經,穿上青衣,跟了來人往府裡看愛姐不題。這西門慶寫了帖兒,送了一腔羊、一罈酒謝了崔中書,把一口豬、一罈酒、兩盒點心擡到後邊孝順老公公。何千戶拜謝道:「長官,你我一家,如何這等計較!」

且說王經到府內,請出韓愛姐,外廳拜見了。打扮的如瓊林玉樹一般,比在家出落自是不同,長大了好些。問了回家中事務,管待了酒飯,見王經身上單薄,與了一件天青紵絲貂鼠氅衣兒,又與了五兩銀子,拿來回覆西門慶話。

西門慶大喜。正與何千戶下棋,忽聞綽道之聲,門上人來報:「夏老爹來拜,拿進兩箇拜帖兒。」兩箇忙迎接到廳叙禮,何千戶又謝昨日房子之事。夏公具了兩分段帕酒禮,奉賀二公。西門慶與何千戶再三致謝,令左右收了。夏公又賞了賁四、玳安、王經十兩銀子,一面分賓主坐下。茶罷,共叙寒溫。夏公道:「請老公公拜見。」何千戶道:「家公公進內去了。」夏公又留下了一箇雙紅拜帖兒,說道:「多頂上老公公,拜遲,恕罪!」言畢,起身去了。何千戶隨即也具一分賀禮,一疋金段,差人送去,不在言表。到晚夕,何千戶又在花園暖閣中擺酒與西門慶共酌,家樂歌唱,到二更方寢。西門慶因昨日夢遺之事,晚夕令王經拿鋪蓋來書房地平上睡。半夜叫上床,摟在被窩內。{\pangpi{一味好淫。}}兩箇口吐丁香,舌融甜唾。正是:不能得與鶯鶯會,且把紅娘去解饞。一晚題過。到次日,起五更與何千戶一行人跟隨進朝。先到待漏院伺候,等的開了東華門進入。但見:

\begin{myquote}
星斗依稀禁漏殘,禁中環珮响珊珊。\\欲知今日天顏喜,遙睹蓬萊紫氣皤。
\end{myquote}

少頃,只聽九重門啟,鳴噦噦之鸞聲;閶闔天開,睹巍巍之袞冕。當時天子祀畢南郊回來,文武百官聚集,等候設朝。須臾鐘响,天子駕出大殿,受百官朝賀。須臾,香毬撥轉,簾捲扇開。正是:

\begin{myquote}
晴日明開青鎖闥,天風吹下御爐香。\\千條瑞靄浮金闕,一朵紅雲捧玉皇。
\end{myquote}

這皇帝生得堯眉舜目,禹背湯肩;才俊過人,口工詩韻;善寫墨君竹,能揮薛稷書;通三教之書,曉九流之典。朝歡暮樂,依稀似劍閣孟商王;愛色貪花,彷彿如金陵陳後主。{\meipi{稱堯眉舜目,忽接到孟商王、陳後主,又似贊,又似貶,可見敗亡之主,何嘗不具聖人之姿?即孟子所謂「堯舜與人同」之意。}}當下駕坐寶位,靜鞭响罷,文武百官秉簡當胸,向丹墀五拜三叩頭,進上表章。已而有殿頭官口傳聖旨道:「朕今即位二十禩矣。艮嶽於茲告成,上天降瑞,今値覆端之慶,與卿共之。」言未畢,班首中閃過一員大臣來,朝靴踏地响,袍袖列風生。視之,乃左丞相崇政殿大學士兼吏部尚書太師魯國公蔡京也。襆頭象簡,俯伏金堦,口稱:「萬歲,萬歲,萬萬歲!臣等誠惶誠恐,稽首頓首,恭惟皇上御極二十禩以來,海宇清寧,天下豐稔,上天降鑑,禎祥疊見。三邊永息兵戈,萬國來朝天闕。銀嶽排空,玉京挺秀。寶籙膺頒於昊闕,絳宵深聳於乾宮。{\pangpi{即好頌語,也覺無謂。}}臣等何幸,欣逢盛世,交際明良,永效華封之祝,常沾日月之光。不勝瞻天仰聖,激切屏營之至!謹獻頌以聞。」{\meipi{據頌所稱,過於賡歌遠矣。而然乎?否乎?可悟國家一聞此,便非好訊息。}}良久,聖旨下來:「賢卿獻頌,益見忠誠,朕心嘉悅。詔改明年為重和元年,正月元旦受定命寶,肄赦覃賞有差。」蔡太師承旨下來。殿頭官口傳聖旨:「有事出班早奏,無事捲簾退朝。」言未畢,見一人出離班部,倒笏躬身,緋袍象簡,玉帶金魚,跪在金堦,口稱:「光祿大夫掌金吾衛事太尉太保兼太子太保臣朱勔,引天下提刑官員章隆等二十六員,例該考察,已更改補,繳換箚付,合當引奏。未敢擅便,請旨定奪。」於是二十六員提刑官都跪在後面。不一時,聖旨傳下來:「照例給領。」朱太尉承旨下來。天子袍袖一展,羣臣皆散,駕即回宮。百官皆從端禮門兩分而出。那十二象不待牽而先走,鎭將長隨紛紛而散。朝門外車馬縱橫,侍仗羅列。人喧呼,海沸波翻;馬嘶喊,山崩地裂。衆提刑官皆出朝上馬,都來本衙門伺候。良久,只見知印拿了印牌來,傳道:「老爺不進衙門了,已往蔡爺、李爺宅內拜冬去了。」以此衆官都散了。西門慶與何千戶回到家中。又過了一夕,到次日,衙門中領了箚付,又掛了號,又拜辭了翟管家,打點殘裝,收拾行李,與何千戶一同起身。何太監晚夕置酒餞行,囑咐何千戶:「凡事請教西門大人,休要自專,差了禮數。」從十一月二十日東京起身,兩家也有二十人跟隨,竟往山東大道而來。已是數九嚴寒之際,點水滴凍之時,一路上見了些荒郊野路,枯木寒鴉。疎林淡日影斜暉,暮雪凍雲迷晚渡。一山未盡一山來,後村已過前村望。比及剛過黃河,到水關八角鎭,驟然撞遇天起一陣大風。{\pangpi{畧處偏詳。}}但見:

\begin{myquote}
非干虎嘯,豈是龍吟?卒律律寒飆撲面,急颼颼冷氣侵人。初時節無蹤無影,次後來捲霧收雲。吹花擺柳白茫茫,走石揚砂昏慘慘。刮得那大樹連聲吼,驚得那孤雁落深濠。須臾,砂石打地,塵土遮天。砂石打地,猶如滿天驟雨即時來;塵土遮天,好似百萬貔貅捲土至。這風大不大?眞箇是吹折地獄門前樹,刮起酆都頂上塵;嫦娥急把蟾宮閉,列子空中叫救人。險些兒玉皇住不得崑崙頂,只刮得大地乾坤上下搖。
\end{myquote}

西門慶與何千戶坐着兩頂氊幃暖轎,被風刮得寸步難行。又見天色漸晚,恐深林中撞出小人來,西門慶分付手下:「快尋那裡安歇一夜,明日風住再行罷。」抓尋了半日,遠遠望見路旁一座古剎,數株疎柳,半堵橫墻。但見:

\begin{myquote}
石砌碑橫夢草遮,迴廊古殿半欹斜。\\夜深宿客無燈火,月落安禪更可嗟。
\end{myquote}

西門慶與何千戶忙入寺中投宿,上題着「黃龍寺」。見方丈內幾箇僧人在那裡坐禪,又無燈火,房舍都毀壞,半用籬遮。{\pangpi{此方是眞正枯禪。}}長老出來問訊,旋吹火煮茶,伐草根喂馬。煮出茶來,西門慶行囊中帶得乾雞臘肉菓餅之類,晚夕與何千戶胡亂食得一頓。長老爨一鍋荳粥吃了,過得一宿。次日風止天晴,與了和尚一兩銀子相謝,作辭起身往山東來。正是:

\begin{myquote}
王事驅馳豈憚勞,關山迢遞赴京朝。\\夜投古寺無烟火,解使行人心內焦。
\end{myquote}

