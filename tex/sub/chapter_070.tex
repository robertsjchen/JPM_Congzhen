\includepdf[pages={139,140},fitpaper=false]{tst.pdf}
\chapter*{第七十回 老太監引酌朝房 二提刑庭參太尉}
\addcontentsline{toc}{chapter}{第七十回 老太監引酌朝房 二提刑庭參太尉}
\markboth{{\titlename}卷之七}{第七十回 老太監引酌朝房 二提刑庭參太尉}


詩曰:

\begin{myquote} 
帝曰簡才能,旌賢在股肱。\\文章體一變,禮樂道逾弘。\\芸閣英華入,賓門鵷鷺登。\\恩筵過所望,聖澤實超恆。
\end{myquote} 

話說西門慶自此與李桂姐斷絕不題。卻說走差人到懷慶府林千戶處打聽訊息,林千戶將陞官邸報封付與來人,又賞了五錢銀子,連夜來遞與提刑兩位官府。當廳夏提刑拆開,同西門慶先觀本衛行來考察官員照會,其畧曰:

\begin{myquote}[\markfont]
兵部一本:

尊明旨,嚴考核,以昭勸懲,以光聖治事:先該金吾衛提督官校太尉太保兼太子太保朱題前事,考察禁衛官員,除堂上官自陳外,其餘兩廂詔獄緝捕、內外提刑所指揮千百戶、鎭撫等官,各挨次格,從公舉劾,甄別賢否,具題上請,等因。

奉聖旨:

兵部知道,欽此欽遵。抄出到部。看得太尉朱題前事,遵奉舊例,委的本官殫力致忠,公於考核,皆出聞見之實,而無偏執之私。足以勵人心而孚公議,無容臣等再喙。但恩威賞罰,出自朝廷,合候命下之日,一體照例施行等因。續奉欽依擬行。

內開:

山東提刑所正千戶夏延齡,資望既久,才練老成,昔視典牧而坊隅安靜,今理齊刑而綽有政聲,宜加獎勵,以冀甄陞,可備鹵簿之選者也。貼刑副千戶西門慶,才幹有為,精察素著。家稱殷實而在任不貪,國事克勤而臺工有績。翌神運而分毫不索,司法令而齊民果仰。宜加轉正,以掌刑名者也。懷慶提刑千戶所正千戶林承勳,年清優學,佔籍武科,繼祖職抱負不凡,提刑獄詳明有法,可加獎勵簡任者也。副千戶謝恩,年齒既殘,昔在行猶有可觀,今任理刑罹軟尤甚,宜罷黜革任者也。
\end{myquote} 

西門慶看了他轉正千戶掌刑,心中大悅。夏提刑見他陞指揮,管鹵簿,大半日無言,面容失色。於是又展開工部工完的本觀看,上面寫道:

\begin{myquote}[\markfont]
工部一本:

神運屆京,天人胥慶,懇乞天恩,俯加渥典,以蘇民困,以廣聖澤事。

奉聖旨:

這神運奉迎大內,奠安艮嶽,以承天眷,朕心嘉悅。你每既效有勤勞,副朕事玄至意。所經過地方,委的小民困苦,着行撫按衙門,查勘明白,着行蠲免今歲田租之半。所毀壩閘,着部裡差官會同巡按御史,即行修理。完日還差內侍孟昌齡前去致祭。蔡京、李邦彥、王煒、鄭居中、高俅,輔弼朕躬,直贊內廷,勳勞茂著,京加太師,邦彥加柱國太子太師,王煒太傅,鄭居中、高俅太保,各賞銀五十兩、四表禮。蔡京還蔭一子為殿中監。國師林靈素,佐國宣化,遠致神運,北伐虜謀,實與天通,加封忠孝伯,食祿一千石,賜坐龍衣一襲,肩輿入內,賜號玉眞教主,加淵澄玄妙廣德眞人、金門羽客、達靈玄妙先生。{\meipi{慶賞如此,可想朝廷法紀。即此可為綱目補遺。}}朱勔、黃經臣,督理神運,忠勤可嘉。勔加太傅兼太子太傅,經臣加殿前都太尉,提督御前人船。各蔭一子為金吾衛正千戶。內侍李彥、孟昌齡、賈祥、何沂、藍從頤,着直延福五位宮近侍,各賜蟒衣玉帶,仍蔭弟侄一人為副千戶,俱見任管事。禮部尚書張邦昌、左侍郎兼學士蔡攸、右侍郎白時中、兵部尚書余深、工部尚書林攄,俱加太子太保,各賞銀四十兩,彩段二表禮。巡撫兩浙僉都御史張閣,陞工部右侍郎。巡撫山東都御史侯蒙,陞太常正卿。巡撫兩浙、山東監察御史尹大諒、宋喬年,都水司郎中安忱、伍訓,各陞俸一級,賞銀二十兩。祗迎神運千戶魏承勳、徐相、楊廷佩、司鳳儀、趙友蘭、扶天澤、西門慶、田九皋等,各陞一級。內侍宋推等,營將王佑等,俱各賞銀十兩。所官薛顯忠等,各賞銀五兩。校尉昌玉等,絹二疋。該衙門知道。
\end{myquote} 

夏提刑與西門慶看畢,各散回家。

後晌時分,有王三官差永定同文嫂拿請書,十一日請西門慶往他府中赴席,少罄謝私之意。西門慶收下,不勝歡喜,以為其妻指日在於掌握。{\jiapi{可惡。}}不期到初十日晚夕,東京本衛經歷司差人行照會:「曉諭各省提刑官員知悉:火速赴京,趕冬節見朝謝恩,毋得違誤取罪。」西門慶看了,到次日衙門中會了夏提刑,各人到家,即收拾行裝,備辦贄見禮物,約早晚起程。西門慶使玳安叫了文嫂兒,教他回王三官:「我今日不得來赴席,要上京見朝謝恩去。」文嫂連忙去回,王三官道:「既是老伯有事,容回來潔誠具請。」西門慶一面叫將賁四來,分付教他跟了去,與他五兩銀子,家中盤纏。留下春鴻看家,帶了玳安、王經跟隨答應。又問周守備討了四名巡捕軍人,四匹小馬,打點馱裝轎馬,排軍擡扛。夏提刑便是夏壽跟隨。兩家共有二十餘人跟從。十二日起身離了清河縣,冬天易晚,晝夜趲行。到了懷西懷慶府會林千戶,千戶已上東京去了。一路天寒坐轎,天暖乘馬,朝登紫陌,暮踐紅塵。正是:

\begin{myquote}
意急款搖青帳幕,心忙敲碎紫絲鞭。
\end{myquote}

話說一日到了東京,進得萬壽門。西門慶主意要往相國寺下。夏提刑不肯,堅執要往他親眷崔中書家投下。西門慶不免先具拜帖拜見。正値崔中書在家,即出迎接,至廳叙禮相見,與夏提刑道及寒溫契闊之情。坐下茶畢,拱手問西門慶尊號。西門慶道:「賤號四泉。」因問:「老先生尊號?」崔中書道:「學生性最愚樸,名閑林下,賤名守愚,拙號遜齋。」因說道:「舍親龍溪久稱盛德,全仗扶持,同心協恭,莫此為厚。」西門慶道:「不敢。在下常領教誨,今又為堂尊,受益恆多,不勝感激。」夏提刑道:「長官如何這等稱呼!便不見相知了。」崔中書道:「四泉說的也是,名分使然。」言畢,彼此笑了。不一時,收拾行李。天晚了,崔中書分付童僕放桌擺飯,無非是菓酌餚饌之類,不必細說。

當日,二人在崔中書家宿歇不題。到次日,各備禮物拜帖,家人跟隨,早往蔡太師府中叩見。那日太師在內閣還未出來,府前官吏人等如蜂屯蟻聚,擠匝不開。西門慶與夏提刑與了門上官吏兩包銀子,拿揭帖稟進去。翟管家見了,即出來相見,讓他到外邊私宅。先是夏提刑先見畢,然後西門慶叙禮,彼此道及往還酬答之意,各分賓位坐下。夏提刑先遞上禮帖:兩疋雲鶴金段、兩疋色段。翟管家是十兩銀子。西門慶禮帖上是一疋大紅絨綵蟒、一疋玄色粧花斗牛補子員領、兩疋京段,另外梯己送翟管家一疋黑綠雲絨、三十兩銀子。{\meipi{此兒大有體面。}}翟謙分付左右:「把老爺禮都收進府中去,上簿籍。」他只受了西門慶那疋雲絨,將三十兩銀子連夏提刑的十兩銀子都不受,說道:「豈有此理。若如此,不見至交親情。」一面令左右放桌兒擺飯,說道:「今日聖上奉艮嶽,新蓋上清寶籙宮,奉安牌匾,該老爺主祭,直到午後纔散。到家同李爺又往鄭皇親家吃酒。只怕親家和龍溪等不的,誤了你每勾當。遇老爺閑,等我替二位稟就是一般。」西門慶道:「蒙親家費心。」翟謙因問:「親家那裡住?」西門慶就把夏龍溪令親家下歇說了。不一時,安放桌席端正,就是大盤大碗,湯飯點心一齊拿上來,都是光祿烹炮,美味極品無加。每人金爵飲酒三盃,就要告辭起身。翟謙款留,令左右又篩上一盃。西門慶因問:「親家,俺每幾時見朝?」翟謙道:「親家,你同不得夏大人。夏大人如今是京堂官,不在此例。你與本衛新陞的副千戶何太監侄兒何永壽,他便貼刑,你便掌刑,與他作同僚了。他先謝了恩,只等着你見朝引奏畢,一同好領箚付。你凡事只會他去。」夏提刑聽了,一聲兒不言語。西門慶道:「請問親家,只怕我還要等冬至郊天回來見朝。」翟謙道:「親家,你等不的冬至聖上郊天回來。那日天下官員上表朝賀,還要排慶成宴,你每怎等的?不如你今日先往鴻臚寺報了名,明日早朝謝了恩,直到那日堂上官引奏畢,領箚付起身就是了。」西門慶謝道:「蒙親家指教,何以為報!」臨起身,翟謙又拉西門慶到側淨處說話,{\meipi{照應。}}甚是埋怨西門慶說:「親家,前日我的書上那等寫了,大凡事要謹密,不可使同僚每知道。親家如何對夏大人說了?{\meipi{照應。}}教他央了林眞人帖子來,立逼着朱太尉來對老爺說,{\meipi{包苴賄賂如此,尚有法守乎?}}要將他情願不管鹵簿,仍以指揮職銜在任所掌刑三年;何太監又在內廷,轉央朝廷所寵安妃劉娘娘的分上,便也傳旨出來,親對老爺和朱太尉說了,要安他侄兒何永壽在山東理刑。兩下人情阻住了,教老爺好不作難!不是我再三在老爺跟前維持,回倒了林眞人,把親家不撐下去了?」慌的西門慶連忙打躬,說道:「多承親家盛情!我並不曾對一人說,此公何以知之?」翟謙道:「自古機事不密則害成,今後親家凡事謹愼些便了。」

西門慶千恩萬謝,與夏提刑作辭出門。來到崔中書家,一面差賁四鴻臚寺報了名。次日同夏提刑見朝,青衣冠帶,正在午門前謝恩出來,剛轉過西闕門來,只見一箇青衣人走向前問道:「那位是山東提刑西門老爹?」賁四問道:「你是那裡的?」那人道:「我是內府匠作監何公公來請老爹說話。」言未畢,只見一箇太監,身穿大紅蟒衣,頭戴三山帽,脚下粉底皁靴,從御街定聲叫道:「西門大人請了!」西門慶遂與夏提刑分別,被這太監用手一把拉在旁邊一所値房內,相見作揖,慌的西門慶倒身還禮不迭。這太監說道:「大人,你不認的我,在下是匠作監太監何沂,見在延寧第四宮端妃馬娘娘位下近侍。昨日內工完了,蒙萬歲爺爺恩典,將侄兒何永壽陞受金吾衛副千戶,見在貴處提刑所理刑管事,與老大人作同僚。」西門慶道:「原來是何老太監,學生不知,恕罪,恕罪!」一面又作揖說道:「此禁地,不敢行禮,容日到老太監外宅進拜。」於是叙禮畢,讓坐,家人捧茶來吃了。茶畢,就揭桌盒蓋兒,桌上許多湯飯餚品,拿盞筯兒來安下。何太監道:「不消小盃了,我曉的大人朝下來,天氣寒冷,拿箇小盞來,沒甚餚饌,褻瀆大人,且吃箇頭腦兒罷。」西門慶道:「不當厚擾。」何太監於是滿斟上一大盃,遞與西門慶,西門慶道:「承老太監所賜,學生領下。只是出去還要見官拜部,若吃得面紅,不成道理。」何太監道:「吃兩盞兒燙寒何害!」因說道:「舍侄兒年幼,不知刑名,望乞大人看我面上,同僚之間,凡事教導他教導。」西門慶道:「豈敢。老太監勿得太謙,令侄長官雖是年幼,居氣養體,自然福至心靈。」何太監道:「大人好說。常言『學到老不會到老』,天下事如牛毛,孔夫子也只識的一腿。{\meipi{妙語,非老太監不能道。}}恐有不到處,大人好歹說與他。」西門慶道:「學生謹領。」因問:「老大監外宅在何處?學生好來奉拜長官。」何太監道:「舍下在天漢橋東,文華坊雙獅馬臺就是。」亦問:「大人下處在那裡?我教做官的先去叩拜。」西門慶道:「學生暫借崔中書家下。」彼此問了住處,西門慶吃了一大盃就起身。何太監送出門,拱着手說道:「適間所言,大人凡事看顧看顧。他還等着你一答兒引奏,好領箚付。」西門慶道:「老太監不消分付,學生知道。」於是出朝門,又到兵部,又遇見了夏提刑,同拜了部官來。比及到本衛參見朱太尉,遞履歷手本,繳箚付,又拜經歷司並本所官員,已是申刻時分。夏提刑改換指揮服色,另具手本參見了朱太尉,免行跪禮,擇日南衙到任。剛出衙門,西門慶還等着,遂不敢與他同行,讓他先上馬。夏延齡那裡肯?定要同行。西門慶趕着他呼「堂尊」,夏指揮道:「四泉,你我同僚在先,為何如此稱呼?」西門慶道:「名分已定,自然之理,何故太謙。」因問:「堂尊高陞美任,不還山東去了,寶眷幾時搬取?」夏延齡道:「欲待搬來,那邊房舍無人看守。如今且在舍親這邊權住,直待過年,差人取家小罷了。還望長官早晚看顧一二。房子若有人要,就央長官替我打發,自當報謝。」西門慶道:「學生謹領。請問府上那房價値若干?」夏延齡道:「舍下此房原是一千三百兩買的,後邊又蓋了一層,使了二百兩,如今賣原價也罷了。」

二人歸到崔宅,王經向前稟說:「新陞何老爹來拜,下馬到廳。小的回部中還未來家。何老爹說多拜上夏老爹、崔老爹,都投下帖。午間又差人送了兩疋金段來。」宛紅帖兒拿與西門慶看,上寫着:「謹具段帕二端,奉引贄敬。寅侍教生何永壽頓首拜。」西門慶看了,連忙差王經封了兩疋南京五彩獅補員領,寫了禮帖。吃了飯,連忙往何家回拜去。到於廳上,何千戶忙出來迎接,烏紗皁履,年紀不上二十歲,生的面如傅粉,唇若塗朱,趨下堦來揖讓,退遜謙恭特甚。二人到廳上叙禮,西門慶令玳安捧上贄見之禮,拜下去,說道:「適承光顧,兼領厚儀,又失迎迓。今早又蒙老公公値房賜饌,感德不盡。」何千戶忙還禮說:「學生叨受微職,忝與長官同例,早晚得領教益,實為三生有幸。適間進拜不遇,又承垂顧,蓬篳光生。」令左右收下去,一面扯椅兒分賓主坐下,左右捧茶上來。吃茶之間,彼此問號,西門慶道:「學生賤號四泉。」何千戶道:「學生賤號天泉。」又問:「長官今日拜畢部堂了?」西門慶道:「從內裡蒙公公賜酒出來,拜畢部,又到本衙門見堂,繳了箚付,拜了所司。出來就要奉謁長官,不知反先辱長官下顧。」何千戶因問:「長官今日與夏公都見朝來?」西門慶道:「夏龍溪已陞了指揮直駕,今日都見朝謝恩在一處,只到衙門見堂之時,他另具手本參見。」說畢,何千戶道:「咱每還是先與本主老爹進禮,還是先領箚付?」西門慶道:「依着舍親說,咱每先在衛主宅中進了禮,然後大朝引奏,還在本衙門到堂同衆領箚付。」何千戶道:「既是如此,咱每明早備禮進了罷。」於是都會下各人禮數,何千戶是兩疋蟒衣、一束玉帶,西門慶是一疋大紅麒麟金段、一疋青絨蟒衣、一柄金鑲玉縧環,各金華酒四罈。明早在朱太尉宅前取齊。約會已定,茶湯兩換,西門慶告辭而回,並不與夏延齡題此事。一宿晚景題過。

到次日,早到何千戶家。何千戶又預備頭腦小席,大盤大碗,齊齊整整,聯手下人飽餐一頓,然後同往大尉宅門前來。賁四同何家人押着禮物。那時正値朱太尉新加太保,徽宗天子又差使往南壇視牲未回,各家餽送賀禮並參見官吏人等,黑壓壓在門首等候。何千戶同西門慶下了馬,在左近一相識人家坐的,差人打聽老爺道子響就來通報。直等到午後,忽見一人飛馬而來,傳報道:「老爺視牲回來,進南薰門了。」分付閑雜人開啟。不一時,又騎報回來,傳:「老爺過天漢橋了。」少頃,只見官吏軍士各打執事旗牌,一對一對傳呼,走了半日,纔遠遠望見朱太尉八擡八簇肩輿明轎,頭戴烏紗,身穿猩紅斗牛絨袍,腰橫荊山白玉,懸掛太保牙牌、黃金魚鑰,好不顯赫威嚴!執事到了宅門首,都一字兒擺開,喝的肅靜迴避,無一人聲嗽。那來見的官吏人等,黑壓壓一羣跪在街前。良久,太尉轎到跟前,左右喝聲:「起來伺候!」那衆人一齊應諾,誠然聲震雲霄。

只聽東邊咚咚鼓樂响動,原來本衙門六員太尉堂官,見朱太尉新加光祿大夫、太保,又蔭一子為千戶,都各備大禮,治酒慶賀,故有許多教坊伶官在此動樂。太尉纔下轎,樂就止了。各項官吏人等,預備進見。忽然一聲道子響,一青衣承差手拿兩箇紅拜帖,飛走而來,遞及閘上人說:「禮部張爺與學士蔡爺來拜。」連忙稟報進去。須臾轎在門首,尚書張邦昌與侍郎蔡攸,都是紅吉服孔雀補子,一箇犀帶,一箇金帶,進去拜畢,待茶畢,送出來。又是吏部尚書王祖道與左侍郎韓侶、右侍郎尹京也來拜,朱太尉都待茶送了。又是皇親喜國公、樞密使鄭居中、駙馬掌宗人府王晉卿,都是紫花玉帶來拜。唯鄭居中坐轎,這兩箇都騎馬。送出去,方是本衙堂上六員太尉到了:頭一位是提督管兩廂捉察使孫榮,第二位管機察梁應龍,第三管內外觀察典牧皇畿童大尉侄兒童天胤,第四提督京城十三門巡察使黃經臣,第五管京營衛緝察皇城使竇監,第六督管京城內外巡捕使陳宗善。都穿大紅,頭戴貂蟬,惟孫榮是太子太保玉帶,餘者都是金帶。下馬進去。各家都有金幣禮物。{\meipi{一路寫得威儀顯赫,可見勢利不可一刻無。}}少頃,裡面樂聲响動,衆太尉插金花,與朱太尉把盞遞酒,堦下一派簫韶盈耳,兩行絲竹和鳴。端的食前方丈,花簇錦筵。怎見得太尉的富貴?但見:

\begin{myquote}
官居一品,位列三臺。赫赫公堂,潭潭相府。虎符玉節,門庭甲仗生寒;象板銀箏,磈礧排場熱鬧。終朝謁見,無非公子王孫;逐歲追遊,盡是侯門戚里。那裡解調和燮理,一味能趨諂逢迎。端的談笑起干戈,眞箇吹嘘驚海嶽。假旨令八位大臣拱手,巧辭使九重天子點頭。督擇花石,江南淮北盡災殃;進獻黃楊,國庫民財皆匱竭。
\end{myquote}

正是:

\begin{myquote}
輦下權豪第一,人間富貴無雙。
\end{myquote}

須臾遞畢,安席坐下。一班兒五箇俳優,朝上箏𥱧琵琶,方響箜篌,紅牙象板,唱了一套「享富貴,受皇恩」。當時酒進三巡,歌吟一套,六員太尉起身,朱太尉親送出來,回到廳,樂聲暫止,管家稟事,各處官員進見。朱太尉令左右擡公案,當廳坐下,分付出來,先令各勳戚中貴仕宦家人送禮的進去。須臾打發出來,纔是本衛紀事、南北衛兩廂、五所、七司捉察、譏察、觀察、巡察、典牧、直駕、提牢、指揮、千百戶等官,各具手本呈遞。然後纔傳出來,叫兩淮、兩浙、山東、山西、關東、關西、河東、河北、福建、廣南、四川十三省提刑官挨次進見。西門慶與何千戶在第五起上,擡進禮物去,管家接了禮帖,鋪在書案上,二人立在堦下,等上邊叫名字。西門慶擡頭見正面五間廠廳,上面朱紅牌匾,懸着徽宗皇帝御筆欽賜「執金吾堂」斗大四箇金字,甚是顯赫。須臾叫名,二人應諾陞堦,到滴水簷前躬身參謁,四拜一跪,聽發放。朱太尉道:「那兩員千戶,怎的又叫你家太監送禮來?」令左右收了,分付:「在地方謹愼做官,我這裡自有公道。伺候大朝引奏畢,來衙門中領箚赴任。」二人齊聲應諾。左右喝:「起去!」由左角門出來。剛出大門來,尋見賁四等擡担出來,正要走,忽見一人拿宛紅帖飛馬來報,說道:「王爺、高爺來了。」西門慶與何千戶閃在人家門裡觀看。須臾,軍牢喝道,只見總督京營八十萬禁軍隴西公王燁,同提督神策御林軍總兵官太尉高俅,俱大紅玉帶,坐轎而至。那各省參見官員一湧出來,又不得見了。西門慶與何千戶走到僻處,呼跟隨人扯過馬來,二人方騎上馬回寓。正是:

\begin{myquote}
權奸誤國禍機深,開國承家戒小人。\\逆賊深誅何足道,奈何二聖遠蒙塵。
\end{myquote}

