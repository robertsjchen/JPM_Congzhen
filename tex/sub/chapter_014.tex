\includepdf[pages={27,28},fitpaper=false]{tst.pdf}
\chapter*{第十四回 花子虛因氣䘮身 李瓶兒迎奸赴會}
\addcontentsline{toc}{chapter}{第十四回 花子虛因氣䘮身 李瓶兒迎奸赴會}
\markboth{{\titlename}卷之二}{第十四回 花子虛因氣䘮身 李瓶兒迎奸赴會}


詩曰:

\begin{myquote}
眼意心期未即休,不堪拈弄玉搔頭。\\春回笑臉花含媚,黛蹙娥眉柳帶愁。\\粉暈桃腮思伉儷,寒生蘭室盼綢繆。\\何如得遂相如意,不讓文君詠白頭。
\end{myquote}

話說一日吳月娘心中不快,吳大妗子來看,月娘留他住兩日。正陪在房中坐的,忽見小厮玳安抱進氊包來,說:「爹來家了。」吳大妗子便往李嬌兒房裡去了。西門慶進來,脫了衣服坐下。小玉拿茶來也不吃。月娘見他面色改常,便問:「你今日會茶,來家恁早?」西門慶道:「今該常二哥會,他家沒地方,請俺們在城外永福寺去耍子。有花二哥邀了應二哥,俺們四五箇,往院裡鄭愛香兒家吃酒。正吃着,忽見幾箇做公的進來,不繇分說,把花二哥拿的去了。把衆人嚇了一驚。我便走到李桂姐家躲了半日,不放心,使人打聽。原來是花二哥內臣家房族中告家財,在東京開封府遞了狀子,批下來,着落本縣拿人。俺們纔放心,各人散歸家來。」月娘聞言,便道:「這是正該的,你整日跟着這夥人,不着箇家,只在外邊胡撞;今日只當弄出事來,纔是箇了手。你如今還不心死。到明日不吃人爭鋒厮打,羣到那裡,是箇爛羊頭,你肯斷絕了這條路兒!正經家裡老婆的言語說着你肯聽?只是院裡淫婦在你跟前說句話兒,你到着箇驢耳朵聽他。正是:家人說着耳邊風,外人說着金字經。」西門慶笑道:「誰人敢七箇頭八箇膽打我!」{\pangpi{口角肖甚。}}月娘道:「你這行貨子,只好家裡嘴頭子罷了。」

正說着,只見玳安走來說:「隔壁花二娘使天福兒來,請爹過去說話。」這西門慶聽了,趔趄脚兒就往外走。月娘道:「明日沒的教人扯你把。」西門慶道:「切隣間不防事。我去到那裡,看他有甚麼話說。」當下走過花子虛家來,李瓶兒使小厮請到後邊說話,只見婦人羅衫不整,粉面慵粧,從房裡出來,臉嚇的蠟渣也似黃,跪着西門慶,再三哀告道:「大官人沒奈何,不看僧面看佛面,{\meipi{以佛面自許,妙甚。皆映帶瓶兒醇厚處。}}常言道:『家有患難,隣里相助。』因他不聽人言,把着正經家事兒不理,只在外邊胡行。今日吃人暗算,弄出這等事來。這時節方對小厮說將來,教我尋人情救他。我一箇婦人家,沒脚蠏,那裡尋那人情去!發狠起來,想着他恁不依說,拿到東京,打的他爛爛的,也不虧他。{\meipi{恨中作轉想,全不念及夫妻,子虛危矣。}}只是難為過世老公公的姓字。奴沒奈何,請將大官人過來,央及大官人,把他不要提起罷,千萬看奴薄面,有人情好歹尋一箇兒,只不教他吃淩逼便了。」西門慶見婦人下禮,連忙道:「嫂子請起來,不妨,我還不知為了甚勾當。」婦人道:「正是一言難盡。俺過世老公公有四箇侄兒,大侄兒喚做花子繇,第三箇喚花子光,第四箇叫花子華,俺這箇名花子虛,都是老公公嫡親的。雖然老公公掙下這一分錢財,見我這箇兒不成器,從廣南迴來,把東西只交付與我手裡收着。着緊還打儻棍兒,那三箇越發打的不敢上前。去年老公公死了,這花大、花三、花四,也分了些床帳家伙去了,只現一分銀子兒沒曾分得。我常說,多少與他些也罷了,他通不理一理兒。今日下,暗不通風,卻教人弄下來了。」說畢,放聲大哭。西門慶道:「嫂子放心,我只道是甚麼事來,原來是房分中告家財事,這箇不打緊。既是嫂子分付,哥的事就是我的事一般,隨問怎的,我在下謹領。」婦人說道:「官人若肯時又好了。請問尋分上,要用多少禮兒,奴好預備。」西門慶道:「也用不多,聞得東京開封府楊府尹,乃蔡太師門生。蔡太師與我這四門親家楊提督,都是當朝天子面前說得話的人。拿兩箇分上,齊對楊府尹說,有箇不依的!不拘多大事情也了了。如今倒是蔡太師用些禮物。那提督楊爺與我舍下有親,他肯受禮?」婦人便往房中開箱子,搬出六十錠大元寶,共計三千兩,教西門慶收去尋人情,上下使用。西門慶道:「只一半足矣,何消用得許多!」婦人道:「多的大官人收了去。奴床後還有四箱櫃蟒衣玉帶,帽頂縧環,都是値錢珍寶之物,亦發大官人替我收去,放在大官人那裡,奴用時來取。{\meipi{世上許多不顧名義者,皆此一念壞之,不獨一瓶兒也。}}趁這時,奴不思箇防身之計,信着他,往後過不出好日子來。眼見得三拳敵不得四手,到明日,沒的把這些東西兒吃人暗算了去,坑閃得奴三不歸!」西門慶道:「只怕花二哥來家尋問怎了?」婦人道:「這都是老公公在時,梯己交與奴收着之物,他一字不知。大官人只顧收去。」西門慶說道:「既是嫂子恁說,我到家教人來取。」於是一直來家,與月娘商議。月娘說:「銀子便用食盒叫小厮擡來。那箱籠東西,若從大門裡來,教兩邊街坊看着不惹眼?必須夜晚打墻上過來方隱密些。」西門慶聽言大喜,即令玳安、來旺、來興、平安四箇小厮,兩架食盒,把三千兩銀子先擡來家。然後到晚夕月上時分,李瓶兒那邊同迎春、綉春放桌櫈,把箱櫃捱到墻上。西門慶這邊,止是月娘、金蓮、春梅,用梯子接着。墻頭上鋪襯氊條,一箇箇打發過來,都送到月娘房中去了。正是:

\begin{myquote}
富貴自是福來投,利名還有利名憂。\\命裡有時終須有,命裡無時莫強求。
\end{myquote}

西慶收下他許多細軟金銀寶物,隣舍街坊俱不知道。連夜打點馱裝停當,求了他親家陳宅一封書,差家人來保上東京。送上楊提督書禮,轉求內閣蔡太師柬帖,下與開封府楊府尹。這府尹名喚楊時,別號龜山,乃陝西弘農縣人氏,由癸未進士陞大理寺卿,今推開封府尹,極是清廉。況蔡太師是他舊時座主,楊戩又是當道時臣,如何不做分上!{\meipi{以龜山一清廉猶聽分上,況其他乎!然此等分上,亦不必不聽。}}當日楊府尹陞廳,監中提出花子虛來,一干人上廳跪下,審問他家財下落。此時花子虛已有西門慶稍書知會了,口口只說:「自從老公公死了,發送念經,都花費了。止有宅舍兩所、庄田一處見在,其餘床帳家伙物件,俱被族人分散一空。」楊府尹道:「你們內官家財,無可稽考,得之易,失之易。既是花費無存,批仰清河縣委官將花太監住宅二所、庄田一處,估價變賣,分給花子繇等三人回繳。」花子繇等又上前跪稟,還要監追子虛,要別項銀兩。被楊府尹大怒,都喝下來,說道:「你這厮少打!當初你那內相一死之時,你每不告做甚麼來?如今事情已往,又來騷擾。」於是把花子虛一下兒也沒打,批了一道公文,押發清河縣前來估計庄宅,不在話下。

來保打聽這訊息,星夜回來,報知西門慶。西門慶聽見分上準了,放出花子虛來家,滿心歡喜。這裡李瓶兒請過西門慶去計議,要叫西門慶拿幾兩銀子,買了這所住的宅子:「到明日,奴不久也是你的人了。」{\pangpi{已先拿定,怕人。}}西門慶歸家與吳月娘商議。月娘道:「你若要他這房子,恐怕他漢子一時生起疑心來,怎了?」西門慶聽記在心。那消幾日,花子虛來家,清河縣委下樂縣丞丈估:太監大宅一所,坐落大街安慶坊,値銀七百兩,賣與王皇親為業;南門外庄田一處,値銀六百五十兩,賣與守備周秀為業。止有住居小宅,値銀五百四十兩,因在西門慶緊隔壁,沒人敢買。花子虛再三使人來說,西門慶只推沒銀子,不肯上帳。縣中緊等要回文書,李瓶兒急了,暗暗使馮媽媽來對西門慶說,教拿他寄放的銀子兌五百四十兩買了罷。這西門慶方纔依允。當官交兌了銀兩,花子繇都畫了字。連夜做文書回了上司,共該銀一千八百九十五兩,三人均分訖。

花子虛打了一場官司出來,沒分的絲毫,把銀兩、房舍、庄田又沒了,兩箱內三千兩大元寶又不見蹤影,心中甚是焦躁。因問李瓶兒查算西門慶使用銀兩下落,今還剩多少,好湊着買房子。反吃婦人整罵了四五日,罵道:「呸!魍魎混沌,{\pangpi{罵得當。}}你成日放着正事兒不理,在外邊眠花臥柳,只當被人弄成圈套,拿在牢裡,使將人來教我尋人情。奴是箇女婦人家,大門邊兒也沒走,曉得甚麼?認得何人?那裡尋人情?渾身是鐵,打得多少釘兒?替你添羞臉,到處求爹爹告奶奶。多虧了隔壁西門大官人,{\meipi{慢慢說到西門慶身上,一些不露相,妙甚。}}看日前相交之情,大冷天,颳得那黃風黑風,使了家下人往東京去,替你把事兒幹得停停當當的。你今日了畢官司,兩脚站在平川地,得命思財,瘡好忘痛,來家到問老婆找起後帳兒來了,還說有也沒有。你寫來的帖子現在,沒你的手字兒,我擅自拿出你的銀子尋人情,抵盜與人便難了!」{\pangpi{虛心病,偏有膽說破,妙甚。}}{\meipi{在花子虛跟前,便有許多饒舌,蓋拿定子虛無可奈何故耳。}}花子虛道:「可知是我的帖子來說,{\pangpi{膿疱口角。}}實指望還剩下些,咱湊着買房子過日子。」婦人道:「呸!濁蠢才!我不好罵你的。你早仔細好來,囷頭兒上不算計,圈底兒下卻算計。千也說使多了,萬也說使多了,你那三千兩銀子能到的那裡?蔡太師、楊提督好小食腸兒!不是恁大人情,平白拿了你一場,當官蒿條兒也沒曾打在你這忘八身上,好好兒放出來,教你在家裡恁說嘴!人家不屬你管轄,你是他甚麼着疼的親?平白怎替你南上北下走跳,使錢救你!你來家也該擺席酒兒,請過人來,知謝人一知謝兒,還一掃帚掃得人光光的,到問人找起後帳兒來了!」幾句連搽帶罵,罵的子虛閉口無言。

到次日,西門慶使玳安送了一分禮來與子虛壓驚。子虛這裡安排了一席,請西門慶來知謝,就要問他銀兩下落。依着西門慶,還要找過幾百兩銀子與他湊買房子。到是李瓶兒不肯,{\pangpi{太甚。}}暗地使馮媽媽過來對西門慶說:「休要來吃酒,只開送一篇花帳與他,說銀子上下打點都使沒了。」花子虛不識時,還使小厮再三邀請。西門慶躲的一徑往院裡去了,只回不在家。花子虛氣的發昏,只是跌脚。

看官聽說:大凡婦人更變,不與男子漢一心,隨你咬折鐵釘般剛毅之夫,也難測其暗地之事。自古男治外而女治內,往往男子之名都被婦人壞了者,為何?皆繇御之不得其道。要之,在乎容德相感,緣分相投,夫唱婦隨,庶可保其無咎。若似花子虛落魄飄風,謾無紀律,而欲其內人不生他意,豈可得乎?正是:

\begin{myquote}
自意得其墊,無風可動搖。
\end{myquote}

話休饒舌。後來子虛只擯湊了二百五十兩銀子,買了獅子街一所房屋居住。得了這口重氣,剛搬到那裡,又不幸害了一場傷寒,從十一月初旬,睡倒在床上,就不曾起來。初時還請太醫來看,後來怕使錢,只挨着。一日兩,兩日三,捱到二十頭,嗚呼哀哉,斷氣身亡,亡年二十四歲。{\meipi{浪子下場頭,往往如此。}}那手下的大小厮天喜兒,從子虛病倒之時,就拐了五兩銀子走的無蹤。子虛一倒了頭,李瓶兒就使馮媽媽請了西門慶過去,與他商議買棺入殮,念經發送,到墳上安葬。那花大、花三、花四一般兒男婦,也都來弔孝送殯。{\pangpi{伏。}}西門慶那日也教吳月娘辦了一張桌席,與他山頭祭奠。當日婦人轎子歸家,也設了一箇靈位,供養在房中。雖是守靈,一心只想着西門慶。從子虛在日,就把兩箇丫頭教西門慶耍了,子虛死後,越發通家往還。一日,正値正月初九,李瓶兒打聽是潘金蓮生日,未曾過子虛五七,李瓶兒就買禮物坐轎子,穿白綾襖兒,藍織金裙,白紵布鬏髻,珠子箍兒,來與金蓮做生日。馮媽媽抱氊包,天福兒跟轎。進門先與月娘磕了四箇頭,說道:「前日山頭多勞動大娘受餓,又多謝重禮。」拜了月娘,又請李嬌兒、孟玉樓拜見了。然後潘金蓮來到,說道:「這位就是五娘?」{\pangpi{寫出神交之久。}}又要磕下頭去,一口一聲稱呼:「姐姐,{\pangpi{應前。}}請受奴一禮兒。」金蓮那裡肯受,相讓了半日,兩箇還平磕了頭。金蓮又謝了他壽禮。又有吳大妗子、潘姥姥一同見了。{\meipi{拜見先後輕重節次,字字有心,直從太史公筆法化來。}}李瓶兒便請西門慶拜見。月娘道:「他今日往門外玉皇廟打醮去了。」一面讓坐了,喚茶來吃了。良久,只見孫雪娥走過來。李瓶兒見他粧飾少次於衆人,便起身來問道:「此位是何人?奴不知,不曾請見得。」月娘道:「此是他姑娘哩。」李瓶兒就要行禮。月娘道:「不勞起動二娘,只是平拜拜兒罷。」於是彼此拜畢,月娘就讓到房中,換了衣裳,分付丫鬟,明間內放桌兒擺茶。須臾,圍爐添炭,酒泛羊羔,安排上酒來。讓吳大妗子、潘姥姥、李瓶兒上坐,月娘和李嬌兒主席,孟玉樓和潘金蓮打橫。孫雪娥回廚下照管,不敢久坐。月娘見李瓶兒鍾鍾酒都不辭,於是親自遞了一遍酒,又令李嬌兒衆人各遞酒一遍,因嘲問他話兒道:「花二娘搬的遠了,俺姊妹們離多會少,好不思想。二娘狠心,就不說來看俺們看見?」孟玉樓便道:「二娘今日不是因與六姐做生日,還不來哩!」李瓶兒道:「好大娘,三娘,蒙衆娘擡舉,{\pangpi{不敢惡識一箇。}}奴心裡也要來,一者熱孝在身,二者家下沒人。昨日纔過了他五七,不是怕五娘怪,還不敢來。」因問:「大娘貴降在幾時?」月娘道:「賤日早哩。」潘金蓮接過來道:「大娘生日是八月十五,二娘好歹來走走。」李瓶兒道:「不消說,已定都來。」孟玉樓道:「二娘今日與俺姊妹相伴一夜兒,不往家去罷了。」李瓶兒道:「奴可知也要和衆位娘叙些話兒。不瞞衆位娘說,小家兒人家,初搬到那裡,自從他沒了,家下沒人,奴那房子後墻緊靠着喬皇親花園,好不空!{\pangpi{伏。}}晚夕常有狐狸拋磚掠瓦,奴又害怕。原是兩箇小厮,那箇大小厮又走了,止是這箇天福兒小厮看守前門,後半截通空落落的。倒虧了這箇老馮,是奴舊時人,常來與奴漿洗些衣裳。」月娘因問:「老馮多少年紀?且是好箇恩實媽媽兒,高大言也沒句兒。」李瓶兒道:「他今年五十六歲,男花女花都沒,只靠說媒度日。我這裡常管他些衣裳。昨日拙夫死了,叫過他來與奴做伴兒,晚夕同丫頭一炕睡。」潘金蓮嘴快,說道:「既有老馮在家裡看家,二娘在這裡過一夜也不妨,左右你花爹沒了,有誰管着你!」{\pangpi{尖。}}玉樓道:「二娘只依我,叫老馮回了轎子,不去罷。」那李瓶兒只是笑,不做聲。{\pangpi{心事可想。}}話說中間,酒過數巡。潘姥姥先起身往前邊去了。潘金蓮隨跟着他娘往房裡去了。李瓶兒再三辭道:「奴的酒勾了。」李嬌兒道:「花二娘怎的,在他大娘、三娘手裡肯吃酒,偏我遞酒,二娘不肯吃?顯的有厚薄。」遂拿箇大盃斟上。李瓶兒道:「好二娘,奴委的吃不去了,豈敢做假!」月娘道:「二娘,你吃過此盃,畧歇歇兒罷。」那李瓶兒方纔接了,放在面前,只顧與衆人說話。孟玉樓見春梅立在旁邊,便問春梅:「你娘在前邊做甚麼哩?{\meipi{已沒得說,又別生枝。}}你去連你娘、潘姥姥快請來,就說大娘請來陪你花二娘吃酒哩。」春梅去不多時,回來道:「姥姥害身上疼,睡哩。俺娘在房裡勻臉,就來。」月娘道:「我倒也沒見,他倒是箇主人家,把客人丟了,三不知往房裡去了。諸般都好,只是有這些孩子氣。」{\meipi{妒之若怪,愛之最嫣。}}有詩為證:

\begin{myquote}
倦來汗濕羅衣徹,樓上人扶上玉梯。\\歸到院中重洗面,金盆水裡發紅泥。
\end{myquote}

正說着,只見潘金蓮走來。玉樓在席上看見他艷抹濃粧,從外邊搖擺將來,戲道:「五丫頭,你好人兒!今日是你箇驢馬畜,把客人丟在這裡,你躲到房裡去了,你可成人養的!」那金蓮笑嘻嘻向他身上打了一下。{\pangpi{媚致可想。}}玉樓道:「好大膽的五丫頭!你還來遞一鍾兒。」李瓶兒道:「奴在三娘手裡吃了好少酒兒,也都勾了。」金蓮道:「他手裡是他手裡帳,我也敢奉二娘一鍾兒。」於是滿斟一大鍾遞與李瓶兒。李瓶兒只顧放着不肯吃。月娘因看見金蓮鬢上撇着一根金壽字簪兒,便問:「二娘,你與六姐這對壽字簪兒,是那裡打造的?倒好樣兒。到明日俺每人照樣也配恁一對兒戴。」李瓶兒道:「大娘既要,奴還有幾對,到明日每位娘都補奉上一對兒。此是過世老公公御前帶出來的,外邊那裡有這樣範!」月娘道:「奴取笑鬬二娘耍子。俺姐妹們人多,那裡有這些相送!」衆女眷飲酒歡笑。看看日西時分,馮媽媽在後邊雪娥房裡管待酒,吃的臉紅紅的出來,催逼李瓶兒道:「起身不起身?好打發轎子回去。」月娘道:「二娘不去罷,叫老馮回了轎子家去罷。」李瓶兒說:「家裡無人,改日再奉看衆位娘,有日子住哩。」孟玉樓道:「二娘好執古,俺衆人就沒些兒分上?如今不打發轎子,等住回他爹來,少不的也要留二娘。」{\meipi{玉樓亦有此毒語,然而隱隱湊趣。}}自這說話,逼迫的李瓶兒就把房門鑰匙遞與馮媽媽,說道:「既是他衆位娘再三留我,顯的奴不識敬重。分付轎子回去,教他明日來接罷。你和小厮家去,仔細門戶。」又教馮媽媽附耳低言:「教大丫頭迎春,拿鑰匙開我床房裡頭一箇箱子,小描金頭面匣兒裡,拿四對金壽字簪兒。你明日早送來,我要送四位娘。」那馮媽媽得了話,拜辭了月娘,一面出門,不在話下。

少頃,李瓶兒不肯吃酒,月娘請到上房,同大妗子一處吃茶坐的。忽見玳安抱進氊包,西門慶來家,掀開簾子進來,說道:「花二娘在這裡!」慌的李瓶兒跳起身來,兩箇見了禮,坐下。月娘叫玉簫與西門慶接了衣裳。西門慶便對吳大妗子、李瓶兒說道:「今日門外玉皇廟聖誕打醮,該我年例做會首,與衆人在吳道官房裡算帳。七担八柳纏到這咱晚。」因問:「二娘今日不家去罷了?」玉樓道:「二娘再三不肯,要去,被俺衆姐妹強着留下。」李瓶兒道:「家裡沒人,奴不放心。」西門慶道:「沒的扯淡,這兩日好不巡夜的甚緊,怕怎的!但有些風吹草動,拿我箇帖兒送與周大人,點到奉行。」又道:「二娘怎的冷清清坐着?{\pangpi{開端妙。}}{\meipi{分明一句閑話,又及時又攤眼。說來妙不容言。}}用了些酒兒不曾?」孟玉樓道:「俺衆人再三勸二娘,二娘只是推不肯吃。」西門慶道:「你們不濟,等我勸二娘。二娘好小量兒!」{\pangpi{插入無痕。}}李瓶兒口裡雖說:「奴吃不去了。」只不動身。一面分付丫鬟,從新房中放桌兒,都是留下伺候西門慶的嗄飯菜蔬、細巧菓仁,擺了一張桌子。吳大妗子知局,推不用酒,因往李嬌兒房裡去了。{\pangpi{打發得乾淨。}}當下李瓶兒上坐,西門慶關席,吳月娘在炕上跐着爐壺兒。{\pangpi{錯綜得妙。}}孟玉樓、潘金蓮兩邊打橫。五人坐定,把酒來斟,也不用小鍾兒,都是大銀衢花鍾子,你一盃,我一盞。常言:「風流茶說合,酒是色媒人。」吃來吃去,吃的婦人眉黛低橫,秋波斜視。{\meipi{一低字,一斜字,寫出女人醉態。}}正是:

\begin{myquote}
兩朵桃花上臉來,眉眼施開眞色相。
\end{myquote}

月娘見他二人吃得餳成一塊,言頗涉邪,看不上,{\meipi{飲酒中不序一語,只用「餳成一塊」十一字包括,而當時嬉笑狎暱情景宛然。人知其煩,而不知其簡之妙如此。}}往那邊房裡陪吳大妗子坐去了,由着他四箇吃到三更時分。李瓶兒星眼乜斜,立身不住,拉金蓮往後邊淨手。西門慶走到月娘房裡,亦東倒西歪,問月娘打發他那裡歇。月娘道:「他來與那箇做生日,就在那箇房兒裡歇。」西門慶道:「我在那裡歇?」{\pangpi{緊接,妙。}}月娘道:「隨你那裡歇,再不你也跟了他一處去歇罷。」西門慶忍不住笑道:「豈有此理!」因叫小玉來脫衣:「我在這房裡睡了。」{\pangpi{扯白得妙。}}月娘道:「就別要汗邪,休要惹我那沒好口的罵出來!你在這裡,他大妗子那裡歇?」西門慶道:「罷,罷!{\meipi{「罷,罷」不得已死心之辭也。至此方死心,不知心先想着何處?}}我往孟三兒房裡歇去罷,」於是往玉樓房中歇了。潘金蓮引着李瓶兒淨了手,同往他前邊來,就和姥姥一處歇臥。到次日起來,臨鏡梳粧,春梅伏侍。他因見春梅靈變,知是西門慶用過的丫頭,與了他一副金三事兒。{\meipi{處處收拾人心,瓶兒亦自不俗。}}那春梅連忙就對金蓮說了。金蓮謝了又謝,說道:「又勞二娘賞賜他。」李瓶兒道:「不枉了五娘有福,好箇姐姐!」梳粧畢,金蓮領着他同潘姥姥,叫春梅開了花園門,各處遊看。李瓶兒看見他那邊墻頭開了箇便門,通着他那壁,便問:「西門爹幾時起蓋這房子?」金蓮道:「前者陰陽看來,說到這二月間興工動土,要把二娘那房子開啟,通做一處,前面蓋山子捲棚,展一箇大花園;後面還蓋三間玩花樓,與奴這三間樓做一條邊。」這李瓶兒聽了在心。

只見月娘使了小玉來,請後邊吃茶。三人同來到上房。吳月娘、李嬌兒、孟玉樓陪着吳大妗子,擺下茶等着哩。衆人正吃點心,只見馮媽媽進來,向袖中取出一方舊汗巾,包着四對金壽字簪兒,遞與李瓶兒。李瓶兒先奉了一對與月娘,然後李嬌兒、孟玉樓、孫雪娥每人都是一對。月娘道:「多有破費二娘,這箇卻使不得。」李瓶兒笑道:「好大娘,甚麼稀罕之物,胡亂與娘們賞人便了。」月娘衆人拜謝了,方纔各人插在頭上。月娘道:「聞說二娘家門首就是燈市,好不熱鬧。到明日我們看燈,就往二娘府上望望,休要推不在家。」李瓶兒道:「奴到那日,奉請衆位娘。」金蓮道:「姐姐還不知,奴打聽來,這十五日是二娘生日。」月娘道:「今日說過,若是二娘貴降的日子,俺姊妹一箇也不少,來與二娘祝壽。」李瓶兒笑道:「蝸居小室,娘們肯下降,奴已定奉請。」不一時吃罷早飯,擺上酒來飲酒。看看留連到日西時分,轎子來接,李瓶兒告辭歸家。衆姐妹款留不住。臨出門,請西門慶拜見。月娘道:「他今日早起身,出門與人家送行去了。」婦人千恩萬謝,方纔上轎來家。正是:

\begin{myquote}
合歡核桃眞堪愛,裡面原來別有仁。
\end{myquote}

