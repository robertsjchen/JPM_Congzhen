\includepdf[pages={77,78},fitpaper=false]{tst.pdf}
\chapter*{第三十九回 寄法名官哥穿道服 散生日敬濟拜冤家}
\addcontentsline{toc}{chapter}{第三十九回 寄法名官哥穿道服 散生日敬濟拜冤家}
\markboth{{\titlename}卷之四}{第三十九回 寄法名官哥穿道服 散生日敬濟拜冤家}


詩曰:

\begin{myquote}
漢武清齋夜築壇,自斟明水醮仙官。\\殿前玉女移香案,雲際金人捧露盤。\\絳節幾時還入夢?碧桃何處更驂鸞?\\茂陵烟雨埋弓劍,石馬無聲蔓草寒。
\end{myquote}

話說當日西門慶在潘金蓮房中歇了一夜。那婦人恨不的鑽入他腹中,在枕畔千般貼戀,萬種牢籠,淚搵鮫鮹,語言溫順,實指望買住漢子心。不料西門慶外邊又刮剌上了王六兒,替他獅子街石橋東邊,使了一百二十兩銀子,買了一所房屋居住。門面兩間,到底四層,一層做客位,一層供養佛像祖先,一層做住房,一層做廚房。自從搬過來,那街坊隣舍知他是西門慶夥計,不敢怠慢,都送茶盒與他,又出人情慶賀。那中等人家稱他做韓大哥、韓大嫂。以下者趕着以叔嬸稱之。西門慶但來他家,韓道國就在鋪子裡上宿,教老婆陪他自在頑耍。朝來暮往,街坊人家也都知道這件事,懼怕西門慶有錢有勢,誰敢惹他!見一月之間,西門慶也來行走三四次,與王六兒打的一似火炭般熱。看看臘月時分,西門慶在家亂着送東京並府縣、軍衛、本衛衙門中節禮。有玉皇廟吳道官使徒弟送了四盒禮物,並天地疏、新春符、謝竈誥。西門慶正在上房吃飯,玳安兒拿進帖來,上寫着:「玉皇廟小道吳宗嚞頓首拜。」西門慶看了說道:「出家人,又教他費心。」分付玳安,叫書童兒封一兩銀子拿回帖與他。月娘在旁,因話題起道:「一箇出家人,你要便年頭節尾受他的禮物,到把前日你為李大姐生孩兒許的願醮,就叫他打了罷。」西門慶道:「早是你題起來,我許下一百二十分醮,我就忘死了。」月娘道:「原來你是箇大謅答子貨!誰家願心是忘記的?你便有口無心許下,神明都記着。嗔道孩兒成日恁啾啾唧唧的,想就是這願心未還,壓的他。」{\meipi{是婦人信神口角。}}西門慶道:「既恁說,正月裡就把這醮願,在吳道官廟裡還了罷。」月娘道:「昨日李大姐說,這孩子有些病痛兒的,要問那裡討箇外名。」西門慶道:「又往那裡討外名?就寄名在吳道官廟裡就是了。」因問玳安:「他廟裡有誰在這裡?」玳安道:「是他第二箇徒弟應春跟禮來的。」西門慶一面走出外邊來,那應春連忙磕頭說道:「家師父多拜上老爹,沒什麼孝順,使小徒弟來送這天地疏並些微禮兒,與老爹賞人。」西門慶止還了半禮,說道:「多謝你師父厚禮。」一面讓他坐。應春道:「小道怎麼敢坐!」西門慶道:「你坐了,我有話和你說。」那道士頭戴小帽,身穿青布直裰,謙遜數次,方纔把椅兒挪到旁邊坐下,問道:「老爹有甚釣語分付?」西門慶道:「正月裡,我有些醮願,要煩你師父替我還還兒,就要送小兒寄名,不知你師父閑不閑?」徒弟連忙立起身來說道:「老爹分付,隨問有甚經事不敢應承?請問老爹,訂在正月幾時?」西門慶道:「就訂在初九,爺旦日罷。」徒弟道:「此日正是天誕。又《玉匣記》上我請律爺交慶,五福駢臻,修齋建醮甚好。請問老爹多少醮款?」西門慶道:「今歲七月,為生小兒許了一百二十分清醮。」徒弟又問:「那日延請多少道衆?」西門慶道:「請十六衆罷。」說畢,左右放桌兒待茶。先封十五兩經錢,另外又是一兩酬答他的節禮,又說:「道衆的襯施,你師父不消備辦,我這裡連阡張香燭一事帶去。」喜歡的道士屁滾尿流,臨出門謝了又謝,磕了頭兒又磕。

到正月初八日,先使玳安兒送了一石白米、一担阡張、十斤官燭、五斤沉檀馬牙香、十六疋生眼布做襯施,又送了一對京段、兩罈南酒、四隻鮮鵝、四隻鮮雞、一對豚蹄、一脚羊肉、十兩銀子,與官哥兒寄名之禮。西門慶預先發帖兒,請下吳大舅、花大舅、應伯爵、謝希大四位相陪。陳敬濟騎頭口,先到廟中替西門慶瞻拜。到初九日,西門慶也沒往衙門中去,絕早冠帶,騎大白馬,僕從跟隨,前呼後擁,竟出東門往玉皇廟來。遠遠望見結綵寶幡,過街榜棚。須臾至山門前下馬,睜眼觀看,果然好座廟宇。但見:

\begin{myquote}
青松鬱鬱,翠柏森森。金釘朱戶,玉橋低影軒宮;碧瓦雕簷,繡幙高懸寶檻。七間大殿,中懸勅額金書;兩廡長廊,彩畫天神帥將。三天門外,離婁與師曠猙獰,左右堦前,白虎與青龍猛勇。八寶殿前,侍立是長生玉女,九龍床上,坐着箇不壞金身。金鐘撞處,三千世界盡皈依;玉磬鳴時,萬象森羅皆拱極。朝天閣上,天風吹下步虛聲;演法壇中,夜月常聞仙珮響。自此便為眞紫府,更於何處覓蓬萊?
\end{myquote}

西門慶繇正門而入,見頭一座流星門上,七尺高朱紅牌架,列着兩行門對,大書:

\begin{myquote}
黃道天開,祥啟九天之閶闔,迓金輿翠蓋以延恩;

玄壇日麗,光臨萬聖之幡幢,誦寶笈瑤章而闡化。
\end{myquote}

到了寶殿上,懸着二十四字齋題,大書着:「靈寶答天謝地報國酬恩九轉玉樞酬盟寄名吉祥普滿齋壇。」兩邊一聯:

\begin{myquote}
先天立極,仰大道之巍巍,庸申至悃;

昊帝尊居,鑒清修之翼翼,上報洪恩。
\end{myquote}

西門慶進入壇中香案前,旁邊一小童捧盆中盥手畢,鋪排跪請上香。西門慶行禮叩壇畢,只見吳道官頭戴玉環九陽雷巾,身披天青二十八宿大袖鶴氅,腰繫絲帶,忙下經筵來,與西門慶稽首道:「小道蒙老爹錯愛,迭受重禮,使小道卻之不恭,受之有愧。就是哥兒寄名,小道禮當叩祝,增延壽命,何以有叨老爹厚賞,誠有愧赧。經襯又且過厚,令小道愈不安。」西門慶道:「厚勞費心辛苦,無物可酬,薄禮表情而已。」叙禮畢,兩邊道衆齊來稽首。一面請去外方丈——三間廠廳,名曰松鶴軒,那裡待茶。西門慶剛坐下,就令棋童兒:「拿馬接你應二爹去。只怕他沒馬,如何這咱還沒來?」玳安道:「有姐夫騎的驢子還在這裡。」西門慶道:「也罷,快騎接去。」棋童應諾去了。吳道官誦畢經,下來遞茶,陪西門慶坐,叙話:「老爹敬神一點誠心,小道都從四更就起來,到壇諷誦諸品仙經,今日三朝九轉玉樞法事,都是整做。又將官哥兒的生日八字,另具一文書,奏名於三寶面前,起名叫做吳應元。永保富貴遐昌。小道這裡,又添了二十四分答謝天地,{\pangpi{感謝。}}十二分慶贊上帝,二十四分薦亡,共列一百八十分醮款。」西門慶道:「多有費心。」不一時,打動法鼓,請西門慶到壇看文書。西門慶從新換了大紅五彩獅補吉服,腰繫蒙金犀角帶,到壇,有絳衣表白在旁,先宣念齋意:

\begin{myquote}[\markfont]
大宋國山東清河縣縣牌坊居住,奉道祈恩,酬醮保安,信官西門慶,本命丙寅年七月廿八日子時建生,同妻吳氏,本命戊辰年八月十五日子時建生。

\kaishu{表白道:「還有寶眷,小道未曾添上。」西門慶道:「你只添上箇李氏,辛未年正月十五日卯時建生,同男官哥兒,丙申年七月廿三日申時建生罷。」表白文宣過一遍,接念道:}

領家眷等,即日投誠,拜干洪造。伏念慶一介微生,三才末品。出入起居,每感龍天之護佑;迭遷寒暑,常蒙神聖以匡扶。職列武班,叨承禁衛,沐恩光之寵渥,享符祿之豐盈。是以修設清醮,共二十四分位,答報天地之洪恩,酬祝皇王之巨澤。又修清醮十二分位,茲逢天誕,慶贊帝眞。介五福以遐昌,迓諸天而下邁。慶又於去歲七月二十三日,因為側室李氏生男官哥兒,要祈坐蓐無虞,臨盆有慶。又願將男官哥兒寄於三寶殿下,賜名吳應元,告許清醮一百二十分位,續箕裘之胤嗣,保壽命之延長。附薦西門氏門中三代宗親等魂:祖西門京良,祖妣李氏;先考西門達,妣夏氏;故室人陳氏,及前亡後化,昇墜罔知。是以修設清醮十二分位,恩資道力,均證生方。共列仙醮一百八十分位,仰干化單,俯賜勾銷。謹以宣和三年正月初九日天誕良辰,特就大慈玉皇殿,仗延官道,修建靈寶,答天謝地,報國酬盟,慶神保安,寄名轉經,吉祥普滿大齋一晝夜。延三境之司尊,迓萬天之帝駕。一門長叨均安,四序公和迪吉。統資道力,介福方來。謹意。
\end{myquote}

宣畢齋意,鋪設下許多文書符命、表白,一一請看,共有一百八九十道,甚是齊整詳細。又是官哥兒三寶蔭下寄名許多文書、符索、牒劄,不暇細覽。西門慶見吳道官十分費心,于是向案前炷了香,畫了文書,叫左右捧一疋尺頭,與吳道官畫字。吳道官固辭再三,方令小童收了。然後一箇道士向殿角頭咕碌碌擂動法鼓,有若春雷相似。合堂道衆,一派音樂響起。吳道官身披大紅五彩法氅,脚穿朱履,手執牙笏,關發文書,登壇召將。兩邊鳴起鍾來。鋪排引西門慶進壇裡,向三寶案左右兩邊上香。西門慶睜眼觀看,果然鋪設齋壇齊整。但見:

\begin{myquote}
位按五方,壇分八級。上供三清四御,旁分八極九霄,中列山川嶽瀆,下設幽府冥官。香騰瑞靄,千枝畫燭流光;花簇錦筵,百盞銀燈散彩。天地亭,高張羽蓋;玉帝堂,密佈幢幡。金鐘撞處,高功躡步奏虛皇;玉珮鳴時,都講登壇朝玉帝。絳綃衣,星辰燦爛;美蒙冠,金碧交加。監壇神將猙獰,直日功曹猛勇。青龍隱隱來黃道,白鶴翩翩下紫宸。
\end{myquote}

西門慶剛遶壇拈香下來,被左右就請到松鶴軒閣兒裡,地鋪錦毯,爐焚獸炭,那裡坐去了。不一時,應伯爵、謝希大來到。唱畢喏,每人封了一星折茶銀子,說道:「實告要送些茶兒來,路遠。這些微意,權為一茶之需。」西門慶也不接,說道:「奈煩!自恁請你來陪我坐坐,又幹這營生做什麼?吳親家這裡點茶,我一總都有了。」應伯爵連忙又唱喏,說:「哥,眞箇?俺每還收了罷。」因望着謝希大說道:「都是你幹這營生!我說哥不受,拿出來,倒惹他訕兩句好的。」{\pangpi{收拾得妙。}}良久,吳大舅、花子繇都到了。每人兩盒細茶食來點茶,西門慶都令吳道官收了。吃畢茶,一同擺齋,鹹食齋饌,點心湯飯,甚是豐潔。西門慶同吃了早齋。原來吳道官叫了箇說書的,說西漢評話《鴻門會》。{\pangpi{寫出道家行徑。}}吳道官發了文書,走來陪坐,問:「哥兒今日來不來?」西門慶道,「正是,小頑還小哩,房下恐怕路遠諕着他,來不的。到午間,拿他穿的衣服來,三寶面前,攝受過就是一般。」吳道官道:「小道也是這般計較,最好。」西門慶道:「別的倒也罷了,他只是有些小膽兒。家裡三四箇丫鬟連養娘輪流看視,只是害怕。貓狗都不敢到他跟前。」{\pangpi{冷脈。}}吳大舅道:「孩兒們好容易養活大……」正說着,只見玳安進來說:「裡邊桂姨、銀姨使了李銘、吳惠送茶來了。」西門慶道:「叫他進來。」李銘、吳惠兩箇拿着兩箇盒子跪下,揭開都是頂皮餅、松花餅、白糖萬壽糕、玫瑰搽穰捲兒。西門慶俱令吳道官收了,因問李銘:「你每怎得知道?」李銘道:「小的早晨路見陳姑夫騎頭口,問來,纔知道爹今日在此做好事。歸家告訴桂姐、三媽說,旋約了吳銀姐,纔來了。多上覆爹,本當親來,不好來得,這粗茶兒與爹賞人罷了。」西門慶分付:「你兩箇等着吃齋。」吳道官一面讓他二人下去,自有坐處,聯手下人都飽食一頓。

話休饒舌。到了午朝拜表畢,吳道官預備了一張大插桌,又是一罈金華酒,又是哥兒的一頂青段子綃金道髻,一件玄色紵絲道衣,一件綠雲段小襯衣,一雙白綾小襪,一雙青潞紬衲臉小履鞋,一根黃絨線縧,一道三寶位下的黃線索,一道子孫娘娘面前紫線索,一付銀項圈條脫,刻着「金玉滿堂,長命富貴」,一道朱書闢非黃綾符,上書着「太乙司命,桃延合康」八字,就紮在黃線索上,都用方盤盛着,又是四盤羹菓,擺在桌上。差小童經袱內包着宛紅紙經疏,將三朝做過法事,一一開載節次,請西門慶過了目,方纔裝入盒担內。共約八擡,送到西門慶家。西門慶甚是歡喜,快使棋童兒家去,叫賞道童兩方手帕、一兩銀子。

且說那日是潘金蓮生日,有吳大妗子、潘姥姥、楊姑娘、郁大姐,都在月娘上房坐的。見廟裡送了齋來,又是許多羹菓插卓禮物,擺了四張桌子,還擺不下,都亂出來觀看。金蓮便道:「李大姐,你還不快出來看哩!你家兒子師父廟裡送禮來了,又有他的小道冠髻,道衣兒。噫,{\pangpi{宛然。}}你看,又是小履鞋兒!」孟玉樓走向前,拿起來手中看,說道:「大姐姐,你看道士家也恁精細,這小履鞋,白綾底兒,都是倒扣針兒方勝兒,鎖的這雲兒又且是好。我說他敢有老婆!不然,怎的扣捺的恁好針脚兒?」吳月娘道:「沒的說。他出家人,那裡有老婆!{\pangpi{畢竟老實。}}想必是顧人做的。」潘金蓮接過來說:「道士有老婆,相王師父和大師父會挑的好汗巾兒,莫不是也有漢子?」{\meipi{玉樓因針線之細而想及道士有老婆;金蓮又因老婆一語想及尼姑有漢子。一層深一層,二美何等穎悟。}}王姑子道:「道士家,掩上箇帽子,那裡不去了!似俺這僧家,行動就認出來。」{\meipi{王姑子似微露眞情,又似明作戲謔,說得帶水拖泥,妙甚。}}金蓮說道:「我聽得說,你住的觀音寺背後就是玄明觀。常言道:男僧寺對着女僧寺,沒事也有事。」月娘道:「這六姐,好恁羅說白道的!」金蓮道:「這箇是他師父與他娘娘寄名的紫線鎖。又是這箇銀脖項符牌兒,上面銀打的八箇字,帶着且是好看。背面墜着他名字,吳什麼元?」棋童道:「此是他師父起的法名吳應元。」金蓮道:「這是箇『應』字。」{\meipi{識字淺,方傳金蓮之神,知此則知前後寄詞題詩,未免墜小傳說也。}}叫道:「大姐姐,道士無禮,怎的把孩子改了他的姓?」{\pangpi{大議論反為情所掩,可悲。}}月娘道:「你看不知禮!」因使李瓶兒:「你去抱了你兒子來,穿上這道衣,俺每瞧瞧好不好?」李瓶兒道:「他纔睡下,又抱他出來?」金蓮道:「不妨事,你揉醒他。」那李瓶兒眞箇去了。

這潘金蓮識字,取過紅紙袋兒,扯出送來的經疏,看見上面西門慶底下同室人吳氏,旁邊只有李氏,再沒別人,心中就有幾分不忿,拿與衆人瞧:「你說賊三等兒九格的強人!你說他偏心不偏心?這上頭只寫着生孩子的,把俺每都是不在數的,都打到贅字型大小裡去了。」孟玉樓問道:「可有大姐姐沒有?」金蓮道:「沒有大姐姐倒好笑。」{\pangpi{答得畧過一層,妙甚。}}{\meipi{便無月娘,或又作別語矣。}}月娘道:「也罷了,有了一箇,也就是一般。莫不你家有一隊伍人,也都寫上,惹的道士不笑話麼?」金蓮道:「俺每都是劉湛兒鬼兒麼?比那箇不出材的,那箇不是十箇月養的哩!」正說着,李瓶兒從前邊抱了官哥兒來。孟玉樓道:「拿過衣服來,等我替哥哥穿。」李瓶兒抱着,孟玉樓替他戴上道髻兒,套上項牌和兩道索,諕的那孩子只把眼兒閉着,半日不敢出氣兒。玉樓把道衣替他穿上。吳月娘分付李瓶兒:「你把這經疏,拿箇阡張頭兒,親往後邊佛堂中,自家燒了罷。」那李瓶兒去了。玉樓抱弄孩子說道:「穿着這衣服,就是箇小道士兒。」金蓮接過來說道:「什麼小道士兒,倒好相箇小太乙兒!」被月娘正色說了兩句道:「六姐,你這箇什麼話,孩兒們面上,快休恁的。」那金蓮訕訕的不言了。{\meipi{陰毒人必不以口嘴傷人,金蓮一味口嘴傷人,畢竟還淺,吾故辨其蓄貓陰害官哥,為未必然也。}}一回,那孩子穿着衣服害怕,就哭起來。李瓶兒走來,連忙接過來,替他脫衣裳時,就拉了一抱裙奶屎。孟玉樓笑道:「好箇吳應元,原來拉屎也有一托盤。」月娘連忙叫小玉拿草紙替他抹。不一時,那孩子就磕伏在李瓶兒懷裡睡着了。李瓶兒道:「小大哥原來困了,媽媽送你到前邊睡去罷。」

吳月娘一面把桌面都散了,請大妗子、楊姑娘、潘姥姥衆人出來吃齋。看看晚來。

原來初八日,西門慶因打醮,不用葷酒。潘金蓮晚夕就沒曾上的壽,直等到今晚來家與他遞酒,來到大門站立。不想等到日落時分,只陳敬濟和玳安自騎頭口來家。潘金蓮問:「你爹來了?」敬濟道:「爹怕來不成了,我來時,醮事還未了,纔拜懺,怕不弄到起更!道士有箇輕饒素放的,還要謝將吃酒。」金蓮聽了,一聲兒沒言語,使性子回到上房裡,對月娘說:「『賈瞎子傳操——幹起了箇五更』,『隔墻掠肝腸——死心塌地』,『兜肚斷了帶子——沒得絆了』。{\meipi{用方言處,不減引經。}}剛纔在門首站了一回,見陳姐夫騎頭口來了,說爹不來了,醮事還沒了,先打發他來家。」月娘道:「他不來罷,咱每自在,晚夕聽大師父、王師父說因果、唱佛曲兒。」{\meipi{只一語,便遞入宣卷,捷甚。}}正說着,只見陳敬濟掀簾進來,已帶半酣兒,說:「我來與五娘磕頭。」問大姐:「有鍾兒,尋箇兒篩酒,與五娘遞一鍾兒。」大姐道:「那裡尋鍾兒去?只恁與五娘磕箇頭兒。到住回,等我遞罷。你看他醉的腔兒,恰好今日打醮,只好了你,吃的恁憨憨的來家。」月娘便問道:「你爹眞箇不來了?玳安那奴才沒來?」陳敬濟道:「爹見醮事還沒了,恐怕家裡沒人,先打發我來了,留下玳安在那裡答應哩。吳道士再三不肯放我,強死強活拉着吃了兩三大鍾酒,{\meipi{道士最好吃人酒,藉口寫出,可謂空中一閣。}}纔了。」月娘問:「今日有那幾箇在那裡?」敬濟道:「今日有大舅和門外花大舅、應二叔、謝三叔,又有李銘、吳惠兩箇小優兒。不知纏到多咱晚。只吳大舅來了。門外花大舅叫爹留住了,也是過夜的數。」金蓮沒見李瓶兒在跟前,便道:「陳姐夫,你也叫起花大舅來?是那門兒親,死了的知道罷了。你叫他李大舅纔是。」{\meipi{花大舅,李瓶兒大伯也,而謂之大舅,名分原糊塗甚矣。金蓮道破,雖毀之而未為過也。}}敬濟道:「五娘,你老人『家鄉里姐姐嫁鄭恩——睜着箇眼兒,閉着箇眼兒罷了』。」大姐道:「賊囚根子,快磕了頭,趁早與我外頭挺去!{\pangpi{隱隱為瓶兒。}}又口裡恁汗邪胡說了!」敬濟于是請金蓮轉上,踉踉蹌蹌磕了四箇頭,{\pangpi{畫。}}往前邊去了。不一時,掌上燈燭,放桌兒,擺上菜兒,請潘姥姥、楊姑娘、大妗子與衆人來。金蓮遞了酒,打發坐下,吃了面。吃到酒闌,收了家活,擡了桌出去。月娘分付小玉把儀門關了,炕上放下小桌兒,衆人圍定兩箇姑子,正在中間焚下香,秉着一對蠟燭,聽着他說因果。先是大師父講說,講說的乃是西天第三十二祖下界降生東土,傳佛心印的佛法因果,直從張員外家豪大富說起,漫漫一程一節,直說到員外感悟佛法難聞,棄了家園富貴,竟到黃梅寺修行去。說了一回,王姑子又接念偈言。念了一回,吳月娘道:「師父餓了,且把經請過,吃些甚麼。」一面令小玉安排了四碟兒素菜鹹食,又四碟薄脆、蒸酥糕餅,請大妗子、楊姑娘、潘姥姥陪二位師父吃。大妗子說:「俺每都剛吃的飽了,教楊姑娘陪箇兒罷,他老人家又吃着箇齋。」月娘連忙用小描金碟兒,每樣揀了點心,放在碟兒裡,先遞與兩位師父,然後遞與楊姑娘,說道:「你老人家陪二位請些兒。」婆子道:「我的佛爺,老身吃的勾了。」又道:「這碟兒裡是燒骨朵,姐姐你拿過去,只怕錯揀到口裡。」把衆人笑的了不得。月娘道:「奶奶,這箇是廟上送來托葷鹹食。你老人家只顧用,不妨事。」楊姑娘道:「既是素的,等老身吃。老身乾淨眼花了,只當做葷的來。」正吃着,只見來興兒媳婦子惠香走來。月娘道:「賊臭肉,你也來什麼?」惠香道:「我也來聽唱曲兒。」月娘道:「儀門關着,你打那裡進來了?」玉簫道:「他廚房封火來。」月娘道:「嗔道恁鼻兒烏嘴兒黑的,成精鼓搗,來聽什麼經!」{\meipi{此一段似可省而不省,文情纖回之妙正在此。}}

當下衆丫鬟婦女圍定兩箇姑子,吃了茶食,收過家活去,搽抹經桌乾淨。月娘從新剔起燈燭來,炷了香。兩箇姑子打動擊子兒,又高念起來。從張員外在黃梅山寺中修行,白日長跪聽經,夜夜參禪打坐。四祖禪師見他不凡,收留做了徒弟,與了他三樁寶貝,教他往濁河邊投胎奪舍,直說到千金小姐在濁河邊洗濯衣裳,見一僧人借房兒住,不合答了他一聲,那老人就跳下河去了。潘金蓮熬的磕困上來,就往房裡睡去了。{\pangpi{必至之情。}}少頃,李瓶兒房中綉春來叫,說官哥兒醒了,也去了。只剩下李嬌兒、孟玉樓、潘姥姥、孫雪娥、楊姑娘、大妗子守着。又聽到河中漂過一箇大鱗桃來,小姐不合吃了,歸家有孕,懷胎十月。王姑子又接唱了一箇《耍孩兒》。唱完,大師父又念了四偈言:

\begin{myquote}
五祖一佛性,投胎在腹中。\\權住十箇月,轉凡度衆生。
\end{myquote}

念到此處,月娘見大姐也睡去了,大妗子𢱉在月娘裡間床上睡着了,楊姑娘也打起欠呵來,{\meipi{一房睏倦,情景宛然。}}桌上蠟燭也點盡了兩根,問小玉:「這天有多少晚了?」小玉道:「已是四更天氣,雞叫了。」月娘方令兩位師父收拾經卷。楊姑娘便往玉樓房裡去了。郁大姐在後邊雪娥房裡宿歇。月娘打發大師父和李嬌兒一處睡去了。王姑子和月娘在炕上睡。兩箇還等着小玉頓了一瓶子茶,吃了纔睡。大妗子在裡間床上和玉簫睡。月娘因問王姑子:「後來這五祖長大了,怎生成正果?」{\meipi{睡上床還要問完,妙得其情。}}王姑子復從爹娘怎的把千金小姐趕出,小姐怎的逃生,來到仙人庄;又怎的降生五祖,落後五祖養活到六歲;又怎的一直走到濁河邊,取了三樁寶貝,逕往黃梅寺聽四祖說法;又怎的遂成正果,後來還度脫母親生天;直說完了纔罷。月娘聽了,越發好信佛法了。有詩為證:

\begin{myquote}
聽法聞經怕無常,紅蓮舌上放毫光。\\何人留下禪空話?留取尼僧化飯糧!
\end{myquote}

 

