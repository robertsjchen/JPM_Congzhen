\includepdf[pages={131,132},fitpaper=false]{tst.pdf}
\chapter*{第六十六回 翟管家寄書致賻 黃眞人發牒薦亡}
\addcontentsline{toc}{chapter}{第六十六回 翟管家寄書致賻 黃眞人發牒薦亡}
\markboth{{\titlename}卷之七}{第六十六回 翟管家寄書致賻 黃眞人發牒薦亡}


詞曰:

\begin{myquote} 
胸中千種愁,掛在斜陽樹。綠葉陰陰自得春,草滿鶯啼處。不見淩波步,空想如簧語。門外重重疊疊山,遮不斷愁來路。

\raggedleft{——右調《卜筭子》\rightquadmargin}
\end{myquote} 

話說西門慶陪吳大舅、應伯爵等飲酒中間,因問韓道國:「客夥中標船幾時起身?咱好收拾打包。」韓道國道:「昨日有人來會,也只在二十四日開船。」西門慶道:「過了二十念經,打包便了。」伯爵問道:「這遭起身,那兩位去?」西門慶道:「三箇人都去。明年先打發崔大哥押一船杭州貨來,他與來保還往松江下五處,置買些布貨來賣。家中段貨紬綿都還有哩。」{\meipi{西門慶只以生意為本,不盡改換門閭,大是高處,恐今人有不及者矣。}}伯爵道:「哥主張極妙。常言道:要的般般有,纔是買賣。」說畢,已有起更時分,吳大舅起身說:「姐夫連日辛苦,俺每酒已勾了,告回,你可歇息歇息。」西門慶不肯,還留住,令小優兒奉酒唱曲,每人吃三鍾纔放出門。西門慶賞小優四人六錢銀子,再三不敢接,說:「宋爺出票叫小的每來,官身如何敢受老爹重賞?」西門慶道:「雖然官差,此是我賞你,怕怎的!」四人方磕頭領去。西門慶便歸後邊歇去了。

次日早起,往衙門中去。早有吳道官差了一箇徒弟、兩名鋪排,來大廳上鋪設壇場,鋪設的齊齊整整。西門慶來家看見,打發徒弟鋪排齋食吃了回去。隨即令溫秀才寫帖兒,請喬大戶、吳大舅、吳二舅、花大舅、沈姨夫、孟二舅、應伯爵、謝希大、常峙節、吳舜臣許多親眷並堂客,明日念經。家中廚役落作,治辦齋供不題。次日五更,道衆皆來,進入經壇內,明燭焚香,打動響樂,諷誦諸經,鋪排大門首掛起長幡,懸弔榜文,兩邊黃紙門對一聯,大書:

\begin{myquote} 
東極垂慈,仙識乘晨而超登紫府;

南丹赦罪,淨魄受練而逕上朱陵。
\end{myquote} 

大廳經壇,懸掛齋題二十字,大書:「青玄救苦,頒符告簡,五七轉經,水火練度,薦揚齋壇。」即日,黃眞人穿大紅,坐牙轎,系金帶,左右圍隨,儀從暄喝,日高方到。吳道官率衆接至壇所,行禮畢,然後西門慶着素衣絰巾,拜見遞茶畢。洞案旁邊安設經筵法席,大紅銷金桌圍,粧花椅褥,二道童侍立左右。發文書之時,西門慶備金段一疋;登壇之時,換了九陽雷巾,大紅金雲白百鶴法氅。先是表白宣畢齋意,齋官沐手上香。然後黃眞人焚香淨壇,飛符召將,關發一應文書符命,啟奏三天,告盟十地。三獻禮畢,打動音樂,化財行香。西門慶與陳敬濟執手爐跟隨,排軍喝路,前後四把銷金傘、三對纓絡挑搭。

行香回來,安請監齋畢,又動音樂,往李瓶兒靈前攝召引魂,朝參玉陛,旁設几筵,聞經悟道。到了午朝,高功冠裳,步罡踏斗,拜進朱表,遣差神將,飛下羅酆。原來黃眞人年約三旬,儀表非常,粧束起來,午朝拜表,儼然就是箇活神仙。但見:

\begin{myquote} 
星冠攢玉葉,鶴氅縷金霞。神清似長江皓月,貌古如太華喬松。踏罡朱履進丹霄,步虛琅函浮瑞氣。長髯廣頰,修行到無漏之天;皓齒明眸,佩籙掌五雷之令。三更步月鸞聲遠,萬里乘雲鶴背高。就是都仙太史臨凡世,廣惠眞人降下方。
\end{myquote} 

拜了表文,吳道官當壇頒生天寶籙神虎玉劄。行畢午香,捲棚內擺齋。黃眞人前,大桌面定勝;吳道官等,稍加差小;其餘散衆,俱平頭桌席。黃眞人、吳道官皆襯段尺頭、四對披花、四疋絲紬,散衆各布一疋。桌面俱令人擡送廟中,散衆各有手下徒弟收入箱中,不必細說。吃畢午齋,都往花園內遊玩散食去了。一面收下家伙,從新擺上齋饌,請吳大舅等衆親朋夥計來吃。

正吃之間,忽報:「東京翟爺那裡差人下書。」西門慶即出廳上,請來人進來。只見是府前承差幹辦,青衣窄褲,萬字頭巾,乾黃靴,全副弓箭,向前施禮。西門慶答禮相還。那人向身邊取出書來遞上,又是一封折賻儀銀十兩。問來人上姓,那人道:「小人姓王名玉,蒙翟爺差遣,送此書來。不知老爹這邊有䘮事,安老爹書到纔知。」西門慶問道:「你安老爹書幾時到的?」那人說:「十月纔到京。因催皇木一年已滿,陞都水司郎中。如今又奉勅修理河道,直到工完回京。」西門慶問了一遍,即令來保廂房中管待齋飯,分付明日來討回書。那人問:「韓老爹在那裡住?宅內稍信在此。小的見了,還要趕往東平府下書去。」西門慶即喚出韓道國來見那人,陪吃齋飯畢,同往家中去了。西門慶拆看書中之意,於是乘着喜歡,將書拿到捲棚內教溫秀才看。{\meipi{一喜便泄,方知安石鎭物之難。}}說:「你照此修一封回書答他,就稍寄十方縐紗汗巾、十方綾汗巾、十副揀金挑牙、十箇烏金酒盃作回奉之禮。他明日就來取回書。」溫秀才接過書來觀看,其書曰:

\begin{myquote}[\markfont]
寓京都眷生翟謙頓首,書奉即擢大錦堂西門四泉親家大人門下:自京邸話別之後,未得從容相叙,心甚歉然。其領教之意,生已於家老爺前悉陳之矣。邇者,安鳳山書到,方知老親家有鼓盆之嘆,但恨不能一弔為悵,奈何,奈何!伏望以禮節哀可也。外具賻儀,少表微忱,希管納。又久仰貴任榮修德政,舉民有五絝之歌,境內有三留之譽,今歲考績,必有甄陞。昨日神運都功,兩次工上,生已對老爺說了,安上親家名字。工完題奏,必有恩典,親家必有掌刑之喜。夏大人年終類本,必轉京堂指揮列銜矣。謹此預報,伏惟高照,不宣。

{\kaishu{附云:}}此書可自省覽,不可使聞之於渠。謹密,謹密!

{\kaishu{又云:}}

楊老爺前月二十九日卒于獄。{\pangpi{又完冷案。}}

\raggedleft{冬上澣具。\rightquadmargin}
\end{myquote}

溫秀才看畢,纔待袖,早被應伯爵取過來,觀看了一遍,還付與溫秀才收了。說道:「老先生把回書千萬加意做好些。翟公府中人才極多,休要教他笑話。」溫秀才道:「貂不足,狗尾續。學生匪才,焉能在班門中弄大斧!不過乎塞責而已。」西門慶道:「溫老先他自有箇主意,你這狗才曉的甚麼!」須臾,吃罷午齋,西門慶分付來興兒打發齋饌,送各親眷街隣。又使玳安回院中李桂姐、吳銀兒、鄭愛月兒、韓釧兒、洪四兒、齊香兒六家香儀人情禮去。每家回答一疋大布、一兩銀子。後晌,就叫李銘、吳惠、鄭奉三箇小優兒來伺候。良久,道衆陞壇發擂,上朝拜懺觀燈,解壇送聖。天色漸晚。比及設了醮,就有起更天氣。門外花大舅被西門慶留下不去了,喬大戶、沈姨夫、孟二舅告辭回家。止有吳大舅、二舅、應伯爵、謝希大、溫秀才、常峙節並衆夥計在此,晚夕觀看水火練度。就在大廳棚內搭高座,紮綵橋,安設水池火沼,放擺斛食。李瓶兒靈位另有几筵幃幕,供獻齊整。旁邊一首魂幡、一首紅幡、一首黃幡,上書「制魔保舉,受練南宮」。先是道衆音樂兩邊列座,持節捧盂劍,四箇道童侍立兩邊。黃眞人頭戴黃金降魔冠,身披絳綃雲霞衣,登高座,口中念念有詞。宣偈云:

\begin{myquote} 
太乙慈尊降駕來,夜壑幽關次第開。\\童子雙雙前引導,死魂受練步雲堦。
\end{myquote} 

宣偈畢,又薰沐焚香,念曰:

\begin{myquote}[\markfont]
伏以玄皇闡教,廣開度於冥途;正一垂科,俾練形而昇舉。恩沾幽爽,澤被飢嘘。謹運眞香,志誠上請東極大慈仁者,太乙救苦天尊、十方救苦諸眞人聖衆,仗此眞香,來臨法會。切以人處塵凡,日縈俗務,不知有死,惟欲貪生。鮮能種於善根,多隨入於惡趣,昏迷弗省,恣欲貪嗔。將謂自己長存,豈信無常易到!一朝傾逝,萬事皆空。業障纏身,冥司受苦。今奉道伏為亡過室人李氏靈魂,一棄塵緣,久淪長夜。若非薦拔於愆辜,必致難逃於苦報。恭惟天尊秉好生之仁,救尋聲之苦。灑甘露而普滋羣類,放瑞光而遍燭昏衢。命三官寬考較之條,詔十殿閣推研之筆。開囚釋禁,宥過解冤。各隨符使,盡出幽關。咸令登火池之沼,悉盪滌黃華之形。凡得更生,俱歸道岸。茲焚靈寶練形眞符,謹當宣奏:

太微迴黃旗,無英命靈幡,\\攝召長夜府,開度受生魂。

\end{myquote} 

道衆先將魂幡安於水池內,焚結靈符,換紅幡;次於火沼內焚鬱儀符,換黃幡。高功念:「天一生水,地二生火,水火交練,乃成眞形。」練度畢,請神主冠帔步金橋,朝參玉陛,皈依三寶,朝玉清,衆舉《五供養》。舉畢,高功曰:「既受三皈,當宣九戒。」九戒畢,道衆舉音樂,宣念符命並《十類孤魂》。練度已畢,黃眞人下高座,道衆音樂送至門外,化財焚燒箱庫。回來,齋功圓滿,{\meipi{眞人舉動宣念,仍是衆道之舉動宣念,別無玄妙。想玄妙處不可以語言求也。}}道衆都換了冠服,鋪排收捲道像。西門慶又早大廳上畫燭齊明,酒筵羅列。三箇小優彈唱,衆親友都在堂前。西門慶先與黃眞人把盞,左右捧着一疋天青雲鶴金段、一疋色段、十兩白銀,叩首下拜道:「亡室今日賴我師經功救拔,得遂超生,均感不淺,微禮聊表寸心。」黃眞人道:「小道謬忝冠裳,濫膺玄教,有何德以達人天?皆賴大人一誠感格,而尊夫人已駕景朝元矣。此禮若受,實為赧顏。」西門慶道:「此禮甚薄,有褻眞人,伏乞笑納!」黃眞人方令小童收了。西門慶遞了眞人酒,又與吳道官把盞,乃一疋金段、五兩白銀,又是十兩經資。吳道官只受經資,餘者不肯受,說:「小道素蒙厚愛,自恁效勞誦經,追拔夫人往生仙界,以盡其心。受此經資尚為不可,又豈敢當此盛禮乎!」西門慶道:「師父差矣。眞人掌壇,其一應文簡法事,皆乃師父費心。此禮當與師父酬勞,何為不可?」

吳道官不得已,方領下,再三致謝。西門慶與道衆遞酒已畢,然後吳大舅、應伯爵等上來與西門慶散福遞酒。吳大舅把盞,伯爵執壺,謝希大捧菜,一齊跪下。伯爵道:「嫂子今日做此好事,幸請得眞人在此,又是吳師父費心,嫂子自得好處。此雖賴眞人追薦之力,實是哥的虔心,嫂子的造化。」於是滿斟一盃送與西門慶。西門慶道:「多蒙列位連日勞神,言謝不盡。」{\pangpi{謝得妙。}}說畢,一飲而盡。伯爵又斟一盞,說:「哥,吃箇雙盃,不要吃單盃。」謝希大慌忙遞一筯菜來吃了。西門慶回敬衆人畢,安席坐下。小優彈唱起來,廚役上割道。當夜在席前猜拳行令,品竹彈絲,直吃到二更時分,西門慶已帶半酣,衆人方作辭起身而去。西門慶進來賞小優兒三錢銀子,往後邊去了。正是:

\begin{myquote}
人生有酒須當醉,一滴何曾到九泉。
\end{myquote}

