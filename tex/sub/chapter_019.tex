\includepdf[pages={37,38},fitpaper=false]{tst.pdf}
\chapter*{第十九回 草裡蛇邏打蔣竹山 李瓶兒情感西門慶}
\addcontentsline{toc}{chapter}{第十九回 草裡蛇邏打蔣竹山 李瓶兒情感西門慶}
\markboth{{\titlename}卷之二}{第十九回 草裡蛇邏打蔣竹山 李瓶兒情感西門慶}


詩曰:

\begin{myquote}
人靡不有初,想君能終之。\\別來歷年歲,舊恩何可期。\\重新而忘故,君子所猶譏。\\寄身雖在遠,豈忘君須臾。\\既厚不為薄,想君時見思。
\end{myquote}

話說西門慶起蓋花園捲棚,約有半年光陰,裝修油漆完備,前後煥然一新。慶房的整吃了數日酒,俱不在話下。一日,八月初旬,與夏提刑做生日,在新買庄上擺酒。叫了四箇唱的、一起樂工、雜耍步戲。西門慶從巳牌時分,就騎馬去了。

吳月娘在家,整置了酒餚細菓,約同李嬌兒、孟玉樓、孫雪娥、大姐、潘金蓮衆人,開了新花園門遊賞。裡面花木庭臺,一望無際,端的好座花園。但見:

\begin{myquote}
正面丈五高,周圍二十板。當先一座門樓,四下幾間臺榭。假山眞水,翠竹蒼松。高而不尖謂之臺,巍而不峻謂之榭。四時賞玩,各有風光:春賞燕遊堂,桃李爭妍;夏賞臨溪館,荷蓮鬬彩;秋賞疊翠樓,黃菊舒金;冬賞藏春閣,白梅橫玉。更有那嬌花籠淺徑,芳樹壓雕欄,弄風楊柳縱蛾眉,帶雨海棠陪嫩臉。燕遊堂前,燈光花似開不開;藏春閣後,白銀杏半放不放。湖山側纔綻金錢,寶檻邊初生石筍。翩翩紫燕穿簾幙,嚦嚦黃鶯度翠陰。也有那月窻雪洞,也有那水閣風亭。木香棚與荼䕷架相連,千葉桃與三春柳作對。松墻竹徑,曲水方池,映堦蕉棕,向日葵榴。遊魚藻內驚人,粉蝶花間對舞。
\end{myquote}

正是:

\begin{myquote}
芍藥展開菩薩面,荔枝擎出鬼王頭。
\end{myquote}

當下吳月娘領着衆婦人,或攜手遊芳徑之中,或鬬草坐香茵之上。一箇臨軒對景,戲將紅荳擲金鱗;一箇伏檻觀花,笑把羅紈驚粉蝶。{\meipi{不減西園雅集。}}月娘於是走在一箇最高亭子上,名喚臥雲亭,和孟玉樓、李嬌兒下棋。潘金蓮和西門大姐、孫雪娥都在翫花樓望下觀看。見樓前牡丹花畔芍藥圃、海棠軒、薔薇架、木香棚,又有耐寒君子竹,欺雪大夫松。端的四時有不謝之花,八節有長春之景。觀之不足,看之有餘。不一時擺上酒來,吳月娘居上,李嬌兒對席,兩邊孟玉樓、孫雪娥、潘金蓮、西門大姐,各依序而坐。月娘道:「我忘了請姐夫來坐坐。」{\meipi{處處是月娘作俑。}}一面使小玉:「前邊快請姑夫來。」不一時,敬濟來到,頭上天青羅帽,身穿紫綾深衣,脚下粉頭皁靴,向前作揖,就在大姐跟前坐下。傳盃換盞,吃了一回酒,吳月娘還與李嬌兒、西門大姐下棋。孫雪娥與孟玉樓卻上樓觀看。惟有金蓮,且在山子前花池邊,用白紗團扇撲蝴蝶為戲。不妨敬濟悄悄在他背後戲說道:「五娘,你不會撲蝴蝶兒,等我替你撲。這蝴蝶兒忽上忽下,心不定,有些走滾。」{\pangpi{挑逗亦妙。}}那金蓮扭回粉頸,斜瞅了他一眼,罵道:「賊短命,人聽着,你待死也!我曉得你也不要命了。」{\meipi{罵得狠甚,卻又情甚,眞千金不能移易一字。}}那敬濟笑嘻嘻撲近他身來,摟他親嘴。被婦人順手只一推,把小夥兒推了一交。卻不想玉樓在翫花樓遠遠瞧見,叫道:「五姐,你走這裡來,我和你說話。」金蓮方纔撇了敬濟,上樓去了。原來兩箇蝴蝶到沒曾捉得住,到訂了燕約鶯期,則做了蜂須花嘴。正是:

\begin{myquote}
狂蜂浪蝶有時見,飛入梨花沒尋處。
\end{myquote}

敬濟見婦人去了,默默歸房,心中怏怏不樂。口占《折桂令》一詞,以遣其悶:

\begin{myquote}
我見他斜戴花枝,朱唇上不抹胭脂,似抹胭脂。前日相逢,今日相逢,似有私情,未見私情。欲見許,何曾見許!似推辭,本是不推辭。約在何時?會在何時?不相逢,他又相思;既相逢,我又相思。
\end{myquote}

且不說吳月娘等在花園中飲酒。單表西門慶從門外夏提刑庄子上吃了酒回家,打南瓦子巷裡頭過。平昔在三街兩巷行走,搗子們都認的——宋時謂之「搗子」,今時俗呼為「光棍」——內中有兩箇,一名草裡蛇魯華,一名過街鼠張勝,常受西門慶資助,乃雞竊狗盜之徒。西門慶見他兩箇在那裡耍錢,就勒住馬,上前說話。二人連忙走到跟前,打箇半跪道:「大官人,這咱晚往那裡去來?」西門慶道:「今日是提刑所夏老爹生日,門外庄上請我們吃了酒來。我有一樁事央煩你們,依我不依?」二人道:「大官人,沒的說,小人平昔受恩甚多,如有使令,雖赴湯蹈火,萬死何辭!」西門慶道:「既是恁說,明日來我家,我有話分付你。」二人道:「那裡等的到明日!你老人家說與小人罷,端的有甚麼事?」西門慶附耳低言,便把蔣竹山要了李瓶兒之事說了一遍:「只要你弟兄二人替我出這口氣兒便了!」因在馬上摟起衣底順袋中,還有四五兩碎銀子,都倒與二人。便道:「你兩箇拿去打酒吃。只要替我幹得停當,還謝你二人。」魯華那裡肯接,說道:「小人受你老人家恩還少哩!我只道教俺兩箇往東洋大海里拔蒼龍頭上角,西華嶽山中取猛虎口中牙,便去不的,這些小之事,有何難哉!這箇銀兩,小人斷不敢領。」西門慶道:「你不收,我也不央及你了。」教玳安接了銀子,打馬就走。又被張勝攔住說:「魯華,你不知他老人家性兒?你不收,恰似咱每推脫的一般。」一面接了銀子,扒到地下磕了頭,說道:「你老人家只顧家裡坐着,不消兩日,管情穩抇抇教你笑一聲。」{\meipi{彼此俱不說破如何出氣,最有養蓄。}}張勝道:「只望大官人到明日,把小人送與提刑夏老爹那裡答應,就勾了小人了。」西門慶道:「這箇不打緊。」後來西門慶果然把張勝送在守備府做了箇親隨。{\pangpi{伏。}}此係後事,表過不題。那兩箇搗子,得了銀子,依舊耍錢去了。

西門慶騎馬來家,已是日西時分。月娘等衆人,聽見他進門,都往後邊去了,只有金蓮在捲棚內看收家活。西門慶不往後邊去,逕到花園裡來,見婦人在亭子上收家伙,便問:「我不在,你在這裡做甚麼來?」金蓮笑道:「俺們今日和大姐姐開門看了看,誰知你來的恁早。」西門慶道:「今日夏大人費心,庄子上叫了四箇唱的,只請了五位客到。我恐怕路遠,來的早。」婦人與他脫了衣裳,因說道:「你沒酒,教丫頭看酒來你吃。」西門慶分付春梅:「把別的菜蔬都收下去,只留下幾碟細菓子兒,篩一壺葡萄酒來我吃。」坐在上面椅子上,因看見婦人上穿沉香色水緯羅對襟衫兒,五色縐紗眉子,下着白碾光絹挑線裙兒,裙邊大紅段子白綾高低鞋兒。頭上銀絲鬏髻,金鑲分心翠梅鈿兒,雲鬢簪着許多花翠。越顯得紅馥馥朱唇、白膩膩粉臉,{\pangpi{金蓮往往以媚勝。}}不覺淫心輒起,攙着他兩隻手兒,摟抱在一處親嘴。

不一時,春梅篩上酒來,兩箇一遞一口兒飲酒咂舌。婦人一面摳起裙子,坐在身上,噙酒哺在他口裡,然後纖手拈了一箇鮮蓮蓬子,{\pangpi{有致。}}與他吃。西門慶道:「澀剌剌的,{\pangpi{俗甚}}吃他做甚麼?」婦人道:「我的兒,你就弔了造化了,娘手裡拿的東西兒你不吃!」又口中噙了一粒鮮核桃仁兒,{\pangpi{嬌態可人。}}送與他,纔罷了。西門慶又要玩弄婦人的胸乳。婦人一面攤開羅衫,露出美玉無瑕、香馥馥的酥胸,緊就就的香乳。揣摸良久,用口舐之,彼此調笑,曲盡於飛。西門慶乘着歡喜,向婦人道:「我有一件事告訴你,到明日,教你笑一聲。你道蔣太醫開了生藥鋪,到明日管情教他臉上開菓子鋪來。」婦人便問怎麼緣故。西門慶悉把今日門外撞遇魯、張二人之事,告訴了一遍。婦人笑道:「你這箇衆生,到明日不知作多少罪業。」又問:「這蔣太醫,不是常來咱家看病的麼?我見他且是謙恭,見了人把頭只低着,可憐見兒的,你這等做作他!」西門慶道:「你看不出他。你說他低着頭兒,他專一看你的脚哩。」婦人道:「汗邪的油嘴!他可哥看人家老婆的脚?我不信,他一箇文墨人兒,{\meipi{鍾情文墨人為甚,惜金蓮未遇耳。}}也幹這箇營生?」{\pangpi{又映出西門慶面目。}}西門慶道:「你看他迎面兒,就誤了勾當,單愛外裝老成內藏奸詐。」兩箇說笑了一回,不吃酒了,收拾了家活,歸房宿歇,不在話下。

卻說李瓶兒招贅了蔣竹山,約兩月光景。初時蔣竹山圖婦人喜歡,修合了些戲藥,買了些景東人事、美女想思套之類,實指望打動婦人。不想婦人在西門慶手裡狂風驟雨經過的,往往幹事不稱其意,漸生憎惡,反被婦人把淫器之物,都用石砸的稀碎,丟掉了。{\meipi{無眞本事人,往往討此沒趣。}}又說:「你本蝦鱔,腰裡無力,{\meipi{語語淫甚,罵竹山,適所以自罵,妙甚。}}平白買將這行貨子來戲弄老娘!把你當塊肉兒,原來是箇中看不中吃臘鎗頭,死王八!」常被婦人半夜三更趕到前邊鋪子裡睡。於是一心只想西門慶,不許他進房。每日躁聒着算帳,查算本錢。

這竹山正受了一肚氣,走在鋪子小櫃裡坐的,只見兩箇人進來,吃的浪浪蹌蹌,楞楞睜睜,走在櫈子上坐下。{\meipi{羅致情景宛然。}}先是一箇問道:「你這鋪中有狗黃沒有?」竹山笑道:「休要作戲。只有牛黃,那有狗黃?」又問:「沒有狗黃,你有冰灰也罷,拿來我瞧,我要買你幾兩。」竹山道:「生藥行只有冰片,是南海波斯國地道出的,那討冰灰來?」那一箇說道:「你休問他,量他纔開了幾日鋪子,那裡有這兩樁藥材?只與他說正經話罷。蔣二哥,你休推睡裡夢裡。你三年前死了娘子兒,問這位魯大哥借的那三十兩銀子,本利也該許多,今日問你要來了。俺們纔進門就先問你要,你在人家招贅了,初開了這箇鋪子,恐怕䘮了你行止,顯的俺們沒陰騭了。故此先把幾句風話來教你認範。你不認範,他這銀子你少不得還他。」竹山聽了,嚇了箇立睜,說道:「我並沒有借他甚麼銀子。」那人道:「你沒借銀,卻問你討?自古『蒼蠅不鑽那沒縫的蛋』,快休說此話!」竹山道:「我不知閣下姓甚名誰,素不相識,如何來問我要銀子?」那人道:「蔣二哥,你就差了!自古『於官不貧,賴債不富』。想着你當初不得地時,串鈴兒賣膏藥,{\pangpi{入情。}}也虧了這位魯大哥扶持,你今日就到這田地來。」這箇人道:「我便姓魯,叫做魯華,{\pangpi{自叫破姓名,妙。}}你某年借了我三十兩銀子,發送妻小,本利該我四十八兩,少不的還我。」竹山慌道:「我那裡借你銀子來?就借你銀子,也有文書保人。」張勝道:「我張勝就是保人。」因向袖中取出文書,與他照了照。把竹山氣的臉臘查也似黃了,罵道:「好殺才,狗男女!你是那裡搗子,走來嚇詐我!」魯華聽了,心中大怒,隔着小櫃,颼的一拳去,早飛到竹山面門上,就把鼻子打歪在半邊,一面把架上藥材撒了一街。竹山大罵:「好賊搗子!你如何來搶奪我貨物?」因叫天福兒來幫助,被魯華一脚踢過一邊,那裡再敢上前。張勝把竹山拖出小櫃來,攔住魯華手,勸道:「魯大哥,你多日子也耽待了,再寬他兩日兒,教他湊過與你便了。蔣二哥,你怎麼說?」竹山道:「我幾時借他銀子來?就是問你借的,也等慢慢好講,如何這等撒野?」張勝道:「蔣二哥,你這回吃了橄欖灰兒——回過味來了?你若好好早這般,我教魯大哥饒讓你些利錢兒,你便兩三限湊了還他,纔是話。你如何把硬話兒不認,莫不人家就不問你要罷?」那竹山聽了道:「氣殺我,我和他見官去!誰借他甚麼錢來!」張勝道:「你又吃了早酒了!」不提防魯華又是一拳,仰八叉跌了一交,險不倒栽入洋溝裡,將髮散開,巾幘都污濁了。竹山大叫「青天白日」起來,被保甲上來,都一條繩子拴了。

李瓶兒在房中聽見外邊人嚷,走來簾下聽覷,見地方拴的竹山去了,氣的箇立睜。使出馮媽媽來,把牌面幌子都收了。街上藥材,被人搶了許多。一面關閉了門戶,家中坐的。早有人把這件事報與西門慶知道,即差人分付地方,明日早解提刑院。這裡又拿帖子,對夏大人說了。次日早,帶上人來,夏提刑陞廳,看了地方呈狀,叫上竹山去,問道:「你是蔣文蕙?如何借了魯華銀子不還,反行毀打他?其情可惡!」竹山道:「小人通不認的此人,並沒借他銀子。小人以理分說,他反不容,亂行踢打,把小人貨物都搶了。」夏提刑便叫魯華:「你怎麼說?」魯華道:「他原借小的銀兩,發送䘮妻,至今三年,延挨不還。小的今日打聽他在人家招贅,做了大買賣,問他理討,他倒百般辱罵小的,說小的搶奪他的貨物。見有他借銀子的文書在此,這張勝就是保人,望爺察情。」一面懷中取出文契,遞上去。夏提刑展開觀看,寫道:

\begin{myquote}[\markfont]
立借票人蔣文蕙,係本縣醫生,為因妻䘮,無錢發送,憑保人張勝,借到魯華名下白銀三十兩,月利三分,入手用度。約至次年,本利交還,不致少欠。恐後無憑,立此借票存照。
\end{myquote}

夏提刑看了,拍案大怒道:「可又來,見有保人、借票,還這等抵賴。看這厮咬文嚼字模樣,就象箇賴債的。」{\meipi{咬文嚼字人會賴債,毒語罵盡天下。}}喝令左右:「選大板,拿下去着實打。」當下三、四箇人,不繇分說,拖翻竹山在地,痛責三十大板,打的皮開肉綻,鮮血淋漓。一面差兩箇公人,拿着白牌,押蔣竹山到家,處三十兩銀子交還魯華。不然,帶回衙門收監。

那蔣竹山打的兩腿剌八着,走到家,哭哭啼啼,哀告李瓶兒,問他要銀子,還與魯華。又被婦人噦在臉上,罵道:「沒羞的忘八,你遞甚麼銀子在我手裡,問我要銀子?我早知你這忘八砍了頭是箇債庄,就瞎了眼也不嫁你這中看不中吃的忘八!」那四箇人聽見屋裡嚷罵,不住催逼叫道:「蔣文蕙,既沒銀子,不消只管挨遲了,趁早到衙門回話去罷。」竹山一面出來安撫了公人,又去裡邊哀告婦人。直蹶兒跪在地上,{\pangpi{此是竹山長技。}}哭哭啼啼說道:「你只當積陰騭,四山五舍齋佛,布施這三十兩銀子罷!不與,這一回去,我這爛屁股上怎禁的拷打?就是死罷了。」婦人不得已,拿出三十兩雪花銀子與他,{\pangpi{還是好人。}}當官交與魯華,扯碎了文書,方纔完事。

這魯華、張勝得了三十兩銀子,逕到西門慶家回話。西門慶留在捲棚下,管待二人酒飯。把前事告訴了一遍。西門慶滿心大喜說:「二位出了我這口氣,足勾了。」魯華把三十兩銀子交與西門慶,西門慶那裡肯收:「你二人收去,買壺酒吃,就是我酬謝你了。後頭還有事相煩。」二人臨起身謝了又謝,拿着銀子,自行耍錢去了。正是:

\begin{myquote} 
常將壓善欺良意,權作尤雲殢雨心。
\end{myquote} 

卻說蔣竹山提刑院交了銀子,歸到家中。婦人那裡容他住,說道:「只當奴害了汗病,把這三十兩銀子問你討了藥吃了。你趁早與我搬出去罷!再遲些時,連我這兩間房子,尚且不勾你還人!」這蔣竹山自知存身不住,{\pangpi{晚矣。}}哭哭啼啼,忍着兩腿疼,自去另尋房兒。但是婦人本錢置的貨物都留下,{\pangpi{也是好人。}}把他原舊的藥材、藥碾、藥篩、藥箱之物,即時催他搬去,兩箇就開交了。臨出門,婦人還使馮媽媽舀了一盆水,趕着潑去,說道:「喜得冤家離眼睛!」當日打發了竹山出門。這婦人一心只想着西門慶,又打聽得他家中沒事,心中甚是懊悔。{\pangpi{勢利語,可笑。}}每日茶飯慵餐,娥眉懶畫,把門兒倚遍,眼兒望穿,白盼不見一箇人兒來。正是:

\begin{myquote} 
枕上言猶在,於今恩愛淪。\\房中人不見,無語自消魂。
\end{myquote} 

不說婦人思想西門慶。單表一日玳安騎馬打門首經過,看見婦人大門關着,藥鋪不開,靜落落的,歸來告訴與西門慶。西門慶道:「想必那矮忘八打重了,在屋裡睡哩,會勝也得半箇月出不來做買賣。」遂把這事情丟下了。一日,八月十五日,吳月娘生日,家中有許多堂客來,在大廳上坐。西門慶因與月娘不說話,一徑來院中李桂姐家坐的,分付玳安:「早回馬去罷,晚上來接我。」旋邀了應伯爵、謝希大來打雙陸。那日桂卿也在家,姐妹兩箇陪侍勸酒。良久,都出來院子內投壺耍子。玳安約至日西時分,勒馬來接。西門慶正在後邊出恭,見了玳安問:「家中無事?」玳安道:「家中沒事。大廳上堂客都散了,止有大妗子與姑奶奶衆人,大娘邀的後邊去了。今日獅子街花二娘那裡,使了老馮與大娘送生日禮來:{\pangpi{瓶兒面皮老甚。}}四盤羹菓、兩盤壽桃面、一疋尺頭,又與大娘做了一雙鞋。大娘與了老馮一錢銀子,說爹不在家了。也沒曾請去。」{\pangpi{月娘有主意。}}西門慶因見玳安臉紅紅的,便問:「你那裡吃酒來?」玳安道:「剛纔二娘使馮媽媽叫了小的去,與小的酒吃。我說不吃酒,強說着叫小的吃了兩鍾,就臉紅起來。{\meipi{亦善辭。}}如今二娘到悔過來,對着小的好不哭哩。前日我告爹說,爹還不信。從那日提刑所出來,就把蔣太醫打發去了。二娘甚是懊悔,一心還要嫁爹,比舊瘦了好些兒,央及小的好歹請爹過去,討爹示下。爹若吐了口兒,還教小的回他一聲。」西門慶道:「賊賤淫婦,既嫁漢子去罷了,又來纏我怎的?既是如此,我也不得閑去。你對他說,甚麼下茶下禮,揀箇好日子,擡了那淫婦來罷。」{\meipi{數語又氣又喜,卻又不敢再緩,妙於立言。}}玳安道:「小的知道了。他那裡還等着小的去回他話哩,教平安、畫童兒這裡伺候爹就是了。」西門慶道:「你去,我知道了。」這玳安出了院門,一直走到李瓶兒那裡,回了婦人話。婦人滿心歡喜,說道:「好哥哥,今日多累你對爹說,成就了此事。」於是親自下廚整理蔬菜,管待玳安,說道:「你二娘這裡沒人,明日好歹你來幫扶天福兒,着人搬家伙過去。」次日顧了五六副扛,整擡運四五日。西門慶也不對吳月娘說,都堆在新蓋的翫花樓上。擇了八月二十日,一頂大轎,一疋段子紅,四對燈籠,派定玳安、平安、畫童、來興四箇跟轎,約後晌時分,方娶婦人過門。婦人打發兩箇丫鬟,教馮媽媽領着先來了,等的回去,方纔上轎。把房子交與馮媽媽、天福兒看守。

西門慶那日不往那裡去,在家新捲棚內,深衣幅巾坐的,單等婦人進門。婦人轎子落在大門首,半日沒箇人出去迎接。{\pangpi{太沒趣。}}孟玉樓走來上房,對月娘說:「姐姐,你是家主,如今他已是在門首,你不去迎接迎接兒,惹的他爹不怪?他爹在捲棚內坐着,轎子在門首這一日了,沒箇人出去,怎麼好進來的?」這吳月娘欲待出去接他,心中惱,又不下氣;欲待不出去,又怕西門慶性子不是好的。沉吟了半晌,於是輕移蓮步,款蹙湘裙,出來迎接。婦人抱着寶瓶,逕往他那邊新房去了。迎春、綉春兩箇丫鬟,又早在房中鋪陳停當,單等西門慶晚夕進房。不想西門慶正因舊惱在心,不進他房去。到次日,叫他出來後邊月娘房裡見面,分其大小,排行他是六娘。一般三日擺大酒席,請堂客會親吃酒,只是不往他房裡去。頭一日晚夕,先在潘金蓮房中。金蓮道:「他是箇新人兒,{\pangpi{未必新。}}纔來頭一日,你就空了他房?」西門慶道:「你不知淫婦有些眼裡火,等我奈何他兩日,慢慢的進去。」到了三日,打發堂客散了,西門慶又不進他房中,往後邊孟玉樓房裡歇去了。這婦人見漢子一連三夜不進他房來,到半夜,打發兩箇丫鬟睡了,飽哭了一場,可憐走到床上,用脚帶弔頸懸梁自縊。{\meipi{熱人一處冷局,便亂矣。}}正是:

\begin{myquote} 
連理未諧鴛帳底,冤魂先到九重泉。
\end{myquote} 

兩箇丫鬟睡了一覺醒來,見燈光昏暗,起來剔燈,猛見床上婦人弔着,嚇慌了手脚。忙走出隔壁叫春梅說:「俺娘上弔哩!」慌的金蓮起來這邊看視,見婦人穿一身大紅衣裳,直掇掇弔在床上。連忙和春梅把脚帶割斷,解救下來。過了半日,吐了一口清涎,方纔甦醒。即叫春梅:「後邊快請你爹來。」西門慶正在玉樓房中吃酒,還未睡哩。先是玉樓勸西門慶說道:「你娶將他來,一連三日不往他房裡去,惹他心中不惱麼?恰似俺們把這樁事放在頭裡一般,頭上末下就讓不得這一夜兒!」西門慶道:「待過三日兒我去。你不知道,淫婦有些吃着碗裡,看着鍋裡。想起來你惱不過我:未曾你漢子死了,相交到如今,甚麼話兒沒告訴我?臨了招進蔣太醫去!我不如那厮?今日卻怎的又尋將我來?」玉樓道:「你惱的是。他也吃人騙了。」正說話間,忽一片聲打儀門。玉樓使蘭香問,說是春梅來請爹:「六娘在房裡上弔哩!」慌的玉樓攛掇西門慶不迭,便道:「我說教你進他房中走走,你不依,只當弄出事來。」於是打着燈籠,走來前邊看視。落後吳月娘、李嬌兒聽見,都起來,到他房中。見金蓮摟着他坐的,說道:「五姐,你灌了他些薑湯兒沒有?」金蓮道:「我救下來時,就灌了些了。」那婦人只顧喉中哽咽了一回,方哭出聲。月娘衆人一塊石頭纔落地,好好安撫他睡下,各歸房歇息。

次日晌午前後,李瓶兒纔吃些粥湯兒。西門慶向李嬌兒衆人說道:「你們休信那淫婦裝死嚇人。我手裡放不過他。到晚夕等我到房裡去,親看着他上箇弔兒我瞧,不然吃我一頓好馬鞭子。賊淫婦!不知把我當誰哩!」衆人見他這般說,都替李瓶兒捏着把汗。到晚夕,見西門慶袖着馬鞭子,進他房去了。玉樓、金蓮分付春梅把門關了,不許一箇人來,都立在角門首兒外悄悄聽着。

且說西門慶見他睡在床上,倒着身子哭泣,{\pangpi{不見景。}}見他進去不起身,心中就有幾分不悅。先把兩箇丫頭都趕去空房裡住了。西門慶走來椅子上坐下,指着婦人罵道:「淫婦!你既然虧心,何消來我家上弔?你跟着那矮忘八過去便了,誰請你來!我又不曾把人坑了,你甚麼緣故,流那𣭈尿怎的?我自來不曾見人上弔,我今日看着你上箇弔兒我瞧!」於是拿一條繩子丟在他面前,叫婦人上弔。那婦人想起蔣竹山說,西門慶是『打老婆的班頭,降婦女的領袖』,思量:我那世裡晦氣,今日大睜眼又撞入火坑裡來了,越發煩惱痛哭起來。{\meipi{時時轉念,寫出瓶兒之淺。}}這西門慶心中大怒,教他下床來脫了衣裳跪着。婦人只顧延挨不脫,被西門慶拖翻在床地平上,袖中取出鞭子來抽了幾鞭子,{\meipi{雖瓶兒自取,然亦非情人舉止。}}婦人方纔脫去上下衣裳,戰兢兢跪在地平上。西門慶坐着,從頭至尾問婦人:「我那等對你說,教你畧等等兒,我家中有些事兒,如何不依我,慌忙就嫁了蔣太醫那厮?你嫁了別人,我倒也不惱!那矮忘八有甚麼起解?你把他倒踏進門去,拿本錢與他開鋪子,在我眼皮子跟前,要撐我的買賣!」婦人道:「奴不說的,悔也是遲了。只因你一去了不見來,朝思暮想,奴想的心斜了。後邊喬皇親花園裡常有狐狸,要便半夜三更假名托姓變做你,來攝我精髓,到天明雞叫就去了。你不信只要問老馮、兩箇丫頭便知。後來看看把奴攝得至死,纔請這蔣太醫來看。奴就象弔在麵糊盆內一般,吃那厮局騙了。說你家中有事,上東京去了,奴不得已,纔幹下這條路。誰知這厮斫了頭是箇債庄,被人打上門來,經動官府。奴忍氣吞聲,丟了幾兩銀子,吃奴即時攆出去了。」{\meipi{始終無一巧言,瓶兒畢竟老實。使金蓮當此,定另有一番妙舌矣}}西門慶道:「說你叫他寫狀子,告我收着你許多東西。{\pangpi{虛心語。}}你如何今日也到我家來了!」婦人道:「你可是沒的說。奴那裡有這話,就把奴身子爛化了。」西門慶道:「就算有,我也不怕。你說你有錢,快轉換漢子,我手裡容你不得!我實對你說罷,前者打太醫那兩箇人,是如此這般使的手段。只畧施小計,教那厮疾走無門,若稍用機關,也要連你掛了到官,弄倒一箇田地。」婦人道:「奴知道是你使的術兒。還是可憐見奴,若弄到那無人烟之處,就是死罷了。」看看說的西門慶怒氣消下些來了。又問道:「淫婦你過來,我問你,我比蔣太醫那厮誰強?」{\pangpi{又自己出路。}}婦人道:「他拿甚麼來比你!你是箇天,他是塊磚;你在三十三天之上,他在九十九地之下。休說你這等為人上之人,只你每日吃用稀奇之物,他在世幾百年還沒曾看見哩!他拿甚麼來比你!莫要說他,就是花子虛在日,若是比得上你時,奴也不恁般貪你了。{\pangpi{此一轉,妙。}}你就是醫奴的藥一般,一經你手,教奴沒日沒夜只是想你。」自這一句話,把西門慶舊情兜起,歡喜無盡,即丟了鞭子,用手把婦人拉將起來,穿上衣裳,摟在懷裡,說道:「我的兒,你說的是。果然這厮他見甚麼碟兒天來大!」即叫春梅:「快放桌兒,後邊取酒菜兒來!」正是:東邊日出西邊雨,道是無情卻有情。有詩為證:

\begin{myquote} 
碧玉破瓜時,郎為情顛倒。\\感君不羞赧,回身就郎抱。
\end{myquote} 

 

