\chapter*{謝肇淛《金瓶梅》跋}
\addcontentsline{toc}{chapter}{謝肇淛《金瓶梅》跋}
\markboth{\titlename}{謝肇淛《金瓶梅》跋}

《金瓶梅》一書,不著作者名代。相傳永陵中有金吾戚里,憑怙奢汰,淫縱無度,而其門客病之,採摭日逐行事,匯以成編,而托之西門慶也。書凡數百萬言,為卷二十,始末不過數年事耳。其中朝野之政務,官私之晉接,閨闥之媟語,市里之猥談,與夫勢交利合之態,心輸背笑之局,桑中濮上之期,尊罍枕蓆之語,駔驗之機械意智,粉黛之自媚爭妍,狎客之從臾逢迎,奴佁之嵇唇淬語,窮極境象,駥意快心。譬之範工摶泥,妍媸老少,人鬼萬殊,不徒肖其貌,且並其神傳之。信稗官之上乘,爐錘之妙手也。其不及《水滸傳》者,以其猥瑣淫媟,無關名理。而或以為過之者,彼猶機軸相放,而此之面目各別,聚有自來,散有自去,讀者意想不到,唯恐易盡。此豈可與褒儒俗士見哉!此書向無鏤版,鈔寫流傳,參差散失。唯弇州家藏者最為完好。余于袁中郎得其十三,于丘諸城得其十五,稍為釐正,而闕所未備,以俟他日。有嗤余誨淫者,余不敢知。然溱洧之音,聖人不刪,則亦中郎帳中必不可無之物也。倣此者有《玉嬌麗》,然而乖彞敗度,君子無取焉。
  
\begin{quotation}
\raggedleft{謝肇淛\rightquadmargin}
\end{quotation}

