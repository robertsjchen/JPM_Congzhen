\includepdf[pages={57,58},fitpaper=false]{tst.pdf}
\chapter*{第二十九回 吳神仙冰鑑定終身 潘金蓮蘭湯邀午戰}
\addcontentsline{toc}{chapter}{第二十九回 吳神仙冰鑑定終身 潘金蓮蘭湯邀午戰}
\markboth{{\titlename}卷之三}{第二十九回 吳神仙冰鑑定終身 潘金蓮蘭湯邀午戰}


詞曰:

\begin{myquote}
新涼睡起,蘭湯試浴郎偷戲。去曾嗔怒,來便生歡喜。奴道無心,郎道奴如此。情如水,易開難斷,若箇知生死。

\raggedleft{——右調《點絳唇》\rightquadmargin}
\end{myquote}

話說到次日,潘金蓮早起,打發西門慶出門。記掛着要做那紅鞋,{\meipi{偏是這些留心。}}拿着針線筐兒,往翡翠軒臺基兒上坐着,描畫鞋扇。使春梅請了李瓶兒來到。李瓶兒問道:「姐姐,你描畫的是甚麼?」金蓮道:「要做一雙大紅素段子白綾平底鞋兒,鞋尖上扣繡鸚鵡摘桃。」李瓶兒道:「我有一方大紅十樣錦段子,也照依姐姐描恁一雙兒。我做高低的罷。」於是取了針線筐,兩箇同一處做。金蓮描了一隻丟下,說道:「李大姐,你替我描這一隻,等我後邊把孟三姐叫了來。他昨日對我說,他也要做鞋哩。」一直走到後邊。玉樓在房中倚着護炕兒,也衲着一隻鞋兒哩。看見金蓮進來,說道:「你早辦!」金蓮道:「我起來的早,打發他爹往門外與賀千戶送行去了。教我約下李大姐,花園裡趕早涼做些生活。我纔描了一隻鞋,教李大姐替我描着,逕來約你同去,咱三箇一搭兒裡好做。」因問:「你手裡衲的是甚麼鞋?」玉樓道:「是昨日你看我開的那雙玄色段子鞋。」金蓮道:「你好漢!又早衲出一隻來了。」玉樓道:「那隻昨日就衲好了,這一隻又衲了好些了。」金蓮接過看了一回,說:「你這箇,到明日使甚麼雲頭子?」玉樓道:「我比不得你每小後生,花花黎黎。{\pangpi{口角入情。}}我老人家了,使羊皮金緝的雲頭子罷,周圍拿紗綠線鎖,好不好?」金蓮道:「也罷。你快收拾,咱去來,李瓶兒那裡等着哩。」玉樓道:「你坐着吃了茶去。」金蓮道:「不吃罷,拿了茶,那裡去吃來。」玉樓分咐蘭香頓下茶送去。兩箇婦人手拉着手兒,袖着鞋扇,逕往外走。吳月娘在上房穿廊下坐,便問:「你每那去?」金蓮道:「李大姐使我替他叫孟三兒去,與他描鞋。」{\pangpi{開口便是謊,妙。}}說着,一直來到花園內。三人一處坐下,拿起鞋扇,你瞧我的,我瞧你的,{\pangpi{必至之情。}}都瞧了一遍。玉樓便道:「六姐,你平白又做平底子紅鞋做甚麼?不如高底好看。你若嫌木底子響脚,也似我用氊底子,卻不好?」{\meipi{分明要說睡鞋,卻從平底、高底慢慢襯入,何等苦心細脈。}}金蓮道:「不是穿的鞋,是睡鞋。他爹因我那隻睡鞋,被小奴才兒偷去弄油了,分咐教我從新又做這雙鞋。」玉樓道:「又說鞋哩,這箇也不是舌頭,李大姐在這裡聽着。昨日因你不見了這隻鞋,他爹打了小鐵棍兒一頓,說把他打的躺在地下,死了半日。惹的一丈青好不在後邊海罵,罵那箇淫婦王八羔子學舌,打了他恁一頓,早是活了,若死了,淫婦、王八羔子也不得清潔!俺再不知罵的是誰。落後小鐵棍兒進來,大姐姐問他:『你爹為甚麼打你?』小厮纔說:『因在花園裡耍子,拾了一隻鞋,問姑夫換圈兒來。不知是甚麼人對俺爹說了,教爹打我一頓。我如今尋姑夫,問他要圈兒去也。』說畢,一直往前跑了。原來罵的『王八羔子』是陳姐夫。{\meipi{隱含着淫婦,罵的是金蓮,卻不說破,妙甚。}}早是只李嬌兒在旁邊坐着,大姐沒在跟前,若聽見時,又是一場兒。」金蓮道:「大姐姐沒說甚麼?」玉樓道:「你還說哩,大姐姐好不說你哩!說:『如今這一家子亂世為王,九條尾狐狸精出世了,把昏君禍亂的貶子休妻,想着去了的來旺兒小厮,好好的從南邊來了,東一帳西一帳,說他老婆養着主子,又說他怎的拿刀弄杖,生生兒禍弄的打發他出去了,把箇媳婦又逼的弔死了。如今為一隻鞋子,又這等驚天動地反亂。你的鞋好好穿在脚上,怎的教小厮拾了?想必吃醉了,在花園裡和漢子不知怎的餳成一塊,纔掉了鞋。如今沒的摭羞,拿小厮頂缸,{\meipi{月娘只不開口,開口亦毒。}}又不曾為甚麼大事。』」金蓮聽了,道:「沒的扯𣭈淡!甚麼是『大事』?殺了人是大事了,奴才拿刀要殺主子!」向玉樓道:「孟三姐,早是瞞不了你,咱兩箇聽見來興兒說了一聲,諕的甚麼樣兒的!你是他的大老婆,倒說這箇話!你也不管,我也不管,教奴才殺了漢子纔好。他老婆成日在你後邊使喚,你縱容着他不管,教他欺大滅小,和這箇合氣,和那箇合氣。各人冤有頭,債有主,你揭條我,我揭條你,弔死了,你還瞞着漢子不說。早是苦了錢,好人情說下來了,不然怎了?{\meipi{壞人多此一念成之。}}你這等推乾淨,說面子話兒,左右是左右我調唆漢子也罷,若不教他把奴才老婆、漢子一條提攆的離門離戶也不算!恆數人挾不到我井裡頭!」玉樓見金蓮粉面通紅,惱了,又勸道:「六姐,你我姐妹都是一箇人,我聽見的話兒,有箇不對你說?說了,只放在你心裡,休要使出來。」{\meipi{告訴了又勸,學舌人往往如此。}}金蓮不依他。到晚等的西門慶進入他房來,一五一十告西門慶說:「來昭媳婦子一丈青怎的在後邊指罵,說你打了他孩子,要邏揸兒和人嚷。」這西門慶不聽便罷,聽了記在心裡。到次日,要攆來昭三口子出門。多虧月娘再三攔勸下,不容他在家,打發他往獅子街房子裡看守,替了平安兒來家守大門。後次月娘知道,甚惱金蓮,不在話下。

西門慶一日正在前廳坐,忽平安兒來報:「守備府周爺差人送了一位相面先生,名喚吳神仙,在門首伺候見爹。」西門慶喚來人進見,遞上守備帖兒,然後道:「有請。」須臾,那吳神仙頭戴青佈道巾,身穿布袍草履,腰繫黃絲雙穗縧,手執龜殼扇子,自外飄然進來。年約四十之上,生得神清如長江皓月,貌古似太華喬松。原來神仙有四般古怪:身如松,聲如鍾,坐如弓,走如風。但見他:

\begin{myquote}
能通風鑑,善究子平。觀乾象,能識陰陽;察龍經,明知風水。五星深講,三命祕談。審格局,決一世之榮枯;觀氣色,定行年之休咎。若非華岳脩眞客,定是成都賣卜人。
\end{myquote}

西門慶見神仙進來,忙降堦迎接,接至廳上。神仙見西門慶,長揖稽首就坐。須臾茶罷。西門慶動問神仙:「高名雅號,仙鄉何處,因何與周大人相識?」那吳神仙欠身道:「貧道姓吳名奭,道號守眞。本貫浙江仙遊人。自幼從師天台山紫虛觀出家。雲遊上國,因往岱宗訪道,道經貴處。周老總兵相約,看他老夫人目疾,特送來府上觀相。」西門慶道:「老仙長會那幾家陰陽?道那幾家相法?」神仙道:「貧道粗知十三家子平,善曉麻衣相法,又曉六壬神課。常施藥救人,不愛世財,隨時住世。」西門慶聽言,益加敬重,誇道:「眞乃謂之神仙也。」一面令左右放桌兒,擺齋管待。神仙道:「貧道未道觀相,豈可先要賜齋。」西門慶笑道:「仙長遠來,已定未用早齋。待用過,看命未遲。」於是陪着神仙吃了些齋食素饌,擡過桌席,拂抹乾淨,討筆硯來。神仙道:「請先觀貴造,然後觀相尊容。」{\meipi{四柱俱不合,想宋時算命如此。}}西門慶便說與八字:「屬虎的,二十九歲了,七月二十八日午時生。」這神仙暗暗十指尋紋,良久說道:「官人貴造:戊寅年,辛酉月,壬午日,丙午時。七月廿三日白露,已交八月算命。月令提剛辛酉,理取傷官格。子平云:傷官傷盡復生財,財旺生官福轉來。立命申宮,七歲行運辛酉,十七行壬戌,二十七癸亥,三十七甲子,四十七乙丑。官人貴造,依貧道所講,元命貴旺,八字清奇,非貴則榮之造。但戊土傷官,生在七八月,身忒旺了。幸得壬午日干,醜中有癸水,水火相濟,乃成大器。丙午時,丙合辛生,後來定掌威權之職。一生盛旺,快樂安然,發福遷官,主生貴子。為人一生耿直,幹事無二,喜則合氣春風,怒則迅雷烈火。一生多得妻財,不少紗帽戴。{\meipi{「不少」二字微詞,寫出不是正路。}}臨死有二子送老。今歲丁未流年,丁壬相合,目下丁火來尅,尅我者為官為鬼,必主平地登雲之喜,添官進祿之榮。大運見行癸亥,戊土得癸水滋潤,定見發生。目下透出紅鸞天喜,定有熊羆之兆。又命宮驛馬臨申,不過七月必見矣。」西門慶問道:「我後來運限如何?」神仙道:「官人休怪我說,但八字中不宜陰水太多,後到甲子運中,將壬午冲破了,又有流星打攪,不出六六之年,主有嘔血流濃之災,骨瘦形衰之病。」西門慶問道:「目下如何?」神仙道:「目今流年,日逢破敗,五鬼在家炒鬧,些小氣惱,不足為災,都被喜氣神臨門冲散了。」西門慶道:「命中還有敗否?」神仙道:「年趕着月,月趕着日,實難矣。」

西門慶聽了,滿心歡喜,便道:「先生,你相我面如何?」神仙道:「請尊容轉正。」西門慶把座兒掇了一掇。神仙相道:「夫相者,有心無相,相逐心生;有相無心,相隨心往。吾觀官人:頭圓項短,定為享福之人;體健觔強,決是英豪之輩;天庭高聳,一生衣祿無虧;地閣方圓,晚歲榮華定取。此幾樁兒好處。還有幾樁不足之處,貧道不敢說。」西門慶道:「仙長但說無妨。」神仙道:「請官人走兩步看。」西門慶眞箇走了幾步。神仙道:「你行如擺柳,必主傷妻;若無刑剋,必損其身。妻宮尅過方好。」西門慶道:「已刑過了。」神仙道:「請出手來看一看。」西門慶舒手來與神仙看。神仙道:「智慧生於皮毛,苦樂觀於手足。細軟豐潤,必享福祿之人也。兩目雌雄,必主富而多詐;眉生二尾,一生常自足歡娛;根有三紋,中歲必然多耗散;奸門紅紫,一生廣得妻財;黃氣發於高曠,旬日內必定加官;紅色起於三陽,今歲間必生貴子。又有一件不敢說,淚堂豐厚,亦主貪花;且喜得鼻乃財星,驗中年之造化;承漿地閣,管來世之榮枯。

\begin{myquote}
承漿地閣要豐隆,準乃財星居正中。\\生平造化皆由命,相法玄機定不容。」
\end{myquote}

神仙相畢,西門慶道:「請仙長相相房下衆人。」一面令小厮:「後邊請你大娘出來。」於是李嬌兒、孟玉樓、潘金蓮、李瓶兒、孫雪娥等衆人都跟出來,在軟屏後潛聽。神仙見月娘出來,連忙道了稽首,也不敢坐,就立在旁邊觀相。端詳了一回,說:「娘子面如滿月,家道興隆;唇若紅蓮,衣食豐足,必得貴而生子;聲響神清,必益夫而發福。請出手來。」月娘從袖中露出十指春蔥來。神仙道:「乾薑之手,女人必善持家,照人之鬢,坤道定須透氣。這幾樁好處。還有些不足之處,休怪貧道直說。」西門慶道:「仙長但說無妨。」「淚堂黑痣,若無宿疾,必刑夫;眼下皺紋,亦主六親若冰炭。

\begin{myquote}
女人端正好容儀,緩步輕如出水龜。\\行不動塵言有節,無肩定作貴人妻。」
\end{myquote}

相畢,月娘退後。西門慶道:「還有小妾輩,請看看。」於是李嬌兒過來。神仙觀看良久:「此位娘子,額尖鼻小,非側室,必三嫁其夫;肉重身肥,廣有衣食而榮華安享;肩聳聲泣,不賤則孤;鼻梁若低,非貧即夭。{\meipi{只十六字,形容得李嬌兒不堪晤對,下筆惡甚。}}請步幾步我看。」李嬌兒走了幾步。神仙道:

\begin{myquote}
「額尖露背並蛇行,早年必定落風塵。\\假饒不是娼門女,也是屏風後立人。」
\end{myquote}

相畢,李嬌兒下去。吳月娘叫:「孟三姐,你也過來相一相。」神仙觀道:「這位娘子,三停平等,一生衣祿無虧;六府豐隆,晚歲榮華定取。平生少疾,皆因月孛光輝;到老無災,大抵年宮潤秀。請娘子走兩步。」玉樓走了兩步,神仙道:

\begin{myquote}
「口如四字神清澈,溫厚堪同掌上珠。\\威命兼全財祿有,終主刑夫兩有餘。」
\end{myquote}

玉樓相畢,叫潘金蓮過來。那潘金蓮只顧嘻笑,不肯過來。{\meipi{到他便有許多韻致,自令人改觀。}}月娘催之再三,方纔出見。神仙擡頭觀看這箇婦人,沉吟半日,方纔說道:「此位娘子,髮濃髩重,{\pangpi{嫣甚,媚甚。}}光斜視以多淫;臉媚眉彎,身不搖而自顫。面上黑痣,必主刑夫;唇中短促,終須壽夭。

\begin{myquote}
舉止輕浮唯好淫,眼如點漆壞人倫。\\月下星前長不足,雖居大廈少安心。」
\end{myquote}

相畢金蓮,西門慶又叫李瓶兒上來,教神仙相一相。神仙觀看這箇女人:「面板香細,{\pangpi{可愛。}}乃富室之女娘;容貌端莊,乃素門之德婦。只是多了眼光如醉,{\pangpi{畫。}}主桑中之約;眉靨漸生,月下之期難定。觀臥蠶明潤而紫色,必產貴兒;體白肩圓,必受夫之寵愛。常遭疾厄,只因根上昏沉;頻遇喜祥,蓋謂福星明潤。此幾樁好處。還有幾樁不足處,娘子可當戒之:山根青黑,三九前後定見哭聲;法令細繵,雞犬之年焉可過?愼之!愼之!

\begin{myquote}
花月儀容惜羽翰,平生良友鳳和鸞。\\朱門財祿堪依倚,莫把凡禽一樣看。」
\end{myquote}

相畢,李瓶兒下去。月娘令孫雪娥出來相一相。神仙看了,說道:「這位娘子,體矮聲高,額尖鼻小,{\pangpi{八字更醜。}}雖然出谷遷喬,但一生冷笑無情,作事機深內重。只是吃了這四反的虧,後來必主兇亡。夫四反者:唇反無稜,耳反無輪,眼反無神,鼻反不正故也。

\begin{myquote}
燕體蜂腰是賤人,眼如流水不廉眞。\\常時斜倚門兒立,不為婢妾必風塵。」
\end{myquote}

雪娥下去,月娘教大姐上來相一相。神仙道:「這位女娘,鼻梁低露,破祖刑家;聲若破鑼,{\meipi{大姐容貌如此,豈是敬濟對手。}}家私消散。面皮太急,雖溝洫長而壽亦夭;行如雀躍,處家室而衣食缺乏。不過三九,當受折磨。

\begin{myquote}
惟夫反目性通靈,父母衣食僅養身。\\狀貌有拘難顯達,不遭惡死也艱辛。」
\end{myquote}

大姐相畢,教春梅也上來教神仙相相。神仙睜眼兒見了春梅,年約不上二九,頭戴銀絲雲髻兒,白線挑衫兒,桃紅裙子,藍紗比甲兒,纏手纏脚出來,道了萬福。神仙觀看良久,相道:「此位小姐五官端正,骨格清奇。髮細眉濃,稟性要強;神急眼圓,為人急燥。{\pangpi{四語是春梅一幅小像。}}山根不斷,必得貴夫而生子;兩額朝拱,主早年必戴珠冠。行步若飛仙,聲響神清,必益夫而得祿,三九定然封贈。但吃了這左眼大,早年尅父;右眼小,週歲尅娘。{\meipi{神仙諸相雖射覆不失,然過於削直,恐近時術家所難。}}左口角下這一點黑痣,主常沾啾唧之災;右腮一點黑痣,一生受夫敬愛。

\begin{myquote}
天庭端正五官平,口若塗硃行步輕。\\倉庫豐盈財祿厚,一生常得貴人憐。」
\end{myquote}

神仙相畢,衆婦女皆咬指以為神相。西門慶封白銀五兩與神仙,又賞守備府來人銀五錢,拿拜帖回謝。吳神仙再三辭卻,說道:「貧道雲遊四方,風餐露宿,要這財何用?決不敢受。」西門慶不得已,拿出一疋大布:「送仙長一件大衣如何?」神仙方纔受之,令小童接了,稽首拜謝。西門慶送出大門,飄然而去。正是:

\begin{myquote}
柱杖兩頭挑日月,葫蘆一箇隱山川。
\end{myquote}

西門慶回到後廳,問月娘:「衆人所相何如?」月娘道:「相的也都好,只是三箇人相不着。」西門慶道:「那三箇相不着?」月娘道:「相李大姐有實疾,到明日生貴子,他見今懷着身孕,這箇也罷了。相咱家大姐到明日受磨折,不知怎的磨折?相春梅後來也生貴子,或者你用好他,各人子孫也看不見。我只不信,說他後來戴珠冠,有夫人之分。端的咱家又沒官,那討珠冠來?就有珠冠,也輪不到他頭上。」西門慶笑道:「他相我目下有平地登雲之喜,加官進祿之榮,我那得官來?他見春梅和你俱站在一處,又打扮不同,戴着銀絲雲髻兒,只當是你我親生女兒一般,或後來匹配名門,招箇貴婿,故說有珠冠之分。{\meipi{此等議論,揆情度勢,可謂十得其九,然俱是暗中揣摹,毫不着。讀此可銷人炎涼輕薄之念。}}自古『算的着命,算不着好』,相逐心生,相隨心滅。周大人送來,咱不好囂了他的,教他相相除疑罷了。」說畢,月娘房中擺下飯,打發吃了飯。西門慶手拿芭蕉扇兒,信步閑遊。來花園大捲棚聚景堂內,周圍放下簾櫳,四下花木掩映。正値日午,只聞綠陰深處一派蟬聲,忽然風送花香,襲人撲鼻。有詩為證:

\begin{myquote}
綠樹蔭濃夏日長,樓臺倒影入池塘。\\水晶簾動微風起,一架薔薇滿院香。
\end{myquote}

西門慶坐於椅上以扇搖涼。只見來安兒、畫童兒兩箇小厮來井上打水。西門慶道:「教一箇來。」來安兒忙走向前,西門慶分咐:「到後邊對你春梅姐說,有梅湯提一壺來我吃。」來安兒應諾去了。半日,只見春梅家常戴着銀絲雲髻兒,手提一壺蜜煎梅湯,笑嘻嘻走來,{\meipi{相得歡喜,故笑。}}問道:「你吃了飯了?」西門慶道:「我在後邊吃了。」春梅說:「嗔道不進房裡來。說你要梅湯吃,等我放在冰裡湃一湃你吃。」{\pangpi{知趣。}}西門慶點頭兒。春梅湃上梅湯,走來扶着椅兒,取過西門慶手中芭蕉扇兒替他打扇,{\pangpi{更趣。}}問道:「頭裡大娘和你說甚麼?」{\pangpi{問得有成心。}}西門慶道:「說吳神仙相面一節。」春梅道:「那道士平白說戴珠冠,教大娘說『有珠冠,只怕輪不到他頭上』。常言道『凡人不可貌相,海水不可斗量』,從來『旋的不圓,砍的圓』,各人裙帶上衣食,怎麼料得定?莫不長遠只在你家做奴才罷!」{\pangpi{直與後出門不哭相應。}}{\meipi{春梅心眼只寬,非一味說大話。}}西門慶笑道:「小油嘴兒,你若到明日有了娃兒,就替你上了頭。」於是把他摟到懷裡,手扯着手兒頑耍,問道:「你娘在那裡?怎的不見?」春梅道:「娘在屋裡,教秋菊熱下水要洗浴。等不的,就在床上睡了。」西門慶道:「等我吃了梅湯,鬼混他一混去。」於是春梅向冰盆內倒了一甌兒梅湯,與西門慶呷了一口,湃骨之涼,透心沁齒,如甘露灑心一般。須臾吃畢,搭伏着春梅肩膀兒,轉過角門來到金蓮房中。看見婦人睡在正面一張新買的螺鈿床上。原是因李瓶兒房中安着一張螺鈿敞廳床,婦人旋教西門慶使了六十兩銀子,替他也買了這一張螺鈿有欄干的床。兩邊槅扇都是螺鈿攢造花草翎毛,掛着紫紗帳幔,錦帶銀鉤。婦人赤露玉體,止着紅綃抹胸兒,{\pangpi{銷魂。}}蓋着紅紗衾,枕着鴛鴦枕,在涼蓆之上,睡思正濃。西門慶一見,不覺淫心頓起,令春梅帶上門出去,悄悄脫了衣褲,上的床來,掀開紗被,見他玉體相互掩映,戲將兩股輕開,按麈柄徐徐插入牝中,比及星眼驚欠之際,已抽拽數十度矣。婦人睜開眼,笑道:「怪強盜,三不知多咱進來?奴睡着了,就不知道。奴睡的甜甜的,摑混死了我!」{\pangpi{違心語。}}西門慶道:「我便罷了,若是箇生漢子進來,你也推不知道罷?」{\pangpi{想當然耳。}}婦人道:「我不好罵的,誰人七箇頭八箇膽,敢進我這房裡來!只許你恁沒大沒小的罷了。」原來婦人因前日西門慶在翡翠軒誇獎李瓶兒身上白淨,就暗暗將茉莉花蕊兒攪酥油定粉,把身上都搽遍了,{\pangpi{婦人邀寵亦不易。}}搽的白膩光滑,異香可愛,欲奪其寵。西門慶見他身體雪白,穿着新做的兩隻大紅睡鞋。一面蹲踞在上,兩手兜其股,極力而提之,垂首觀其出入之勢。婦人道:「怪貨,只顧端詳甚麼?奴的身上黑,不似李瓶兒的身上白就是了。他懷着孩子,你便輕憐痛惜,俺每是拾的,由着這等掇弄。」{\meipi{開口便夾酸帶妬,所以為妙。}}西門慶問道:「說你等着我洗澡來?」婦人問道:「你怎得知道來?」西門慶道:「是春梅說的。」婦人道:「你洗,我叫春梅掇水來。」不一時把浴盆掇到房中,注了湯。二人下床來,同浴蘭湯,共效魚水之歡。洗浴了一回,西門慶乘興把婦人仰臥在浴板之上,兩手執其雙足跨而提之,掀騰𢵞幹,何止二三百回,其聲如泥中螃蠏一般響之不絕。婦人恐怕香雲拖墜,一手扶着雲髩,{\pangpi{好描畫。}}一手扳着盆沿,口中燕語鶯聲,百般難述。{\pangpi{好描畫。}}怎見這場交戰?但見:

\begin{myquote}
華池盪漾波紋亂,翠幃高捲秋雲暗。才郎情動逞風流,美女心歡顯手段。𥐙𥐙へへ弄聲響,砰砰ぺぺ成一片。滑滑𣺥𣺥怎住停,攔攔濟濟難存站。一箇逆水撐船將玉股搖;一箇艄公把舵將金蓮揝。拖泥帶水兩情癡,殢雨尤雲都不辨。任他錦帳鳳鸞交,不似蘭湯魚水戰。
\end{myquote}

二人水中戰鬬了一回,西門慶精泄而止。拭抹身體乾淨,撤去浴盆。止着薄纊短襦上床,安放炕桌菓酌飲酒。教秋菊:「取白酒來與你爹吃。」又拿菓餡餅與西門慶吃,恐怕他肚中飢餓。只見秋菊半日拿上一銀注子酒來。婦人纔斟了一鍾,摸了摸冰涼的,就照着秋菊臉上只一潑,潑了一頭一臉,罵道:「好賊少死的奴才!我分咐教你燙了來,如何拿冷酒與爹吃?你不知安排些甚麼心兒?」叫春梅:「與我把這奴才採到院子裡跪着去。」春梅道:「我替娘後邊捲裹脚去來,一些兒沒在跟前,你就弄下硶兒了。」那秋菊把嘴谷都着,口裡喃喃吶吶說道:「每日爹娘還吃冰湃的酒兒,誰知今日又改了腔兒。」婦人聽見罵道:「好賊奴才,你說甚麼?與我採過來!」叫春梅每邊臉上打與他十箇嘴巴。春梅道:「皮臉,沒的打污濁了我手。娘只教他頂着石頭跪着罷。」於是不繇分說,拉到院子裡,教他頂着塊大石頭跪着,不在話下。婦人從新叫春梅煖了酒來,陪西門慶吃了幾鍾,掇去酒桌,放下紗帳子來,分咐拽上房門,兩箇抱頭交股,體倦而寢。正是:

\begin{myquote}
若非羣玉山頭見,多是陽臺夢裡尋。
\end{myquote}


