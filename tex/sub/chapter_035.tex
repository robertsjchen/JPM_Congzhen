\includepdf[pages={69,70},fitpaper=false]{tst.pdf}
\chapter*{第三十五回 西門慶為男寵報仇 書童兒作女粧媚客}
\addcontentsline{toc}{chapter}{第三十五回 西門慶為男寵報仇 書童兒作女粧媚客}
\markboth{{\titlename}卷之四}{第三十五回 西門慶為男寵報仇 書童兒作女粧媚客}


詩曰:

\begin{myquote} 
娟娟遊冶童,結束類妖姬。\\揚歌倚箏瑟,艷舞逞媚姿。\\貴人一蠱惑,飛騎爭相追。\\婉孌邀恩寵,百態隨所施。
\end{myquote} 

話說西門慶早到衙門,先退廳與夏提刑說:「車淡四人再三尋人情來說,交將就他。」夏提刑道:「也有人到學生那邊,不好對長官說。既是這等,如今提出來,戒飭他一番,放了罷。」西門慶道:「長官見得有理。」即陞廳,令左右提出車淡等犯人跪下。生怕又打,只顧磕頭。西門慶也不等夏提刑開言,就道:「我把你這起光棍,如何尋這許多人情來說!本當都送問,且饒你這遭,若再犯了我手裡,都活監死。出去罷!」連韓二都喝出來了,往外金命水命,走投無命。這裡處斷公事不題。

且說應伯爵拿着五兩銀子,尋書童兒問他討話,悄悄遞與他銀子。書童接的袖了。那平安兒在門首拿眼兒睃着他。書童於是如此這般:「昨日我替爹說了,今日往衙門裡發落去了。」伯爵道:「他四箇父兄再三說,恐怕又責罰他。」書童道:「你老人家只顧放心去,管情兒一下不打他。」那伯爵得了這訊息,急急走去,回他們話去了。到早飯時分,四家人都到家,箇箇撲着父兄家屬放聲大哭。每人去了百十兩銀子,落了兩腿瘡,再不敢妄生事了。{\pangpi{還是好人。}}正是:

\begin{myquote} 
禍患每從勉強得,煩惱皆因不忍生。
\end{myquote} 

卻說那日西門慶未來家時,書童兒在書房內,叫來安兒掃地,向食盒內,把人家送的桌面上響糖與他吃。{\pangpi{都是為嘴起,妙。}}那小厮千不合萬不合,叫:「書童哥,我有句話兒告你說。昨日俺平安哥接五娘轎子,在路上好不學舌,說哥的過犯。」書童問道:「他說我甚麼來?」來安兒道:「他說哥攬的人家幾兩銀子,大膽買了酒肉,送在六娘房裡,吃了半日出來。又在前邊鋪子裡吃,不與他吃。{\pangpi{禍根。}}又說你在書房裡,和爹幹什麼營生。」這書童聽了,暗記在心,也不題起。到次日,西門慶早晨約會了,不往衙門裡去,都往門外永福寺,置酒與須坐營送行去了。直到下午纔來家,下馬就分咐平安:「但有人來,只說還沒來家。」說畢,進到廳上,書童兒接了衣裳。西門慶因問:「今日沒人來?」書童道:「沒有。管屯的徐老爹送了兩包螃蠏、十斤鮮魚。小的拿回帖打發去了,與了來人一錢銀子。又有吳大舅送了六箇帖兒,明日請娘們吃三日。」原來吳大舅子吳舜臣,娶了喬大戶娘子姪女兒鄭三姐做媳婦兒,西門慶送了茶去,他那裡來請。

西門慶到後邊,月娘拿了帖兒與他瞧,西門慶說道:「明日你們都收拾了去。」說畢,出來到書房裡坐下。書童連忙拿炭火爐內燒甜香餅兒,雙手遞茶上去。{\pangpi{趣人。}}西門慶擎茶在手。他慢慢挨近站立在桌邊。{\pangpi{可意。}}良久,西門慶𢫓了箇嘴兒,使他把門關上,用手摟在懷裡,一手捧着他的臉兒。西門慶吐舌頭,那小郎口裡噙着鳳香餅兒遞與他,下邊又替他弄玉莖。西門慶問道:「我兒,外邊沒人欺負你?」那小厮乘機就說:「小的有樁事,不是爹問,小的不敢說。」西門慶道:「你說不妨。」書童就把平安一節告說一遍:「前日爹叫小的在屋裡,他和畫童在窻外聽覷,小的出來舀水與爹洗手,親自看見。他又在外邊對着人罵小的蠻奴才,百般欺負小的。」西門慶聽了,心中大怒,說道:「我若不把奴才腿卸下來也不算!」這裡書房中說話不題。

且說平安兒專一打聽這件事,三不知走去報與金蓮。金蓮使春梅前邊來請西門慶說話。剛轉過松墻,只見畫童兒在那裡弄松虎兒,{\meipi{寫出稚子神情。}}便道:「姐來做什麼?爹在書房裡。」被春梅頭上鑿了一下。西門慶在裡面聽見裙子響,就知有人來,連忙推開小厮,走在床上睡着。那書童在桌上弄筆硯,春梅推門進來,見了西門慶,咂嘴兒說道:「你們悄悄的在屋裡,把門兒關着,敢守親哩!娘請你說話。」西門慶仰睡在枕頭上,便道:「小油嘴兒,他請我說什麼話?你先行,等我畧倘倘兒就去!」那春梅那裡容他,說道:「你不去,我就拉起你來!」西門慶怎禁他死拉活拉,拉到金蓮房中。金蓮問:「他在前頭做什麼?」春梅道:「他和小厮兩箇在書房裡,把門兒插着,捏殺蠅子兒是的,知道幹的甚麼繭兒,恰是守親的一般。我進去,小厮在桌子跟前推寫字,他便倘剌在床上,拉着再不肯來。」潘金蓮道:「他進來我這屋裡,只怕有鍋鑊吃了他是的。{\pangpi{應是皮的。}}賊沒廉恥的貨,你想有箇廉恥,大白日和那奴才平白關着門做什麼來?左右是奴才臭屁股門子,鑽了,到晚夕還進屋裡,和俺每沾身睡,好乾淨兒!」{\meipi{將就。}}西門慶道:「你信小油嘴兒胡說,我那裡有此勾當!我看着他寫禮帖兒來,我便𢱉在床上。」金蓮道:「巴巴的關着門兒寫禮帖?什麼機密謠言,什麼三隻腿的金剛、兩箇觫角的象,{\pangpi{出口便是一串,妙。}}怕人瞧見?明日吳大妗子家做三日,掠了箇帖子兒來,不長不短的,也尋件甚麼子與我做拜錢。你不與,莫不教我和野漢子要!大姐姐是一套衣裳、五錢銀子,別人也有簪子的,也有花的。只我沒有,我就不去了!」西門慶道:「前邊廚櫃內拿一疋紅紗來,與你做拜錢罷。」金蓮道,「我就去不成,也不要那囂紗片子,拿出去倒沒的教人笑話!」西門慶道:「你休亂,等我往那邊樓上,尋一件什麼與他便了。如今往東京送賀禮,也要幾疋尺頭,一答兒尋下來罷。」於是走到李瓶兒那邊樓上,尋了兩疋玄色織金麒麟補子尺頭、兩箇南京色段、一疋大紅斗牛紵絲、一疋翠藍雲段。因對李瓶兒說:「要尋一件雲絹衫與金蓮做拜錢,如無,拿帖段子鋪討去罷。」李瓶兒道:「你不要鋪子裡取去,我有一件織金雲絹衣服哩!大紅衫兒、藍裙,留下一件也不中用,俺兩箇都做了拜錢罷。」一面向箱中取出來。李瓶兒親自拿與金蓮瞧:「隨姐姐揀,衫兒也得,裙兒也得,咱兩箇一事包了做拜錢倒好,省得又取去。」金蓮道:「你的,我怎好要?」李瓶兒道:「好姐姐,怎生恁說話!」推了半日,金蓮方纔肯了。又出去教陳敬濟換了腰封,寫了二人名字在上,不題。

且說平安兒正在大門首,只見白賚光走來問道:「大官人在家麼?」平安兒道:「俺爹不在家了。」那白賚光不信,逕入裡面廳上,{\pangpi{老到。}}見槅子關着,說道:「果然不在家。往那裡去了?」平安道:「今日門外送行去了,還沒來。」白賚光道:「既是送行,這咱晚也該來家了。」平安道:「白大叔有甚話說下,待爹來家,小的稟就是了。」白賚光道:「沒什麼活,只是許多時沒見,閑來望望。既不在,我等等罷。」{\pangpi{更老到。}}平安道:「只怕來晚了,你老人家等不得。」白賚光不依,把槅子推開,進入廳內,在椅子上就坐了。衆小厮也不理他,繇他坐去。不想天假其便,西門慶教迎春抱着尺頭,從後邊走來,剛轉過軟壁,頂頭就撞見白賚光在廳上坐着。迎春兒丟下段子,往後走不迭。白賚光道:「這不是哥在家!」一面走下來唱喏。西門慶見了,推辭不得,須索讓坐。睃見白賚光頭戴着一頂出洗覆盔過的、恰如太山游到嶺的舊羅帽兒,{\meipi{畫出。}}身穿着一件壞領磨襟救火的硬漿白布衫,脚下靸着一雙乍板唱曲兒前後彎絕戶綻的皁靴,{\meipi{可憐。}}裡邊插着一雙一碌子蠅子打不到、黃絲轉香馬櫈襪子。坐下,也不叫茶,見琴童在旁伺候,就分咐:「把尺頭抱到客房裡,教你姐夫封去。」那琴童應諾,抱尺頭往廂房裡去了。白賚光舉手道:「一向欠情,沒來望的哥。」西門慶道:「多謝掛意。我也常不在家,日逐衙門中有事。」白賚光道:「哥這衙門中也日日去麼?」西門慶道:「日日去兩次,每日坐廳問事。到朔望日子,還要拜牌,畫公座,大發放,地方保甲番役打卯。歸家便有許多窮冗,無片時閑暇。今日門外去,因須南溪新陞了新平寨坐營,衆人和他送行,只剛到家。明日管皇庄薛公公家請吃酒,路遠去不成。後日又要打聽接新巡按。又是東京太師老爺四公子又選了駙馬,童太尉侄男童天新選上大堂,陞指揮使僉書管事。兩三層都要賀禮。這連日通辛苦的了不得。」說了半日語,來安兒纔拿上茶來。白賁光纔拿在手裡呷了一口,只見玳安拿着大紅帖兒往裡飛跑,報道:「掌刑的夏老爹來了!外邊下馬了。」西門慶就往後邊穿衣服去了。白賁光躲在西廂房內,打簾裡望外張看。良久,夏提刑進到廳上,西門慶冠帶從後邊迎將來。兩箇叙禮畢,分賓主坐下。不一時,棋童兒拿了兩盞茶來吃了。夏提刑道:「昨日所言接大巡的事,今日學生差人打聽,姓曾,乙未進士,牌已行到東昌地方。他列位每都明日起身遠接。你我雖是武官,系領勑衙門提點刑獄,比軍衛有司不同。咱後日起身,離城十里尋箇去所,預備一頓飯,那裡接見罷!」{\meipi{因後被參,先叙得疎虞,妙。}}西門慶道:「長官所言甚妙,也不消長官費心,學生這裡着人尋箇庵觀寺院,或是人家庄園亦好,教箇廚役早去整理。」夏提刑謝道:「這等又教長官費心。」說畢,又吃了一道茶,夏提刑起身去了。西門慶送了進來,寬去衣裳。那白賁光還不去,走到廳上又坐下了。對西門慶說:「自從哥這兩箇月沒往會裡去,把會來就散了。老孫雖年紀大,主不得事。應二哥又不管。昨日七月內,玉皇廟打中元醮,連我只三四箇人到,沒箇人拿出錢來,都打撒手兒。{\meipi{的眞扯淡。落運人語言無味者,如此。}}難為吳道官,晚夕謝將,又叫了箇說書的,甚是破費他。他雖故不言語,各人心上不安。不如那咱哥做會首時,還有箇張主。不久還要請哥上會去。」西門慶道:「你沒的說。散便散了罷,那裡得工夫幹此事?遇閑時,在吳先生那裡一年打上箇醮,答報答報天地就是了。隨你們會不會,不消來對我說。」幾句話搶白的白賚光沒言語了。又坐了一回,西門慶見他不去,只得喚琴童兒廂房內放桌兒,拿了四碟小菜,牽葷連素,一碟煎麵觔、一碟燒肉。{\meipi{吃物數種,寫世炎涼惡態,使人慾涕欲笑。}}西門慶陪他吃了飯。篩酒上來,西門慶又討副銀鑲大鍾來,斟與他。吃了幾鍾,白賚光纔起身。西門慶送到二門首,說道:「你休怪我不送你,我戴着小帽,不好出去得。」那白賚光告辭去了。

西門慶回到廳上,拉了把椅子坐下,就一片聲叫平安兒。那平安兒走到跟前,西門慶罵道:「賊奴才,還站着?」叫答應的,就是三四箇排軍在旁伺候。那平安不知甚麼緣故,諕的臉蠟查黃,跪下了。西門慶道:「我進門就分咐你,但有人來,答應不在。你如何不聽?」平安道:「白大叔來時,小的回說爹往門外送行去了,沒來家。他不信,強着進來了。小的就跟進來問他:『有話說下,待爹來家,小的稟就是了。』他又不言語,自家推開廳上槅子坐下。落後,不想出來就撞見了。」西門慶罵道:「你這奴才,不要說嘴!你好小膽子兒?人進來,你在那裡耍錢吃酒去來,不在大門首守着!」令左右:「你聞他口裡。」那排軍聞了一聞,稟道:「沒酒氣。」西門慶分咐:「叫兩箇會動刑的上來,與我着實拶這奴才!」當下兩箇伏侍一箇,套上拶指,只顧擎起來。拶的平安疼痛難忍,叫道:「小的委實回爹不在,他強着進來。」那排軍拶上,把繩子綰住,跪下稟道:「拶上了。」西門慶道:「再與我敲五十敲。」旁邊數着,敲到五十上住了手。西門慶分咐:「打二十棍!」須臾打了二十,打的皮開肉綻,滿腿血淋。西門慶喝令:「與我放了。」兩箇排軍向前解了拶子,解的直聲呼喚。西門慶罵道:「我把你這賊奴才!你說你在大門首,想說要人家錢兒,在外邊壞我的事,休吹到我耳朵內,把你這奴才腿卸下來!」那平安磕了頭起來,提着褲子往外去了。西門慶看見畫童兒在旁邊,說道:「把這小奴才拿下去,也拶他一拶子。」一面拶的小厮殺豬兒似怪叫。這裡西門慶在前廳拶人不題。

單說潘金蓮從房裡出來往後走,剛走到大廳後儀門首,只見孟玉樓獨自一箇在軟壁後聽覷。金蓮便問:「你在此聽甚麼兒哩?」玉樓道:「我在這裡聽他爹打平安兒,連畫童小奴才也拶了一拶子,不知為什麼。」一回棋童兒過來,玉樓叫住問他:「為什麼打平安兒?」棋童道:「爹嗔他放進白賚光來了。」金蓮接過來道:「也不是為放進白賚光來,敢是為他打了象牙來,不是打了象牙,平白為什麼打得小厮這樣的!賊沒廉恥的貨,亦發臉做了主了。想有些廉恥兒也怎的!」那棋童就走了。玉樓便問金蓮:「怎的打了象牙?」金蓮道:「我要告訴你,還沒告訴你。我前日去俺媽家做生日去了,不在家,蠻秫秫小厮攬了人家說事幾兩銀子,買兩盒嘎飯,又是一罈金華酒,掇到李瓶兒房裡,和小厮吃了半日酒,小厮纔出來。沒廉恥貨來家,也不言語,還和小厮在花園書房裡,插着門兒,兩箇不知幹着什麼營生。平安這小厮拿着人家帖子進去,見門關着,就在窻下站着了。蠻小厮開門看見了,想是學與賊沒廉恥的貨,今日挾仇,打這小厮打的膫子成。那怕蠻奴才到明日把一家子都收拾了,管人弔脚兒事!」玉樓笑道:「好說,雖是一家子,有賢有愚,{\pangpi{自己也在裡頭。}}莫不都心邪了罷?」金蓮道:「不是這般說,等我告訴你。如今這家中,他心肝肐蒂兒,偏歡喜的只兩箇人,一箇在裡,一箇在外,成日把魂恰似落在他身上一般,見了說也有,笑也有。俺們是沒時運的,行動就是烏眼雞一般。賊不逢好死變心的強盜!通把心狐迷住了,更變的如今相他哩!三姐你聽着,到明日弄出什麼八怪七喇出來!今日為拜錢,又和他合了回氣。但來家,就在書房裡。

今日我使春梅叫他來,誰知大白日裡和賊蠻奴才關着門兒哩!春梅推門入去,諕的一箇箇眼張失道的。{\meipi{如見。}}到屋裡,教我盡力數罵了幾句。他只顧左遮右掩的。先拿一疋紅紗與我做拜錢,我不要。落後往李瓶兒那邊樓上尋去。賊人膽兒虛,自知理虧,拿了他箱內一套織金衣服來,親自來盡我,我只是不要。他慌了,說:『姐姐,怎的這般計較!姐姐揀衫兒也得,裙兒也得。看了,好拿到前邊,教陳姐夫封寫去。』盡了半日,我纔吐了口兒。他讓我要了衫子。」玉樓道:「這也罷了,也是他的儘讓之情。」金蓮道:「你不知道,不要讓了他。如今年世,只怕睜着眼兒的金剛,不怕閉着眼兒的佛!老婆漢子,你若放些鬆兒與他,王兵馬的皁隸,還把你不當㒲的。」玉樓戲道,「六丫頭,你是屬麵觔的,倒且是有靳道。」說着,兩箇笑了。

只見小玉來請:「三娘、五娘,後邊吃螃蠏哩!我去請六娘和大姑娘去。」兩箇手拉着手兒進來,月娘和李嬌兒正在上房穿廊下坐,說道:「你兩箇笑什麼?」金蓮道:「我笑他爹打平安兒。」月娘道:「嗔他恁亂蝍䗫叫喊的,只道打什麼人?原來打他。為什麼來,」金蓮道:「為他打折了象牙了。」月娘老實,便問:「象牙放在那裡來,怎的教他打折了?」那潘金蓮和孟玉樓兩箇嘻嘻哈哈,只顧笑成一塊。月娘道:「不知你每笑什麼,不對我說。」玉樓道:「姐姐你不知道,爹打平安為放進白賚光來了。」月娘道:「放進白賚光便罷了,怎麼說道打了象牙?也沒見這般沒稍幹的人,在家閉着膫子坐,平白有要沒緊來人家撞些什麼!」來安道:「他來望爹來了。」月娘道:「那箇弔下炕來了?望,沒的扯臊淡,不說來挄嘴吃罷了。」良久,李瓶兒和大姐來到,衆人圍遶吃螃蠏。月娘分咐小玉:「屋裡還有些葡萄酒,篩來與你娘每吃。」金蓮快嘴,說道:「吃螃蠏得些金華酒吃纔好!」又道:「只剛一味螃蠏就着酒吃,得只燒鴨兒撕了來下酒。」月娘道:「這咱晚那裡買燒鴨子去!」李瓶兒聽了,把臉飛紅了。正是:話頭兒包含着深意,題目兒哩暗蓄着留心。那月娘是箇誠實的人,怎曉的話中之話。這裡吃螃蠏不題。

且說平安兒被責,來到外邊,賁四、來興衆人都亂來問平安兒:「爹為甚麼打你?」平安哭道:「我知為甚麼!」來興兒道:「爹嗔他放進白賚光來了。」平安道,「早是頭裡你看着,我那等攔他,他只強着進去了。不想爹從後邊出來撞見了,又沒甚話,吃了茶,再不起身。只見夏老爹來了,我說他去了,他還躲在廂房裡又不去。直等拿酒來吃了纔去。倒惹的打我這一頓,你說我不造化低!我沒攔他?又說我沒攔他。他強自進來,管我腿事!打我!教那箇賊天殺男盜女娼的狗骨禿,吃了俺家這東西,打背梁脊下過!」來興兒道:「爛折脊梁骨,倒好了他往下撞!」平安道:「教他生噎食病,把顙根軸子爛弔了。天下有沒廉恥皮臉的,不相這狗骨禿沒廉恥,來我家闖的狗也不咬。賊雌飯吃花子㒲的,再不爛了賊忘八的屁股門子!」來興笑道:「爛了屁股門子,人不知道,只說是臊的。」衆人都笑了。平安道:「想必是家裡沒晚米做飯,老婆不知餓的怎麼樣的。閑的沒的幹,來人家抹嘴吃。圖家裡省了一頓,也不是常法兒。不如教老婆養漢,做了忘八倒硬朗些,不教下人唾罵。」玳安在鋪子裡篦頭,篦了,打發那人錢去了,走出來說:「平安兒,我不言語,憋的我慌。虧你還答應主子,當家的性格,你還不知道?你怎怪人?常言養兒不要屙金溺銀,只要見景生情。比不的應二叔和謝叔來,答應在家不在家,他彼此都是心甜厚間便罷了。以下的人,他又分咐你答應不在家,你怎的放人來?不打你卻打誰!」賁四戲道:「平安兒從新做了小孩兒,才學閑閑,他又會頑,成日只踢毬兒耍子。」衆人又笑了一回。賁四道:「他便為放人進來,這畫童兒卻為什麼,也陪拶了一拶子?是甚好吃的菓子,陪吃箇兒?吃酒吃肉也有箇陪客,十箇指頭套在拶子上,也有箇陪的來?」那畫童兒揉着手,只是哭。玳安戲道:「我兒少哭,你娘養的你忒嬌,把饊子兒拿繩兒拴在你手兒上,你還不吃?」這裡前邊小厮熱亂不題。

西門慶在廂房中,看着陳敬濟封了禮物尺頭,寫了揭帖,次日早打發人上東京,送蔡駙馬、童堂上禮,不在話下。到次日,西門慶往衙門裡去了。吳月娘與衆房,共五頂轎子,頭戴珠翠,身穿錦綉,來興媳婦一頂小轎跟隨,往吳大妗家做三日去了。止留下孫雪娥在家中,和西門大姐看家。早間韓道國送禮相謝:一罈金華酒,一隻水晶鵝,一副蹄子,四隻燒鴨,四尾鰣魚。帖子上寫着「晚生韓道國頓首拜」。書童因沒人在家,不敢收,連盒担留下,待的西門慶衙門回來,拿與西門慶瞧。西門慶使琴童兒鋪子裡旋叫了韓夥計來,甚是說他:「沒分曉,又買這禮來做甚麼!我決然不受!」那韓道國拜說:「小人蒙老爹莫大之恩,可憐見與小人出了氣,小人舉家感激不盡。無甚微物,表一點窮心。望乞老爹好歹笑納。」西門慶道:「這箇使不得。你是我門下夥計,如同一家,我如何受你的禮!即令原人與我擡回去。」韓道國慌了,央說了半日。

西門慶分咐左右,只受了鵝酒,別的禮都令擡回去了。教小厮拿帖兒,請應二爹和謝爹去,對韓道國說:「你後晌叫來保看着鋪子,你來坐坐。」韓道國說:「禮物不受,又教老爹費心。」應諾去了。西門慶又添買了許多菜蔬,後晌時分,在翡翠軒捲棚內,放下一張八仙桌兒。應伯爵、謝希大先到了。西門慶告他說:「韓夥計費心,買禮來謝我,我再三不受他,他只顧死活央告,只留了他鵝酒。我怎好獨享,請你二位陪他坐坐。」伯爵道:「他和我討較來,要買禮謝。我說你大官府那裡稀罕你的,休要費心,你就送去,他決然不受。如何?我恰似打你肚子裡鑽過一遭的,果然不受他的。」說畢,吃了茶,兩箇打雙陸。

不一時,韓道國到了,二人叙禮畢坐下。應伯爵、謝希大居上,西門慶關席,韓道國打橫。登時四盤四碗拿來,桌上擺了許多下飯,把金華酒分咐來安兒就在旁邊開啟,用銅甑兒篩熱了拿來,教書童斟酒。伯爵分咐書童兒:「後邊對你大娘房裡說,怎的不拿出螃蠏來與應二爹吃?你去說我要螃蠏吃哩。」西門慶道:「傻狗才,那裡有一箇螃蠏!實和你說,管屯的徐大人送了我兩包螃蠏,到如今娘們都吃了,剩下醃了幾箇。」分咐小厮:「把醃螃蠏𢵞幾箇來。今日娘們都往吳妗子家做三日去了。」不一時,畫童拿了兩盤子醃蠏上來。那應伯爵和謝希大兩箇搶着,吃的淨光。因見書童兒斟酒,說道:「你應二爹一生不吃啞酒,自誇你會唱的南曲,我不曾聽見。今日你好歹唱箇兒,我纔吃這鍾酒。」那書童纔待拍着手唱,伯爵道:「這等唱一萬箇也不算。你裝龍似龍,裝虎似虎,下邊搽畫裝扮起來,相箇旦兒的模樣纔好。」{\meipi{伯爵差排指勒處,節節多端,然而正中主人之好,此其所以莫逆也。}}那書童在席上,把眼只看西門慶的聲色兒。西門慶笑罵伯爵:「你這狗才,專一歪厮纏人!」因向書童道:「既是他索落你,教玳安兒前邊問你姐要了衣服,下邊粧扮了來。」玳安先走到前邊金蓮房裡問春梅要,春梅不與。旋往後問上房玉蕭要了四根銀簪子,一箇梳背兒,面前一件仙子兒,一雙金鑲假青石頭墜子,大紅對衿絹衫兒,綠重絹裙子,紫銷金箍兒。要了些脂粉,在書房裡搽抹起來,儼然就如箇女子,打扮的甚是嬌娜。走在席邊,雙手先遞上一盃與應伯爵,頓開喉音,在旁唱《玉芙蓉》道:

\begin{myquote} 
殘紅水上飄,梅子枝頭小。這些時,眉兒淡了誰描?因春帶得愁來到,春去緣何愁未消?人別後,山遙水遙。我為你數歸期,畫損了掠兒稍。
\end{myquote} 

伯爵聽了,誇獎不已,說道:「相這大官兒,不在了與他碗飯吃。你看他這喉音,就是一管蕭。說那院裡小娘兒便怎的,那些唱都聽熟了。怎生如他這等滋潤!哥,不是俺們面獎,似你這般的人兒在你身邊,你不喜歡!」西門慶笑了。伯爵道:「哥,你怎的笑?我到說的正經話。{\meipi{說得正正經經,何等侃鑿。}}你休虧這孩子,凡事衣類兒上,另着箇眼兒看他。難為李大人送了他來,也是他的盛情。」西門慶道:「正是。如今我不在家,書房中一應大小事,都是他和小婿。小婿又要鋪子裡兼看看。」應伯爵飲過,又斟雙盃。伯爵道:「你替我吃些兒。」書童道:「小的不敢吃,不會吃。」伯爵道:「你不吃,我就惱了。我賞你待怎的?」書童只顧把眼看西門慶。西門慶道:「也罷,應二爹賞你,你吃了。」那小厮打了箇僉兒,慢慢低垂粉頸,呷了一口。餘下半鍾殘酒,用手擎着,與伯爵吃了。方纔轉過身來,遞謝希大酒,又唱了箇曲兒。謝希大問西門慶道:「哥,書官兒青春多少?」西門慶道:「他今年纔交十六歲。」問道:「你也會多少南曲?」書童道:「小的也記不多幾箇曲子,胡亂答應爹們罷了。」希大道:「好箇乖覺孩子!」亦照前遞了酒。下來遞韓道國。道國道:「老爹在上,小的怎敢欺心。」西門慶道:「今日你是客。」韓道國道:「那有此理!還是從老爹上來,次後纔是小人吃酒。」書童下席來遞西門慶酒,又唱了一箇曲兒。西門慶吃畢,到韓道國跟前。韓道國慌忙立起身來接酒。伯爵道:「你坐着,教他好唱。」韓道國方纔坐下。書童又唱了箇曲兒。韓道國未等詞終,連忙一飲而盡。

正飲酒中間,只見玳安來說:「賁四叔來了,請爹說話。」西門慶道:「你叫他來這裡說罷。」不一時,賁四進來,向前作了揖,旁邊安頓坐了。玳安又取一雙鍾筯放下。西門慶令玳安後邊取菜蔬。西門慶因問他:「庄子上收拾怎的樣了?」賁四道:「前一層纔蓋瓦,後邊捲棚昨日纔打的基,還有兩邊廂房與後一層住房的料,都沒有。客位與捲棚漫地尺二方磚,還得五百,那舊的都使不得。砌墻的大城角也沒了。墊地脚帶山子上土,也添勾了百多車子。灰還得二十兩銀子的。」西門慶道:「那灰不打緊,我明日衙門裡分咐灰戶,教他送去。昨日你磚廠劉公公說送我些磚兒。你開箇數兒,封幾兩銀子送與他,須是一半人情兒回去。只少這木植。」賁四道:「昨日老爹分咐,門外看那庄子,今早同張安兒去看,原來是向皇親家庄子。大皇親沒了,如今向五要賣神路明堂。咱們不要他的,講過只拆他三間廳、六間廂房、一層羣房就勾了。他口氣要五百兩。到跟前拿銀子和他講,三百五十兩上,也該拆他的。休說木料,光磚瓦連土也値一二百兩銀子。」應伯爵道:「我道是誰來!是向五的那庄子。向五被人爭地土,告在屯田兵備道,打官司使了好多銀子。又在院裡包着羅存兒。如今手裡弄的沒錢了。你若要,與他三百兩銀子,他也罷了。冷手撾不着熱饅頭。」西門慶分咐賁四:「你明日拿兩錠大銀子,同張安兒和他講去,若三百兩銀子肯,拆了來罷。」賁四道:「小人理會。」良久,後邊拿了一碗湯、一盤蒸餅上來,賁四吃了。斟上,陪衆人吃酒。

書童唱了一遍,下去了。應伯爵道:「這等吃的酒沒趣。取箇骰盆兒,俺們行箇令兒吃纔好。」西門慶令玳安:「就在前邊六娘屋裡取箇骰盆來。」不一時,玳安取了來,放在伯爵跟前,悄悄走到西門慶耳邊說:「六娘房裡哥哭哩。迎春姐叫爹着箇人兒接接六娘去。」西門慶道:「你放下壺,快叫箇小厮拿燈籠接去!」因問:「那兩箇小厮在那裡?」玳安道:「琴童與棋童兒先拿兩箇燈籠接去了。」伯爵見盆內放着六箇骰兒,即用手拈着一箇,說:「我擲着點兒,各人要骨牌名一句兒,見合着點數兒,如說不過來,罰一大盃酒。下家唱曲兒,不會唱曲兒說笑話兒,兩樁兒不會,定罰一大盃。」西門慶道:「怪狗才,忒韶刀了!」伯爵道:「令官放箇屁,也欽此欽遵。你管我怎的!」叫來安:「你且先斟一盃,罰了爹,然後好行令。」西門慶笑而飲之。伯爵道:「衆人聽着,我起令了!說差了也罰一盃。」說道:「張生醉倒在西廂。吃了多少酒?一大壺,兩小壺,」果然是箇麼。西門慶叫書童兒上來斟酒,該下家謝希大唱。希大拍着手兒道:「我唱箇《折桂令》兒你聽罷。」唱道:

\begin{myquote} 
可人心二八嬌娃,百件風流,所事撐達。眉蹙春山,眼橫秋水,髩綰着烏鴉。乾相思,撇不下一時半霎;咫尺間,如隔着海角天涯。瘦也因他,病也因他。誰與做箇成就了姻緣,便是那救苦難的菩薩。
\end{myquote} 

伯爵吃了酒,過盆與謝希大擲,輪着西門慶唱。謝希大拿過骰兒來說:「多謝紅兒扶上床。甚麼時候?三更四點。」可是作怪,擲出箇四來。伯爵道:「謝子純該吃四盃。」希大道:「折兩盃罷,我吃不得。」書童兒滿斟了兩盃,先吃了頭一盃,等他唱。席上伯爵二人把一碟子荸薺都吃了。西門慶道:「我不會唱,說箇笑話兒罷。」說道:「一箇人到菓子鋪問:「可有榧子麼?」那人說有。取來看,那買菓子的不住的往口裡放。賣菓子的說:『你不買,如何只顧吃?』那人道:『我圖他潤肺。』那賣的說:『你便潤了肺,我卻心疼。』」衆人都笑了。伯爵道:「你若心疼,再拿兩碟子來。我『媒人婆拾馬糞——越發越晒』。」{\meipi{妙。}}謝希大吃了。第三該西門慶擲。說:「留下金釵與表記。多少重?五六七錢。」西門慶拈起骰兒來,擲了箇五。書童兒也只斟上兩鍾半酒。謝希大道:「哥大量,也吃兩盃兒,沒這箇理。哥吃四鍾罷,只當俺一家孝順一鍾兒。」該韓夥計唱。韓道國讓:「賁四哥年長。」賁四道:「我不會唱,說箇笑話兒罷。」西門慶吃過兩鍾,賁四說道:「一官問姦情事。問:『你當初如何姦他來?』那男子說:『頭朝東,脚也朝東姦來。』官云:『胡說!那裡有箇缺着行房的道理!』旁邊一箇人走來跪下,說道:『告稟,若缺刑房,待小的補了罷!』」應伯爵道:「好賁四哥,你便益不失當家!你大官府又不老,別的還可說,你怎麼一箇行房,你也補他的?」{\meipi{毒極。}}賁四聽見此言,諕的把臉通紅了,說道:「二叔,什麼話!小人出於無心。」伯爵道:「什麼話?檀木靶,沒了刀兒,只有刀鞘兒了。」{\meipi{惡極。}}那賁四在席上終是坐不住,去又不好去,如坐針氊相似。西門慶飲畢四鍾酒,就輪該賁四擲。賁四纔待拿起骰子來,只見來安兒來請:「賁四叔,外邊有人尋你。我問他,說是窯上人。」這賁四巴不得要去,聽見這一聲,一箇金蟬脫殼走了。西門慶道:「他去了,韓夥計你擲罷。」韓道國舉起骰兒道:「小人遵令了。」說道:「夫人將棒打紅娘。打多少?八九十下。」伯爵道:「該我唱,我不唱罷,我也說箇笑話兒。教書童合席都篩上酒,連你爹也篩上。聽我這箇笑話:一箇道士,師徒二人往人家送疏。行到施主門首,徒弟把縧兒鬆了些,垂下來。師父說:『你看那樣!倒相沒屁股的。』徒弟回頭答道:『我沒屁股,師父你一日也成不得。』」西門慶罵道:「你這歪狗才,狗口裡吐出什麼象牙來!」這裡飲酒不題。

且說玳安先到前邊,又叫了畫童,拿着燈籠,來吳大妗子家接李瓶兒。瓶兒聽見說家裡孩子哭,也等不得上拜,留下拜錢,就要告辭來家。吳大妗、二妗子那裡肯放:「好歹等他兩口兒上了拜兒!」月娘道:「大妗子,你不知道,倒教他家去罷。家裡沒人,孩子好不尋他哭哩!俺多坐回兒不妨事。」那吳大妗子纔放了李瓶兒出門。玳安丟下畫童,和琴童兒兩箇隨轎子先來家了。落後,上了拜,堂客散時,月娘等四乘轎子,只打着一箇燈籠,況是八月二十四日,月黑時分。月娘問:「別的燈籠在那裡,如何只一箇?」棋童道:「小的原拿了兩箇來。玳安要了一箇,和琴童先跟六娘家去了。」月娘便不問,就罷了。潘金蓮有心,便問棋童:「你們頭裡拿幾箇來?」棋童道:「小的和琴童拿了兩箇來,落後玳安與畫童又要了一箇去,把畫童換下,和琴童先跟了六娘去了。」金蓮道:「玳安那囚根子,他沒拿燈籠來?」{\pangpi{問得精細。}}畫童道:「我和他又拿了一箇燈籠來了。」金蓮道:「既是有一箇就罷了,怎的又問你要這箇?」棋童道:「我那等說,他強着奪了去。」金蓮便叫吳月娘:「姐姐,你看玳安恁賊獻勤的奴才!等到家和他答話。」月娘道:「奈煩,孩子家裡緊等着,叫他打了去罷了。」{\meipi{佛。}}金蓮道:「姐姐,不是這等說。俺便罷了,你是箇大娘子,沒些家法兒,晴天還好,這等月黑,四頂轎子只點着一箇燈籠,顧那些兒的是?」

說着轎子到了門首。月娘、李嬌兒便往後邊去了。金蓮和孟玉樓一答兒下轎,進門就問,「玳安兒在那裡?」平安道:「在後邊伺候哩!」剛說着,玳安出來,被金蓮罵了幾句:「我把你獻勤的囚根子!明日你只認清了,單揀着有時運的跟,只休要把脚兒踢踢兒。有一箇燈籠打着罷了,信那斜汗世界一般又奪了箇來。又把小厮也換了來。他一頂轎子,倒佔了兩箇燈籠,俺們四頂轎子,反打着一箇燈籠,俺們不是爹的老婆?」玳安道:「娘錯怪小的了。爹見哥兒哭,教小的:『快打燈籠接你六娘先來家罷,恐怕哭壞了哥兒。』莫不爹不使我,我好乾着接去來!」金蓮道:「你這囚根子,不要說嘴!他教你接去,沒教你把燈籠都拿了來。哥哥,你的雀兒只揀旺處飛,休要認差了,冷竈上着一把兒、熱竈上着一把兒纔好。俺們天生就是沒時運的來?」玳安道:「娘說的什麼話!小的但有這心,騎馬把脯子骨撞折了!」金蓮道:「你這欺心的囚根子!不要慌,我洗淨眼兒看着你哩!」說着,和玉樓往後邊去了。那玳安對着衆人說:「我精晦氣的營生,平自爹使我接去,卻被五娘罵了恁一頓。」

玉樓、金蓮二人到儀門首,撞見來安兒,問:「你爹在那裡哩?」來安道:「爹和應二爹、謝爹、韓大叔還在捲棚內吃酒。書童哥裝了箇唱的,在那裡唱哩,娘每瞧瞧去。」二人間走到捲棚槅子外,往裡觀看。只見應伯爵在上坐着,把帽兒歪挺着,醉的只相線兒提的。謝希大醉的把眼兒通睜不開。書童便粧扮在旁邊斟酒唱南曲。西門慶悄悄使琴童兒抹了伯爵一臉粉,又拿草圈兒從後邊悄悄兒弄在他頭上作戲。把金蓮和玉樓在外邊忍不住只是笑,罵:「賊囚根子,到明日死了也沒罪了,把醜都出盡了!」西門慶聽見外邊笑,使小厮出來問是誰,二人纔往後邊去了。散時,已一更天氣了。

西門慶那日往李瓶兒房裡睡去了。金蓮歸房,因問春梅:「李瓶兒來家說甚麼話來?」春梅道:「沒說甚麼。」金蓮又問:「那沒廉恥貨,進他屋裡去來沒有?」春梅道:「六娘來家,爹往他房裡還走了兩遭。」金蓮道:「眞箇是因孩子哭接他來?」春梅道:「孩子後晌好不怪哭的,抱着也哭,放下也哭,再沒法處。前邊對爹說了,纔使小厮接去。」金蓮道:「若是這等也罷了。我說又是沒廉恥的貨,三等兒九般使了接去。」又問:「書童那奴才,穿的是誰的衣服?」春梅道:「先來問我要,教我罵了玳安出去。落後,和玉簫借了。」金蓮道:「再要來,休要與秫秫奴才穿。」說畢,見西門慶不來,使性兒關門睡了。

且說應伯爵見賁四管工,在庄子上賺錢,明日又拿銀子買向五皇親房子,少說也有幾兩銀子背。正行令之間,可可見賁四不防頭,說出這箇笑話兒來。伯爵因此錯他這一錯,使他知道。賁四果然害怕,次日封了三兩銀子,親到伯爵家磕頭。伯爵反打張驚兒,說道:「我沒曾在你面上盡得心,何故行此事?」賁四道:「小人一向缺禮,早晚只望二叔在老爹面前扶持一二,足感不盡!」伯爵於是把銀子收了,待了一鍾茶,打發賁四出門。拿銀子到房中,與他娘子兒說:「老兒不發狠,婆兒沒布裙。賁四這狗啃的,我舉保他一場,他得了買賣,扒自飯碗兒,就不用着我了。大官人教他在庄子上管工,明日又托他拿銀子成向五家庄子,一向賺的錢也勾了。我昨日在酒席上,拿言語錯了他錯兒,他慌了,不怕他今日不來求我。送了我三兩銀子,我且買幾疋布,勾孩子們冬衣了。」正是:

\begin{myquote} 
祗恨閒愁成懊惱,豈知伶俐不如癡。
\end{myquote} 

