\part*{附錄}
\addcontentsline{toc}{part}{附錄}

\chapter*{金瓶梅詞話序}
\addcontentsline{toc}{chapter}{欣欣子《金瓶梅詞話序》}
\markboth{\titlename}{金瓶梅詞話序}


竊謂蘭陵笑笑生作《金瓶梅傳》,寄意於時俗,蓋有謂也。人有七情,憂鬱為甚。上智之士,與化俱生,霧散而冰裂,是故不必言矣。次焉者,亦知以理自排,不使為累。惟下焉者,既不出了於心胸,又無詩書道腴可以撥遣,然則不致于坐病者幾希!吾友笑笑生為此,爰罄平日所蘊者,著斯傳,凡一百回。其中語句新奇,膾炙人口。無非明人倫、戒淫奔、分淑慝、化善惡,知盛衰消長之機,取報應輪迴之事,如在目前;始終如脈絡貫通,如萬絲迎風而不亂也。使觀者庶幾可以一哂而忘憂也。其中未免語涉俚俗,氣含脂粉。余則曰:不然。《關雎》之作,樂而不淫,哀而不傷。富與貴,人之所慕也,鮮有不至于淫者;哀與怨,人之所惡也,鮮有不至于傷者。吾嘗觀前代騷人,如盧景暉之《剪燈新話》、元徽之之《鶯鶯傳》、趙君弼之《效顰集》、羅貫中之《水滸傳》、丘瓊山之《鍾情麗集》、盧梅湖之《懷春雅集》、周靜軒之《秉燭清談》,其後《如意傳》、《于湖記》,其間語句文確,讀者往往不能暢懷,不至終篇而掩棄之矣。此一傳者,雖市井之常談,閨房之碎語,使三尺童子聞之,如飫天漿而拔鯨牙,洞洞然易曉。雖不比古之集理趣,文墨綽有可觀。其它關繋世道風化,懲戒善惡,滌慮洗心,不無小補。譬如房中之事,人皆好之,人皆惡之。人非堯舜聖賢,鮮不為所躭。富貴善良,是以搖動人心,蕩其素志。觀其高堂大廈,雲窻霧閣,何深沉也;金屏綉褥,何美麗也;鬢雲斜軃,春酥滿胸,何嬋娟也;雄鳳雌凰迭舞,何慇懃也;錦衣玉食,何侈費也;佳人才子,嘲風咏月,何綢繆也;雞舌含香,唾圓流玉,何溢度也;一雙玉腕綰復綰,兩隻金蓮顛倒顛,何猛浪也。旣其樂矣,然樂極必悲生:如離别之機將興,憔悴之容必見者,所不能免也;折梅逢驛使,尺素寄魚書,所不能無也;患難迫切之中,顛沛流離之頃,所不能脫也;陷命於刀劔,所不能逃也;陽有王法,幽有鬼神,所不能逭也。至于淫人妻子,妻子淫人,祸因惡積,福緣善慶,種種皆不出循環之機。故天有春夏秋冬,人有悲歡離合,莫怪其然也。合天時者,遠則子孫悠久,近則安享終身;逆天時者,身名罹喪,祸不旋踵。人之䖏世,雖不出乎世運代謝,然不經兇祸,不蒙耻辱者,亦幸矣。吾故曰:笑笑生作此傳者,蓋有所謂也。

\begin{quotation}
\raggedleft{欣欣子書于明賢里之軒\rightquadmargin}
\end{quotation}

