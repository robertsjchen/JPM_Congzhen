\includepdf[pages={151,152},fitpaper=false]{tst.pdf}
\chapter*{第七十六回 春梅嬌撒西門慶 畫童哭躲溫葵軒}
\addcontentsline{toc}{chapter}{第七十六回 春梅嬌撒西門慶 畫童哭躲溫葵軒}
\markboth{{\titlename}卷之八}{第七十六回 春梅嬌撒西門慶 畫童哭躲溫葵軒}


詩曰:

\begin{myquote} 
相勸頻攜金粟杯,莫將閒事繋柔懷。\\年年只是人依舊,處處何曾花不開。\\歌詠且添詩酒興,醉酣還命管絃來。\\尊前百事皆如昨,簡點惟無溫秀才。
\end{myquote} 

話說西門慶見月娘半日不出去,又親自進來催促,見月娘穿衣裳,方纔請任醫官進明間內坐下。{\meipi{人情之常。}}少頃,月娘從房內出來,望上道了萬福,慌的任醫官躲在旁邊,屈身還禮。月娘就在對面椅上坐下。琴童安放桌兒錦茵,月娘向袖口邊伸玉腕,露青蔥,教任醫官診脈。良久診完,月娘又道了箇萬福。抽身回房去了。房中小厮拿出茶來。吃畢茶,任醫官說道:「老夫人原來稟的氣血弱,尺脈來的浮澀。雖是胎氣,有些榮衛失調,易生嗔怒,又動了肝火。如今頭目不清,中膈有些阻滯煩悶,四肢之內,血少而氣多。」{\meipi{明醫。}}月娘使出琴童來說:「娘如今只是有些頭疼心脹,胳膊發麻,肚腹往下墜着疼,腰痠,吃飲食無味。」任醫官道:「我已知道,說得明白了。」西門慶道:「不瞞後溪說,房下如今見懷臨月身孕,因着氣惱,不能運轉,滯在胸膈間。望乞老先生留神加減一二,足見厚情。」{\meipi{是誰之過歟?}}任醫官道:「豈勞分付,學生無不用心。此去就奉過安胎理氣、和中養榮蠲痛之劑來。老夫人服過,要戒氣惱,{\pangpi{此第一着。}}就厚味也少吃。」西門慶道:「望乞老先生把他這胎氣好生安一安。」任醫官道:「已定安胎理氣,養其榮衛,不勞分付,學生自有斟酌。」西門慶復說:「學生第三房下有些肚疼,望乞有暖宮丸藥,並見賜些。」任醫官道:「學生謹領,就封過來。」說畢起身,走到前廳院內,見許多教坊樂工伺候,因問:「老翁,今日府上有甚事?」西門慶道:「巡按宋公連兩司官,請巡撫侯石泉老先生,在舍擺酒。」這任醫官聽了,越發駭然尊敬,在前門揖讓上馬,打了恭又打恭,比尋常不同,倍加敬重。西門慶送他回來,隨即封了一兩銀子,兩方手帕,使琴童騎馬討藥去。

李嬌兒、孟玉樓衆人,都在月娘房裡裝定菓盒,搽抹銀器。因說:「大娘,你頭裡還要不出去,怎麼他看了就知道你心中的病?」月娘道:「甚麼好成樣的老婆,由他死便死了罷,可是他說的:『你是我婆婆?無故只是大小之分罷了。我還大他八箇月哩,漢子疼我,你只好看我一眼兒罷了。』他不討了他口裡話,他怎麼和我大嚷大鬧?{\meipi{揣摹而成,月娘亦有蘇、張之口。}}若不是你們攛掇我出去,我後十年也不出去。隨他死,教他死去!常言道:『一雞死,一雞鳴,新來雞兒打鳴忒好聽。』{\meipi{猛然念及瓶兒。}}我死了,把他立起來,也不亂,也不嚷,纔『拔了蘿蔔地皮寬」。」玉樓道:「大娘,耶嚛,耶嚛!那裡有此話,俺每就替他賭箇大誓。這六姐,不是我說他,有些不知好歹,行事要便勉強,恰似咬羣出尖兒的一般,一箇大有口沒心的行貨子。{\pangpi{未必。}}大娘你惱他,可知錯惱了哩。」{\meipi{玉樓善於說詞。}}月娘道:「他是比你沒心?他一團兒心機。他怎的會悄悄聽人,行動拿話兒譏諷人。」玉樓道:「娘,你是箇當家人,惡水缸兒,不恁大量些,卻怎樣兒的!常言一箇君子待了十箇小人。{\meipi{千古格言,不獨為金蓮解釋。}}你手放高些,他敢過去了;你若與他一般見識起來,他敢過不去。」月娘道:「只有了漢子與他做主兒着,那大老婆且打靠後。」玉樓道:「哄那箇哩?如今像大娘心裡恁不好,他爹敢往那屋裡去麼!」月娘道:「他怎的不去?可是他說的,他屋裡拿豬毛繩子套,他不去?一箇漢子的心,如同沒籠頭的馬一般,他要喜歡那一箇,只喜歡那箇。誰敢攔他攔,他又說是浪了。」{\meipi{丈夫怒時言語,出口便忘;一到女子,偏偏記得。筆之所至,何所不有?}}玉樓道:「罷麼,大娘,你已是說過,通把氣兒納納兒。等我教他來與娘磕頭,賠箇不是。趁着他大妗子在這裡,你們兩箇笑開了罷。你不然,教他爹兩下里不作難?就行走也不方便。但要往他屋裡去,又怕你惱;若不去,他又不敢出來。今日前邊恁擺酒,俺們都在這裡定菓盒,忙的了不得,他到落得在屋裡躲猾兒。俺每也饒不過他。大妗子,我說的是不是?」大妗子道:「姑娘,也罷,他三娘也說的是。不爭你兩箇話差,只顧不見面,教他姑夫也難,兩下里都不好行走的。」月娘通一聲也不言語。

孟玉樓抽身往前走。{\pangpi{乖。}}月娘道:「孟三姐,不要叫他去,隨他來不來罷。」玉樓道:「他不敢不來,若不來,我可拿豬毛繩子套了他來。」{\pangpi{趣。}}{\meipi{亦自有致。}}一直走到金蓮房中,見他頭也不梳,把臉黃着,坐在炕上。玉樓道:「五姐,你怎的裝憨兒?把頭梳起來,今日前邊擺酒,後邊恁忙亂,你也進去走走兒,怎的只顧使性兒起來?剛纔如此這般,俺每勸了他這一回。你去到後邊,把惡氣兒揣在懷裡,將出好氣兒來,看怎的與他下箇禮,賠箇不是兒罷。你我既在矮簷下,怎敢不低頭。常言:『甜言美語三冬暖,惡語傷人六月寒。』你兩箇已是見過話,只顧使性兒到幾時?人受一口氣,佛受一爐香,你去與他賠箇不是兒,天大事都了了。不然,你不教爹兩下里也難。待要往你這邊來,他又惱。」{\meipi{此一語足動金蓮。刺心語一兩言便了,千古說法也。}}金蓮道:「耶嚛,耶嚛!我拿甚麼比他?{\pangpi{原是。}}可是他說的,他是眞材寔料,正經夫妻,你我都是趁來的露水,能有多大湯水兒?比他的脚指頭兒也比不的兒。」玉樓道:「你又說,我昨日不說的,一棒打三四箇人。就是後婚老婆,也不是趁將來的,當初也有箇三媒六證,難道只恁就跟了往你家來!砍一枝,損百株,就是六姐惱了你,還有沒惱你的。有勢休要使盡,有話休要說盡。凡事看上顧下,留些兒防後纔好。不管蜢蟲、螞蚱,一例都說着。對着他三位師父、郁大姐。人人有面,樹樹有皮,俺每臉上就沒些血兒?他今日也覺不好意思的。只是你不去,卻怎樣兒的?少不的逐日唇不離腮,還有一處兒。你快些把頭梳了,咱兩箇一答兒到後邊去。」那潘金蓮見他恁般說,尋思了半日,忍氣吞聲,{\meipi{可憐英雄失勢時,不知為此四字束縛多少。}}鏡臺前拿過抿鏡,只抿了頭,戴上鬏髻,穿上衣裳,同玉樓徑到後邊上房來。

玉樓掀開簾兒先進去,說道:「我怎的走了去就牽了他來!他不敢不來!」{\meipi{玉樓戲臉。}}便道:「我兒,還不過來與你娘磕頭!」在旁邊便道:「親家,孩兒年幼,不識好歹,冲撞親家。高擡貴手,將就他罷,饒過這一遭兒。到明日再無禮,{\meipi{雖謔語,然道着金蓮實病。}}犯到親家手裡,隨親家打,我老身也不敢說了。」{\pangpi{戲臉。}}那潘金蓮與月娘磕了四箇頭,跳起來,趕着玉樓打道:「汗邪了你這麻淫婦,你又做我娘來了。」連衆人都笑了,那月娘忍不住也笑了。{\pangpi{好人。}}玉樓道:「賊奴才,你見你主子與了你好臉兒,就抖毛兒打起老娘來了。」大妗子道:「你姐妹們笑開,恁歡喜歡喜卻不好?就是俺這姑娘一時間一言半語咭咶你們,大家厮擡厮敬,儘讓一句兒就罷了。常言:『牡丹花兒雖好,還要綠葉扶持。』」月娘道:「他不言語,那箇好說他?」金蓮道:「娘是箇天,俺每是箇地。娘容了俺每,俺每骨禿叉着心裡。」玉樓打了他肩背一下,說道:「我的兒,你這回纔像老娘養的。且休要說嘴,俺每做了這一日話,也該你來助助忙兒。」這金蓮便向炕上與玉樓裝定菓盒,不在話下。

琴童討將藥來,西門慶看了藥貼,就叫送進來與月娘、玉樓。月娘便問玉樓:「你也討藥來?」玉樓道:「還是前日那根兒,下首裡只是有些怪疼,我教他爹對任醫官說,稍帶兩服丸子藥來我吃。」月娘道:「你還是前日空心掉了冷氣了,那裡管下寒的是!」

按下後邊。卻說前廳宋御史先到了,西門慶陪他在捲棚內坐。宋御史深謝其爐鼎之事:「學生還當奉價。」西門慶道:「奉送公祖,猶恐見卻,豈敢云價。」宋御史道:「這等,何以克當?」一面又作揖致謝。茶罷,因說起地方民情風俗一節,西門慶大畧可否而答之。次問及有司官員,{\meipi{黜陟賢否,朝廷矩典,乃諮及市井之人。甚矣,錢神可畏,而官箴可笑也。}}西門慶道:「卑職只知本府胡正堂民望素著,李知縣吏事克勤。其餘不知其詳,不敢妄說。」宋御史問道:「守備周秀曾與執事相交,為人卻也好不好?」西門慶道:「周總兵雖歷練老成,還不如濟州荊都監,青年武舉出身,才勇兼備,公祖倒看他看。」宋御史道:「莫不是都監荊忠?執事何以相熟?」西門慶道:「他與我有一面之交,昨日遞了箇手本與我,望乞公祖青盼一二。」宋御史道:「我也久聞他是箇好將官。」又問其次者,西門慶道:「卑職還有妻兄吳鎧,見任本衙右所正千戶之職。昨日委管修義倉,例該陞指揮,亦望公祖提拔,實卑職之沾恩惠也。」宋御史道:「既是令親,到明日類本之時,不但加陞本等職級,我還保舉他見任管事。」西門慶連忙作揖謝了,因把荊都監並吳大舅履歷手本遞上。宋御史看了,即令書吏收執,分付:「到明日類本之時,呈行我看。」那吏典收下去了。西門慶又令左右悄悄遞了三兩銀子與他,不在話下。

正說話間,前廳鼓樂响,左右來報:「兩司老爺都到了。」慌的西門慶即出迎接,到廳上叙禮。這宋御史慢慢纔走出花園角門。衆官見禮畢數,觀看正中擺設大插卓一張,五老定勝方糖,高頂簇盤,甚是齊正,周圍卓席俱豐勝,心中大悅。都望西門慶謝道:「生受,容當奉補。」宋御史道:「分資誠為不足,四泉看我分上罷了,諸公不消奉補。」西門慶道:「豈有此理。」一面各分次序坐下,左右拿上茶來。衆官又一面差官邀去。看看等到午後,只見一匹報馬來到說:「侯爺來了。」這裡兩邊鼓樂一齊响起,衆官都出大門迎接。宋御史只在二門裡相候。不一時,藍旗馬道過盡,侯巡撫穿大紅孔雀,戴貂鼠暖耳,渾金帶,坐四人大轎,直至門首下轎。衆官迎接進來。宋御史亦換了大紅金雲白豸員領,犀角帶,相讓而入。到於大廳上,叙畢禮數,各官廷參畢,然後是西門慶拜見。侯巡撫因前次擺酒請六黃太尉,認得西門慶。即令官吏拿雙紅友生侯蒙單拜貼,遞與西門慶。西門慶雙手接了,分付家人捧上去。一面參拜畢,寬衣上坐。衆官兩旁僉坐,宋御史居主位。奉畢茶,堦下動起樂來。宋御史遞酒簪花,捧上尺頭,隨即擡下卓席來,裝在盒內,差官吏送到公廳去了。然後上坐,獻湯飯,割獻花豬,俱不必細說。先是教坊弔隊舞,撮弄百戲,十分齊整。然後纔是海鹽子弟上來磕頭,呈上關目揭貼。侯公分付搬演《裴晉公還帶記》。唱了一折下來,又割錦纏羊。端的花簇錦攢,吹彈歌舞,簫韶盈耳,金貂滿座。{\meipi{亦富麗。}}有詩為證:

\begin{myquote} 
華堂非霧亦非烟,歌遏行雲酒滿筵。\\不但紅娥垂玉珮,果然綠鬢插金蟬。
\end{myquote} 

侯巡撫只坐到日西時分,酒過數巡,歌唱兩折下來,令左右拿五兩銀子,分賞廚役、茶酒、樂工、脚下人等,就穿衣起身。衆官俱送出大門,看着上轎而去。回來,宋御史與衆官謝了西門慶,亦告辭而歸。

西門慶送了回來,打發樂工散了。因見天色尚早,分付把卓席休動。一面使小厮請吳大舅並溫秀才、應伯爵、傅夥計、甘夥計、賁第傳、陳敬濟來坐,聽唱。又拿下兩卓酒餚,打發子弟吃了。等的人來,教他唱《四節記•冬景•韓熙載夜宴陶學士》,擡出梅花來,放在兩邊卓上,賞梅飲酒。先是三夥計來旁邊坐下。不一時,溫秀才也過來了,吳大舅、吳二舅、應伯爵都來了。應伯爵與西門慶唱喏:「前日空過衆位嫂子,又多謝重禮。」西門慶笑罵道:「賊天殺的狗材,你打窻戶眼兒內偷瞧的你娘們好!」{\pangpi{應出,趣甚。}}伯爵道:「你休聽人胡說,豈有此理。我想來也沒人。」{\meipi{無意中卻便供出。}}指王經道:「就是你這賊狗骨禿兒,乾淨來家就學舌。我到明日把你這小狗骨禿兒肉也咬了。」說畢,吃了茶。

吳大舅要到後邊,西門慶陪下來,向吳大舅如此這般說:「對宋大巡已替大舅說,他看了揭貼,交付書辦收了。我又與了書辦三兩銀子,連荊大人的都放在一處。他親口許下,到明日類本之時,自有意思。」吳大舅聽了,滿心歡喜,連忙與西門慶唱喏:「多累姐夫費心。」西門慶道:「我就說是我妻兄,他說既是令親,我已定見過分上。」於是同到房中,見了月娘。月娘與他哥道萬福。大舅向大妗子說道:「你往家去罷了,家裡沒人,如何只顧不去了?」大妗子道:「三姑娘留下,教我過了初三日去哩。」吳大舅道:「既是姑娘留你,到初四日去便了。」說畢,來到前邊,同衆坐下飲酒。不一時,下邊戲子鑼鼓响動,搬演《韓熙載夜宴•郵亭佳遇》。正在熱鬧處,忽見玳安來說:「喬親家爹那裡,使了喬通在下邊請爹說話。」西門慶隨即下席見喬通。喬通道:「爹說昨日空過親家。爹使我送那援納例銀子來,一封三十兩,另外又拿着五兩與吏房使用。」西門慶道:「我明日早封過與胡大尹,他就與了箚付來。又與吏房銀子做甚麼?你還帶回去。」一面分付玳安拿酒飯點心,管待喬通,打發去了。

話休饒舌。當日唱了《郵亭》兩折,有一更時分,西門慶前邊人散了,看收了家伙,就進入月娘房來。大妗子正坐的,見西門慶進來,連忙往那邊屋裡去了。西門慶因向月娘說:「我今日替你哥如此這般對宋巡按說,他許下除加陞一級,還教他見任管事,就是指揮僉事。我剛纔已對你哥說了,他好不喜歡,只在年終就題本。」月娘便道:「沒的說,他一箇窮衛家官兒,那裡有二三百銀子使?」西門慶道:「誰問他要一百文錢兒。我就對宋御史說是我妻兄,他親口既許下,無有箇不做分上的。」月娘道:「隨你與他幹,我不管你。」西門慶便問玉簫:「替你娘煎了藥,拿來我瞧着,打發你娘吃了罷。」月娘道:「你去,休管他,等我臨睡自家吃。」那西門慶纔待往外走,被月娘又叫回來,問道:「你往那裡去?若是往前頭去,趁早兒不要去。他頭裡與我陪過不是了,只少你與他陪不是去哩。」{\meipi{金蓮之陪禮為此着也,偏不許去,大煞風景。}}西門慶道:「我不往他屋裡去。」月娘道:「你不往他屋裡去,往誰屋裡去?那前頭媳婦子跟前也省可去。惹的他昨日對着大妗子,好不拿話兒咂我,說我縱容着你要他,圖你喜歡哩。你又恁沒廉恥的。」西門慶道:「你理那小淫婦兒怎的!」{\pangpi{罵處多露愛心。}}月娘道:「你只依我說,今日偏不要你往前邊去,也不要你在我這屋裡,你往下邊李嬌姐房裡睡去。隨你明日去不去,我就不管了。」西門慶見恁說,無法可處,只得往李嬌兒房裡歇了一夜。

到次日,臘月初一日,早往衙門中同何千戶發牌陞廳畫卯,發放公文。一早晨纔來家,又打點禮物豬酒,並三十兩銀子,差玳安往東平府送胡府尹去。胡府尹收下禮物,即時封過箚付來。西門慶在家,請了陰陽徐先生,廳上擺設豬羊酒菓,燒紙還願心畢,打發徐先生去了。因見玳安到了,看了回貼,箚付上面用着許多印信,塡寫喬洪本府義官名目。一面使玳安送兩盒胙肉與喬大戶家,就請喬大戶來吃酒,與他箚付瞧。又分送與吳大舅、溫秀才、應伯爵、謝希大並衆夥計,每人都是一盒,不在話下。一面又發貼兒,初三日請周守備、荊都監、張團練、劉、薛二內相、何千戶、范千戶、吳大舅、喬大戶、王三官兒,共十位客,叫一起雜耍樂工,四箇唱的。

那日,孟玉樓攢了帳,遞與西門慶,就交代與金蓮管理,他不管了。因來問月娘道:「大娘,你昨日吃了藥兒,可好些?」月娘道:「怪不的人說怪浪肉,平白教人家漢子捏了捏手,今日好了。{\meipi{月娘亦有此妙想。}}頭也不疼,心口也不發脹了。」玉樓笑道:「大娘,你原來只少他一捏兒。」連大妗子也笑了。西門慶拿了攢的帳來,又問月娘。月娘道:「該那箇管,你交與那箇就是了。來問我怎的,誰肯讓的誰?」{\meipi{想見西門慶百種虛心,月娘一番冷臉,如畫如睹。}}這西門慶方打帳兌三十兩銀子,三十吊錢,交與金蓮管理,不在話下。

良久,喬大戶到了。西門慶陪他廳上坐的,如此這般拿胡府尹箚付與他看。看見上寫義官喬洪名字:「援例上納白米三十石,以濟邊餉」,滿心歡喜,連忙向西門慶打恭致謝:「多累親家費心,容當叩謝。」{\meipi{應轉前番,一語作結。}}因叫喬通:「好生送到家去。」又說:「明日若親家見招,在下有此冠帶,就敢來陪。」西門慶道:「初三日親家好歹早些下降。」一面吃茶畢,分付琴童,西廂書房裡放卓兒。「親家請那裡坐,還暖些。」同到書房,纔坐下,只見應伯爵到了。斂了幾分人情,交與西門慶,說:「此是列位奉賀哥的分資。」西門慶接了,看頭一位就是吳道官,其次應伯爵、謝希大、祝實念、孫寡嘴、常峙節、白賚光、李智、黃四、杜三哥,共十分人情。西門慶道:「我這邊還有吳二舅、沈姨夫,門外任醫官、花大哥並三箇夥計、溫蔡軒,也有二十多人,就在初四日請罷。」一面令左右收進人情去,使琴童兒:「拿馬請你吳大舅來,陪你喬家親爹坐。」因問:「溫師父在家不在?」來安兒道:「溫師父不在家,望朋友去了。」不一時,吳大舅來到,連陳敬濟五人共坐,把酒來斟。卓上擺列許多下飯。飲酒中間,西門慶因向吳大舅說:「喬親家恭喜的事,今日已領下箚付來了。容日我這裡備禮寫文軸,咱每從府中迎賀迎賀。」喬大戶道:「惶恐,甚大職役,敢起動列位親家費心。」忽有本縣衙差人送歷日來了,共二百五十本。西門慶拿回貼賞賜,打發來人去了。應伯爵道:「新曆日俺每不曾見哩。」西門慶把五十本拆開,與喬大戶、吳大舅、伯爵三人分開。伯爵看了看,開年改了重和元年,該閏正月。

不說當日席間猜枚行令。飲酒至晚,喬大戶先告家去。西門慶陪吳大舅、伯爵坐到起更時分方散。分付伴當:「早伺候備馬,邀你何老爹到我這裡起身,同往郊外送侯爺,留下四名排軍,與來安、春鴻兩箇,跟大娘轎往夏家去。」說畢,就歸金蓮房中來。那婦人未等他進房,就先摘了冠兒,亂挽烏雲,花容不整,朱粉懶施,渾衣兒𢱉在床上,叫着只不做聲。西門慶便坐在床上問道:「怪小油嘴,你怎的恁箇腔兒?」也不答應。被西門慶用手拉起他來,說道:「你如何悻悻的?」那婦人便做出許多喬張致來,把臉扭着,止不住紛紛香腮上滾下淚來。那西門慶就是鐵石人,也把心腸軟了。{\pangpi{自然。}}連忙一隻手摟着他脖子說:「怪油嘴,好好兒的,平白你兩箇合甚麼氣?」那婦人半日方回說道:「誰和他合氣來?他平白尋起箇不是,對着人罵我是攔漢精,趁漢精,{\meipi{罵得不差。}}趁了你來了。他是眞材寔料,正經夫妻。誰教你又到我這屋裡做甚麼!你守着他去就是了,省的我把攔着你。說你來家,只在我這房裡纏,早是肉身聽着,你這幾夜只在我這屋裡睡來?白眉赤眼兒的嚼舌根。一件皮襖,也說我不問他,擅自就問漢子討了。我是使的奴才丫頭,莫不往你屋裡與你磕頭去?為這小肉兒罵了那賊瞎淫婦,也說不管,偏有那些聲氣的。你是箇男子漢,若是有主張,一拳柱定,那裡有這些閑言帳語。怪不的俺每自輕自賤,常言道:『賤裡買來賤裡賣,容易得來容易捨。』趁將你家來,與你家做小老婆,不氣長。你看昨日,生怕氣了他,在屋裡守着的是誰?請太醫的是誰?在跟前攛撥侍奉的是誰?苦惱俺每這陰山背後,就死在這屋裡,也沒箇人兒來偢問。這箇就是出那人的心來了!還教我含着眼淚兒,走到後邊與他賠不是。」{\meipi{責備件件都是,然又不得不然,丈夫處此,大費調停,欲娶妾者看樣。}}說着,那桃花臉上止不住又滾下珍珠兒,倒在西門慶懷裡,嗚嗚咽咽,哭的捽鼻涕彈眼淚。西門慶一面摟抱着勸道:「罷麼,我的兒,我連日心中有事,你兩家各省一句兒就罷了。你教我說誰的是?{\pangpi{果然大難。}}昨日要來看你,他說我來與你賠不是,不放我來。我往李嬌兒房裡睡了一夜。雖然我和人睡,一片心只想着你。」婦人道:「罷麼,我也見出你那心來了。一味在我面上虛情假意,倒老還疼你那正經夫妻。他如今替你懷着孩子,俺每一根草兒,拿甚麼比他!」被西門慶摟過脖子來親了箇嘴,道:「小油嘴,休要胡說。」只見秋菊拿進茶來。西門慶便道:「賊奴才,好乾淨兒,如何教他拿茶?」因問:「春梅怎的不見?」婦人道:「你還問春梅哩,他餓的還有一口遊氣兒,那屋裡躺着不是。帶今日三四日沒吃點湯水兒了,一心只要尋死在那裡。說他大娘,對着人罵了他奴才,氣生氣死,整哭了三四日了。」這西門慶聽了,說道:「眞箇?」婦人道:「莫不我哄你不成,你瞧去不是!」

這西門慶慌過這邊屋裡,只見春梅容粧不整,雲髻歪斜,睡在炕上。西門慶叫道:「怪小油嘴,你怎的不起來?」叫着他,只不做聲,推睡。被西門慶雙關抱將起來。那春梅從酩子裡伸腰,一箇鯉魚打挺,險些兒沒把西門慶掃了一交,早是抱的牢,有護炕倚住不倒。春梅道:「達達,放開了手。你又來理論俺每這奴才做甚麼?也玷辱了你這兩隻手。」{\meipi{娘兒一派,甚有傳授。}}西門慶道:「小油嘴兒,你大娘說了你兩句兒罷了,只顧使起性兒來了。說你這兩日沒吃飯?」春梅道:「吃飯不吃飯,你管他怎的!左右是奴才貨兒,死便隨他死了罷。我做奴才,也沒幹壞了甚麼事,並沒教主子罵我一句兒,打我一下兒,做甚麼為這㒲遍街搗遍巷的賊瞎婦,教大娘這等罵我,嗔俺娘不管我,莫不為瞎淫婦打我五板兒?{\meipi{遷怒大奇,然婦人女子恆情如此。}}等到明日,韓道國老婆不來便罷,若來,你看我指着他一頓好罵。原來送了這瞎淫婦來,就是箇禍根。」西門慶道:「就是送了他來,也是好意,誰曉的為他合起氣來。」春梅道:「他若肯放和氣些,我好罵他?他小量人家!」西門慶道:「我來這裡,你還不倒鍾茶兒我吃?那奴才手不乾淨,我不吃他倒的茶。」春梅道:「死了王屠,連毛吃豬。我如今走也走不動在這裡,還教我倒甚麼茶?」西門慶道:「怪小油嘴兒,誰教你不吃些甚麼兒?」因說道:「咱每往那邊屋裡去。我也還沒吃飯哩,教秋菊後邊取菜兒,篩酒,烤菓餡餅兒,炊鮮湯咱每吃。」於是不繇分訴,拉着春梅手到婦人房內。分付秋菊:「拿盒子後邊取吃飯的菜兒去。」不一時,拿了一方盒菜蔬來。西門慶分付春梅:「把肉鮓拆上幾絲雞肉,加上酸筍韭菜,和成一大碗香噴噴餛飩湯來。」放下卓兒擺上,一面盛飯來。又烤了一盒菓餡餅兒。西門慶和金蓮並肩而坐,春梅也在旁陪着同吃。三箇你一盃,我一盃,吃到一更方睡。

到次日,西門慶起早,約會何千戶來到,吃了頭腦酒,起身同往郊外送侯巡撫去了。吳月娘先送禮往夏指揮家去,然後打扮,坐大轎,排軍喝道,來安、春鴻跟隨來吃酒,看他娘子兒,不在話下。

且說玳安、王經看家,將到晌午時分,只見縣前賣茶的王媽媽領着何九,來大門首尋問玳安:「老爹在家不在家?」玳安道:「何老人家、王奶奶稀罕,今日那陣風兒吹你老人家來這裡走走?」王婆子道:「沒勾當怎好來踅門踅戶?今日不因老九,為他兄弟的事,要央煩你老爹,老身還不敢來。」玳安道:「老爺今日與侯爺送行去了,俺大娘也不在家。你老人家站站,等我進去對五娘說聲。」進入不多時出來,說道:「俺五娘請你老人家進去哩。」王婆道:「我敢進去?你引我引兒,只怕有狗。」那玳安引他進入花園金蓮房門首,掀開簾子,王婆進去。見婦人家常戴着臥免兒,穿着一身錦段衣裳,搽抹的粉粧玉琢,正在炕上脚登着爐臺兒坐的。進去不免下禮,慌的婦人答禮,說道:「老王免了罷。」{\meipi{口角輕薄。}}那婆子見畢禮,坐在炕邊頭。婦人便問:「怎的一向不見你?」王婆子道:「老身心中常想着娘子,只是不敢來親近。」問:「添了哥哥不曾?」婦人道:「有倒好了。小產過兩遍,白不存。」問:「你兒子有了親事來?」王婆道:「還不曾與他尋。他跟客人淮上來家這一年多,家中積攢了些,買箇驢兒,胡亂磨些麵兒賣來度日。」因問:「老爹不在家了?」婦人道:「他今日往門外與撫按官送行去了,他大娘也不在家,有甚話說?」王婆道:「何老九有樁事,央及老身來對老爹說:他兄弟何十吃賊攀了,{\meipi{又應出何九。}}見拿在提刑院老爹手裡問。攀他是窩主。本等與他無干,望乞老爹案下與他分豁分豁。賊若指攀,只不准他就是了。何十出來,到明日買禮來重謝老爹,有箇說貼兒在此。」一面遞與婦人。婦人看了,說道:「你留下,等你老爹來家,我與他瞧。」婆子道:「老九在前邊伺候着哩,明日教他來討話罷。」

婦人一面叫秋菊看茶來,須臾,秋菊拿了一盞茶來,與王婆吃了。那婆子坐着,說道:「娘子,你這般受福勾了。」{\pangpi{此語未免居功。}}婦人道:「甚麼勾了,不惹氣便好,成日歐氣不了在這裡。」婆子道:「我的奶奶,你飯來張口,水來濕手,這等插金戴銀,呼奴使婢,又惹甚麼氣?」婦人道:「常言說得好,三窩兩塊,大婦小妻,一箇碗內兩張匙,不是湯着就抹着。如何沒些氣兒?」婆子道:「好奶奶,你比那箇不聰明!趁着老爹這等好時月,你受用到那裡是那裡。」說道:「我明日使他來討話罷。」於是拜辭起身。婦人道:「老王,你多坐回去不是?」那婆子道:「難為老九,只顧等我,不坐罷。改日再來看你。」婦人也不留他留兒,就放出他來了。到了門首,又叮嚀玳安。玳安道:「你老人家去,我知道,等俺爹來家我就稟。」何九道:「安哥,我明日早來討話罷。」於是和王婆一路去了。

至晚,西門慶來家。玳安便把此事稟知。西門慶到金蓮房看了貼子,交付與答應的收着:「明日到衙門中稟我。」一面又令陳敬濟發初四日請人貼子。瞞着春梅,又使琴童兒送了一兩銀子並一盒點心到韓道國家,對着他說:「是與申二姐的,教他休惱。」那王六兒笑嘻嘻接了,說:「他不敢惱。多上覆爹娘,冲撞他春梅姑娘。」俱不在言表。

至晚,月娘來家,先拜見大妗子衆人,然後見西門慶,道了萬福,就告訴:「夏大人娘子見了我去,好不喜歡。今日也有許多親隣堂客。原來夏大人有書來了,也有與你的書,明日送來與你。{\pangpi{伏着老溫一案。}}也只在這初六、七起身,搬取家小上京。說了又說,好歹央賁四送他到京就回來。賁四的那孩子長兒,今日與我磕頭,好不出跳的好箇身段兒。嗔道他旁邊捧着茶把眼只顧偷瞧我。我也忘了他,倒是夏大人娘子叫他改換的名字,叫做瑞雲,『過來與你西門奶奶磕頭』,他纔放下茶托兒,與我磕了四箇頭。我與了他兩枝金花兒。夏大人娘子好不喜歡,擡舉他,也不把他當房裡人,只做親兒女一般看他。」西門慶道:「還是這孩子有福,若是別人家手裡,怎麼容得,不罵奴才少椒末兒,又肯擡舉他!」{\meipi{慧心人面前,帶㧓話原說不得。}}被月娘瞅了一眼,說道:「硶說嘴的貨,是我罵了你心愛的小姐兒了!」西門慶笑了,說道:「他借了賁四押家小去,我線鋪子教誰看?」月娘道:「關兩日也罷了。」西門慶道:「關兩日,阻了買賣,近年近節,紬絹絨線正快,如何關閉了鋪子?到明日再處。」說畢,月娘進裡間脫衣裳摘頭,走到那邊房內,和大妗子坐的。家中大小都來參見磕頭。是日,西門慶在後邊雪娥房中歇了一夜,早往衙門中去了。只見何九走來問玳安討信,與了玳安一兩銀子。玳安道:「昨日爹來家,就替你說了。今日到衙門中,敢就開出你兄弟來了。你往衙門首伺候。」何九聽言,滿心歡喜,一直走到衙門前去了。西門慶到衙門中坐廳,提出強盜來,每人又是一夾,二十大板,把何十開出來,放了。另拿了弘化寺一名和尚頂缺,說強盜曾在他寺內宿了一夜。{\meipi{近來刑獄,大抵如斯。}}正是:張公吃酒李公醉,桑樹上脫枝柳樹上報。有詩為證:

\begin{myquote} 
宋朝氣運已將終,執掌提刑甚不公。\\畢竟難逃天下眼,那堪激濁與揚清。
\end{myquote} 

那日西門慶家中叫了四箇唱的:吳銀兒、鄭愛月兒、洪四兒、齊香兒,日頭晌午就來了,都到月娘房內,與月娘、大妗子衆人磕頭。月娘擺茶與他們吃了。正彈着樂器,唱曲兒與衆人聽,忽見西門慶從衙門中來家,進房來。四箇唱的都放了樂器,笑嘻嘻向前,與西門慶磕頭。坐下,月娘便問:「你怎的衙門中這咱纔來?」西門慶告訴:「今日向理好幾樁事情。」因望着金蓮說:「昨日王媽媽來說何九那兄弟,今日我已開除來放了。那兩名強盜還攀扯他,教我每人打了二十,夾了一夾,拿了門外寺裡一箇和尚頂缺,明日做文書送過東平府去。又是一起姦情事,是丈母養女婿的。那女婿不上二十多歲,名喚宋得,原與這家是養老不歸宗女婿。落後親丈母死了,娶了箇後丈母周氏,不上一年,把丈人死了。這周氏年小,守不得,就與這女婿暗暗通姦,後因為責使女,被使女傳於兩隣,纔首告官。今日取了供招,都一日送過去了。這一到東平府,姦妻之母,系緦麻之親,兩箇都是絞罪。」潘金蓮道:「要着我,把學舌的奴才打的爛糟糟的,問他箇死罪也不多。你穿青衣抱黑柱,一句話就把主子弄了。」{\meipi{關着敬濟,便言之激烈乃爾。}}{\pangpi{便伏秋菊案。}}西門慶道:「也吃我把那奴才拶了幾拶子好的。為你這奴才,一時小節不完,䘮了兩箇人性命。」月娘道:「大不正則小不敬。母狗不掉尾,公狗不上身。大凡還是女人心邪,若是那正氣的,誰敢犯他!」{\meipi{月娘詞氣侃侃,足寒金蓮之膽。}}四箇唱的都笑道:「娘說的是。就是俺裡邊唱的,接了孤老的朋友還使不的,休說外頭人家。」說畢,擺飯與西門慶吃了。

忽聽前廳鼓樂响,荊都監來了。西門慶連忙冠帶出迎,接至廳上叙禮,分賓主坐下。茶罷,如此這般告說:「宋巡按收了說貼,已慨然許下,執事恭喜,必然在邇。」荊都監聽了,又下坐作揖致謝:「老翁費心,提攜之力,銘刻難忘。」西門慶又說起:「周老總兵,生也薦言一二,宋公必有主意。」談話間,忽然劉薛二公公到。鼓樂迎接進來,西門慶相讓入廳,叙禮。二內相皆穿青縲絨蟒衣,寶石縧環,正中間坐下。次後周守備到了,一處叙話。荊都監又向周守備說:「四泉厚情,昨日宋公在尊府擺酒,曾稱頌公之才猷。宋公已留神於中,高轉在即。」周守備亦欠身致謝不盡。落後張團練、何千戶、王三官、范千戶、吳大舅、喬大戶陸續都到了。喬大戶冠帶青衣,四箇伴當跟隨,{\meipi{便有氣勢。}}進門見畢諸公,與西門慶拜了四拜。衆人問其恭喜之事,西門慶道:「舍親家在本府援例新受恩榮義官之職。」周守備道:「四泉令親,吾輩亦當奉賀。」喬大戶道:「蒙列位老爹盛情,{\pangpi{妙。}}豈敢動勞。」說畢,各分次序坐下。遍遞了一道茶,然後遞酒上坐。錦屏前玳筵羅列,畫堂內寶玩爭輝,堦前動一派笙歌,席上堆滿盤異菓。良久,遞酒安席畢,各歸席坐下。王三官再三不肯上來坐,西門慶道:「尋常罷了,今日在舍,權借一日陪諸公上坐。」王三官必不得已,左邊垂首坐了。須臾,上罷湯飯,下邊教坊撮弄雜耍百戲上來。良久,纔是四箇唱的,拿着銀箏玉板,放嬌聲當筵彈唱。正是:

\begin{myquote} 
舞裙歌板逐時新,散盡黃金只此身。\\寄與富兒休暴殄,儉如良藥可醫貧。
\end{myquote} 

當日劉內相坐首席,也賞了許多銀子。飲酒為歡,至一更時分方散。西門慶打發樂工賞錢出門。四箇唱的都在月娘房內彈唱,月娘留下吳銀兒過夜,打發三箇唱的去。臨去,見西門慶在廳上,拜見拜見。西門慶分付鄭愛月兒:「你明日就拉了李桂姐,兩箇還來唱一日。」鄭愛月兒就知今日有王三官兒,不叫李桂姐來唱,{\meipi{乖。}}笑道:「爹,你『兵馬司倒了墻——賊走了』?」又問:「明日請誰吃酒?」西門慶道:「都是親朋。」鄭愛月兒道:「有應二那花子,我不來,我不要見那醜冤家怪物。」西門慶道:「明日沒有他。」愛月兒道:「沒有他纔好。若有那怪攮刀子的,俺們不來。」說畢,磕了頭去了。西門慶看着收了家伙,回到李瓶兒那邊,和如意兒睡了。一宿晚景題過。

次日,早往衙門送問那兩起人犯過東平府去。回來家中擺酒,請吳道官、吳二舅、花大舅、沈姨父、韓姨夫、任醫官、溫秀才、應伯爵,並會衆人李智、黃四、杜三哥並家中三箇夥計,十二張桌兒。席中止是李桂姐、吳銀兒、鄭愛月兒三箇粉頭遞酒,李銘、吳惠、鄭奉三箇小優兒彈唱。正遞酒中間,忽平安兒來報:「雲二叔新襲了職,來拜爹,送禮來。」西門慶聽言,忙道:「有請。」只見雲理守穿着青紵絲補服員領,冠冕着,腰繫金帶,後面伴當擡着禮物,先遞上揭貼,與西門慶觀看。上寫:「新襲職山東清河右衛指揮同知門下生雲理守頓首百拜。謹具土儀:貂鼠十箇,海魚一尾,蝦米一包,臘鵝四隻,臘鴨十隻,油低簾二架,少申芹敬。」西門慶即令左右收了,連忙致謝。雲理守道:「在下昨日纔來家,今日特來拜老爹。」於是四雙八拜,說道:「蒙老爹莫大之恩,些少土儀,表意而已。」然後又與衆人叙禮拜見。西門慶見他居官,就待他不同,安他與吳二舅一桌坐了,連忙安鍾筯,下湯飯。脚下人俱打發攢盤酒肉。因問起發䘮替職之事,這雲理守一一數言:「蒙兵部余爺憐先兄在鎭病亡,祖職不動,還與了箇本衛見任僉書。」西門慶歡喜道:「恭喜恭喜,容日已定來賀。」當日衆人席上每位奉陪一盃,又令三箇唱的奉酒,須臾把雲理守灌的醉了。那應伯爵在席上,如線兒提的一般,起來坐下,又與李桂姐、鄭月兒彼此互相戲罵不絕。當日酒筵笑聲,花攢錦簇,觥籌交錯,耍頑至二更時分方纔席散。打發三箇唱的去了,西門慶歸上房宿歇。

到次日起來遲,正在上房擺粥吃了,穿衣要拜雲理守。只見玳安來說:「賁四在前邊請爹說話。」西門慶就知為夏龍溪送家小之事,一面出來廳上。只見賁四向袖中取出夏指揮書來呈上,說道:「夏老爹要教小人送送家小往京裡去,小人稟問老爹去不去?」西門慶看了書中言語,無非是叙其闊別,謝其早晚看顧家小,又借賁四攜送家小之事,因說道:「他既央你,你怎的不去!」因問:「幾時起身?」賁四道:「今早他大官兒叫了小人去,分付初六日家小準起身。小人也得半月纔回來。」說畢,把獅子街鋪內鑰匙交遞與西門慶。西門慶道:「你去,我教你吳二舅來,替你開兩日罷。」那賁四方纔拜辭出門,往家中收拾行裝去了。西門慶就冠冕着出門,拜雲指揮去了。

那日大妗子家去,叫下轎子門首伺候。也是合當有事,月娘裝了兩盒子茶食點心下飯,送出門首上轎。只見畫童兒小厮躲在門房,大哭不止。那平安兒只顧扯他,那小厮越扯越哭起來。{\pangpi{今日肯哭者誰?}}{\meipi{畫童較近時未冠,覺有操守。}}被月娘等聽見,送出大妗子去了,便問平安兒:「賊囚,你平白扯他怎的?惹的他恁怪哭。」平安道:「溫師父那邊叫扯,他白不去,只是罵小的。」月娘道:「你教他好好去罷。」因問道:「小厮,你師父那邊叫,去就是了,怎的哭起來?」那畫童嚷平安道:「又不關你事,我不去罷了,你扯我怎的?」月娘道:「你因何不去?」那小厮又不言語。金蓮道:「這賊小囚兒,就是箇肉佞賊。你大娘問你,怎的不言語?」被平安向前打了一箇嘴巴,{\pangpi{奇。}}那小厮越發大哭了。月娘道:「怪囚根子,你平白打他怎的?你好好教他說,怎的不去?」

正問着,只見玳安騎了馬進來。月娘問道:「你爹來了?」玳安道:「被雲二叔留住吃酒哩。使我送衣裳來了,要還氊巾去。」看見畫童兒哭,便問:「小大官兒,怎的号啕痛也是的?」平安道:「對過溫師父叫他不去,反哭罵起我來了。」玳安道:「我的哥哥,溫師父叫,你仔細,有名的溫屁股,他一日沒屁股也成不的。你每常怎麼挨他的,今日又躲起來了?」{\meipi{如今沒屁股過不得的甚多,安得盡以溫屁股名之也?}}月娘罵道:「怪囚根子,怎麼溫屁股?」玳安道:「娘只問他就是。」潘金蓮得不的風兒就是雨兒,{\pangpi{留心此道。}}一面叫過畫童兒來,只顧問他:「小奴才,你寔說他叫你做甚麼?你不說,看我教你大娘打你。」逼問那小厮急了,說道:「他只要哄着小的,把他那行貨子放在小的屁股裡,弄的脹脹的疼起來。我說你還不快拔出來,他又不肯拔,只顧來回動。且教小的拿出,跑過來,他又來叫小的。」月娘聽了便喝道:「怪賊小奴才兒,還不與我過一邊去!也有這六姐,只管審問他,說的硶死了。我不知道,還當是好話兒,側着耳朵兒聽他。這蠻子也是箇不上蘆帚的行貨子,人家小厮與你使,卻背地幹這箇營生。」金蓮道:「大娘,那箇上蘆帚的肯幹這營生,{\meipi{列位先生請看:小使且不可,況門生乎?伏侯林兒。}}冷鋪睡的花子纔這般所為。」{\pangpi{金蓮獨不記討紗裙時耶!}}孟玉樓道:「這蠻子,他有老婆,怎生這等沒廉恥?」金蓮道:「他來了這一向,俺們就沒見他老婆怎生樣兒。」平安道:「娘每會勝也不看見他。他但往那邊去就鎖了門。住了這半年,我只見他坐轎子往娘家去了一遭,沒到晚就來家了。往常幾時出箇門兒來,只好晚夕門首倒榪子走走兒罷了。」金蓮道:「他那老婆也是箇不長俊的行貨子,嫁了他,怕不的也沒見箇天日兒,敢每日只在屋裡坐天牢哩。」說了回,月娘同衆人回後邊去了。

西門慶約莫日落時分來家,到上房坐下。月娘問道:「雲夥計留你坐來?」西門慶道:「他在家,見我去,旋放桌兒留我坐,開啟一罈酒和我吃。如今衛中荊南崗陞了,他就挨着掌印。明日連他和喬親家,就是兩分賀禮,衆同僚都說了,要與他掛軸子,少不得教溫葵軒做兩篇文章,買軸子寫。」月娘道:「還纏甚麼溫葵軒、鳥葵軒哩!平白安紮恁樣行貨子,沒廉恥,傳出去教人家知道,把醜來出盡了。」西門慶聽言,諕了一跳,{\meipi{不繇他不吃驚。}}便問:「怎麼的?」月娘道:「你別要來問我,你問你家小厮去。」西門慶道:「是那箇小厮?」金蓮道:「情知是誰?畫童賊小奴才,俺去送大妗子去,他正在門首哭,如此這般,溫蠻子弄他來。」西門慶聽了,還有些不信,便道:「你叫那小奴才來,等我問他。」一面使玳安兒前邊把畫童兒叫到上房,跪下,西門慶要拿拶子拶他,{\pangpi{何必。}}便道:「賊奴才,你寔說,他叫你做甚麼?」畫童兒道:「他叫小的,要灌醉了小的,幹那小營生兒。{\meipi{外冠裳而內穿窬者,不止溫秀才一箇。}}今日小的害疼,躲出來了,不敢去。他只顧使平安叫,又打小的,教娘出來看見了。他常時問爹家中各娘房裡的事,小的不敢說。昨日爹家中擺酒,他又教唆小的偷銀器家伙與他。又某日他望倪師父去,拿爹的書稿兒與倪師父瞧,倪師父又與夏老爺瞧。」這西門慶不聽便罷,聽了便道:「畫虎畫皮難畫骨,知人知面不知心。我把他當箇人看,誰知他人皮包狗骨東西,要他何用?」{\meipi{一箇疑團到此結出,有意無意之中何等冷雋。}}一面喝令畫童起去,分付:「再不消過那邊去了。」

那畫童磕了頭,起來往前邊去了。西門慶向月娘道:「怪道前日翟親家說我機事不密則害成,我想來沒人,原來是他把我的事透泄與人,我怎的曉得?這樣的狗骨禿東西,平白養在家做甚麼?」月娘道:「你和誰說?你家又沒孩子上學,平白招攬箇人在家養活,只為寫禮貼兒,饒養活着他,還教他弄乾坤兒。」西門慶道:「不消說了,明日教他走道兒就是了。」一面叫將平安來,分付:「對過對他說,家老爹要房子堆貨,教溫師父轉尋房兒便了。等他來見我,你在門首,只回我不在家。」那平安兒應諾去了。

西門慶告月娘說:「今日賁四來辭我,初六日起身,與夏龍溪送家小往東京去。我想來,線鋪子沒人,倒好教二舅來替他開兩日兒。好不好?」月娘道:「好不好,隨你叫他去。我不管你,省的人又說照顧了我的兄弟。」西門慶不聽,於是使棋童兒:「請你二舅來。」不一時,請吳二舅到,在前廳陪他吃酒坐的,把鑰匙交付與他:「明日同來昭早往獅子街開鋪子去。」不在話下。

卻說溫秀才見畫童兒一夜不過來睡,心中省恐。到次日,平安走來說:「家老爹多上覆溫師父,早晚要這房子堆貨,教師父別尋房兒罷。」這溫秀才聽了,大驚失色,就知畫童兒有甚話說,穿了衣巾,要見西門慶說話。平安道:「俺爹往衙門中去了,還未來哩。」比及來,這溫秀才又衣巾過來伺候,具了一篇長柬,{\meipi{中有何說。}}遞與琴童兒。琴童又不敢接,說道:「俺爹纔從衙門中回家,辛苦,後邊歇去了,俺每不敢稟。」這溫秀才就知疎遠他,一面走到倪秀才家商議,還搬移家小往舊處住去了。正是:

\begin{myquote} 
誰人汲得西江水,難洗今朝一面羞。\\靡不有初鮮克終,交情似水淡長濃。\\自古人無千日好,果然花無摘下紅。
\end{myquote} 

