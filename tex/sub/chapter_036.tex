\includepdf[pages={71,72},fitpaper=false]{tst.pdf}
\chapter*{第三十六回 翟管家寄書尋女子 蔡狀元留飲借盤纏}
\addcontentsline{toc}{chapter}{第三十六回 翟管家寄書尋女子 蔡狀元留飲借盤纏}
\markboth{{\titlename}卷之四}{第三十六回 翟管家寄書尋女子 蔡狀元留飲借盤纏}


詩曰:

\begin{myquote} 
既傷千里目,還驚遠去魂。\\豈不憚跋涉?深懷國士恩。\\季布無一諾,侯嬴重一言。\\人生感意氣,黃金何足論。
\end{myquote} 

話說次日,西門慶早與夏提刑接了新巡按,又到庄上犒勞做活的匠人。至晚來家,平安進門就稟:「今日有東昌府下文書快手,往京裡順便稍了一封書帕來,說是太師爺府裡翟大爹寄來與爹的。小的接了,交進大娘房裡去了。那人明日午後來討回書。」西門慶聽了,走到上房,取書拆開觀看,上面寫着:

\begin{myquote}[\markfont]
京都侍生翟謙頓首書拜

即擢大錦堂西門大人門下:久仰山斗,未接丰標,屢辱厚情,感愧何盡!前蒙馳諭,生銘刻在心。凡百於老爺左右,無不盡力扶持。所有小事,曾托盛价煩瀆,想已為我處之矣。今日鴻便,薄具帖金十兩奉賀,兼候起居。伏望俯賜回音,生不勝感激之至。外新狀元蔡一泉,乃老爺之假子,奉勑回籍省視,道經貴處,仍望留之一飯,彼亦不敢有忘也。至祝至祝!秋後一日信。
\end{myquote} 

西門慶看畢,只顧諮嗟不已,說道:「快叫小厮叫媒人去。我什麼營生,就忘死了。」{\meipi{為人之事,雖感德如雲峯,亦要忘了,況其他乎?}}吳月娘問:「甚麼勾當?」西門慶道:「東京太師老爺府裡翟管家,前日有書來,說無子,央及我這裡替他尋箇女子。不拘貧富,不限財禮,只要好的,他要圖生長。粧奩財禮,該使多少,教我開了去,他一一還我,往後他在老爺面前,一力扶持我做官。我一向亂着上任,七事八事,就把這事忘死了。來保又日逐往鋪子裡去了,又不題我。

今日他老遠的教人稍書來,問尋的親事怎樣了。又寄了十兩折禮銀子賀我。明日差人就來討回書,你教我怎樣回答他?教他就怪死了!叫了媒人,你分咐他,好歹上緊替他尋着,不拘大小人家,只要好女兒,或十五六、十七八的也罷,該多少財禮,我這裡與他。再不,把李大姐房裡綉春,倒好模樣兒,與他去罷。」月娘道:「我說你是箇火燎腿行貨子!這兩三箇月,你早做什麼來?人家央你一場,替他看箇眞正女子去也好。那丫頭你又收過他,怎好打發去的!你替他當箇事幹,他到明日也替你用的力。如今急水發,怎麼下得漿?比不得買什麼兒,拿了銀子到市上就買的來了。一箇人家閨門女子,好歹不同,也等着媒人慢慢踏看將來。你倒說的好自在話兒!」西門慶道:「明日他來要回書,怎麼回答他?」月娘道:「虧你還斷事!這些勾當兒,便不會打發人?等那人明日來,你多與他些盤纏,寫書回覆他,只說女子尋下了,只是衣服粧奩未辦,還待幾時完畢,這裡差人送去。打發去了,你這裡教人替他尋也不遲。此一舉兩得其便,纔幹出好事來,也是人家托你一場。」西門慶笑道:「說的有理!」一面叫將陳敬濟來,隔夜修了回書。

次日,下書人來到,西門慶親自出來,問了備細。又問蔡狀元幾時船到,好預備接他。那人道:「小人來時蔡老爹纔辭朝,京中起身。翟爹說:只怕蔡老爹回鄉,一時缺少盤纏,煩老爹這裡多少只顧藉與他。寫書去,翟老爹那裡如數補還。」西門慶道:「你多上覆翟爹,隨他要多少,我這裡無不奉命。」說畢,命陳敬濟讓去廂房內管待酒飯。臨去交割回書,又與了他五兩路費。那人拜謝,歡喜出門,長行去了。

看官聽說:當初安忱取中頭甲,被言官論他是先朝宰相安惇之弟,系黨人子孫,不可以魁多士。徽宗不得已,把蔡蘊擢為第一,做了狀元。投在蔡京門下,做了假子。陞祕書省正事,給假省親。且說月娘家中使小厮叫了老馮、薛嫂兒並別的媒人來,分咐各處打聽人家有好女子,拿帖兒來說,不在話下。

一日,西門慶使來保往新河口,打聽蔡狀元船隻,原來就和同榜進士安忱同船。這安進士亦因家貧未續親,東也不成,西也不就,辭朝還家續親,因此二人同船來到新河口。來保拿着西門慶拜帖來到船上見,就送了一分下程,酒面、雞鵝、下飯、鹽醬之類。蔡狀元在東京,翟謙已預先和他說了:「清河縣有老爺門下一箇西門千戶,乃是大巨家,富而好禮。亦是老爺擡舉,見做理刑官。你到那裡,他必然厚待。」這蔡狀元牢記在心,{\meipi{為己之事,便牢記在心。}}見西門慶差人遠來迎接,又餽送如此大禮,心中甚喜。次日就同安進士進城來拜。西門慶已是預備下酒席。因在李知縣衙內吃酒,看見有一起蘇州戲子唱的好,旋叫了四箇來答應。

蔡狀元那日封了一端絹帕、一部書、一雙雲履。安進士亦是書帕二事、四袋芽茶、四柄杭扇。各具宮袍烏紗,先投拜帖進去。西門慶冠冕迎接至廳上,叙禮交拜。獻畢贄儀,然後分賓主而坐。先是蔡狀元舉手欠身說道:「京師翟雲峯,{\meipi{觀此,稱雲峯以為榮,寫出仕途之穢。}}甚是稱道賢公閥閱名家,清河巨族。久仰德望,未能識荊,今得晉拜堂下,為幸多矣!」西門慶答道:「不敢!昨日雲峯書來,具道二位老先生華輈下臨,理當迎接,奈公事所羈,望乞寬恕。」因問:「二位老先生仙鄉、尊號?」蔡狀元道:「學生本貫滁州之匡廬人也。賤號一泉,僥倖狀元,官拜祕書正字,給假省親。」安進士道:「學生乃浙江錢塘縣人氏。賤號鳳山。見除工部觀政,亦給假還鄉續親。敢問賢公尊號?」西門慶道:「在下卑官武職,何得號稱。」詢之再三,方言:「賤號四泉,累蒙蔡老爺擡舉,雲峯扶持,襲錦衣千戶之職。見任理刑,實為不稱。」蔡狀元道:「賢公抱負不凡,雅望素著,休得自謙。」叙畢禮話,請去花園捲棚內寬衣。蔡狀元辭道:「學生歸心匆匆,行舟在岸,就要回去。既見尊顏,又不遽捨,奈何奈何!」{\meipi{口角留連得妙。}}西門慶道:「蒙二公不棄蝸居,伏乞暫住文旆,少留一飯,以盡芹獻之情。」蔡狀元道:「既是雅情,學生領命。」一面脫去衣服,二人坐下。左右又換了一道茶上來。蔡狀元以目瞻顧園池臺館,花木深秀,一望無際,心中大喜,極口稱羨道:「誠乃蓬瀛也!」於是擡過棋桌來下棋。西門慶道:「今日有兩箇戲子在此伺候,以供宴賞。」安進士道:「在那裡?何不令來一見?」不一時,四箇戲子跪下磕頭。蔡狀元問道:「那兩箇是生旦?叫甚名字?」內中一箇答道:「小的粧生,叫苟子孝。那一箇裝旦的叫周順。一箇貼旦叫袁琰。那一箇裝小生的叫胡慥。」安進士問:「你們是那裡子弟?」苟子孝道:「小的都是蘇州人。」安進士道:「你等先粧扮了來,唱箇我們聽。」四箇戲子下邊粧扮去了。西門慶令後邊取女衣釵梳與他,教書童也粧扮起來。共三箇旦、兩箇生,在席上先唱《香囊記》。大廳正面設兩席,蔡狀元、安進士居上,西門慶下邊主位相陪。飲酒中間,唱了一折下來,安進士看見書童兒裝小旦,便道:「這箇戲子是那裡的?」西門慶道:「此是小价書童。」安進士叫上去,賞他酒吃,說道:「此子絕妙而無以加矣!」蔡狀元又叫別的生旦過來,亦賞酒與他吃。因分咐:「你唱箇《朝元歌》『花邊柳邊』。」苟子孝答應,在旁拍手道:

\begin{myquote} 
「花邊柳邊,簷外晴絲捲。山前水前,馬上東風軟。自歎行蹤,有如蓬轉,盼望家鄉留戀。雁杳魚沉,離愁滿懷誰與傳?日短北堂萱,空勞魂夢牽。洛陽遙遠,幾時得上九重金殿?」
\end{myquote} 

唱完了,安進士問書童道:「你們可記的《玉環記》『恩德浩無邊』?」書童答道:「此是《畫眉序》,小的記得。」隨唱道:

\begin{myquote}
恩德浩無邊,父母重逢感非淺。幸終身托與,又與姻緣。風雲會異日飛騰,鸞鳳配今諧繾綣。料應夫婦非今世,前生種玉藍田。
\end{myquote}

原來安進士杭州人,喜尚男風,見書童兒唱的好,拉着他手兒,兩箇一遞一口吃酒。良久,酒闌上來,西門慶陪他復遊花園,向捲棚內下棋。令小厮拿兩箇桌盒,三十樣都是細巧果菜、鮮物下酒。蔡狀元道:「學生們初會,不當深擾潭府,天色晚了,告辭罷。」西門慶道:「豈有此理。」因問:「二公此回去,還到船上?」蔡狀元道:「暫借門外永福寺寄居。」西門慶道:「如今就門外去也晚了。不如老先生把手下從者止留一二人答應,其餘都分咐回去,明日來接,庶可兩盡其情。」蔡狀元道:「賢公雖是愛客之意,其如過擾何!」當下二人一面分咐手下,都回門外寺裡歇去,明日早拿馬來接。衆人應諾去了,不在話下。

二人在捲棚內下了兩盤棋,子弟唱了兩折,恐天晚,西門慶與了賞錢,打發去了。止是書童一人,席前遞酒伏侍。看看吃至掌燈,二人出來更衣,蔡狀元拉西門慶說話:「學生此去回鄉省親,路費缺少。」西門慶道:「不勞老先生分咐。雲峯尊命,已定謹領。」良久,讓二人到花園:「還有一處小亭請看。」把二人一引,轉過粉墻,來到藏春塢雪洞內。裡面煖騰騰掌着燈燭,小琴桌上早已陳設果酌之類,床榻依然,琴書瀟灑。從新復飲,書童在旁歌唱。蔡狀元問道:「大官,你會唱『紅入仙桃』?」書童道:「此是《錦堂月》,小的記得。」於是把酒都斟,拿住南腔,拍手唱了一箇。安進士聽了,喜之不勝,向西門慶道:「此子可愛。」將盃中之酒一吸而飲之。那書童在席間穿着翠袖紅裙,勒着銷金箍兒,高擎玉斝,捧上酒,又唱了一箇。當日直飲至夜分,方纔歇息。西門慶藏春塢、翡翠軒兩處俱設床帳,鋪陳綾錦被褥,就派書童、玳安兩箇小厮答應。西門慶道了安置,方回後邊去了。

到次日,蔡狀元、安進士跟從人夫轎馬來接。西門慶廳上擺酒伺候,饌飲下飯與脚下人吃。教兩箇小厮,方盒捧出禮物。蔡狀元是金段一端,領絹二端,合香五百,白金一百兩。安進士是色段一端,領絹一端,合香三百,白金三十兩。蔡狀元固辭再三,說道:「但假十數金足矣,何勞如此太多,又蒙厚腆!」安進士道:「蔡年兄領受,學生不當。」西門慶笑道:「些須微贐,表情而已。老先生榮歸續親,在下少助一茶之需。」於是兩人俱出席謝道:「此情此德,何日忘之!」一面令家人各收下去,一面與西門慶相別,說道:「生輩此去,暫違臺教。不日旋京,倘得寸進,自當圖報。」安進士道:「今日相別,何年再得奉接尊顏?」西門慶道:「學生蝸居屈尊,多有褻慢,幸惟情恕!本當遠送,奈官守在身,先此告過。」送二人到門首,看着上馬而去。正是:

\begin{myquote}
博得錦衣歸故里,功名方信是男兒。
\end{myquote}

