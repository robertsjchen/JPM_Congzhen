\part*{{\titlename}卷之五}
\addcontentsline{toc}{part}{{\titlename}卷之五}


\includepdf[pages={81,82},fitpaper=false]{tst.pdf}
\chapter*{第四十一回 兩孩兒聯姻共笑嬉 二佳人憤深同氣苦}
\addcontentsline{toc}{chapter}{第四十一回 兩孩兒聯姻共笑嬉 二佳人憤深同氣苦}
\markboth{{\titlename}卷之五}{第四十一回 兩孩兒聯姻共笑嬉 二佳人憤深同氣苦}


詞曰:

\begin{myquote}
瀟灑佳人,風流才子,天然分付成雙。蘭堂綺席,燭影耀熒煌。數幅紅羅,錦繡寶粧。篆金鴨焚香。分明是、芙蕖浪裡,一對鴛鴦。

\raggedleft{——右調《滿庭芳前》\rightquadmargin}
\end{myquote}

話說西門慶在家中,裁縫攢造衣服,那消兩日就完了。到十二日,喬家使人邀請。早晨,西門慶先送了禮去。那日,月娘並衆姊妹、大妗子,六頂轎子一搭兒起身。留下孫雪娥看家。奶子如意兒抱着官哥,又令來興媳婦蕙秀伏侍疊衣服,又是兩頂小轎。

西門慶在家,看着賁四叫了花兒匠來紮縛烟火,在大廳、捲棚內掛燈,使小厮拿帖兒往王皇親宅內定下戲子,俱不必細說。後晌時分,走到金蓮房中。金蓮不在家,春梅在旁伏侍茶飯,放桌兒吃酒。西門慶因對春梅說:「十四日請衆官娘子,你們四箇都打扮出去,與你娘跟着遞酒,也是好處。」春梅聽了,斜靠着桌兒說道:「你若叫,只叫他三箇出去,我是不出去。」西門慶道:「你怎的不出去?」春梅道:「娘們都新做了衣裳,陪侍衆官戶娘子便好看。俺們一箇一箇只像燒糊了卷子一般,平白出去惹人家笑話。」西門慶道:「你們都有各人的衣服首飾、珠翠花朵。」春梅道:「頭上將就戴着罷了,身上有數那兩件舊片子,怎麼好穿出去見人的!到沒的羞剌剌的。」西門慶笑道:「我曉的你這小油嘴兒,見你娘們做了衣裳,卻使性兒起來。不打緊,叫趙裁來,連大姐帶你四箇,每人都裁三件:一套段子衣裳、一件遍地錦比甲。」春梅道:「我不比與他。我還問你要件白綾襖兒,搭襯着大紅遍地錦比甲兒穿。」{\meipi{春梅意見往往高人一頭,可見人品成於居,養者其後,而立志貴早。}}西門慶道:「你要不打緊,少不的也與你大姐裁一件。」春梅道:「大姑娘有一件罷了,我卻沒有,他也說不的。」西門慶于是拿鑰匙開樓門,揀了五套段子衣服、兩套遍地錦比甲兒,一疋白綾裁了兩件白綾對衿襖兒。惟大姐和春梅是大紅遍地錦比甲兒,迎春、玉簫、蘭香,都是藍綠顏色;衣服都是大紅段子織金對衿襖,翠藍邊拖裙,共十七件。一面叫了趙裁來,都裁剪停當。又要一疋黃紗做裙腰,貼裡一色都是杭州絹兒。春梅方纔喜歡了,陪侍西門慶在屋裡吃了一日酒,說笑頑耍不題。

且說吳月娘衆妹妹到了喬大戶家。原來喬大戶娘子那日請了尚舉人娘子,並左隣朱臺官娘子、崔親家母,並兩箇外甥姪女兒——段大姐及吳舜臣媳婦兒鄭三姐。叫了兩箇妓女,席前彈唱。聽見月娘衆姊妹和吳大妗子到了,連忙出儀門首迎接,後廳叙禮。趕着月娘呼姑娘,李嬌兒衆人都排行叫二姑娘、三姑娘,俱依吳大妗子那邊稱呼之禮。又與尚舉人、朱臺官娘子叙禮畢,段大姐、鄭三姐向前拜見了。各依次坐下。丫環遞過了茶,喬大戶出來拜見,謝了禮。他娘子讓進衆人房中去寬衣服,就放桌兒擺茶,請衆堂客坐下吃茶。奶子如意兒和蕙秀在房中看官哥兒,另自管待。須臾,吃了茶到廳,屏開孔雀,褥隱芙蓉,正面設四張桌席。讓月娘坐了首位,其次就是尚舉人娘子、吳大妗子、朱臺官娘子、李嬌兒、孟玉樓、潘金蓮、李瓶兒;喬大戶娘子關席坐位。旁邊放一桌,是段大姐、鄭三姐,共十一位。兩箇妓女在旁邊唱。上了湯飯,廚役上來獻了頭一道水晶鵝,月娘賞了二錢銀子;第二道是頓爛よ蹄兒,月娘又賞了一錢銀子;第三道獻燒鴨,月娘又賞了一錢銀子。

喬大戶娘子下來遞酒,遞了月娘過去,又遞尚舉人娘子。月娘就下來往後房換衣服、勻臉去了。孟玉樓也跟下來,到了喬大戶娘子臥房中,只見奶子如意兒看守着官哥兒,在炕上鋪着小褥子兒躺着。他家新生的長姐,也在旁邊臥着。兩箇你打我下兒,我打你下兒頑耍。{\meipi{其心歡喜。}}把月娘、玉樓見了,喜歡的要不得,說道:「他兩箇倒好相兩口兒。」只見吳大妗子進來,說道:「大妗子,你來瞧瞧,兩箇倒相小倆口兒。」大妗子笑道:「正是。孩兒每在炕上,張手蹬脚兒的,你打我,我打你,小姻緣一對兒耍子。」喬大戶娘子和衆堂客都進房到。吳大妗子如此這般說。喬大戶娘子道:「列位親家聽着,小家兒人家,怎敢攀的我這大姑娘府上?」月娘道:「親家好說,我家嫂子是何人?鄭三姐是何人?我與你愛親做親,就是我家小兒也玷辱不了你家小姐,如何卻說此話?」玉樓推着李瓶兒說道:「李大姐,你怎的說?」那李瓶兒只是笑。{\meipi{玉樓自韻,瓶兒自媚,金蓮獨不在耶?何不出人言也。}}吳妗子道:「喬親家不依,我就惱了。」

尚舉人娘子和朱臺官娘子皆說道:「難為吳親家厚情,喬親家你休謙辭了。」因問:「你家長姐去年十一月生的?」月娘道:「我家小兒六月廿三日生的,原大五箇月,正是兩口兒。」衆人不繇分說,把喬大戶娘子和月娘、李瓶兒拉到前廳,兩箇就割了衫襟。兩箇妓女彈唱着。旋對喬大戶說了,拿出菓盒、三段紅來遞酒。月娘一面分付玳安、琴童快往家中對西門慶說。旋擡了兩罈酒、三疋段子、紅綠板兒絨金絲花、四箇螺甸大菓盒。兩家席前,掛紅吃酒。一面堂中畫燭高擎,花燈燦爛,麝香靉靉,喜笑匆匆。兩箇妓女,啟朱唇,露皓齒,輕撥玉阮,斜抱琵琶唱着。

衆堂客與吳月娘、喬大戶娘子、李瓶兒三人都簪了花,掛了紅,遞了酒,各人都拜了。從新復安席坐人飲酒。廚子上了一道裹餡壽字雪花糕、喜重重滿池嬌並頭蓮湯。月娘坐在上席,滿心歡喜,叫玳安過來,賞一疋大紅與廚役。兩箇妓女每人都是一疋。俱磕頭謝了。喬大戶娘子不放起身,還在後堂留坐,擺了許多勸碟,細菓攢盒。約吃到一更時分,月娘等方纔拜辭回來,說道:「親家,明日好歹下降寒舍那裡坐坐。」喬大戶娘子道:「親家盛情,家老兒說來,只怕席間不好坐的,改日望親家去罷。」月娘道:「好親家,再沒人。親家只是見外。」因留了大妗子:「你今日不去,明日同喬親家一搭兒裡來罷。」大妗子道:「喬親家,別的日子你不去罷,到十五日,你正親家生日,你莫不也不去?」喬大戶娘子道:「親家十五日好日子,我怎敢不去!」月娘道:「親家若不去,大妗子,我交付與你,只在你身上。」于是,生死把大妗子留下了,然後作辭上轎。頭裡兩箇排軍,打着兩箇大紅燈籠;後邊又是兩箇小厮,打着兩箇燈籠。吳月娘在頭裡,李嬌兒、孟玉樓、潘金蓮、李瓶兒一字在中間,如意兒和蕙秀隨後。奶子轎子裡用紅綾小被把官哥兒裹得沿沿的,恐怕冷,脚下還蹬着銅火爐兒。兩邊小厮圜隨。到了家門首下轎,西門慶正在上房吃酒,月娘等衆人進來,道了萬福,坐下。衆丫鬟都來磕了頭。月娘先把今日酒席上結親之話,告訴了一遍。西門慶聽了道:「今日酒席上有那幾位堂客?」月娘道:「有尚舉人娘子、朱序班娘子、崔親家母、兩箇姪女。」西門慶說:「做親也罷了,只是有些不搬陪。」月娘道:「倒是俺嫂子,見他家新養的長姐和咱孩子在床炕上睡着,都蓋着那被窩兒,你打我一下兒,我打你一下兒,恰是小倆口兒一般,纔叫了俺們去,說將起來,酒席上就不因不由做了這門親。我方纔使小厮來對你說,擡送了花紅菓盒去。」西門慶道:「既做親也罷了,只是有些不搬陪些。喬家雖有這箇家事,他只是箇縣中大戶白衣人。你我如今見居着這官,又在衙門中管着事,到明日會親酒席間,他戴着小帽,與俺這官戶怎相處?甚不雅相。就是前日,荊南岡央及營裡張親家,再三趕着和我做親,說他家小姐今纔五箇月兒,也和咱家孩子同歲。我嫌他沒娘母子,是房裡生的,所以沒曾應承他。不想到與他家做了親。」潘金蓮在旁接過來道:「嫌人家是房裡養的,誰家是房外養的?就是喬家這孩子,也是房裡生的。正是『險道神撞着壽星老兒——你也休說我長,我也休嫌你短』。」西門慶聽了此言,心中大怒,罵道:「賊淫婦,還不過去!人這裡說話,也插嘴插舌的。有你甚麼說處!」金蓮把臉羞的通紅了,抽身走出來,說道:「誰說這裡有我說處?可知我沒說處哩!」{\meipi{金蓮揖變,恆以取愛尖利,亦以招尤,讀此可惕然於三縬之銘。}}看官聽說:今日潘金蓮在酒席上,見月娘與喬大戶家做了親,李瓶兒都披紅簪花遞酒,心中甚是氣不憤,來家又被西門慶罵了這兩句,越發急了,走到月娘這邊屋裡哭去了。西門慶因問:「大妗子怎的不來?」月娘道:「喬親家母明日見有衆官娘子,說不得來。我留下他在那裡,教明日同他一搭兒裡來。」西門慶道:「我說只這席間坐次上不好相處,到明日怎麼厮會?」說了回話,只見孟玉樓也走到這邊屋裡來,見金蓮哭泣,說道:「你只顧惱怎的?隨他說幾句罷了。」金蓮道:「早是你在旁邊聽着,我說他什麼歹話來?他說別家是房裡養的,我說喬家是房外養的?也是房裡生的。那箇紙包兒包着,瞞得過人?賊不逢好死的強人,就睜着眼罵起我來。罵的人那絕情絕義。怎的沒我說處?改變了心,教他明日現報在我的眼裡!多大的孩子,一箇懷抱的尿泡種子,平白扳親家,有錢沒處施展的,爭破臥單沒的蓋,『狗咬尿胞——空歡喜』!如今做濕親家還好,到明日休要做了乾親家纔難。『吹殺燈擠眼兒——後來的事看不見』。做親時人家好,過三年五載方了的纔一箇兒!」玉樓道:「如今人也賊了,不幹這箇營生。論起來也還早哩。纔養的孩子,割甚麼衫襟?無過只是圖往來扳陪着耍子兒罷了。」{\meipi{既然曉的,又何必怒。}}金蓮道:「你便浪𢵞着圖扳親家耍子,平白教賊不合鈕的強人罵我。」玉樓道:「誰教你說話不着箇頭項兒就說出來?他不罵你罵狗?」金蓮道:「我不好說的,他不是房裡,是大老婆?就是喬家孩子,是房裡生的,還有喬老頭子的些氣兒。你家失迷家鄉,還不知是誰家的種兒哩!」玉樓聽了,一聲兒沒言語。坐了一回,金蓮歸房去了。

李瓶兒見西門慶出來了,從新花枝招颭與月娘磕頭,說道:「今日孩子的事,累姐姐費心。」{\meipi{一到瓶兒開口,不使人愛,便使人憐。}}那月娘笑嘻嘻,也倒身還下禮去,說道:「你喜呀!」李瓶兒道:「與姐姐同喜。」磕畢頭起來,與月娘、李嬌兒坐着說話。只見孫雪娥、大姐來與月娘磕頭,與李嬌兒、李瓶兒道了萬福。小玉拿茶來,正吃茶,只見李瓶兒房裡丫鬟綉春來請,說:「哥兒屋裡尋哩,爹使我請娘來了。」李瓶兒道:「奶子慌的三不知就抱的屋裡去了。一搭兒去也罷了,只怕孩子沒箇燈兒。」月娘道:「頭裡進門,到是我叫他抱的房裡去。恐怕晚了。」小玉道:「頭裡如意兒抱着他,來安兒打着燈籠送他來。」李瓶兒道:「這等也罷了。」于是,作辭月娘,回房中來。只見西門慶在屋裡,官哥兒在奶子懷裡睡着了。因說:「你如何不對我說就抱了他來?」如意兒道:「大娘見來安兒打着燈籠,就趁着燈兒來了。哥哥哭了一口,纔拍着他睡着了。」西門慶道:「他尋了這一回,纔睡了。」李瓶兒說畢,望着他笑嘻嘻說道:「今日與孩兒定了親,累你,我替你磕箇頭兒。」于是,插燭也似磕下去。喜歡的西門慶滿面堆笑,連忙拉起來,做一處坐的。一面令迎春擺下酒兒,兩箇吃酒。

且說潘金蓮到房中使性子,沒好氣,明知道西門慶在李瓶兒這邊,因秋菊開的門遲了,進門就打了兩箇耳刮子,高聲罵道:「賊淫婦奴才!怎的叫了恁一日不開?你做甚麼來?我且不和你答話。」于是走到屋裡坐下。春梅走來磕頭遞茶。婦人問他:「賊奴才他在屋裡做什麼來?」春梅道:「在院子裡坐着來。我這等催他,還不理。」婦人道:「我知道他和我兩箇慪氣。黨太尉吃匾食,他也學人照樣兒欺負我。」待要打他,又恐西門慶聽見;不言語,心中又氣。{\meipi{尖嘴人常受此氣,余亦多坐此病。}}一面卸了濃粧,春梅與他搭了鋪,上床就睡了。

到次日,西門慶衙門中去了。婦人把秋菊叫他頂着大塊柱石,跪在院子裡。跪的他梳了頭,叫春梅扯了他褲子,拿大板子要打他。春梅道:「好乾淨的奴才,叫我扯褲子,到沒的污濁了我的手!」走到前邊,旋叫了畫童兒扯去秋菊的衣。婦人打着他罵道:「賊奴才淫婦,你從幾時就恁大來?別人興你,我卻不興你。姐姐,你知我見的,將就膿着些兒罷了。平白撐着頭兒,逞什麼強?姐姐,你休要倚着,我到明日洗着兩箇眼兒看着你哩!」{\meipi{沒些把控。}}一面罵着又打,打了又罵,打的秋菊殺豬也似叫。李瓶兒那邊纔起來,正看着奶子打發官哥兒睡着了,又諕醒了。明明白白聽見金蓮這邊打丫鬟,罵的言語兒有因,一聲兒不言語,諕的只把官哥兒耳朵握着。一面使綉春:「去對你五娘說休打秋菊罷。哥兒纔吃了些奶睡着了。」金蓮聽了,越發打的秋菊狠了,罵道:「賊奴才,你身上打着一萬把刀子,這等叫饒。我是恁性兒,你越叫,我越打。{\meipi{這樣可恨。}}莫不為你拉斷了路行人?人家打丫頭,也來看着。你好姐姐對漢子說,把我別變了罷!」李瓶兒這邊分明聽見指罵的是他,把兩隻手氣的冰冷,忍氣吞聲,敢怒而不敢言。早晨茶水也沒吃,摟着官哥兒在炕上就睡着了。等到西門慶衙門中回家,入房來看官哥兒,見李瓶兒哭的眼紅紅的,睡在炕上,問道:「你怎的這咱還不梳頭?上房請你說話。你怎揉的眼恁紅紅的?」李瓶兒也不題金蓮指罵之事,只說:「我心中不自在。」西門慶告說:「喬親家那裡,送你的生日禮來了。一疋尺頭、兩罈南酒、一盤壽桃、一盤壽麵、四樣下飯。又是哥兒送節的兩盤元宵、四盤蜜食、四盤細菓、兩掛珠子弔燈、兩座羊皮屏風燈、兩疋大紅官段、一頂青段㩟的金八吉祥帽兒、兩雙男鞋、六雙女鞋。咱家倒還沒往他那裡去,他又早與咱孩兒送節來了。如今上房的請你計較去。他那裡使了箇孔嫂兒和喬通押了禮來。大妗子先來了,說明日喬親家母不得來,直到後日纔來。他家有一門子做皇親的喬五太太,聽見和咱們做親,好不喜歡!到十五日,也要來走走,咱少不得補箇帖兒請去。」李瓶兒聽了,方慢慢起來梳頭,走了後邊,拜了大妗子。孔嫂兒正在月娘房裡待茶,禮物擺在明間內,都看了。一面打發回盒起身,與了孔嫂兒、喬通每人兩方手帕,五錢銀子,寫了回帖去了。正是:但將鐘鼓悅和愛,好把犬羊為國羞。有詩為證:

\begin{myquote}
西門獨富太驕矜,襁褓孩兒結做親。\\不獨資財如糞土,也應嗟嘆後來人。
\end{myquote}
