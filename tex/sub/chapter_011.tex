\part*{{\titlename}卷之二}
\addcontentsline{toc}{part}{{\titlename}卷之二}


\includepdf[pages={21,22},fitpaper=false]{tst.pdf}
\chapter*{第十一回 潘金蓮激打孫雪娥 西門慶梳籠李桂姐}
\addcontentsline{toc}{chapter}{第十一回 潘金蓮激打孫雪娥 西門慶梳籠李桂姐}
\markboth{{\titlename}卷之二}{第十一回 潘金蓮激打孫雪娥 西門慶梳籠李桂姐}


詩曰:

\begin{myquote}
六街簫鼓正喧闐,初月今朝一線添。\\睡去烏衣驚玉剪,鬬來宵燭渾朱簾。\\香綃染處紅餘白,翠黛攢來苦未甜。\\阿姐當年曾似此,從他戲汝不須嫌。
\end{myquote}

話說潘金蓮在家恃寵生驕,顛寒作熱,鎭日夜不得箇寧靜。性極多疑,專一聽籬察壁。那箇春梅,又不是十分耐煩的。一日,金蓮為些零碎事情不湊巧,罵了春梅幾句。春梅沒處出氣,走往後邊廚房下去,槌檯拍櫈鬧狠狠的模樣。那孫雪娥看不過,假意戲他道:「怪行貨子!想漢子便別處去想,怎的在這裡硬氣?」{\pangpi{禍從此一戲罵起。}}春梅正在悶時,聽了這句,不一時暴跳起來:「那箇歪斯纏我哄漢子?」雪娥見他性不順,只做不聽得。春梅便使性做幾步走到前邊來,一五一十,又添些話頭,道:「他還說娘教爹收了我,俏一幫兒哄漢子。」挑撥與金蓮知道。金蓮滿肚子不快活。因送吳月娘出去送殯,起身早些,有些身子倦,睡了一覺,走到亭子上。只見孟玉樓搖颭的走來,{\pangpi{沒心人多少快活。}}笑嘻嘻道:「姐姐如何悶悶的不言語?」金蓮道:「不要說起,今早倦的了不得。三姐你在那裡去來?」玉樓道:「纔到後面廚房裡走了走來。」金蓮道:「他與你說些甚麼來?」玉樓道:「姐姐沒言語。」金蓮心雖懷恨,口裡卻不說出。兩箇做了一回針指。只見春梅拿茶來,吃畢,兩箇悶倦,就放桌兒下棋耍子。忽見看園門小厮琴童走來,{\pangpi{伏。}}報道:「爹來了。」慌的兩箇婦人收棋子不迭。西門慶恰進門檻,看見二人家常都帶着銀絲鬏髻,露着四鬢,耳邊青寶石墜子,白紗衫兒,銀紅比甲,挑線裙子,雙灣尖趫,紅鴛瘦小,一箇箇粉粧玉琢,不覺滿面堆笑,戲道:「好似一對兒粉頭,也値百十兩銀子!」潘金蓮說道:「俺們倒不是粉頭,你家正有粉頭在後邊哩!」{\pangpi{劈空插入,尖甚。}}那玉樓抽身就往後走,被西門慶一手拉住,說道:「你往那裡去?我來了,你倒要脫身去了。寔說,我不在家,你兩箇在這裡做甚麼?」金蓮道:「俺倆箇悶的慌,在這裡下了兩盤棋時,沒做賊,誰知道你就來了。」一面替他接了衣服,說道:「你今日送殯來家早。」西門慶道:「今日齋堂裡都是內相同官,天氣又熱,我不耐煩,先來家。」玉樓問道:「他大娘怎的還不來?」西門慶道:「他的轎子也待進城,我先回,使兩箇小厮接去了。」一面坐下。因問:「你兩箇下棋賭些甚麼?」金蓮道:「俺兩箇自下一盤耍子,平白賭什麼?」西門慶道:「等我和你們下一盤,那箇輸了,拿出一兩銀子做東道。」金蓮道:「俺們沒銀子。」西門慶道:「你沒銀子,拿簪子問我當,也是一般。」於是擺下棋子,三人下了一盤。潘金蓮輸了。{\pangpi{輸得妙。}}西門慶纔數子兒,被婦人把棋子撲撒亂了。一直走到瑞香花下,倚着湖山,推掐花兒。西門慶尋到那裡,說道:「好小油嘴兒!你輸了棋子,卻躲在這裡。」那婦人見西門慶來,睨笑不止,說道:「怪行貨子!孟三兒輸了,你不敢禁他,卻來纏我!」將手中花撮成瓣兒,洒西門慶一身。{\meipi{金蓮撤嬌弄癡,事事俱堪入畫,每閱一過,輒令人消魂半晌。}}被西門慶走向前,雙關抱住,按在湖山畔,就口吐丁香,舌融甜唾,戲謔做一處。不防玉樓走到根前,叫道:「六姐,他大娘來家了。咱後邊去來。」這婦人撇了西門慶,說道:「哥兒,我回來和你答話。」{\pangpi{藕斷絲連。}}遂同玉樓到後邊,與月娘道了萬福。月娘問:「你們笑甚麼?」玉樓道:「六姐今日和他爹下棋,輸了一兩銀子,到明日整治東道,請姐姐耍子。」月娘笑了。金蓮只在月娘面前打了箇照面兒,就走來前邊陪伴西門慶。分付春梅房中薰香,預備澡盆浴湯,準備晚間效魚水之歡。

看官聽說:家中雖是吳月娘居大,常有疾病,不管家事。只是人情來往,出入銀錢,都在李嬌兒手裡。孫雪娥單管率領家人媳婦,在廚中上竈,打發各房飲食。譬如西門慶在那房裡宿歇,或吃酒,或吃飯,造甚湯水,俱經雪娥手中整理,那房裡丫頭自往廚下去拿。此不必說。當晚西門慶在金蓮房中,吃了回酒,洗畢澡,兩人歇了。次日,也是合當有事。西門慶許下金蓮,要往廟上替他買珠子穿箍兒戴。早起來,等着要吃荷花餅、銀絲鮓湯,{\pangpi{好名色。}}使春梅往廚下說去。那春梅只顧不動身。金蓮道:「你休使他。{\meipi{一唱一和,都妙。}}有人說我縱容他,教你收了,俏成一幫兒哄漢子。百般指豬罵狗,欺負俺娘兒們。你又使他後邊做甚麼去?」西門慶便問:「是誰說的?你對我說。」婦人道:「說怎的!盆礶都有耳朵,{\pangpi{不便說出,更妙。}}你只不叫他後邊去,另使秋菊去便了。」這西門慶遂叫過秋菊,分付他往廚下對雪娥說去。約有兩頓飯時,婦人已是把桌兒放了,{\pangpi{偏快。}}白不見拿來。急的西門慶只是暴跳。婦人見秋菊不來,使春梅:「你去後邊瞧瞧那奴才,只顧生根長苗的不見來。」春梅有幾分不順,使性子走到廚下。只見秋菊正在那裡等着哩,便罵道:「賊奴才,娘要卸你那腿哩!說你怎的就不去了。爹等着吃了餅,要往廟上去。急的爹在前邊暴跳,叫我採了你去哩!」這孫雪娥不聽便罷,聽了心中大怒,罵道:「怪小淫婦兒!『馬回子拜節——來到的就是』。鍋兒是鐵打的,也等慢慢兒的來,預備下熬的粥兒又不吃,忽剌八新興出來要烙餅做湯。那箇是肚裡蛔蟲!」{\meipi{雪娥殊不自揣。}}春梅不忿他罵,說道:「沒的扯𣭈淡!{\pangpi{妙在不卑不亢。}}主子不使了來,那箇好來問你要。有與沒,俺們到前邊只說的一聲兒,有那些聲氣的?」一隻手擰着秋菊的耳朵,一直往前邊來。雪娥道:「主子奴才,常遠似這等硬氣,有時道着!」春梅道:「有時道沒時道,沒的把俺娘兒兩箇別變了罷!」於是氣狠狠走來。婦人見他臉氣得黃黃的,拉着秋菊進門,便問:「怎的來了?」春梅道:「你問他。我去時還在廚房裡雌着,等他慢條厮禮兒纔和麵兒。我自不是,說了一句『爹在前邊等着,娘說你怎的就不去了?』倒被那小院兒裡的{\pangpi{輕嘴。}}千奴才、萬奴才罵了我恁一頓。說爹『馬回子拜節——走到的就是』,只象那箇調唆了爹一般,預備下粥兒不吃,平白地生發起要甚餅和湯。只顧在廚房裡罵人,不肯做哩。」婦人在旁便道:「我說別要使他去,{\pangpi{挑撥,冷。}}人自恁和他合氣。說俺娘兒兩箇𢺞攔你在這屋裡,只當吃人罵將來。」

西門慶聽了大怒,走到後邊廚房裡,不由分說,向雪娥踢了幾脚,罵道:「賊𢱉剌骨!我使他來要餅,你如何罵他?你罵他奴才,你如何不溺胞尿把自己照照!」{\pangpi{罵得毒。}}雪娥被西門慶踢罵了一頓,敢怒而不敢言。西門慶剛走出廚房外,孫雪娥對着來昭妻一丈青說道:「你看,我今日晦氣!早是你在旁聽,我又沒曾說什麼。他走將來兇神似一般,大喓小喝,把丫頭採的去了,反對主子面前輕事重報,惹的走來平白地恁一場兒。我洗着眼兒,看着主子奴才長遠恁硬氣着,只休要錯了脚兒!」不想被西門慶聽見了,復回來又打了幾拳,{\meipi{往往反覆播弄。}}罵道:「賊奴才淫婦!你還說不欺負他,親耳朵聽見你還罵他。」打的雪娥疼痛難忍,西門慶便往前邊去了。那雪娥氣的在廚房裡兩淚悲流,放聲大哭。

吳月娘正在上房,纔起來梳頭,因問小玉:「廚房裡亂些甚麼?」小玉回道:「爹要餅吃了往廟上去,說姑娘罵五娘房裡春梅來,被爹聽見了,踢了姑娘幾脚,哭起來。」月娘道:「也沒見他,要餅吃連忙做了與他去就罷了,平白又罵他房裡丫頭怎的!」於是使小玉走到廚房,攛掇雪娥和家人媳婦忙造湯水,打發西門慶吃了,往廟上去,不題。

這雪娥氣憤不過,正走到月娘房裡告訴此事。不妨金蓮驀然走來,立於窻下潛聽。見雪娥在房裡對月娘、李嬌兒說他怎的𢺞攔漢子,背地無所不為:「娘,你還不知淫婦,說起來比養漢老婆還浪,一夜沒漢子也不成的。背地幹的那繭兒,人幹不出,他幹出來。{\meipi{雖仇口,卻句句是金蓮實錄。}}當初在家,把親漢子用毒藥擺死了,跟了來。如今把俺們也吃他活埋了。弄的漢子烏眼雞一般,見了俺們便不待見。」月娘道:「也沒見你,他前邊使了丫頭要餅,你好好打發與他去便了。平白又罵他怎的?」孫雪娥道:「我罵他禿也瞎也來?那頃,這丫頭在娘房裡,着緊不聽手,俺沒曾在竈上把刀背打他,{\pangpi{此失時語。}}娘尚且不言語。可哥今日輪到他手裡,便驕貴的這等的了。」正說着,只見小玉走到,說:「五娘在外邊。」{\meipi{小玉又先說一聲,偏在忙中搖擺。}}少傾,金蓮進房,望着雪娥說道:「比如我當初擺死親夫,你就不消叫漢子娶我來家,省得我𢺞攔着他,撐了你的窩兒。{\pangpi{開口絕無一蔓,又突又冷。}}論起春梅,又不是我的丫頭,你氣不憤,還教他伏侍大娘就是了。省得你和他合氣,把我扯在裡頭。那箇好意死了漢子嫁人?{\pangpi{此句難說。}}如今也不難的勾當,等他來家,與我一紙休書,我去就是了。」月娘道:「我也不曉的你們底事。你們大家省言一句兒便了。」孫雪娥道:「娘,你看他嘴似淮洪也一般,隨問誰也辯他不過。{\meipi{呆人沒得說,往往以此二字語扯白。}}明在漢子根前戳舌兒,轉過眼就不認了。依你說起來,除了娘,把俺們都攆,只留着你罷!」那吳月娘坐着,由着他兩箇你一句我一句,只不言語。後來見罵起來,雪娥道:「你罵我奴才!你便是眞奴才!」險些兒不曾打起來。月娘看不上,使小玉把雪娥拉往後邊去。這潘金蓮一直歸到前邊,卸了濃粧,洗了脂粉,烏雲散亂,花容不整,哭得兩眼如桃,躺在床上。{\pangpi{婦人慣用此技。}}到日西時分,西門慶廟上來,袖着四兩珠子,進入房中,一見便問:「怎的來?」婦人放聲号哭起來,問西門慶要休書。如此這般告訴一遍:「我當初又不曾圖你錢財,自恁跟了你來。如何今日教人這等欺負?千也說我擺殺漢子,萬也說我擺殺漢子!沒丫頭便罷了,如何要人房裡丫頭伏侍?吃人指罵!」這西門慶不聽便罷,聽了時,三屍神暴跳,五臟氣冲天。一陣風走到後邊,採過雪娥頭髮來,盡力{\pangpi{二字惡。}}拿短棍打了幾下。多虧吳月娘向前拉住了,說道:「沒的大家省些事兒罷了!{\pangpi{二字公道。}}好交你主子惹氣!」西門慶便道:「好賊𢱉剌骨,我親自聽見你在廚房裡罵,你還攪纏別人。我不把你下截打下來也不算。」看官聽說:不爭今日打了孫雪娥,管教潘金蓮從前作過事,沒興一齊來。正是:

\begin{myquote}
惟有感恩並積恨,萬年千載不生塵。
\end{myquote}

當下西門慶打了雪娥,走到前邊,窩盤住了金蓮,袖中取出廟上買的四兩珠子,遞與他。婦人見漢子與他做主,出了氣,如何不喜。繇是要一奉十,寵愛愈深。

話休饒舌,一日正輪該花子虛家擺酒會茶,這花家就在西門慶緊隔壁。內官家擺酒,甚是豐盛。衆兄弟都到了。因西門慶有事,約午後纔來,都等他,不肯先坐。少頃,西門慶來到,然後叙禮讓坐,東家安西門慶居首席。兩箇妓女,琵琶箏ぬ在席前彈唱。端的說不盡梨園嬌艷,色藝雙全。但見:

\begin{myquote}
羅衣疊雪,寶髻堆雲。櫻桃口,杏臉桃腮;楊柳腰,蘭心蕙性。歌喉宛轉,聲如枝上流鶯;舞態蹁躚,影似花間鳳轉。腔依古調,音出天然。舞回明月墜秦樓,歌遏行雲遮楚館。高低緊慢按宮商,輕重疾徐依格調,箏排雁柱聲聲慢,板拍紅牙字字新。
\end{myquote}

少頃,酒過三巡,歌吟兩套,兩箇唱的放下樂器,向前花枝招颭般來磕頭。西門慶呼玳安書袋內取兩封賞賜,每人二錢,拜謝了下去。因問東家花子虛道:「這位姐兒上姓?端的會唱。」東家未及答應,應伯爵插口道:「大官人多忘事,就不認的了?這ち箏的是花二哥令翠——抅欄後巷吳銀兒。這彈琵琶的,就是我前日說的李三媽的女兒、李桂卿的妹子,小名叫做桂姐。你家中見放着他親姑娘。如何推不認的?」西門慶笑道:「元來就是他,我六年不見,不想就出落得恁般成人了!」{\pangpi{便有意。}}落後酒闌,上席來遞酒。這桂姐殷勤勸酒,情話盤桓。西門慶因問:「你三媽與姐姐桂卿,在家做甚麼?怎的不來我家看看你姑娘?」桂姐道:「俺媽從去歲不好了一場,至今腿脚半邊通動不的,只扶着人走。俺姐姐桂卿被淮上一箇客人包了半年,常接到店裡住,兩三日不放來家。家中好不無人,只靠着我逐日出來供唱,好不辛苦!時常也想着要往宅裡看看姑娘,白不得箇閑。爹許久怎的也不來裡邊走走?幾時放姑娘家去看看俺媽也好。」西門慶見他一團和氣,說話兒乖覺伶變,就有幾分留戀之意,說道:「我今日約兩位好朋友送你家去。你意下何如?」桂姐道:「爹休哄我。你肯貴人脚兒踏俺賤地?」{\pangpi{老到。}}西門慶道:「我不哄你。」便向袖中取出汗巾,連挑牙與香茶盒兒,遞與桂姐收了。桂姐道:「多咱去?如今使保兒先家去說一聲,作箇預備。」西門慶道:「直待人散,一同起身。」少頃,遞畢酒,約掌燈人散時分,西門慶約下應伯爵、謝希大,也不到家,騎馬同送桂姐,逕進抅欄往李家去。正是:

\begin{myquote}
陷人坑,土窖般暗開掘;迷魂洞,囚牢般巧砌疊;撿屍場,屠舖般明排列。整一味死溫存活打劫。招牌兒大字書者:買俏金,哥哥休撦;纏頭錦,婆婆自接;賣花錢,姐姐不賒。
\end{myquote}

西門慶等送桂姐轎子到門首,李桂卿迎門接入堂中。見畢禮數,請老媽出來拜見。不一時,虔婆扶拐而出,半邊肐膊都動彈不得,見了西門慶,道了萬福。說道:「天麼,天麼!姐夫貴人,那陣風兒颳得你到這裡?」西門慶笑道:「一向窮冗,沒曾來得,老媽休怪。」虔婆又嚮應、謝二人說道:「二位怎的也不來走走?」伯爵道:「便是白不得閑,今日在花家會茶,遇見桂姐,因此同西門爹送回來。快看酒來,俺們樂飲三盃。」虔婆讓三位上首坐了。一面點茶,一面打抹春臺,收拾酒菜。少頃,掌上燈燭,酒餚羅列。桂姐從新房中打扮出來,{\pangpi{細。}}旁邊陪坐,免不得姐妹兩箇金樽滿泛,玉阮同調,歌唱遞酒。正是:

\begin{myquote}
琉璃鍾,琥珀濃。小槽酒滴珍珠紅。烹龍炮鳳玉脂泣,羅幃繡幄圍香風。吹龍笛,擊鼉鼓。皓齒歌,細腰舞。況是青春莫虛度,銀缸掩映嬌娥語,不到劉伶墳上去。
\end{myquote}

當下姊妹兩箇唱了一套,席上觥籌交錯飲酒。西門慶向桂卿道:「今日二位在此,久聞桂姐善能歌南曲,何不請歌一詞,奉勸二位一盃兒酒!」應伯爵道:「我又不當起動,借大官人餘光,洗耳願聽佳音。」那桂姐坐着只是笑,半晌不動身。原來西門慶有心要梳籠桂姐,故先索落他唱。那院中婆娘見識精明,早已看破了八九分。桂卿在旁,就先開口說道:「我家桂姐從小兒養得嬌,自來生得腼腆,不肯對人胡亂便唱。」於是西門慶便叫玳安書袋內取出五兩一錠銀子來,放在桌上,說道:「這些不當甚麼,權與桂姐為脂粉之需,改日另送幾套織金衣服。」桂姐連忙起身謝了。{\pangpi{與坐着句相應。}}先令丫鬟收去,方纔下席來唱。這桂姐雖年紀不多,卻色藝過人,當下不慌不忙,輕扶羅袖,擺動湘裙,袖口邊搭剌着一方銀紅撮穗的落花流水汗巾兒,{\pangpi{想宋時北妓如此。}}歌唱道:

\begin{myquote}
{\markfont\small〔駐雲飛〕}舉止從容,壓盡抅欄占上風。行動香風送,頻使人欽重。嗏!玉杵污泥中,豈凡庸?一曲宮商,滿座皆驚動。勝似襄王一夢中,勝似襄王一夢中。
\end{myquote}

唱畢,把箇西門慶喜歡的沒入脚處。分付玳安回馬家去,晚夕就在李桂卿房裡歇了一宿。緊着西門慶要梳籠這女子,又被應伯爵、謝希大兩箇一力攛掇,就上了道兒。次日,使小厮往家去拿五十兩銀子,段舖內討四件衣裳,要梳籠桂姐。那李嬌兒聽見要梳籠他的姪女兒,{\pangpi{映帶。}}如何不喜?連忙拿了一錠大元寶付與玳安,拿到院中打頭面,做衣服,定桌席,吹彈歌舞,花攢錦簇,飲三日喜酒。應伯爵、謝希大又約會了孫寡嘴、祝實念、常峙節,每人出五分分子,都來賀他。鋪的蓋的都是西門慶出。每日大酒大肉,在院中頑耍,不在話下。

\begin{myquote}
舞裙歌板逐時新,散盡黃金只此身。\\寄與富兒休暴殄,儉如良藥可醫貧。
\end{myquote}

