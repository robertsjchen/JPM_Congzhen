\includepdf[pages={51,52},fitpaper=false]{tst.pdf}
\chapter*{第二十六回 來旺兒遞解徐州 宋蕙蓮含羞自縊}
\addcontentsline{toc}{chapter}{第二十六回 來旺兒遞解徐州 宋蕙蓮含羞自縊}
\markboth{{\titlename}卷之三}{第二十六回 來旺兒遞解徐州 宋蕙蓮含羞自縊}


詩曰:

\begin{myquote}
與君形影分吳越,玉枕經年對離別。\\登臺北望烟雨深,回身哭向天邊月。
\end{myquote}

又:

\begin{myquote}
夜深悶到戟門邊,卻遶行廊又獨眠。\\閨中只是空相憶,魂歸漠漠魄歸泉。
\end{myquote}

話說西門慶聽了金蓮之言,又變了卦。到次日,那來旺兒收拾行李伺候,到日中還不見動靜。只見西門慶出來,叫來旺兒到跟前說道:「我夜間想來,你纔打杭州來家多少時兒,又教你往東京去,忒辛苦了,不如叫來保替你去罷。你且在家歇宿幾日,我到明日,家門首生意尋一箇與你做罷。」自古物聽主裁,那來旺兒那裡敢說甚的,只得應諾下來。西門慶就把銀兩書信,交付與來保和吳主管,三月念八日起身往東京去了。不在話下。

這來旺兒回到房中,心中大怒,{\pangpi{應前大喜。}}吃酒醉倒房中,口內胡說,怒起宋蕙蓮來,要殺西門慶。{\meipi{情急而亂,禍臨頭,人往往如此。}}被宋蕙蓮罵了他幾句:「你咬人的狗兒不露齒,是言不是語,墻有縫,壁有耳。𠳹了那黃湯,挺那兩覺。」打發他上床睡了。到次日,走到後邊,串玉簫房裡請出西門慶。兩箇在廚房後墻底下僻靜處說話,玉簫在後門首替他觀風。婆娘甚是埋怨,說道:「你是箇人?你原說教他去,怎麼轉了靶子,又教別人去?你乾淨是箇『毬子心腸——滾上滾下』,{\pangpi{說得切。}}『燈草柺棒兒——原拄不定把』。你到明日蓋箇廟兒,立起箇旗杆來,就是箇謊神爺!我再不信你說話了。我那等和你說了一場,就沒些情分兒!」{\meipi{埋怨中帶戲謔,妙甚。}}西門慶笑道:「到不是此說。我不是也叫他去,恐怕他東京蔡太師府中不熟,所以教來保去了。留下他,家門首尋箇買賣與他做罷!」婦人道:「你對我說,尋箇甚麼買賣與他做?」西門慶道:「我教他搭箇主管,在家門首開酒店。」婦人聽言滿心歡喜,走到屋裡一五一十對來旺兒說了,{\pangpi{此時已明做矣。}}單等西門慶示下。

一日,西門慶在前廳坐下,着人叫來旺兒近前,桌上放下六包銀兩,說道:「孩兒!你一向杭州來家辛苦。教你往東京去,恐怕你蔡府中不十分熟,所以教來保去了。今日這六包銀子三百兩,你拿去搭上箇主管,在家門首開酒店,月間尋些利息孝順我,也是好處。」那來旺連忙趴在地下磕頭,領了六包銀兩。回到房中,告與老婆說:「他倒拿買賣來窩盤我,今日與了我這三百兩銀子,教我搭主管,開酒店做買賣。」老婆道:「怪賊黑囚!你還嗔老婆說。一鍬就掘了井?也等慢慢來。如何今日也做上買賣了!你安分守己,休再吃了酒,口裡六說白道!」來旺兒叫老婆把銀兩收在箱中:「我在街上尋夥計去也!」於是走到街上尋主管。尋到天晚,主管也不成,又吃的大醉來家。老婆打發他睡了,就被玉簫走來,叫到後邊去了。

來旺兒睡了一覺,約一更天氣,酒還未醒,正朦朦朧朧睡着,忽聽的窻外隱隱有人叫他道:「來旺哥!還不起來看看,你的媳婦子又被那沒廉恥的勾引到花園後邊,幹那營生去了。虧你倒睡的放心!」{\meipi{就心事上誘之,不得不應,妙局。}}來旺兒猛可驚醒,睜開眼看看,不見老婆在房裡,只認是雪娥看見甚動靜,來遞信與他,不覺怒從心上起,道:「我在面前就弄鬼兒!」忙跳起身來,開了房門,逕撲到花園中來。剛到廂房中角門首,不防黑影裡丟擲一條櫈子來,把來旺兒絆了一交,只見响喨一聲,一把刀子落地。左右閃過四五箇小厮,大叫:「有賊!」一齊向前,把來旺兒一把捉住了。來旺兒道:「我是來旺兒,進來尋媳婦子,如何把我拿住了?」衆人不由分說,一步一棍,打到廳上。只見大廳上燈燭熒煌,西門慶坐在上面,即叫:「拿上來!」來旺兒跪在地下,說道:「小的睡醒了,不見媳婦在房裡,進來尋他。如何把小的做賊拿?」那來興兒就把刀子放在面前,與西門慶看。西門慶大怒,罵道:「衆生好度人難度,這厮眞是箇殺人賊!我倒見你杭州來家,叫你領三百兩銀子做買賣,如何夤夜進內來要殺我?不然拿這刀子做甚麼?」喝令左右:「與我押到他房中,取我那三百兩銀子來!」衆小厮隨即押到房中。蕙蓮正在後邊同玉簫說話,忽聞此信,忙跑到房裡。看見了,放聲大哭,說道:「你好好吃了酒睡罷,平白又來尋我做甚麼?只當暗中了人的拖刀之計。」一面開箱子,取出六包銀子來,拿到廳上。西門慶燈下開啟觀看,內中止有一包銀兩,餘者都是錫鉛錠子。西門慶大怒,因問:「如何抵換了!我的銀兩往那裡去了?趁早寔說!」那來旺兒哭道:「爹擡舉小的做買賣,小的怎敢欺心抵換銀兩?」西門慶道:「你打下刀子,還要殺我。刀子現在,還要支吾甚麼?」因把來興兒叫來,面前跪下,執證說:「你從某日,沒曾在外對衆發言要殺爹,嗔爹不與你買賣做?」這來旺兒只是嘆氣,張開口兒合不的。西門慶道:「既賍證刀杖明白,叫小厮與我拴鎖在門房內。明日寫狀子,送到提刑所去!」只見宋蕙蓮雲鬟撩亂,衣裙不整,走來廳上向西門慶跪下,說道:「爹,此是你幹的營生!他好好進來尋我,怎把他當賊拿了?你的六包銀子,我收着,原封兒不動,平白怎的抵換了?恁活埋人,也要天理。他為甚麼,你只因他甚麼?{\pangpi{如何說得出。}}打與他一頓。如今拉着送他那裡去?」西門慶見了他,回嗔作喜道:「媳婦兒,關你甚事?你起來。他無禮膽大不是一日,見藏着刀子要殺我,你不得知道。你自安心,沒你之事。」因令來安兒:「好攙扶你嫂子回房去,休要慌嚇他。」{\pangpi{老面皮。}}那蕙蓮只顧跪着不起來,說:「爹好狠心!你不看僧面看佛面,我恁說着,你就不依依兒?{\meipi{人面前說軟媚情語,都不要臉矣。}}他雖故吃酒,並無此事。」纏得西門慶急了,教來安兒搊他起來,勸他回房去了。

到天明,西門慶寫了柬帖,叫來興兒做幹證,揣着狀子,押着來旺兒往提刑院去,說某日酒醉,持刀夤夜殺害家主,又抵換銀兩等情。纔待出門,只見吳月娘走到前廳,向西門慶再三將言勸解,說道:「奴才無禮,家中處分他便了。又要拉出去,驚官動府做甚麼?」西門慶聽言,圓睜二目,喝道:「你婦人家,不曉道理!奴才安心要殺我,你倒還教饒他罷!」{\meipi{月娘不看勢頭好歹就勸,所以討箇沒趣。}}於是不聽月娘之言,喝令左右把來旺兒押送提刑院去了。月娘當下羞赧而退,回到後邊,向玉樓衆人說道:「如今這屋裡亂世為王,九尾狐狸精出世。不知聽信了甚麼人言語,平白把小厮弄出去了。你就賴他做賊,萬物也要箇着實纔好,拿紙棺材糊人,成何道理?恁沒道理昏君行貨!」宋蕙蓮跪在當面哭泣。月娘道:「孩兒你起來,不消哭。你漢子恆數問不的他死罪。賊強人,他吃了迷魂湯了,俺們說話不中聽,『老婆當軍——充數兒罷了』。」玉樓向蕙蓮道:「你爹正在箇氣頭上,待後慢慢的俺每再勸他。你安心回房去罷。」按下這裡不提。

單表來旺兒押到提刑院,西門慶先差玳安送了一百石白米與夏提刑、賀千戶。二人受了禮物,然後坐廳。來興兒遞上呈狀,看了,已知來旺兒先因領銀做買賣,見財起意,抵換銀兩,恐家主查算,夤夜持刀突入後廳,謀殺家主等情。心中大怒,把來旺叫到當廳跪下。這來旺兒告道:「望天官爺察情!容小的說,小的便說;不容小的說,小的不敢說。」夏提刑道:「你這厮!見獲賍證明白,勿得推調,從實與我說來,免我動刑。」來旺兒悉把西門慶初時令某人將藍段子,怎的調戲他媳婦兒宋氏成姦,如今故入此罪,要墊害圖霸妻子一節,訴說一遍。夏提刑大喝了一聲,令左右打嘴巴,說:「你這奴才欺心背主!你這媳婦也是你家主娶的,{\pangpi{妙語。}}配與你為妻,又把資本與你做買賣,你不思報本,卻倚醉夤夜突入臥房,持刀殺害。滿天下人都象你這奴才,也不敢使人了。」來旺兒口還叫冤屈,被夏提刑叫過來興兒過來執證。那來旺兒有口說不得了。正是:

\begin{myquote}
會施天上計,難免目前災。
\end{myquote}

夏提刑即令左右選大夾棍上來,把來旺兒夾了一夾,打了二十大棍,打的皮開肉綻,鮮血淋漓。分咐獄卒,帶下去收監。來興兒、鉞安兒來家,回覆了西門慶話。西門慶滿心歡喜,分咐家中小厮:「鋪蓋、飯食,一些都不許與他送進去。但打了,休來家對你嫂子說,只說衙門中一下兒也沒打他,監幾日便放出來。」衆小厮應諾了。

這宋蕙蓮自從拿了來旺兒去,頭也不梳,臉也不洗,黃着臉兒,只是關閉房門哭泣,茶飯不吃。西門慶慌了,使玉簫並賁四娘子兒再三進房解勸他,說道:「你放心,爹因他吃酒狂言,監他幾日,耐他性兒,不久也放他出來。」蕙蓮不信,使小厮來安兒送飯進監去,回來問他,也是這般說:「哥見官,一下兒也不打。一兩日就來家,教嫂子在家安心。」這蕙蓮聽了此言,方纔不哭了。每日淡掃娥眉,薄施脂粉,出來走跳。西門慶要便來回打房門首走,老婆在簷下叫道:「房裡無人,爹進來坐坐不是!」西門慶進入房裡,與老婆做一處說話。西門慶哄他說道:「我兒,你放心。我看你面上,寫了帖兒對官府說,也不曾打他一下兒。監他幾日,耐耐他性兒,還放他出來,還叫他做買賣。」婦人摟抱着西門慶脖子,說道:「我的親達達!你好歹看奴之面,{\meipi{詞愈親,則情愈疏矣,人多不悟。}}奈何他兩日,放他出來。隨你教他做買賣,不教他做買賣也罷,這一出來,我教他把酒斷了,隨你去近到遠使他,他敢不去?再不你若嫌不自便,替他尋上箇老婆,他也罷了。我常遠不是他的人了。」西門慶道:「我的心肝,你話是了。我明日買了對過喬家房,收拾三間房子與你住,搬你那裡去,咱兩箇自在頑耍。」婦人道:「着來,親親!隨你張主便了。」說畢,兩箇閉了門兒。原來婦人夏月常不穿褲兒,只單弔着兩條裙子,遇見西門慶在那裡,便掀開裙子就幹。於是二人解佩露甄妃之玉,齊眉點漢署之香,雙鳧飛肩,雲雨一席。婦人將身帶的白銀條紗挑線香袋兒——裡邊裝着松柏兒並排草,挑着「嬌香美愛」四箇字,把與西門慶。喜的心中要不的,恨不的與他誓共死生。向袖中即掏出一二兩銀子,與他買菓子吃。再三安撫他:「不消憂慮,只怕憂慮壞了你。我明日寫帖子對夏大人說,就放他出來。」說了一回,西門慶恐有人來,連忙出去了。這婦人得了西門慶此話,到後邊對衆丫鬟媳婦,詞色之間未免輕露。{\pangpi{婦人沒受用在此。}}孟玉樓早已知道,轉來告潘金蓮說,他爹怎的早晚要放來旺兒出來,另替他娶一箇;怎的要買對門喬家房子,把媳婦子弔到那裡去,與他三間房住,又買箇丫頭伏侍他;與他編銀絲鬏髻,打頭面。一五一十說了一遍:「就和你我輩一般,甚麼張致!大姐姐也就不管管兒!」潘金蓮不聽便罷,聽了時:忿氣滿懷無處着,雙腮紅上更添紅。說道:「眞箇繇他,我就不信了!今日與你說的話,我若教賊奴才淫婦,與西門慶放了第七箇老婆——我不喇嘴說——就把『潘』字倒過來!」玉樓道:「漢子沒正條的,大姐姐又不管,咱每能走不能飛,到的那些兒?」{\pangpi{酷肖玉樓口角。}}金蓮道:「你也忒不長俊,要這命做甚麼?活一百歲殺肉吃!他若不依我,拚着這命擯兌在他手裡也不差甚麼!」玉樓笑道:「我是小膽兒,不敢惹他,看你有本事和他纏。」

到晚,西門慶在花園中翡翠軒書房裡坐的,正要教陳敬濟來寫帖子,往夏提刑處說,要放來旺兒出來。被金蓮驀地走到跟前,搭伏着書桌兒,{\pangpi{故作緩致。}}問:「你教陳姐夫寫甚麼帖子?」西門慶不能隱諱,因說道:「我想把來旺兒責打與他幾下,放他出來罷。」婦人止住小厮:「且不要叫陳姐夫來。」坐在旁邊,因說道:「你空耽着漢子的名兒,原來是箇隨風倒舵、順水推船的行貨子!我那等對你說的話兒你不依,倒聽那賊奴才淫婦話兒。隨你怎的逐日沙糖拌蜜與他吃,他還只疼他的漢子。依你如今把那奴才放出來,你也不好要他這老婆了,教他奴才好藉口,你放在家裡不葷不素,當做甚麼人兒看成?待要把他做你小老婆,奴才又見在;待要說道奴才老婆,你見把他逞的恁沒張致的,在人跟前上頭上臉有些樣兒!就算另替那奴才娶一箇,着你要了他這老婆,往後倘忽你兩箇坐在一答裡,那奴才或走來跟前回話,或做甚麼,見了有箇不氣的?老婆見了他,站起來是,不站起來是?先不先,只這箇就不雅相。{\meipi{設出許多木然之想,說得事事可慮,金蓮口嘴殊可畏。}}傳出去,休說六隣親戚笑話,只家中大小,把你也不着在意裡。正是上梁不正下梁歪。你既要幹這營生,不如一狠二狠,把奴才結果了,{\pangpi{活冤家。}}你就摟着他老婆也放心。」{\meipi{此等論頭,似從武大身上得來。}}幾句又把西門慶念翻轉了,反又寫帖子送與夏提刑,教夏提刑限三日提出來,一頓拷打,拷打的通不象模樣。提刑兩位官並上下觀察、緝捕、排軍,監獄中上下,都受了西門慶財物,只要重不要輕。內中有一當案的孔目陰先生,名喚陰騭,乃山西孝義縣人,極是箇仁慈正直之士。因見西門慶要陷害此人,圖謀他妻子,再三不肯做文書送問,與提刑官抵面相講。兩位提刑官以此掣肘難行,延挨了幾日,人情兩盡,只把他當廳責了四十,論箇遞解原籍徐州為民。當查原賍,花費十七兩,鉛錫五包,責令西門慶家人來興兒領回。差人寫箇帖子,回覆了西門慶,隨教即日押發起身。這裡提刑官當廳押了一道公文,差兩箇公人把來旺兒取出來,已是打的稀爛,釘了扭,上了封皮,限即日起程,逕往徐州管下交割。可憐這來旺兒,在監中監了半月光景,沒錢使用,弄的身體狼狽,衣服藍褸,沒處投奔。哀告兩箇公人說:「兩位哥在上,我打了一場屈官司,身上分文沒有,要湊些脚步錢與二位,望你可憐見,押我到我家主處,有我的媳婦兒並衣服箱籠,討出來變賣了,知謝二位,並路途盤費,也討得一步鬆寬。」那兩箇公人道:「你好不知道理!你家主既擺佈了一場,他又肯發出媳婦並箱籠與你?{\meipi{畢竟公人有見識。}}你還有甚親故,俺們看陰師父面上,瞞上不瞞下,領你到那裡,胡亂討些錢米,勾你路上盤費便了。誰指望你甚脚步錢兒!」來旺道:「二位哥哥,你只可憐引我先到我家主門首,我央浼兩三位親隣,替我美言討討兒,無多有少。」兩箇公人道:「也罷,我們就押你去。」這來旺兒先到應伯爵門首,伯爵推不在家。{\pangpi{活賊。}}又央了左隣賈仁清、伊勉慈二人來西門慶家,替來旺兒說討媳婦箱籠。西門慶也不出來,使出五六箇小厮,一頓棍打出來,不許在門首纏擾。把賈、伊二人羞的要不的。他媳婦兒宋蕙蓮,在屋裡瞞的鐵桶相似,並不知一字。西門慶分咐:「那箇小厮走漏訊息,決打二十板!」兩箇公人又同到他丈人——賣棺材的宋仁家。來旺兒如此這般對宋仁哭訴其事,打發了他一兩銀子,與兩箇公人一弔銅錢、一斗米,路上盤纏。哭哭啼啼,從四月初旬離了清河縣,往徐州大道而來。正是:

\begin{myquote}
若得苟全癡性命,也甘飢餓過平生。
\end{myquote}

不說來旺兒遞解徐州去了。且說宋蕙蓮在家,每日只盼他出來。小厮一般的替他送飯,到外邊,衆人都吃了。轉回來蕙蓮問着他,只說:「哥吃了,監中無事。若不是也放出來了,連日提刑老爺沒來衙門中問事,也只在一二日來家。」西門慶又哄他說:「我差人說了,不久即出。」婦人以為信實。一日風裡言風裡語,聞得人說,來旺兒押出來,在門首討衣箱,不知怎的去了。這婦人幾次問衆小厮,都不說。忽見鉞安兒跟了西門慶馬來家,叫住問他:「你旺哥在監中好麼?幾時出來?」鉞安道:「嫂子,我告你知了罷,俺哥這早晚到流沙河了。」蕙蓮問其故,這鉞安千不合萬不合,如此這般:「打了四十板,遞解原籍徐州家去了。只放你心裡,休題我告你說。」這婦人不聽萬事皆休,聽了此言,關閉了房間,放聲大哭道:「我的人嚛!你在他家幹壞了甚麼事來?被人紙棺材暗算計了你!你做奴才一場,好衣服沒曾掙下一件在屋裡。今日只當把你遠離他鄉,弄的去了,坑得奴好苦也!你在路上死活未知。我就如合在缸底下一般,怎的曉得?」{\meipi{蕙蓮既為蔣聰報仇,又欲為來旺死節,雖淫,過金蓮、瓶兒遠矣。}}哭了一回,取一條長手巾拴在臥房門樞上,懸梁自縊。不想來昭妻一丈青,住房正與他相連,從後來聽見他屋裡哭了一回,不見動靜,半日只聽喘息之聲。扣房門叫他不應,慌了手脚,教小厮平安兒撬開窻戶進去。見婦人穿着隨身衣服,在門樞上正弔得好。一面解救下來,並了房門,取薑湯撅灌。須臾,嚷的後邊知道。吳月娘率領李嬌兒、孟玉樓、西門大姐、李瓶兒、玉簫、小玉都來看視,賁四娘子兒也來瞧。一丈青搊扶他坐在地下,只顧哽咽,白哭不出聲來。月娘叫着他,只是低着頭,口吐涎痰,不答應。月娘便道:「原來是箇傻孩子!你有話只顧說便好,如何尋起這條路起來!」又令玉簫扶着他,親叫道:「蕙蓮孩兒,你有甚麼心事,越發老實叫上幾聲,不妨事。」{\pangpi{動人苦衷。}}問了半日,那婦人哽咽了一回,大放聲排手拍掌哭起來。月娘叫玉簫扶他上炕,他不肯上炕。月娘衆人勸了半日,回後邊去了。止有賁四嫂同玉簫相伴在屋裡。只見西門慶掀簾子進來,看見他坐在冷地下哭泣,令玉簫:「你搊他炕上去罷。」玉簫道:「剛纔娘教他上去,他不肯去。」西門慶道:「好強孩子,冷地下冰着你。你有話對我說,如何這等拙智!」蕙蓮把頭搖着說道:「爹,你好人兒,你瞞着我幹的好勾當兒!還說甚麼孩子不孩子!你原來就是箇弄人的劊子手,把人活埋慣了,害死人還看出殯的!{\meipi{半是想來旺,半是恨西門慶不聽己言,故執念不回,並作態以要寵也。}}你成日間只哄着我,今日也說放出來,明日也說放出來。只當端的好出來。你如遞解他,也和我說聲兒,暗暗不通風,就解發遠遠的去了。你也要合憑箇天理!你就信着人幹下這等絕戶計,把圈套兒做的成成的,你還瞞着我。你就打發,兩箇人都打發了,如何留下我做甚麼?」{\pangpi{語太無情。}}西門慶笑道:「孩兒,不關你事。那厮壞了事,所以打發他。你安心,我自有處。」因令玉簫:「你和賁四娘子相伴他一夜兒,我使小厮送酒來你每吃。」說畢,往外去了。賁四嫂良久扶他上炕坐的,和玉簫將話兒勸解他。西門慶到前邊鋪子裡,問傅夥計支了一弔錢,買了一錢酥燒,拿盒子盛了,又是一瓶酒,使來安兒送到蕙蓮屋裡,說道:「爹使我送這箇與嫂子吃。」蕙蓮看見,一頭罵:「賊囚根子!趁早與我拿了去,省的我摔一地。」{\meipi{此時送此物來,自惹人氣。}}來安兒道:「嫂子收了罷,我拿回去,爹又要打我。」便就放在桌子上。蕙蓮跳下來,把酒拿起來,纔待趕着摔了去,被一丈青攔住了。那賁四嫂看着一丈青咬指頭兒。正相伴他坐的,只見賁四嫂家長兒走來,叫他媽道:「爹門外頭來家,要吃飯。」賁四嫂和一丈青走出來。到一丈青門首,只見西門大姐在那裡,和來保兒媳婦惠祥說話。因問賁四嫂那裡去,賁四嫂道:「俺家的門外頭來了,要飯吃。我到家瞧瞧就來。我只說來看看,吃他大爹再三央,陪伴他坐坐兒,誰知倒把我掛住了。」惠祥道:「剛纔爹在屋裡,他說甚麼來?」賁四嫂只顧笑,說道:「看不出他旺官娘子,原來也是箇辣菜根子,{\pangpi{借旁人口襯出。}}和他大爹白搽白折的平上。誰家媳婦兒有這箇道理!」惠祥道:「這箇媳婦兒比別的媳婦兒不同,從公公身上拉下來的媳婦兒,{\pangpi{妙語。}}這一家大小誰如他?」說畢惠祥去了。一丈青道:「四嫂,你到家快來。」賁四嫂道:「甚麼話,我若不來,惹他大爹就怪死了。」卻說西門慶白日教賁四嫂和一丈青陪他坐,晚夕教玉簫伴他睡,慢慢將言詞勸他,說道:「宋大姐,你是箇聰明的,趁恁妙齡之時,一朵花初開,主子愛你,也是緣法相投。你如今將上不足,比下有餘,守着主子,強如守着奴才。他已是去了,你恁煩惱不打緊,一時哭的有好歹,卻不虧負了你的性命?常言道:做一日和尚撞一日鍾,往後貞節輪不到你身上了。」{\meipi{說得花花鬨哄,雖鐵人亦動。古今名理,不知被此等言語害了多少。}}那蕙蓮聽了,只是哭泣,每日粥飯也不吃。{\meipi{雖非貞節,然能於死生貴賤之際,感戀不忘其情,亦自可悲。}}玉簫回了西門慶話。西門慶又令潘金蓮親來對他說,也不依。金蓮惱了,向西門慶道:「賊淫婦,他一心只想他漢子,千也說一夜夫妻百夜恩,萬也說相隨百步,也有箇徘徊意,這等貞節的婦人,卻拿甚麼拴的住他心?」西門慶笑道:「你休聽他摭說,他若早有貞節之心,當初只守着廚子蔣聰,不嫁來旺兒了。」{\meipi{如此語使金蓮聞之應自愧,故宜枘鑿也。嫁來旺為報蔣聰仇也。今來旺之仇誰報?雖然蔣聰之仇由來旺而報,來旺無西門則亦不能報。然則來旺之仇死之可也,不死之亦可也。}}一面坐在前廳上,把衆小厮都叫到跟前審問:「來旺兒遞解去時,是誰對他說來?趁早舉出來,我也一下不打他。不然,我打聽出來,每人三十板,即與我離門離戶。」忽有畫童跪下,說道:「那日小的聽見鉞安跟了爹馬來家,在夾道內,嫂子問他,他走了口對嫂子說。」西門慶聽了大怒,一片聲使人尋鉞安兒。這鉞安早知訊息,一直躲到潘金蓮房裡去。金蓮正洗臉,小厮走到屋裡,跪着哭道:「五娘救小的則箇!」金蓮罵道:「賊囚!猛可走來,嚇我一跳!你又不知幹下甚麼事!」鉞安道:「爹因為小的告嫂子說了旺哥去了,要打我。娘好歹勸勸爹。若出去,爹在氣頭裡,小的就是死罷了!」金蓮道:「怪囚根子,唬的鬼也似的!我說甚麼勾當來,恁驚天動地的?原來為那奴才淫婦。」分咐:「你在我這屋裡,不要出去。」於是藏在門背後。西門慶見叫不將鉞安去,在前廳暴叫如雷。一連使了兩替小厮來金蓮房裡尋,都被金蓮罵的去了。落後,西門慶一陣風自家走來,手裡拿着馬鞭子,問:「奴才在那裡?」金蓮不理他,被西門慶遶屋尋遍,從門背後採出鉞安來要打。吃金蓮向前,把馬鞭子奪了,掠在床頂上。{\pangpi{金蓮頗有膽氣。}}說道:「沒廉恥的貨兒,你臉做主了!那奴才淫婦想他漢子上弔,羞急拿小厮來煞氣,關小厮甚事!」那西門慶氣的睜睜的。{\meipi{此等情節不堪說破,說破則西門慶自開口動手不得。}}金蓮叫小厮:「你往前頭幹你那營生去,不要理他。等他再打你,有我哩!」那鉞安得手,一直往前去了。正是:

\begin{myquote}
兩手劈開生死路,翻身跳出是非門。
\end{myquote}

這潘金蓮見西門慶留意在宋蕙蓮身上,乃心生一計。在後邊唆調孫雪娥,說來旺兒媳婦子怎的說你要了他漢子,備了他一篇是非,他爹惱了,纔把他漢子打發了:「前日打了你那一頓,拘了你頭面衣服,都是他過嘴告說的。」這孫雪娥聽了箇耳滿心滿。掉了雪娥口氣兒,走到前邊,向蕙蓮又是一樣話說,說孫雪娥怎的後邊罵你是蔡家使喝的奴才,積年轉主子養漢,不是你背養主子,你家漢子怎的離了他家門?說你眼淚留着些脚後跟。說的兩下都懷仇恨。一日,也是合當有事。四月十八日,李嬌兒生日,院中李媽媽並李桂姐,都來與他做生日。吳月娘留他同衆堂客在後廳飲酒,西門慶往人家赴席不在家。這宋蕙蓮吃了飯兒,從早晨在後邊打了箇幌兒,走到屋裡直睡到日西。繇着後邊一替兩替使了丫鬟來叫,只是不出來。雪娥尋不着這箇繇頭兒,走來他房裡叫他,說道:「嫂子做了玉美人了,怎的這般難請?」那蕙蓮也不理他,只顧面朝裡睡。這雪娥又道:「嫂子,你思想你家旺官兒哩。早思想好來!不得你他也不得死,還在西門慶家裡。」{\pangpi{雪娥來得殊無文理,後之一死適足以償。}}這蕙蓮聽了他這一句話,打動潘金蓮說的那情繇,翻身跳起來,望雪娥說道:「你沒的走來浪聲顙氣!他便因我弄出去了。你為甚麼來?打你一頓,攆的不容上前。得人不說出來,大家將就些便罷了,何必撐着頭兒來尋趁人!」這雪娥心中大怒,罵道:「好賊奴才,養漢淫婦!如何大膽罵我?」蕙蓮道:「我是奴才淫婦,你是奴才小婦!我養漢養主子,強如你養奴才!{\pangpi{罵得痛快。}}你倒背地偷我漢子,你還來倒自家掀騰?」這幾句話,說的雪娥急了,宋蕙蓮不防,被他走向前,一箇巴掌打在臉上,打的臉上通紅。說道:「你如何打我?」於是一頭撞將去,兩箇就揪扭打在一處。慌的來昭妻一丈青走來勸解,把雪娥拉的後走,兩箇還罵不絕口。吳月娘走來罵了兩句:「你每都沒些規矩兒!不管家裡有人沒人,都這等家反宅亂的!等你主子回來,看我對你主子說不說!」當下雪娥就往後邊去了。月娘見蕙蓮頭髮揪亂,便道:「還不快梳了頭,往後邊來哩!」蕙蓮一聲兒不答話。打發月娘後邊去了,走到房內,倒插了門,哭泣不止。哭到掌燈時分,衆人亂着,後邊堂客吃酒,可憐這婦人忍氣不過,{\meipi{四字春秋得妙,以見非為節也。}}尋了兩條脚帶,拴在門楹上,自縊身死,亡年二十五歲。正是:

\begin{myquote}
世間好物不堅牢,彩雲易散琉璃脆。
\end{myquote}

落後,月娘送李媽媽、桂姐出來,打蕙蓮門首過,房門關着,不見動靜,心中甚是疑影。打發李媽媽娘兒上轎去了,回來叫他門不開,都慌了手脚。還使小厮打窻戶內跳進去,割斷脚帶,解卸下來,撅救了半日,不知多咱時分,嗚呼哀哉死了。但見:

\begin{myquote}
四肢冰冷,一氣燈殘。香魂眇眇,已赴望鄉臺;星眼瞑瞑,屍猶橫地下。不知精爽逝何處,疑是行雲秋水中。
\end{myquote}

月娘見救不活,慌了。連忙使小厮來興兒,騎頭口往門外請西門慶來家。雪娥恐怕西門慶來家拔樹尋根,歸罪於己,在上房打旋磨兒跪着月娘,教休題出和他嚷鬧來。月娘見他嚇得那等腔兒,心中又下般不得,因說道:「此時你恁害怕,當初大家省言一句兒便了。」至晚,等的西門慶來家,只說蕙蓮因思想他漢子,哭了一日,趕後邊人亂,不知多咱尋了自盡。西門慶便道:「他恁箇拙婦,原來沒福。」{\meipi{只深淡一語作結便了,蓋無情以繫心也。作者一絲不亂。}}一面差家人遞了一紙狀子,報到縣主李知縣手裡,只說本婦因本家請堂客吃酒,他管銀器家伙,因失落一件銀鍾,恐家主查問見責,自縊身死。又送了知縣三十兩銀子。知縣自恁要作分上,胡亂差了一員司吏帶領幾箇仵作來看了。自買了一具棺材,討了一張紅票,賁四、來興兒同送到門外地藏寺。與了火家五錢銀子,多架些柴薪。纔待發火燒燬,不想他老子賣棺材宋仁打聽得知,走來攔住,叫起屈來。說他女兒死的不明白,稱西門慶因倚強姦他:「我女貞節不從,威逼身死。我還要撫按告狀,誰敢燒化屍首!」那衆火家都亂走了,不敢燒。賁四、來興少不的把棺材停在寺裡來回話。正是:

\begin{myquote}
青龍與白虎同行,吉兇事全然未保。
\end{myquote}

