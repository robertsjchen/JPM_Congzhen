\includepdf[pages={25,26},fitpaper=false]{tst.pdf}
\chapter*{第十三回 李瓶姐墻頭密約 迎春兒隙底私窺}
\addcontentsline{toc}{chapter}{第十三回 李瓶姐墻頭密約 迎春兒隙底私窺}
\markboth{{\titlename}卷之二}{第十三回 李瓶姐墻頭密約 迎春兒隙底私窺}


詞曰:

\begin{myquote}
綉面芙蓉一笑開,斜飛寶鴨襯香腮。眼波纔動被人猜。一面風情深有韻,半箋嬌恨寄幽懷。月移花影約重來。

\raggedleft{——右調《山花子》\rightquadmargin}
\end{myquote}

話說一日西門慶往前邊走來,到月娘房中。月娘告說:「今日花家使小厮拿帖來,請你吃酒。」西門慶觀看帖子,寫着:「即午院中吳銀家一叙,希即過我同往,萬萬!」少頃,打選衣帽,叫了兩箇跟隨,騎匹駿馬,先逕到花家。不想花子虛不在家了。他渾家李瓶兒,夏月間戴着銀絲鬏髻,金鑲紫瑛墜子,藕絲對衿衫,白紗挑線鑲邊裙,裙邊露一對紅鴛鳳嘴、尖尖趫趫小脚,立在二門裡臺基上。那西門慶三不知走進門,兩下撞了箇滿懷。{\meipi{此一撞,可謂五百年風流業冤。}}這西門慶留心已久,雖故庄上見了一面,不曾細玩。今日對面見了,見他生的甚是白淨,五短身才,瓜子面兒,細灣灣兩道眉兒,不覺魂飛天外,忙向前深深作揖。婦人還了萬福,轉身入後邊去了。使出一箇頭髮齊眉的丫鬟來,名喚綉春,請西門慶客位內坐。他便立在角門首,半露嬌容說:「大官人少坐一時。他適纔有些小事出去了,便來也。」丫鬟拿出一盞茶來,西門慶吃了。婦人隔門說道:「今日他請大官人往那邊吃酒去,好歹看奴之面,{\pangpi{托熟得妙。}}勸他早些回家。兩箇小厮又都跟去了,止是這兩箇丫鬟和奴,家中無人。」{\pangpi{深心語。}}西門慶便道:「嫂子見得有理,哥家事要緊。嫂子既然分付在下,在下已定伴哥同去同來。」

正說着,只見花子虛來家,婦人便回房去了。花子虛見西門慶叙禮說道:「蒙哥下降,小弟適有些不得已小事出去,失迎,恕罪!」於是分賓主坐下,便叫小厮看茶。須臾,茶罷。又分付小厮:「對你娘說,看菜兒來,我和西門爹吃三盃起身。今日六月二十四,是院內吳銀姐生日,請哥同往一樂。」西門慶道:「二哥何不早說?」即令玳安:「快家去,討五錢銀子封了來。」花子虛道:「哥何故又費心?小弟到不是了。」西門慶見左右放桌兒,說道:「不消坐了,咱往裡邊吃去罷。」花子虛道:「不敢久留,哥畧坐一回。」少傾,就是齊整餚饌拿將上來,銀高脚葵花鍾,每人三鍾,又是四箇捲餅,吃畢收下來與馬上人吃。少傾,玳安取了分資來,一同起身上馬,逕往吳四媽家,與吳銀兒做生日。到那裡,花攢錦簇,歌舞吹彈,飲酒至一更時分方散。西門慶留心,把子虛灌得酩酊大醉。又因李瓶兒央浼之言,相伴他一同來家。小厮叫開大門,扶到他客位坐下。李瓶兒同丫鬟掌着燈燭出來,把子虛攙扶進去。西門慶交付明白,就要告回。婦人旋走出來,拜謝西門慶,說道:「拙夫不才貪酒,多累看奴薄面,{\pangpi{就肯認帳,妙。}}姑待來家,官人休要笑話。」那西門慶忙屈身還喏,說道:「不敢。嫂子這裡分付,在下敢不銘心刻骨,同哥一搭裡來家!非獨嫂子耽心,顯的在下幹事不的了。方纔哥在他家,被那些人纏住了,我強着催哥起身。走到樂星堂兒門首粉頭鄭愛香兒家,{\pangpi{妙想。}}小名叫做鄭觀音,生的一表人物,哥就要往他家去,被我再三攔住,勸他說道:『恐怕家中嫂子放心不下。』方纔一直來家。若到鄭家,便有一夜不來。{\meipi{討好綽趣,無一語不在箇中。}}嫂子在上,不該我說,哥也糊塗,嫂子又青年,{\pangpi{尖。}}偌大家室,如何就丟了,成夜不在家?是何道理!」婦人道:「正是如此,奴為他這等在外胡行,不聽人說,奴也氣了一身病痛在這裡。{\meipi{語語情見乎辭。瓶兒雖淫,畢竟醇厚。}}往後大官人但遇他在院中,好歹看奴薄面,勸他早早回家。奴恩有重報,不敢有忘。」{\pangpi{一語覺瓊瑤,木桃猶淺。}}這西門慶是頭上打一下脚底板響的人,積年風月中走,甚麼事兒不知道?今日婦人到明明開了一條大路,教他入港,豈不省腔!於是滿面堆笑道:「嫂子說那裡話!相交朋友做甚麼?我已定苦心諫哥,嫂子放心。」婦人又道了萬福,又叫小丫鬟拿了一盞菓仁泡茶來。西門慶吃畢茶,說道:「我回去罷,嫂子仔細門戶。」遂告辭歸家。自此西門慶就安心設計,圖謀這婦人,屢屢安下應伯爵、謝希大這夥人,把子虛掛住在院裡飲酒過夜。他便脫身來家,一徑在門首站立。這婦人亦常領着兩箇丫鬟在門首。西門慶看見了,便揚聲咳嗽,一回走過東來,又往西去,{\pangpi{舊手段。}}或在對門站立,把眼不住望門裡睃盼。婦人影身在門裡,見他來便閃進裡面,見他過去了,又探頭去瞧。兩箇眼意心期,已在不言之表。一日,西門慶正站在門首,忽見小丫鬟綉春來請。西門慶故意問道:「姐姐請我做甚麼?你爹在家裡不在?」綉春道:「俺爹不在家,娘請西門慶爹問句話兒。」這西門慶得不的一聲,連忙走過來,到客位內坐下。良久,婦人出來,道了萬福,便道:「前日多承官人厚意,奴銘刻於心,知感不盡。他從昨日出去,一連兩日不來家了,不知官人曾會見他來不曾?」西門慶道:「他昨日同三四箇在鄭家吃酒,我偶然有些小事就來了。今日我不曾得進去,不知他還在那裡沒在。若是我在那裡,恐怕嫂子憂心,有箇不催促哥早早來家的?」婦人道:「正是這般說。奴吃煞他不聽人說、在外邊眠花臥柳,不顧家事的虧。」西門慶道:「論起哥來,仁義上也好,只是有這一件兒。」說着,小丫鬟拿茶來吃了。西門慶恐子虛來家,不敢久戀,就要告歸。婦人又千叮萬囑,央西門慶:「不拘到那裡,好歹勸他早來家,奴已定恩有重報,決不敢忘官人!」{\meipi{好歹只此一語,愈見瓶兒醇厚。}}西門慶道:「嫂子沒的說,我與哥是那樣相交!」說畢,西門慶家去了。

到次日,花子虛自院中回家,婦人再三埋怨說道:「你在外邊貪酒戀色,多虧隔壁西門大官人,兩次三番顧睦你來家。你買分禮兒謝謝他,方不失了人情。」那花子虛連忙買了四盒禮物,一罈酒,使小厮天福兒送到西門慶家。西門慶收下,厚賞來人去了。吳月娘便問說:「花家如何送你這禮?」西門慶道:「花二哥前日請我們在院中與吳銀兒做生日,醉了,被我攙扶了他來家;又見常時院中勸他休過夜,早早來家。他娘子兒因此感我的情,想對花二哥說,故買此禮來謝我。」{\meipi{一人開口,便着一人之痛癢,所以為妙。}}吳月娘聽了,與他打箇問訊,說道:「我的哥哥,你自顧了你罷,又泥佛勸土佛!你也成日不着箇家,在外養女調婦,反勸人家漢子!」又道:「你莫不白受他這禮?」因問:「他帖上兒寫着誰的名字?若是他娘子的名字,今日寫我的帖兒,請他娘子過來坐坐,他也只恁要來咱家走走哩。{\meipi{瓶兒蓄意已久,此語恍惚為瓶兒傳神。}}若是他男子漢名字,隨你請不請,我不管你。」西門慶道:「是花二哥名字,我明日請他便了。」次日,西門慶果然治酒,請過花子虛來,吃了一日酒。歸家,李瓶兒說:「你不要差了禮數。咱送了他一分禮,他到請你過去吃了一席酒,你改日還該治一席酒請他,只當回席。」

光陰迅速,又早九月重陽。花子虛假着節下,叫了兩箇妓者,具柬請西門慶過來賞菊。又邀應伯爵、謝希大、祝實念、孫天化四人相陪。傳花擊鼓,歡樂飲酒。有詩為證:

\begin{myquote}
烏兔循環似箭忙,人間佳節又重陽。\\千枝紅樹粧秋色,三徑黃花吐異香。\\不見登高烏帽客,還思捧酒綺羅娘。\\綉簾瑣闥私相覷,從此恩情兩不忘。
\end{myquote}

當日,衆人飲酒到掌燈之後,西門慶忽下席來外邊解手。不防李瓶兒正在遮槅子邊站立偷覷,兩箇撞了箇滿懷,{\meipi{此一撞未必無心。}}西門慶迴避不及。婦人走到西角門首,暗暗使綉春黑影裡走到西門慶跟前,低聲說道:「俺娘使我對西門爹說,少吃酒,早早回家。晚夕,娘如此這般要和西門爹說話哩。」西門慶聽了,歡喜不盡。小解回來,到席上連酒也不吃,唱的左右彈唱遞酒,只是裝醉不吃。看看到一更時分,那李瓶兒不住走來廉外,見西門慶坐在上面,只推做打盹。那應伯爵、謝希大,如同釘在椅子上,白不起身。熬的祝實念、孫寡嘴也去了,他兩箇還不動。把箇李瓶兒急的要不的。西門慶已是走出來,被花子虛再不放,說道:「今日小弟沒敬心,哥怎的白不肯坐?」西門慶道:「我本醉了,吃不去。」於是故意東倒西歪,教兩箇扶歸家去了。應伯爵道:「他今日不知怎的,白不肯吃酒,吃了不多酒就醉了。既是東家費心,難為兩箇姐兒在此,拿大鍾來,咱每再週四五十輪,散了罷。」李瓶兒在簾外聽見,罵「涎臉的囚根子」不絕。暗暗使小厮天喜兒請下花子虛來,分付說:「你既要與這夥人吃,趁早與我院裡吃去。休要在家裡聒噪。我半夜三更,熬油費火,我那裡耐煩!」花子虛道:「這咱晚我就和他們院裡去,也是來家不成,你休再麻犯我。」婦人道:「你去,我不麻犯便了。」這花子虛得不的這一聲,走來對衆人說:「我們往院裡去。」應伯爵道:「眞箇?休哄我。你去問聲嫂子來,咱好起身。」子虛道:「房下剛纔已是說了,教我明日來家。」謝希大道:「可是來,自吃應花子這等嘮叨。哥剛纔已是討了老脚來,咱去的也放心。」於是連兩箇唱的,都一齊起身進院。此時已是二更天氣,天福兒、天喜兒跟花子虛等三人,從新又到後巷吳銀兒家去吃酒,不題。

單表西門慶推醉到家,走到金蓮房裡,剛脫了衣裳,就往前邊花園裡去坐,單等李瓶兒那邊請他。良久,只聽得那邊趕狗關門。{\pangpi{寫出驚心。}}少傾,只見丫鬟迎春黑影裡扒着墻,推叫貓,{\meipi{趕狗叫貓,俗事一經點染,覺竹聲花影無此韻致。}}看見西門慶坐在亭子上,遞了話。這西門慶就掇過一張桌櫈來踏着,暗暗扒過墻來,這邊已安下梯子。李瓶兒打發子虛去了,已是摘了冠兒,亂挽烏雲,素體濃粧,立在穿廊下。{\pangpi{悄悄冥冥。}}看見西門慶過來,歡喜無盡,忙迎接進房中。燈燭下,早已安排一桌齊整酒餚菓菜,壺內滿貯香醪。婦人雙手高擎玉斝,親遞與西門慶,深深道箇萬福:「奴一向感謝官人,蒙官人又費心酬答,使奴家心下不安。今日奴自治了這盃淡酒,請官人過來,聊盡奴一點薄情。{\meipi{此何時,又作酬酢語,不幾迂而可笑。然此迂而可笑處,正隱隱畫出瓶兒之為人,不然,則又一金蓮矣。}}又撞着兩箇天殺的涎臉,只顧坐住了,急的奴要不的。剛纔吃我都打發到院裡去了。」西門慶道:「只怕二哥還來家麼?」婦人道:「奴已分付過夜,不來了。兩箇小厮都跟去了。家裡再無一人,只是這兩箇丫頭,一箇馮媽媽看門首,他是奴從小兒養娘,心腹人。前後門都已關閉了。」西門慶聽了,心中甚喜。兩箇於是並肩疊股,交盃換盞,飲酒做一處。迎春旁邊斟酒,綉春往來拿菜兒。吃得酒濃時,錦帳中香薰鴛被,設放珊瑚,兩箇丫鬟撤開酒桌,拽上門去了。兩人上床交歡。

原來大人家有兩層窻寮,外面為窻,裡面為寮。婦人打發丫鬟出去,關上裡面兩扇窻寮,房中掌着燈燭,外邊通看不見。這迎春丫頭,今年已十七歲,頗知事體,見他兩箇今夜偷期,悄悄向窻下,用頭上簪子挺籤破窻寮上紙,往裡窺覷。端的二人怎樣交接?但見:

\begin{myquote}
燈光影裏,鮫鮹帳中,一箇玉臂忙搖,一箇金蓮高舉。一箇鶯聲嚦嚦,一箇燕語喃喃。好似君瑞遇鶯娘,猶若宋玉偷神女。山盟海誓,依稀耳中;蝶戀蜂恣,未能即罷。正是:被翻紅浪,靈犀一點透酥胸;帳挽銀鉤,眉黛兩彎垂玉臉。
\end{myquote}

房中二人雲雨,不料迎春在窻外,聽看得明明白白。聽見西門慶問婦人多少青春。李瓶兒道:「奴今年二十三歲。」因問:「他大娘貴庚?」西門慶道:「房下二十六歲了。」婦人道:「原來長奴三歲,到明日買分禮兒過去,看看大娘,只怕不好親近。」{\meipi{自是一片結識深情,非枕邊閒語也。}}西門慶道:「房下自來好性兒。」婦人又問:「你頭裡過這邊來,他大娘知道不知?儻或問你時,你怎生回答?」西門慶道:「俺房下都在後邊第四層房子裡,惟有我第五箇小妾潘氏,在這前邊花園內,獨自一所樓房居住,他不敢管我。」婦人道:「他五娘貴庚多少?」西門慶道:「他與大房下同年。」婦人道:「又好了,若不嫌奴有玷,奴就拜他五娘做箇姐姐罷。到明日,討他大娘和五娘的鞋樣兒來,奴親自做兩雙鞋兒過去,以表奴情。」說着,又將頭上關頂的金簪兒撥下兩根來,替西門慶帶在頭上,說道:「若在院裡,休要叫花子虛看見。」西門慶道:「這理會得。」當下二人如膠似漆,盤桓到五更時分。窻外雞叫,東方漸白,西門慶恐怕子虛來家,整衣而起,照前越墻而過。兩箇約定暗號兒,但子虛不在家,這邊就使丫鬟在墻頭上暗暗以咳嗽為號,或先丟塊瓦兒,見這邊無人,方纔上墻,這邊西門慶便用梯櫈扒過墻來。兩箇隔墻酬和,竊玉偷香,不繇大門行走,街坊隣舍怎的曉得?有詩為證:

\begin{myquote}
月落花陰夜漏長,相逢疑是夢高堂。\\夜深偷把銀缸照,猶恐憨奴瞰隙光。
\end{myquote}

卻說西門慶扒過墻來,走到潘金蓮房裡。金蓮還睡未起,因問:「你昨日也不知又往那裡去了這一夜?也不對奴說一聲兒。」西門慶道:「花二哥又使小厮邀我往院裡去,吃了半夜酒,纔脫身走來家。」金蓮雖故信了,還有幾分疑影在心。一日,同孟玉樓飯後在花園亭子上做針指,猛可見一塊瓦兒打在面前。那孟玉樓低着頭納鞋,沒看見。這潘金蓮單單把眼四下觀看,影影綽綽,只見隔壁墻頭上一箇白面探了一探,就下去了。金蓮忙推玉樓,指與他瞧,說道:「三姐姐,你看這箇,是隔壁花家那大丫頭,想是上墻瞧花兒,看見俺們在這裡,他就下去了。」說畢,也就罷了。到晚夕,西門慶自外赴席來家,進金蓮房中。金蓮與他接了衣裳,問他,飯不吃,茶也不吃,趔趄着脚兒,只往前邊花園裡走。{\pangpi{心虛偏有此景。}}這潘金蓮賊留心,暗暗看着他。坐了好一回,只見先頭那丫頭在墻頭上打了箇照面,這西門慶就踏着梯櫈過墻去了。那邊李瓶兒接入房中,兩箇厮會,不題。

這潘金蓮歸到房中,翻來覆去,通一夜不曾睡。將到天明,只見西門慶過來,推開房門,婦人睡在床上,不理他。那西門慶先帶幾分愧色,挨近他床上坐下。婦人見他來,跳起來坐着,一手撮着他耳朵,罵道:「好負心的賊!你昨日端的那裡去來?把老娘氣了一夜!你原來幹的那繭兒,我已是曉得不耐煩了!{\meipi{映出聽籬察壁心腸。}}趁早寔說,從前已往,與隔壁花家那淫婦偷了幾遭?一一說出來,我便甘休。但瞞着一字兒,到明日你前脚兒過去,後脚我就喓喝起來,教你負心的囚根子死無葬身之地!你安下人標住他漢子在院裡過夜,卻這裡要他老婆。我教你吃不了包着走!嗔道昨日大白日裡,我和孟三姐在花園裡做生活,只見他家那大丫頭在墻那邊探頭舒腦的,原來是那淫婦使的勾使鬼來勾你來了。你還哄我老娘!前日他家那忘八,半夜叫了你往院裡去,原來他家就是院裡!」西門慶聽了,慌的裝矮子,只跌脚跪在地下,笑嘻嘻央及,{\meipi{散言碎語都有根據,始知從前一字不可減。}}說道:「怪小油嘴兒,禁聲些!實不瞞你,{\meipi{寫慌處,妙在是喜處。}}他如此這般問了你兩箇的年紀,到明日討了鞋樣去,每人替你做雙鞋兒,要拜認你兩箇做姐姐,他情願做妹子。」金蓮道:「我是不要那淫婦認甚哥哥姐姐的。他要了人家漢子,又來獻小殷勤兒,我老娘眼裡是放不下砂子的人,肯叫你在我跟前弄了鬼兒去!」說着一隻手把他褲子扯開,{\pangpi{又深意層。}}只見那話軟仃當,銀托子還帶在上面,問道:「你寔說,與淫婦弄了幾遭?」西門慶道:「弄到有數兒的,只一遭。」婦人道:「你賭箇誓,{\meipi{妬甚,氣甚,恨甚。}}一遭就弄的他恁軟如鼻涕,濃如醬,卻如風癱了一般的!有些硬朗氣兒也是人心。」說着把托子一揪,掛下來,罵道:「沒羞的強盜,嗔道教我那裡沒尋,原來把這行貨子悄地帶出,和那淫婦㒲搗去了。」{\meipi{寫人無恥,卻帶出自家無恥,妙甚。}}西門慶滿臉兒陪笑說道:「怪小淫婦兒,麻犯人死了,他再三教我稍了上覆來,他到明日過來與你磕頭,{\pangpi{消氣在此。}}還要替你做鞋。昨日使丫頭替了吳家的樣子去了。今日教我稍了這一對壽字簪兒送你。」於是除了帽子,向頭上拔將下來,遞與金蓮。金蓮接在手內觀看,卻是兩根番石青塡地、金玲瓏壽字簪兒,乃御前所制,宮裡出來的,甚是奇巧。金蓮滿心歡喜,說道:「既是如此,我不言語便了。{\meipi{金蓮大都要強,非盡愛小便宜也。}}等你過那邊去,我這裡與你兩箇觀風,教你兩箇自在㒲搗。你心下如何?」那西門慶歡喜的雙手摟抱着說道:「我的乖乖的兒,正是如此。不枉的養兒不在屙金溺銀,只要見景生情。我到明日梯己買一套粧花衣服謝你。」婦人道:「我不信那蜜嘴糖舌,既要老娘替你二人周旋,要依我三件事。」西門慶道:「不拘幾件,我都依。」婦人道:「頭一件不許你往院裡去;第二件要依我說話;第三件你過去和他睡了,來家就要告我說,一字不許你瞞我。」{\meipi{三件事俱帶孩子氣,妙不失美人心性。}}西門慶道:「這箇不打緊,都依你便了。」自此為始,西門慶過去睡了來,就告婦人說:「李瓶兒怎的生得白淨,身軟如綿花,好風月,又善飲。俺兩箇帳子裡放着菓盒,看牌飲酒,常頑耍半夜不睡。」又向袖中取出一箇物件兒來,遞與金蓮瞧,道:「此是他老公公內府畫出來的,俺兩箇點着燈,看着上面行事。」金蓮接在手中,展開觀看。有詞為證:

\begin{myquote}
內府衢花綾裱,牙籤錦帶粧成。大青小綠細描金,鑲嵌斗方乾淨。女賽巫山神女,男如宋玉郎君,雙雙帳內慣交鋒。解名二十四,春意動關情。
\end{myquote}

金蓮從前至尾看了一遍,不肯放手,就交與春梅道:「好生收在我箱子內,早晚看着耍子。」{\pangpi{老氣得妙。}}西門慶道:「你看兩日,還交與我。此是人的愛物兒,我借了他來家瞧瞧,還與他。」金蓮道:「他的東西,如何到我家?我又不曾從他手裡要將來。就是打,也打不出去。」{\meipi{字字寫金蓮狡猾。}}西門慶道:「怪小奴才兒,休要耍。」因趕着奪那手卷。金蓮道:「你若奪一奪兒,賭箇手段,我就把他扯得稀爛,大家看不成。」{\meipi{即相如持璧睨柱意。}}西門慶笑道:「我也沒法了,隨你看完了與他罷麼。你還了他這箇去,他還有箇稀奇物件兒哩,到明日我要了來與你。」金蓮道:「我兒,誰養得你恁乖?{\pangpi{狡甚。}}你拿了來,我方與你這手卷去。」兩箇絮聒了一回。晚夕,金蓮在房中香薰鴛被,款設銀燈,艷粧澡牝,與西門慶展開手卷,在錦帳之中效「於飛」之樂。{\meipi{瓶兒之物轉同金蓮戲弄,則瓶兒不言可知,文章說一是兩之妙。}}看觀聽說:巫蠱魘昧之物,自古有之。金蓮自從叫劉瞎子回背之後,不上幾時,使西門慶變嗔怒而為寵愛,化憂辱而為歡娛,再不敢制他。正是:饒你奸似鬼,也吃洗脚水。有詞為證:

\begin{myquote}
記得書齋乍會時,雲蹤雨跡少人知。曉來鸞鳳棲雙枕,剔盡銀燈半吐輝。思往事,夢魂迷,今宵喜得效于飛。顛鸞倒鳳無窮樂,從此雙雙永不離。
\end{myquote}

 

