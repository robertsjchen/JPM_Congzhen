\includepdf[pages={5,6},fitpaper=false]{tst.pdf}
\chapter*{第三回 定挨光虔婆受賄 設圈套浪子私挑}
\addcontentsline{toc}{chapter}{第三回 定挨光虔婆受賄 設圈套浪子私挑}
\markboth{{\titlename}卷之一}{第三回 定挨光虔婆受賄 設圈套浪子私挑}


詩曰:

\begin{myquote} 
乍對不相識,徐思似有情。\\盃前交一面,花底戀雙睛。\\傞俹驚新態,含糊問舊名。\\影含今夜燭,心意幾交橫。
\end{myquote} 

話說西門慶央王婆,一心要會那雌兒一面,便道:「乾娘,你端的與我說這件事成,我便送十兩銀子與你。」王婆道:「大官人,你聽我說:但凡『挨光』的兩箇字最難。怎的是『挨光』?比如今俗呼『偷情』就是了。要五件事俱全,方纔行的。第一要潘安的貌;第二要驢大行貨;第三要鄧通般有錢;第四要青春少小,就要綿裡針一般,軟款忍耐;第五要閑工夫。此五件,喚做『潘驢鄧小閑』。都全了,此事便獲得着。」西門慶道:「實不瞞你說,這這五件事我都有。第一件,我的貌雖比不得潘安,也充得過;第二件,我小時在三街兩巷遊串,也曾養得好大龜;第三,我家裡也有幾貫錢財,雖不及鄧通,也頗得過日子;第四,我最忍耐;他便打我四百頓,休想我回他一拳;第五,我最有閑工夫,不然如何來得恁勤。乾娘,你自作成,完備了時,我自重重謝你。」王婆道:「大官人,你說五件事都全,我知道還有一件事打攪,也多是成不得。」西門慶道:「且說,甚麼一件事打攪?」王婆道:「大官人休怪老身直言,但凡挨光最難十分,有使錢到九分九釐,也有難成處。我知你從來慳吝,不肯胡亂便使錢,只這件打攪。」西門慶道:「這箇容易,我只聽你言語便了。」王婆道:「若大官人肯使錢時,老身有一條妙計,須交大官人和這雌兒會一面。」西門慶道:「端的有甚妙計?」王婆笑道:「今日晚了,且回去,過半年三箇月來商量。」西門慶央及道:「乾娘,你休撒科!自作成我則箇,恩有重報。」王婆笑哈哈道:「大官人卻又慌了。老身這條計,雖然入不得武成王廟,端的強似孫武子教女兵——十捉八九着。今日實對你說了罷:這箇雌兒來歷,雖然微末出身,卻倒百伶百俐,會一手好彈唱,針指女工,百家歌曲,雙陸象棋,無所不知。小名叫做金蓮,娘家姓潘,原是南門外潘裁的女兒,賣在張大戶家學彈唱。後因大戶年老,打發出來,不要武大一文錢,白白與了他為妻。這雌兒等閑不出來,老身無事常過去與他閑坐。他有事亦來請我理會,他也叫我做乾娘。武大這兩日出門早。大官人如幹此事,便買一疋藍紬、一疋白紬、一疋白絹,再用十兩好綿,都把來與老身。老身卻走過去問他借歷日,央及他揀箇好日期,叫箇裁縫來做。他若見我這般說,揀了日期,不肯與我來做時,此事便休了;他若歡天喜地說:『我替你做。』不要我叫裁縫,這光便有一分了。{\meipi{計料如指掌。}}我便請得他來做,就替我縫,這光便二分了。他若來做時,午間我卻安排些酒食點心請他吃。他若說不便當,定要將去家中做,此事便休了;他不言語吃了時,這光便有三分了。這一日你也莫來,直至第三日,晌午前後,你整整齊齊打扮了來,以咳嗽為號,你在門前叫道:『怎的連日不見王乾娘?我買盞茶吃。』我便出來請你入房裡坐吃茶。他若見你便起身來,走了歸去,難道我扯住他不成?此事便休了。他若見你入來,不動身時,這光便有四分了。坐下時,我便對雌兒說道:『這箇便是與我衣服施主的官人,虧殺他。』我便誇大官人許多好處,你便賣弄他針指。若是他不來兜攬答應時,此事便休了;他若口中答應與你說話時,這光便有五分了。我便道:『卻難為這位娘子與我作成出手做,虧殺你兩施主,一箇出錢,一箇出力。不是老身路歧相央,難得這位娘子在這裡,官人做箇主人替娘子澆澆手。』你便取銀子出來,央我買。若是他便走時,難道我扯住他?此事便休了。他若是不動身時,事務易成,這光便有六分了。我卻拿銀子,臨出門時對他說:{\meipi{井井有條。}}『有勞娘子相待官人坐一坐。』他若起身走了家去,我終不成阻擋他?此事便休了。若是他不起身,又好了,這光便有七分了。待我買得東西提在桌子上,便說:『娘子且收拾過生活去,且吃一盃兒酒,難得這官人壞錢。』他不肯和你同桌吃,去了,此事便休了。若是他不起身,此事又好了,這光便有八分了。待他吃得酒濃時,正說得入港,我便推道沒了酒,再交你買,你便拿銀子,又央我買酒去,並菓子來配酒。我把門拽上,關你兩箇在屋裡。他若焦燥跑了歸去時,此事便休了;他若繇我拽上門,不焦躁時,這光便有九分,只欠一分了。只是這一分倒難。大官人你在房裡,便着幾句甜話兒說入去,卻不可燥爆,便去動手動脚打攪了事,那時我不管你。你先把袖子向桌子上拂落一雙筯下去,只推拾筯,將手去他脚上捏一捏。他若鬧吵起來,我自來搭救。此事便休了,再也難成。若是他不做聲時,此事十分光了。這十分光做完備,你怎的謝我?」西門慶聽了大喜道:「雖然上不得淩烟閣,乾娘你這條計,端的絕品好妙計!」王婆道:卻不要忘了許我那十兩銀子。」{\pangpi{要緊的。}}西門慶道:「便得一片橘皮吃,切莫忘了洞庭湖。這條計,乾娘幾時可行?」婆道:「只今晚來有回報。我如今趁武大未歸,過去問他借歷日,細細說與他。你快使人送將紬絹綿子來,休要遲了。」西門慶道:「乾娘,這是我的事,如何敢失信。」于是作別了王婆,離了茶肆,就去街上買了紬絹三疋並十兩清水好綿。家裡叫了玳安兒用氊包包了,一直送入王婆家來。王婆歡喜收下,打發小厮回去。正是:

\begin{myquote} 
巫山雲雨幾時就,莫負襄王築楚臺。
\end{myquote} 

當下王婆收了紬絹綿子,開了後門,走過武大家來。那婦人接着,走去樓上坐的。王婆道:「娘子怎的這兩日不過貧家吃茶?」那婦人道:「便是我這幾日身子不快,懶走動的。」王婆道:「娘子家裡有歷日,借與老身看一看,要箇裁衣的日子。」婦人道:「乾娘裁甚衣服?」王婆道:「便是因老身十病九痛,怕一時有些山高水低,我兒子又不在家。」婦人道:「大哥怎的一向不見?」王婆道:「那厮跟了箇客人在外邊,不見箇音信回來,老身日逐耽心不下。」婦人道:「大哥今年多少年紀?」王婆道:「那厮十七歲了。」婦人道:「怎的不與他尋箇親事,{\pangpi{工夫全做在此等處!}}與乾娘也替得手?」王婆道:「因是這等說,家中沒人。待老身東梬西補的來,早晚要替他尋下箇兒。等那厮來,卻再理會。見如今老身白日黑夜只發喘咳嗽,身子打碎般睡不倒的只害疼,一時先要預備下送終衣服。難得一箇財主官人,常在貧家吃茶,但凡他宅裡看病,買使女,說親,見老身這般本分,大小事兒無不管顧老身。又布施了老身一套送終衣料,{\meipi{說起便及送終,此亦王婆之諱。}}紬絹表裡俱全,又有若干好綿,放在家裡一年有餘,不能勾做得。今年覺得好生不濟,不想又撞着閏月,趁着兩日倒閑,要做又被那裁縫勒掯,只推生活忙,不肯來做。老身說不得這苦也!」那婦人聽了,笑道:「只怕奴家做得不中意。{\pangpi{煞是技癢。}}若是不嫌時,奴這幾日倒閑,出手與乾娘做如何?」那婆子聽了,堆下笑來說道:「若得娘子貴手做時,老身便死也得好處去。久聞娘子好針指,只是不敢來相央。」那婦人道:「這箇何妨!既是許了乾娘,務要與乾娘做了,將歷日去交人揀了黃道好日,奴便動手。」王婆道:「娘子休推老身不知,你詩詞、百家、曲兒內字樣,你不知識了多少,如何交人看歷日?」婦人微笑道:「奴家自幼失學。」婆子道:「好說,好說。」便取歷日遞與婦人。婦人接在手內,看了一回,道:「明日是破日,後日也不好,直到外後日方是裁衣日期。」王婆一把手取過歷頭來,掛在墻上,便道:「若得娘子肯與老身做時,就是一點福星。{\pangpi{老奸甚巧。}}何用選日!老身也曾央人看來,說明日是箇破日,老身只道裁衣日不用破日,我不忌他。」那婦人道:「歸壽衣服,正用破日便好。」王婆道:「既是娘子肯作成,老身膽大,只是明日起動娘子,到寒家則箇。」婦人道:「何不將過來做?」王婆道:「便是老身也要看娘子做生活,又怕門首沒人。」婦人道:「既是這等說,奴明日飯後過來。」那婆子千恩萬謝下樓去了,當晚回覆了西門慶話,約定後日準來。當夜無話。

次日清晨,王婆收拾房內乾淨,預備下針線,安排了茶水,在家等候。且說武大吃了早飯,挑着担兒自出去了。那婦人把簾兒掛了,分付迎兒看家,從後門走過王婆家來。那婆子歡喜無限,接入房裡坐下,便濃濃點一盞胡桃松子泡茶,與婦人吃了。抹得桌子乾淨,便取出那紬絹三疋來。婦人量了長短,裁得完備,縫將起來。婆子看了,口裡不住喝采道:「好手段,老身也活了六七十歲,眼裡眞箇不曾見這般好針指!」那婦人縫到日中,王婆安排些酒食請他,又下了一筯麵與那婦人吃。再縫一歇,將次晚來,便收拾了生活,自歸家去。恰好武大挑担兒進門,婦人拽門下了簾子。武大入屋裡,看見老婆面色微紅,問道:「你那裡來?」婦人應道:「便是間壁乾娘央我做送終衣服,日中安排些酒食點心請我吃。」武大道:「你也不要吃他的纔是,我們也有央及他處。他便央你做得衣裳,你便自歸來吃些點心,不値得甚麼,便攪撓他。你明日再去做時,帶些錢在身邊,也買些酒食與他回禮。常言道:遠親不如近隣,休要失了人情。他若不肯交你還禮時,你便拿了生活來家,做還與他便了。」正是:

\begin{myquote} 
阿母牢籠設計深,大郎愚鹵不知音。\\帶錢買酒酬奸詐,卻把婆娘自送人。{\pangpi{語俗而眞。}}
\end{myquote} 

婦人聽了武大言語,當晚無話。

次日飯後,武大挑担兒出去了,王婆便踅過來相請。婦人去到他家屋裡,取出生活來,一面縫來。王婆忙點茶來與他吃了茶。看看縫到日中,那婦人向袖中取出三百文錢來,向王婆說道:「乾娘,奴和你買盞酒吃。」王婆道:「啊呀,那裡有這箇道理。老身央及娘子在這裡做生活,如何交娘子倒出錢,婆子的酒食,不到吃傷了哩!」{\meipi{三字非謙,卻是定論。}}那婦人道:「卻是拙夫分付奴來,若是乾娘見外時,只是將了家去,做還乾娘便了。」那婆子聽了道:「大郎直恁地曉事!既然娘子這般說時,老身且收下。」這婆子生怕打攪了事,自又添錢去買好酒好食來,殷勤相待。看官聽說:但凡世上婦人,繇你十分精細,被小意兒縱十箇九箇着了道兒。這婆子安排了酒食點心,和那婦人吃了。再縫了一歇,看看晚來,千恩萬謝歸去了。

話休絮煩。第三日早飯後,王婆只張武大出去了,便走過後後門首叫道:「娘子,老身大膽。」那婦人從樓上應道:「奴卻待來也。」兩箇厮見了,來到王婆房裡坐下,取過生活來縫。那婆子點茶來吃,自不必說。婦人看看縫到晌午前後。卻說西門慶巴不到此日,{\pangpi{也好一等了。}}打選衣帽齊齊整整,身邊帶着三五兩銀子,手裡拿着洒金川扇兒,搖搖擺擺逕往紫石街來。到王婆門首,便咳嗽道:「王乾娘,連日如何不見?」那婆子瞧科,便應道:「兀的誰叫老娘?」西門慶道:「是我。」那婆子趕出來看了,笑道:「我只道是誰,原來是大官人!你來得正好,且請入屋裡去看一看。」把西門慶袖子只一拖,拖進房裡來,對那婦人道:「這箇便是與老身衣料施主官人。」西門慶睜眼看着那婦人:雲鬟疊翠,粉面生春,上穿白布衫兒,桃紅裙子,藍比甲,正在房裡做衣服。見西門慶過來,便把頭低了。{\pangpi{媚致。}}這西門慶連忙向前屈身唱喏。那婦人隨即放下生活,還了萬福。王婆便道:「難得官人與老身段疋紬絹,放在家一年有餘,不曾得做,虧殺隣家這位娘子,出手與老身做成全了。眞箇是布機也似好針線,縫的又好又密,眞箇難得!大官人,你過來且看一看。」西門慶拿起衣服來看了,一面喝采,口裡道:「這位娘子,傳得這等好針指,神仙一般的手段!」那婦人低頭笑道:「官人休笑話。」{\pangpi{來了。}}西門慶故問王婆道:「乾娘,不敢動問,這位娘子是誰家宅上的娘子?」王婆道:「你猜。」西門慶道:「小人如何猜得着。」王婆哈哈笑道:「大官人你請坐,我對你說了罷。」那西門慶與婦人對面坐下。那婆子道:「好交大官人得知罷,你那日屋簷下走,打得正好。」西門慶道:「就是那日在門首叉竿打了我的?倒不知是誰家宅上娘子?」婦人分外把頭低了一低,笑道:「那日奴誤冲撞,官人休怪!」{\meipi{嬌情慾絕。}}西門慶連忙應道:「小人不敢。」王婆道:「就是這位,卻是間壁武大娘子。」西門慶道:「原來如此,小人失瞻了。」王婆因望婦人說道:「娘子你認得這位官人麼?」婦人道:「不識得。」婆子道:「這位官人,便是本縣裡一箇財主,知縣相公也和他來往,叫做西門大官人。家有萬萬貫錢財,在縣門前開生藥鋪。家中錢過北斗,米爛成倉,黃的是金,白的是銀,圓的是珠,放光的是寶,也有犀牛頭上角,大象口中牙。他家大娘子,也是我說的媒,是吳千戶家小姐,生得百伶百俐。」因問:「大官人,怎的不過貧家吃茶?」西門慶道:「便是家中連日小女有人家定了,不得閑來。」婆子道:「大姐有誰家定了?怎的不請老身去說媒?」西門慶道:「被東京八十萬禁軍楊提督親家陳宅定了。他兒子陳敬濟纔十七歲,還上學堂。不是也請乾娘說媒,他那邊有了箇文嫂兒來討帖兒,俺這裡又使常在家中走的賣翠花的薛嫂兒,同做保山,說此親事。乾娘若肯去,到明日下小茶,我使人來請你。」婆子哈哈笑道:「老身哄大官人耍子。俺這媒人們都是狗娘養下來的,他們說親時又沒我,做成的熟飯兒怎肯搭上老身一分?常言道:當行壓當行。到明日娶過了門時,老身胡亂三朝五日,拿上些人情去走走,討得一張半張桌面,到是正經。怎的好和人鬬氣!」兩箇一遞一句說了一回。婆子只顧誇獎西門慶,口裡假嘈,那婦人便低了頭縫針線。

\begin{myquote} 
水性從來是女流,背夫常與外人偷。\\金蓮心愛西門慶,淫蕩春心不自繇。
\end{myquote} 

西門慶見金蓮有幾分情意歡喜,恨不得就要成雙。王婆便去點兩盞茶來,遞一盞西門慶,一盞與婦人,說道:「娘子相待官人吃些茶。」旋又看着西門慶,把手在臉上摸一摸,西門慶已知有五分光了。自古「風流茶說合,酒是色媒人」。王婆便道:「大官人不來,老身也不敢去宅上相請。一者緣法撞遇,二者來得正好。常言道:一客不煩二主。大官人便是出錢的,這位娘子便是出力的,虧殺你這兩位施主。不是老身路歧相煩,難得這位娘子在這裡,官人好與老身做箇主人,拿出些銀子買些酒食來,與娘子澆澆手,如何?」西門慶道:「小人也見不到這裡,有銀子在此。」便向茄袋裡取出來,約有一兩一塊,遞與王婆,交備辦酒食。那婦人便道「不消生受。」口裡說着恰不動身。王婆接了銀子,臨出門便道:「有勞娘子相陪大官人坐一坐,我去就來。」那婦人道:「乾娘免了罷。」卻亦不動身。{\meipi{身不動處;正是心動處。}}王婆便出門去了,丟下西門慶和那婦人在屋裡。這西門慶一雙眼不轉睛,只看着那婦人。那婆娘也把眼來偷睃西門慶,又低着頭做生活。不多時,王婆買了見成肥鵝燒鴨、熟肉鮮鮓、細巧菓子,歸來盡把盤碟盛了,擺在房裡桌子上。看那婦人道:「娘子且收拾過生活,吃一盃兒酒。」那婦人道:「你自陪大官人吃,奴卻不當。」那婆子道:「正是專與娘子澆手,如何卻說這話!」一面將盤饌卻擺在面前,三人坐下,把酒來斟。西門慶拿起酒盞來道:「乾娘相待娘子滿飲幾盃。」婦人謝道:「奴家量淺,吃不得。」王婆道:「老身得知娘子洪飲,且請開懷吃兩盞兒。」那婦人一面接酒在手,向二人各道了萬福。西門慶拿起筯來說道:「乾娘替我勸娘子些菜兒。」那婆子揀好的遞將過來與婦人吃。一連斟了三巡酒,那婆子便去燙酒來。西門慶道:「小人不敢動問,娘子青春多少?」婦人低頭應道:「二十五歲。」西門慶道:「娘子到與家下賤內同庚,也是庚辰屬龍的。他是八月十五日子時。」婦人又迴應道:「將天比地,折殺奴家。」王婆便插口道:「好箇精細的娘子,百伶百俐,又不枉做得一手好針線。諸子百家,雙陸象棋,折牌道字皆通。一筆好寫。」西門慶道:「卻是那裡去討。」王婆道:「不是老身說是非,大官人宅上有許多,那裡討得一箇似娘子的!」西門慶道:「便是這等,一言難盡。只是小人命薄,不曾招得一箇好的在家裡。」王婆道:「大官人先頭娘子須也好。」西門慶道:「休說!我先妻若在時,卻不恁的家無主,屋到豎。如今身邊枉自有三五七口人吃飯,都不管事。」婆子嘈道:「連我也忘了,沒有大娘子得幾年了?」西門慶道:「說不得,小人先妻陳氏,雖是微末出身,卻倒百伶百俐,{\meipi{語俱有意。}}是件都替的我。如今不幸他沒了,已過三年來。今繼娶這箇賤累,又常有疾病,不管事,家裡的勾當都七顛八倒。為何小人只是走了出來?在家裡時,便要嘔氣。」婆子道:「大官人,休怪我直言,你先頭娘子並如今娘子,也沒這大娘子這手針線,這一表人物。」西門慶道:「便是房下們也沒這大娘子一般兒風流。」{\pangpi{妙。}}{\meipi{如此情意,較武二官何如?}}那婆子笑道:「官人,你養的外宅東街上住的,如何不請老身去吃茶?」西門慶道:「便是唱慢曲兒的張惜春。我見他是路歧人,不喜歡。」婆子又道:「官人你和抅欄中李嬌兒卻長久。」西門慶道:「這箇人見今已娶在家裡。若得他會當家時,自冊正了他。」王婆道:「與卓二姐卻相交得好?」西門慶道:「卓丟兒別要說起,我也娶在家做了第三房。近來得了箇細疾,卻又沒了。」婆子道:「耶嚛,耶嚛!若有似大娘子這般中官人意的,來宅上說,不妨事麼?」西門慶道:「我的爹娘俱已沒了,我自主張,誰敢說箇不字?」王婆道:「我自說耍,急切便那裡有這般中官人意的!」{\meipi{遠在千里,近只目前。}}西門慶道:「做甚麼便沒?只恨我夫妻緣分上薄,自不撞着哩。」西門慶和婆子一遞一句,說了一回。王婆道:「正好吃酒,卻又沒了。官人休怪老身差撥,買一瓶兒酒來吃如何?」西門慶便向茄袋內,還有三四兩散銀子,都與王婆,說道:「乾娘,你拿了去,要吃時只顧取來,多的乾娘便就收了。」那婆子謝了起身。睃那粉頭時,三鍾酒下肚,鬨動春心,又自兩箇言來語去,都有意了,只低了頭不起身。正是:

\begin{myquote} 
眼意眉情卒未休,姻緣相湊遇風流。\\王婆貪賄無他技,一味花言巧舌頭。
\end{myquote} 
 

