\includepdf[pages={75,76},fitpaper=false]{tst.pdf}
\chapter*{第三十八回 王六兒棒槌打搗鬼 潘金蓮雪夜弄琵琶}
\addcontentsline{toc}{chapter}{第三十八回 王六兒棒槌打搗鬼 潘金蓮雪夜弄琵琶}
\markboth{{\titlename}卷之四}{第三十八回 王六兒棒槌打搗鬼 潘金蓮雪夜弄琵琶}


詞曰:

\begin{myquote}
銀箏宛轉,促柱調弦,聲遶梁間。巧作秦聲獨自憐。指輕妍,風迴雪旋,緩揚清曲,響奪鈞天。說甚麼別鶴烏啼,試按「羅敷陌上」篇,休按「羅敷陌上」篇。

\raggedleft{——右調《綿搭絮》\rightquadmargin}
\end{myquote}

話說馮婆子走到前廳角門首,看見玳安在廳槅子前,拏着茶盤兒伺候。玳安望着馮媽努嘴兒:「你老人家先往那裡去,俺爹和應二爹說了話就起身。已先使棋童兒送酒去了。」那婆子聽見,兩步做一步走的去了。原來應伯爵來說:「攬頭李智、黃四派了年例三萬香蠟等料錢糧下來,該一萬兩銀子,也有許多利息。上完了批,就在東平府見關銀子,來和你計較,做不做?」西門慶道:「我那裡做他!攬頭以假充眞,買官讓官。我衙門裡搭了事件,還要動他。我做他怎的!」伯爵道:「哥若不做,叫他另搭別人。你只借二千兩銀子與他,每月五分行利,叫他關了銀子還你,你心下何如?」西門慶道:「既是你的分上,我挪一千銀子與他罷。如今我庄子收拾,還沒銀子哩。」伯爵見西門慶吐了口兒,說道:「哥若十分沒銀子,看怎麼再撥五百兩貨物兒,湊箇千五兒與他罷,他不敢少下你的。」西門慶道:「他少下我的,我有法兒處。又一件,應二哥,銀子便與他,只不叫他打着我的旗兒,在外邊東誆西騙。我打聽出來,只怕我衙門監裡放不下他。」伯爵道:「哥說的什麼話,典守者不得辭其責。他若在外邊打哥的旗兒,常沒事罷了,若壞了事,要我做甚麼?哥你只顧放心,但有差池,我就來對哥說。說定了,我明日叫他好寫文書。」西門慶道:「明日不教他來,我有勾當。叫他後日來。」說畢,伯爵去了。

西門慶教玳安伺候馬,帶上眼紗,問棋童去沒有。玳安道:「來了,取挽手兒去了。」不一時,取了挽手兒來,打發西門慶上馬,逕往牛皮巷來。不想韓道國兄弟韓二搗鬼,耍錢輸了,吃的光睜睜兒的,走來哥家,問王六兒討酒吃。袖子裡掏出一條小腸兒來,說道:「嫂,我哥還沒來哩,我和你吃壺燒酒。」那婦人恐怕西門慶來,又見老馮在廚下,不去兜攬他,說道:「我是不吃。你要吃拏過一邊吃去,我那裡耐煩?你哥不在家,招是招非的,又來做什麼?」那韓二搗鬼,把眼兒涎睜着,又不去,看見桌底下一罈白泥頭酒,貼着紅紙帖兒,問道:「嫂子,是那裡酒?開啟篩壺來俺每吃。耶嚛!你自受用!」婦人道:「你趁早兒休動,是宅里老爹送來的,你哥還沒見哩。等他來家,有便倒一甌子與你吃。」韓二道:「等什麼哥?就是皇帝爺的,我也吃一鍾兒!」纔待搬泥頭,被婦人劈手一推,奪過酒來,提到屋裡去了。把二搗鬼仰八叉推了一交,半日扒起來,惱羞變成怒,口裡喃喃吶吶罵道:「賊淫婦,我好意帶將菜兒來,見你獨自一箇冷落落,和你吃盃酒。你不理我,倒推我一交。我教你不要慌,你另叙上了有錢的漢子,不理我了,要把我開啟,故意兒囂我,訕我,又趍我。休叫我撞見,我叫你這不値錢的淫婦,白刀子進去紅刀子出來!」婦人見他的話不妨頭,一點紅從耳邊起,須臾紫脹了雙腮,便取棒槌在手,趕着打出來,{\meipi{棒槌正好搗鬼。}}罵道:「賊餓不死的殺才!你那裡𠳹醉了,來老娘這裡撒野火兒。老娘手裡饒你不過!」那二搗鬼口裡喇喇哩哩罵淫婦,直罵出門去。不想西門慶正騎馬來,見了他,問是誰,婦人道:「情知是誰,是韓二那厮,見他哥不在家,要便耍錢輸了,吃了酒來毆我。有他哥在家,常時撞見打一頓。」那二搗鬼看見,一溜烟跑了。西門慶又道:「這少死的花子,等我明日到衙門裡與他做功德!」婦人道:「又叫爹惹惱。」西門慶道:「你不知,休要慣了他。」婦人道:「爹說的是。自古良善被人欺,慈悲生患害。」一面讓西門慶明間內坐。西門慶分付棋童回馬家去,叫玳安兒:「你在門首看,但掉着那光棍的影兒,就與我鎖在這裡,明日帶到衙門裡來。」玳安道:「他的魂兒聽見爹到,不知走的那裡去了。」西門慶坐下。婦人見畢禮,連忙屋裡叫丫鬟錦兒拏了一盞果仁茶出來,與西門慶吃,就叫他磕頭。西門慶道:「也罷,到好箇孩子,你且將就使着罷。」又道:「老馮在這裡,怎的不替你拏茶?」婦人道:「馮媽媽他老人家,我央及他廚下使着手哩。西門慶又道:「頭裡我使小厮送來的那酒,是箇內臣送我的竹葉清。裡頭有許多藥味,甚是峻利。我前日見你這裡打的酒,都吃不上口,我所以拏的這罈酒來。」婦人又道了萬福,說:「多謝爹的酒,正是這般說,俺每不爭氣,住在這僻巷子裡,又沒箇好酒店,那裡得上樣的酒來吃,只往大街上取去。」西門慶道:「等韓夥計來家,你和他計較,等着獅子街那裡,替你破幾兩銀子買所房子,等你兩口子亦發搬到那裡住去罷。鋪子裡又近,買東西諸事方便。」婦人道:「爹說的是。看你老人家怎的可憐見,離了這塊兒也好。就是你老人家行走,也免了許多小人口嘴,咱行的正,也不怕他。{\pangpi{虧他說得出。}}爹心裡要處自情處,他在家和不在家一箇樣兒,也少不的打這條路兒來。」{\meipi{韓道國明放一着,又反形出夾。}}說一回,房裡放下桌兒,請西門慶進去寬了衣服坐。

須臾,安排酒菜上來,婦人陪定,把酒來斟。不一時,兩箇並肩疊股而飲。吃的酒濃時,兩箇脫剝上床交歡,自在玩耍。婦人早已床炕上鋪的厚厚的被褥,被裡薰的噴鼻香。西門慶見婦人好風月,一徑要打動他。家中袖了一箇錦包兒來,開啟,裡面銀托子、相思套、硫黃圈、藥煮的白綾帶子、懸玉環、封臍膏、勉鈴,一弄兒淫器。那婦人仰臥枕上,玉腿高蹺,口舌內吐。西門慶先把勉鈴教婦人自放牝內,然後將銀托束其根,硫黃圈套其首,臍膏貼於臍上。婦人以手導入牝中,兩相迎湊,漸入大半。婦人呼道:「達達!我只怕你墩的腿痠,拏過枕頭來,你墊着坐,等我淫婦自家動罷。」{\meipi{以淫婦自稱,妙絕。}}又道:「只怕你不自在,你把淫婦腿弔着㒲,你看好不好?」西門慶眞箇把他脚帶解下一條來,拴他一足,弔在床槅子上低着拽,拽的婦人牝中之津如蝸之吐涎,綿綿不絕,又拽出好些白漿子來。西門慶問道:「你如何流這些白?」纔待要抹去,婦人道:「你休抹,等我吮咂了罷。」於是蹲跪在他面前吮吞數次,嗚咂有聲。咂的西門慶淫心輒起,弔過身子,兩箇幹後庭花。龜頭上有硫黃,濡研難澀。婦人蹙眉隱忍,半晌僅沒其稜。西門慶頗作抽送,而婦人用手摸之,漸入大半,把屁股坐在西門慶懷裡,回首流眸,作顫聲叫:「達達!慢着些,後越發粗大,教淫婦怎生挨忍。」西門慶且扶起股,觀其出入之勢,因叫婦人小名:「王六兒,我的兒,你達不知心裡怎的只好這一樁兒,不想今日遇你,正可我之意。我和你明日生死難開。」婦人道:「達達,只怕後來耍的絮煩了,把奴不理怎了?」西門慶道:「相交下來,纔見我不是這樣人。」說話之間,兩箇幹勾一頓飯時。西門慶令婦人沒高低淫聲浪語叫着纔過。婦人在下,一面用手舉股承受其精,樂極情濃,一泄如注。已而抽出那話來,帶着圈子,婦人還替他吮咂淨了,兩箇方纔並頭交股而臥。正是:一般滋味美,好耍後庭花。有詞為證:

\begin{myquote}
美冤家,一心愛折後庭花。尋常只在門前裡走,又被開路先鋒把住了他。放在戶中難禁受。轉絲韁,勒回馬,親得勝弄的我身上麻,蹴損了奴的粉臉那丹霞。
\end{myquote}

西門慶與婦人摟抱到二鼓時分,小厮馬來接,方纔起身回家。到次日,到衙門裡差了兩箇緝捕,把二搗鬼拏到提刑院,只當做掏摸土賊,不繇分說,一夾二十,打的順腿流血。睡了一箇月,險不把命花了。往後嚇的影也再不敢上婦人門纏攪了。正是:

\begin{myquote}
恨小非君子,無毒不丈夫。
\end{myquote}

遲了幾日,來保、韓道國一行人東京回來,備將前事對西門慶說:「翟管家見了女子,甚是歡喜,說爹費心。留俺府裡住了兩日,討了回書。送了爹一匹青馬,封了韓夥計女兒五十兩銀子禮錢,又與了小的二十兩盤纏。」西門慶道:「勾了。」看了回書,書中無非是知感不盡之意。自此兩家都下眷生名字,稱呼親家,{\meipi{外面援着親家,似支離可笑,然於內細思之,實亦不愧。}}不在話下。韓道國與西門慶磕頭拜謝回家。西門慶道:「韓夥計,你還把你女兒這禮錢收去,也是你兩口兒恩養孩兒一場。」韓道國再三不肯收,說道:「蒙老爹厚恩,禮錢是前日有了。這銀子小人怎好又受得?從前累的老爹好少哩!」西門慶道:「你不依,我就惱了。你將回家,不要花了,我有箇處。」那韓道國就磕頭謝了,拜辭回去。

老婆見他漢子來家,滿心歡喜,一面接了行李,與他拂了塵上,問他長短:「孩子到那裡好麼?」這道國把往回一路的話,告訴一遍,說:「好人家,孩子到那裡,就與了三間房,兩箇丫鬟伏侍,衣服頭面不消說。第二日,就領了後邊見了太太。翟管家甚是歡喜,留俺們住了兩日,酒飯連下人都吃不了。又與了五十兩禮錢。我再三推辭,大官人又不肯,還叫我拏回來了。」因把銀子與婦人收了。婦人一塊石頭方落地,因和韓道國說:「咱到明日,還得一兩銀子謝老馮。你不在,虧他常來做作伴兒。大官人那裡,也與了他一兩。」正說着,只見丫頭過來遞茶。韓道國道:「這箇是那裡大姐?」婦人道:「這箇是咱新買的丫頭,名喚錦兒。過來與你爹磕頭!」磕了頭,丫頭往廚下去了。老婆如此這般,把西門慶勾搭之事,告訴一遍,「自從你去了,來行走了三四遭,纔使四兩銀子買了這箇丫頭。但來一遭,帶一二兩銀子來。第二的不知高低,氣不憤走來這裡放水。被他撞見了,拏到衙門裡,打了箇臭死,至今再不敢來了。大官人見不方便,許了要替我每大街上買一所房子,叫咱搬到那裡住去。」韓道國道:「嗔道他頭裡不受這銀子,教我拏回來休要花了,原來就是這些話了。」婦人道:「這不是有了五十兩銀子,他到明日,已定與咱多添幾兩銀子,看所好房兒。也是我輸了身一場,且落他些好供給穿戴。」韓道國道:「等我明日往鋪子裡去了,他若來時,你只推我不知道,休要怠慢了他,凡事奉承他些兒。如今好容易撰錢,怎麼趕的這箇道路!」{\meipi{老婆偷人,難得道國亦不氣苦。予嘗謂好色甚於好財,觀此,則好財又甚於好色矣。}}老婆笑道:「賊強人,倒路死的!你到會吃自在飯兒,你還不知老娘怎樣受苦哩!」兩箇又笑了一回,打發他吃了晚飯,夫妻收拾歇下。到天明,韓道國宅裡討了鑰匙,開鋪子去了,與了老馮一兩銀子謝他。俱不必細說。

一日,西門慶同夏提刑衙門回來。夏提刑見西門慶騎着一匹高頭點子青馬,問道:「長官那匹白馬怎的不騎,又換了這匹馬?到好一匹馬,不知口裡如何?」西門慶道:「那馬在家歇他兩日兒。這馬是昨日東京翟雲峯親家送來的,{\meipi{說得口角津津榮幸。}}是西夏劉參將送他的。口裡纔四箇牙兒,脚程緊慢都有他的。只是有些毛病兒,快護糟踅蹬。初時騎了路上走,把膘跌了許多,這兩日內吃的好些兒。」夏提刑道:「這馬甚是會行,但只好騎着蹗街道兒罷了,不可走遠了他。論起在咱這裡,也値七八十兩銀子。我學生騎的那馬,昨日又瘸了。今早來衙門裡來,旋拏帖兒問舍親借了這匹馬騎來,甚是不方便。」西門慶道:「不打緊,長官沒馬,我家中還有一匹黃馬,送與長官罷。」夏提刑舉手道:「長官下顧,學生奉價過來。」西門慶道:「不須計較。學生到家,就差人送來。」兩箇走到西街口上,西門慶舉手分路來家。到家就使玳安把馬送去。夏提刑見了大喜,賞了玳安一兩銀子,與了回帖兒,說:「多上覆,明日到衙門裡面謝。」

過了兩月,乃是十月中旬時分。夏提刑家中做了些菊花酒,叫了兩名小優兒,請西門慶一叙,以酬送馬之情。西門慶家中吃了午飯,理了些事務,往夏提刑家飲酒。原來夏提刑備辦一席齊整酒餚,只為西門慶一人而設。見了他來,不勝歡喜,降堦迎接,至廳上叙禮。西門慶道:「如何長官這等費心?」夏提刑道:「今年寒家做了些菊花酒,閑中屈執事一叙,再不敢請他客。」於是見畢禮數,寬去衣服,分賓主而坐。茶罷着棋,就席飲酒敍談,兩箇小優兒在旁彈唱。正是:得多少

\begin{myquote}
金尊進酒浮香蟻,象板催箏唱鷓鴣。
\end{myquote}

不說西門慶在夏提刑家飲酒,單表潘金蓮見西門慶許多時不進他房裡來,每日翡翠衾寒,芙蓉帳冷。那一日把角門兒開着,在房內銀燈高點,靠定幃屏,彈弄琵琶。等到二三更,使春梅連瞧數次,不見動靜。正是:銀箏夜久殷勤弄,寂寞空房不忍彈。取過琵琶,橫在膝上,低低彈了箇《二犯江兒水》,唱道:

\begin{myquote}
悶把幃屏來靠,和衣強睡倒。
\end{myquote}

猛聽得房簷上鐵馬兒一片聲響,只道西門慶敲的門環兒響,連忙使春梅去瞧。春梅回道:「娘,錯了,是外邊風起落雪了。」{\meipi{人只知鬲越相思之苦,孰知眼前相思之苦如此。人只知野合想思之苦,孰知閨閫夫妻相思之苦尤甚。可勝嘆息。}}婦人又彈唱道:

\begin{myquote}
聽風聲嘹亮,雪灑窻寮,任冰花片片飄。
\end{myquote}

一回兒燈昏香盡,心裡欲待去剔,見西門慶不來,又意兒懶的動彈了。唱道:

\begin{myquote}
懶把寶燈挑,慵將香篆燒。捱過今宵,怕到明朝。細尋思,這煩惱何日是了?想起來,今夜裡心兒內焦,誤了我青春年少!你撇的人,有上稍來沒下稍。
\end{myquote}

且說西門慶約一更時分,從夏提刑家吃了酒歸來。一路天氣陰晦,空中半雨半雪下來,落在衣服上都化了。不免打馬來家,小厮打着燈籠,就不到後邊,逕往李瓶兒房來。李瓶兒迎着,一面替他拂去身上雪霰,接了衣服。止穿綾敞衣,坐在床上,就問:「哥兒睡了不曾?」李瓶兒道:「小官兒頑了這回,方睡下了。」迎春拏茶來吃了。李瓶兒問,「今夜吃酒來的早?」西門慶道:「夏龍溪因我前日送了他那匹馬,今日為我費心,治了一席酒請我,又叫了兩箇小優兒。和他坐了這一回,見天氣下雪,來家早些。」李瓶兒道:「你吃酒,叫丫頭篩酒來你吃。大雪裡來家,只怕冷哩。」西門慶道:「還有那葡萄酒,你篩來我吃。今日他家吃的是造的菊花酒,我嫌他殽香殽氣的,我沒大好生吃。」

於是迎春放下桌兒,就是幾碟嗄飯、細巧果菜之類。李瓶兒拏杌兒在旁邊坐下。桌下放着一架小火盆兒。這裡兩箇吃酒,潘金蓮在那邊屋裡冷清清,獨自一箇兒坐在床上。懷抱着琵琶,桌上燈昏燭暗。待要睡了,又恐怕西門慶一時來;待要不睡,又是那盹困,又是寒冷。不免除去冠兒,亂挽烏雲,把帳兒放下半邊來,擁衾而坐,正是:

\begin{myquote}
倦倚綉床愁懶睡,低垂錦帳繡衾空。\\早知薄倖輕拋棄,辜負奴家一片心。
\end{myquote}

又唱道:

\begin{myquote}
懊恨薄情輕棄,離愁閑自惱。
\end{myquote}

又喚春梅過來:「你去外邊再瞧瞧,你爹來了沒有?快來回我話。」那春梅走去,良久回來,說道:「娘還認爹沒來哩,爹來家不耐煩了,在六娘房裡吃酒的不是?」{\meipi{數語傷心之極。}}這婦人不聽罷了,聽了如同心上戳上幾把刀子一般,罵了幾句負心賊,繇不得撲簌簌眼中流下淚來。一徑把那琵琶兒放得高高的,口中又唱道:

\begin{myquote}
心癢痛難搔,愁懷悶自焦。讓了甜桃,去尋酸棗。奴將你這定盤星兒錯認了。想起來,心兒裡焦,誤了我青春年少。你撇的人,有上稍來沒下稍。
\end{myquote}

西門慶正吃酒,忽聽見彈的琵琶聲,便問:「是誰彈琵琶?」迎春答道:「是五娘在那邊彈琵琶響。」李瓶兒道:「原來你五娘還沒睡哩。綉春,你快去請你五娘來吃酒。你說俺娘請哩。」那綉春去了。李瓶兒忙分付迎春:「安下箇坐兒,放箇鍾筯在面前。」良久,綉春走來說:「五娘摘了頭,不來哩。」李瓶兒道:「迎春,你再去請五娘去。你說,娘和爹請五娘哩。」不多時,迎春來說:「五娘把角門兒關了,說吹了燈,睡下了。」西門慶道:「休要信那小淫婦兒,等我和你兩箇拉他去,務要把他拉了來。咱和他下盤棋耍子。」於是和李瓶兒同來打他角門。打了半日,春梅把角門子開了。西門慶拉着李瓶兒進入他房中,只見婦人坐在帳中,琵琶放在旁邊。西門慶道:「怪小淫婦兒,怎的兩三轉請着你不去!」金蓮坐在床上,紋絲兒不動,把臉兒沉着,半日說道:「那沒時運的人兒,丟在這冷屋裡,隨我自生自活的,又來瞅採我怎的?沒的空費了你這箇心,留着別處使。」{\meipi{語雖酸甚,臉雖皮甚,然情自可憐。}}西門慶道:「怪奴才!『八十歲媽媽沒牙——有那些唇說的』?李大姐那邊請你和他下盤棋兒,只顧等你不去了。」李瓶兒道:「姐姐,可不怎的。我那屋裡擺下棋子了,咱們閑着下一盤兒,賭盃酒吃。」金蓮道:「李大姐,你們自去,我不去。你不知我心裡不耐煩,我如今睡也,比不的你們心寬閑散。我這兩日只有口遊氣兒,黃湯淡水誰嘗着來?我成日睜着臉兒過日子哩!」{\meipi{當此時此景,金蓮固雖傷心,然西門慶亦難為情。}}西門慶道:「怪奴才,你好好兒的,怎的不好?你若心內不自在,早對我說,我好請太醫來看你。」金蓮道:「你不信,叫春梅拏過我的鏡子來,等我瞧。這兩日,瘦的相箇人模樣哩!」春梅把鏡子眞箇遞在婦人手裡,燈下觀看。正是:

\begin{myquote}
羞對菱花拭粉粧,為郎憔瘦減容光。\\閉門不管閑風月,任你梅花自主張。
\end{myquote}

西門慶拏過鏡子也照了照,說道:「我怎麼不瘦?」金蓮道:「拏甚麼比你!你每日碗酒塊肉,吃的肥胖胖的,專一只奈何人。」被西門慶不繇分說,一屁股挨着他坐在床上,摟過脖子來就親了箇嘴,舒手被裡,摸見他還沒脫衣裳,兩隻手齊插在他腰裡去,說道:「我的兒,是箇瘦了些。」金蓮道:「怪行貨子,好冷手,冰的人慌!莫不我哄了你不成?我的苦惱,誰人知道,眼淚打肚裡流罷了。」亂了一回,西門慶還把他強死強活拉到李瓶兒房內,下了一盤棋,吃了一回酒。臨起身,李瓶兒見他這等臉酸,把西門慶攛掇過他這邊歇了。正是:得多少

\begin{myquote}
腰瘦故知閒事惱,淚痕只為別情濃。
\end{myquote}

