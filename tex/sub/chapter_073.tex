\includepdf[pages={145,146},fitpaper=false]{tst.pdf}
\chapter*{第七十三回 潘金蓮不憤憶吹簫 西門慶新試白綾帶}
\addcontentsline{toc}{chapter}{第七十三回 潘金蓮不憤憶吹簫 西門慶新試白綾帶}
\markboth{{\titlename}卷之八}{第七十三回 潘金蓮不憤憶吹簫 西門慶新試白綾帶}


詞曰:

\begin{myquote}
喚多情,憶多情,誰把多情喚我名?喚名人可憎。為多情,轉多情,死向多情心不平。休教情重輕。

\raggedleft{——右調《長相思》\rightquadmargin}
\end{myquote}

話說應伯爵回家去了。西門慶就在藏春塢坐着,看泥水匠打地炕。墻外燒火,安放花草,庶不至煤烟薰觸。忽見平安拿進帖兒,稟說:「帥府周爺差人送分資來了。」盒內封着五封分資:周守備、荊都監、張團練、劉薛二內相,每人五星,粗帕二方,奉引賀敬。西門慶令左右收入後邊,拿回帖打發去了。

且說那日,楊姑娘與吳大妗子、潘姥姥坐轎子先來了,然後薛姑子、大師父、王姑子,並兩箇小姑子妙趣、妙鳳,並郁大姐,都買了盒兒來,與玉樓做生日。月娘在上房擺茶,衆姊妹都在一處陪侍。須臾吃了茶,各人取便坐了。潘金蓮想着要與西門慶做白綾帶兒,即便走到房裡,拿過針線匣,揀一條白綾兒,將磁盒內顫聲嬌藥末兒裝在裡面,周圍用倒口針兒撩縫的甚是細法,預備晚夕要與西門慶雲雨之歡。不想薛姑子驀地進房來,送那安胎氣的衣胞符藥與他。這婦人連忙收過,一面陪他坐的。薛姑子見左右無人,便悄悄遞與他,{\meipi{此時有他說嘴處矣。}}說道:「你揀箇壬子日空心服,到晚夕與官人在一處,管情一度就成胎氣。你看後邊大菩薩,也是貧僧替他安的胎,今已有了半肚子了。我還說箇法兒與你:縫箇錦香囊,我書道硃砂符兒安在裡面,帶在身邊,管情就是男胎,好不準驗。」{\pangpi{又出一奇,妙。}}這婦人聽了,滿心歡喜,一面接了符藥,藏放在箱內。拿過歷日來看,二十九日是壬子日。於是就稱了三錢銀子送與他,說:「這箇不當什麼,拿到家買菜吃。等坐胎之時,我尋疋絹與你做衣穿。」薛姑子道:「菩薩快休計較,我不象王和尚那樣利心重。{\meipi{一味貪婪,卻夾佛法果報出之。說得似餓鬼,似羅剎,又似活菩薩。此輩可笑可憎,莫不具見。}}前者因過世那位菩薩念經,他說我攙了他的主顧,好不和我嚷鬧,到處拿言語䘮我。我的爺,隨他墮業,我不與他爭執。我只替人家行好事,救人苦難。」婦人道:「薛爺,你只行你的事,各人心地不同。我這勾當,你也休和他說。」薛姑子道:「法不傳六耳,我肯和他說!去年為後邊大菩薩喜事,他還說我背地得多少錢,擗了一半與他纔罷了。一箇僧家,戒行也不知,利心又重,得了十方施主錢糧,不修功果,到明日死後,披毛戴角還不起。」說了回話,婦人教春梅:「看茶與薛爺吃。」那姑子吃了茶,又同他到李瓶兒那邊參了靈,方歸後邊來。約後晌時分,月娘放桌兒炕屋裡,請衆堂客並三箇姑子坐的。又在明間內放八仙桌兒,鋪着火盆擺下案酒,與孟玉樓上壽。不一時,瓊漿滿泛,玉斝高擎,孟玉樓打扮的粉粧玉琢,先與西門慶遞了酒,然後與衆姊妹叙禮,安席而坐。陳敬濟和大姐又與玉樓上壽,行畢禮,就在旁邊坐下。廚下壽麵點心添換,一齊拿上來。衆人纔吃酒,只見來安拿進盒兒來說:「應保送人情來了。」西門慶叫月娘收了,就教來安:「送應二娘帖兒去,就請你應二爹和大舅來坐坐。我曉的他娘子兒,明日也是不來,請你二爹來坐坐罷,改日回人情與他就是了。」來安拿帖兒同應保去了。西門慶坐在上面,不覺想起去年玉樓上壽還有李大姐,今日妻妾五箇,只少了他,繇不得心中痛酸,眼中落淚。{\meipi{遍插茱萸少一人,那得不悲!}}

不一時,李銘和兩箇小優兒進來了。月娘分付:「你會唱『比翼成連理』不會?」韓佐道:「小的記得。」纔待拿起樂器來彈唱,被西門慶叫近前,分付:「你唱一套『憶吹簫』我聽罷。」兩箇小優連忙改調唱《集賢賓》「憶吹簫,玉人何處也。」唱了一回,唱到「他為我褪湘裙杜鵑花上血」,潘金蓮見唱此詞,就知西門慶念思李瓶兒之意。{\pangpi{慧心處可愛。}}及唱到此句,在席上故意把手放在臉兒上,這點兒那點兒羞他,{\pangpi{畫。}}說道:「孩兒,那裡『豬八戒走在冷鋪中坐着——你怎的醜的沒對兒』!一箇後婚老婆,又不是女兒,那裡討『杜鵑花上血』來?好箇沒羞的行貨子!」西門慶道:「怪奴才,聽唱罷麼,我那裡曉得什麼。單管胡枝扯葉的。」只見兩箇小優又唱到:「一箇相府內懷春女,忽剌八拋去也。我怎肯恁隨邪,又去把墻花亂折!」那西門慶只顧低着頭留心細聽。{\pangpi{畫。}}須臾唱畢,這潘金蓮就不憤他,兩箇在席上只顧拌嘴起來。月娘有些看不上,便道:「六姐,你也耐煩,兩箇只顧強什麼?楊姑奶奶和他大妗子丟在屋裡,冷清清的,沒箇人兒陪他,你每着兩箇進去陪他坐坐兒,我就來。」當下金蓮和李嬌兒就往房裡去了。

不一時,只見來安來說:「應二娘帖兒送到了。二爹來了,大舅便來。」西門慶道:「你對過請溫師父來坐坐。」因對月娘說:「你分付廚下拿菜出來,我前邊陪他坐去。」又叫李銘:「你往前邊唱罷。」李銘即跟着西門慶出來,到西廂房內陪伯爵坐的。又謝他人情:「明日請令正好歹來走走。」伯爵道:「他怕不得來,家下沒人。」良久,溫秀才到,作揖坐下。伯爵舉手道:「早晨多有累老先生。」溫秀才道:「豈敢。」吳大舅也到了,相見讓位畢,一面琴童兒秉燭來,四人圍煖爐坐定。來安拿春盛案酒擺在桌上。伯爵燈下看見西門慶白綾襖子上,罩着青段五彩飛魚蟒衣,張牙舞爪,頭角崢嶸,揚須鼓鬣,金碧掩映,蟠在身上,唬了一跳,問:「哥,這衣服是那裡的?」西門慶便立起身來,笑道:「你每瞧瞧,猜是那裡的?」伯爵道:「俺每如何猜得着。」西門慶道:「此是東京何太監送我的。{\meipi{賣弄處,鬚眉俱動。}}我在他家吃酒,因害冷,他拿出這件衣服與我披。這是飛魚,因朝廷另賜了他蟒龍玉帶,他不穿這件,就送我了。此是一箇大分上。」伯爵極口誇道:「這花衣服,少說也値幾箇錢兒。此是哥的先兆,到明日高轉做到都督上,愁沒玉帶蟒衣?何況飛魚!只怕穿過界兒去哩!」{\meipi{贊處妙在深一層,方暢其賣弄之意,富貴人家自少此輩不得。}}說着,琴童安放鍾筯,拿酒上來。李銘在面前彈唱。伯爵道:「也該進去與三嫂遞盃酒兒纔好,如何就吃酒?」西門慶道:「我兒,你既有孝順之心,往後邊與三嫂磕箇頭兒就是了,說他怎的?」伯爵道:「磕頭到不打緊,只怕惹人議論我做大不尊,到不如你替我磕箇兒罷。」被西門慶向他頭上打了一下,罵道:「你這狗才,單管恁沒大小!」伯爵道:「有大小到不教孩兒們打了。」兩箇戲說了一回,琴童拿將壽麵來,西門慶讓他三人吃。自己因在後邊吃了,就遞與李銘吃。那李銘吃了,又上來彈唱。伯爵叫吳大舅:「分付曲兒叫他唱。」大舅道:「不要索落他,隨他揀熟的唱罷。」西門慶道:「大舅好聽《瓦盆兒》這一套。」一面令琴童斟上酒,李銘於是箏排雁柱,款定冰弦,唱了一套「叫人對景無言,終日減芳容」,下邊去了。只見來安上來稟說:「廚子家去,請問爹,明日叫幾名答應?」西門慶分付:「六名廚役、二名茶酒,酒筵共五桌,俱要齊備。」來安應諾去了。吳大舅便問:「姐夫明日請甚麼人?」西門慶悉把安郎中作東請蔡九知府說了。吳大舅道:「既明日大巡在姐夫這裡吃酒,又好了。」西門慶道:「怎的說?」吳大舅道:「還是我修倉的事,要在大巡手裡題本,望姐夫明日說說,教他青目青目,到年終考滿之時保舉一二,就是姐夫情分。」西門慶道:「這不打緊。大舅明日寫箇履歷揭帖來,等我取便和他說。」大舅連忙下來打恭。伯爵道:「老舅,你老人家放心,你是箇都根主子,不替你老人家說,再替誰說?管情消不得吹嘘之力,一箭就上垜。」{\meipi{說得兩人都快活,妙舌。}}前邊吃酒到二更時分散了,西門慶打發李銘等出門,就分付:「明日俱早來伺候。」李銘等應諾去了。小厮收進家伙,上房內擠着一屋裡人,聽見前邊散了,都往那房裡去了。

卻說金蓮,只說往他屋裡去,慌的往外走不迭。不想西門慶進儀門來了,他便藏在影壁邊黑影兒裡,看着西門慶進入上房,悄悄走來窻下聽覷。{\meipi{欲為稍菓子打秋菊線索,偏在忙裡下針,寧與人指之為冗為淡,不與人見其神龍首尾,高文妙法,子長以下所無。}}只見玉簫站在堂屋門首,說道:「五娘怎的不進去?」又問:「姥姥怎的不見?」金蓮道:「老行貨子,他害身上疼,往房裡睡去了。」良久,只聽月娘問道:「你今日怎的叫恁兩箇新小王八子?唱又不會唱,只一味『三弄梅花』。」玉樓道:「只你臨了教他唱『鴛鴦浦蓮開』,他纔依了你唱。好兩箇猾小王八子,不知叫什麼名字,一日在這裡只是頑。」西門慶道:「一箇叫韓佐,一箇叫邵謙。」月娘道:「誰曉的他叫什麼謙兒李兒!」不防金蓮躡足潛蹤進去,立在煖炕兒背後,忽說道:「你問他?正經姐姐分付的曲兒不叫他唱,平白胡枝扯葉的教他唱什麼『憶吹簫』,支使的小王八子亂騰騰的,不知依那箇的是。」玉樓「噦」了一聲,扭回頭看見是金蓮,便道:「這箇六丫頭,你在那裡來?猛可說出話來,倒唬我一跳。{\meipi{金蓮幽蹤與玉樓小膽,又作一笑,映出妙手。}}單愛行鬼路兒。你從多咱走在我背後?」小玉道:「五娘在三娘背後,好少一回兒。」金蓮點着頭兒向西門慶道:「哥兒,你膿着些兒罷了。你那小見識兒,只說人不知道。他是甚『相府中懷春女』?他和我都是一般的後婚老婆。{\pangpi{妙在放倒自己。}}什麼他為你『褪湘裙杜鵑花上血』,三箇官唱兩箇喏,誰見來?孫小官兒問朱吉,別的都罷了,這箇我不敢許。可是你對人說的,自從他死了,好應心的菜兒也沒一碟子兒。『沒了王屠,連毛吃豬』!你日逐只𠳹屎哩?俺們便不是上數的,可不着你那心罷了。一箇大姐姐這般當家立紀,也扶持不過你來,{\meipi{提出月娘作主,不獨題目正大,得樹敵之意,自使西門慶惱不得。}}可哥兒只是他好。他死,你怎的不拉住他?當初沒他來時,你怎的過來?如今就是諸般兒稱不上你的心了。題起他來,就疼的你這心裡格地地的!拿別人當他,借汁兒下面,{\pangpi{忽插入如意,不費一痕氣力,神化之筆。}}也喜歡的你要不的。只他那屋裡水好吃麼?」月娘道:「好六姐,常言道『好人不長壽,禍害一千年』。{\pangpi{月娘出語亦毒。}}自古鏇的不圓,砍的圓。你我本等是遲貨,應不上他的心,隨他說去罷了。」金蓮道:「不是咱不說他,他說出來的話灰人的心。只說人憤不過他。」那西門慶只是笑,罵道:「怪小淫婦兒,胡說了你,我在那裡說這箇話來?」金蓮道:「還是請黃內官那日,{\meipi{六黃太尉何等勢焰,金蓮「黃內官」三字寫得冰冷,可見眞正情婦人、淫婦人,胸中原無富貴。}}你沒對着應二和溫蠻子說?{\pangpi{偏他記得。}}怪不的你老婆都死絕了,就是當初有他在,也不怎麼的。到明日再扶一箇起來,和他做對兒就是了。{\pangpi{倒底不放如意。}}賊沒廉恥撒根基的貨!」說的西門慶急了,跳起來,趕着拿靴脚踢他,那婦人奪門一溜烟跑了。{\meipi{再呆講,便贅矣。走得賊甚,且賊得有線索。}}這西門慶趕出去不見他,只見春梅站在上房門首,就一手搭伏春梅肩背往前邊來。{\pangpi{移花接木,妙。}}月娘見他醉了,巴不的打發他前邊去睡,要聽三箇姑子宣卷。於是教小玉打箇燈籠,送他前邊去。金蓮和玉簫站在穿廊下黑影中,西門慶沒看見,逕走過去。玉簫向金蓮道:「我猜爹管情向娘屋裡去了。」金蓮道:「他醉了,快發訕,{\pangpi{六字簡透。}}繇他先睡,等我慢慢進去。」{\meipi{搶白西門慶一頓,而西門慶又去尋他,要強好勝之心遂矣。復往後邊來,一者湊春梅之趣,二者要顯出由他自睡,不因搶白而小心周旋他也。}}這玉簫便道:「娘,你等等,我取些菓子兒稍與姥姥吃去。」{\pangpi{又映愛小便宜。}}於是走到床房內,拿些菓子遞與婦人,婦人接的袖了,一直走到他前邊。只見小玉送了回來,說道:「五娘在那邊來?爹好不尋五娘。」金蓮到房門首不進去,悄悄向窻眼望裡張覷,看見西門慶坐在床上,正摟着春梅做一處頑耍。恐怕攪擾他,連忙走到那邊屋裡,將菓子交付秋菊。因問:「姥姥睡沒有?」秋菊道:「睡了一大回了。」金蓮囑咐他:「菓子好生收在揀粧內。」又復往後邊來。只見月娘、李嬌兒、孟玉樓、西門大姐、大妗子、楊姑娘,並三箇姑子帶兩箇小姑子,坐了一屋裡人。薛姑子便盤膝坐在月娘炕上,{\pangpi{象箇活佛。}}當中放着一張炕桌兒,炷了香,衆人都圍着他,聽他說佛法。只見金蓮笑掀簾子進來,{\meipi{一「笑」字接前脈,斷而復斷,且寫出滿肚皮賣弄。}}月娘道:「你惹下禍來,他往屋裡尋你去了。你不打發他睡,如何又來了?我還愁他到屋裡要打你。」金蓮笑道:「你問他敢打我不敢?」{\pangpi{自要說嘴。}}月娘道:「你頭裡話出來的忒緊了,他有酒的人,一時激得惱了,不打你打狗不成?俺每倒替你捏兩把汗,原來你到這等潑皮。」金蓮道:「他就惱,我也不怕他,看不上那三等兒九做的。正經姐姐分付的曲兒不教唱,且東溝犁西溝耙,唱他的心事。就是今日孟三姐的好日子,也不該唱這離別之詞。{\meipi{又樹一敵,機鋒圓利。}}人也不知死到那裡去了,偏有那些佯慈悲、假孝順,我是看不上。」大妗子道:「你姐妹每亂了這一回,我還不知因為什麼來。姑夫好好的進來坐着,怎的又出去了?」月娘道:「大妗子,你還不知道,那一箇因想起李大姐來,說年時孟三姐生日還有他,今年就沒他,落了幾點眼淚,教小優兒唱了一套『憶吹簫,玉人兒何處也』。這一箇就不憤他唱這詞,剛纔搶白了他爹幾句。搶白的那箇急了,趕着踢打,這賊就走了。」楊姑娘道:「我的姐姐,你隨官人教他唱罷了,又搶白他怎的?想必每常見姐姐每都全全兒的,今日只不見了李家姐姐,漢子的心怎麼不慘切箇兒。」孟玉樓道:「好奶奶,若是我每,誰嗔他唱!俺這六姐姐平昔曉的曲子裡滋味,見那箇誇死了的李大姐,比古人那箇不如他,又怎的兩箇相交情厚,又怎麼山盟海誓,你為我,我為你。這箇牢成的又不服氣,只顧拿言語搶白他,整厮亂了這半日。」楊姑娘道:「我的姐姐,原來這等聰明!」月娘道:「他什麼曲兒不知道!但題起頭兒,就知尾兒。象我每叫唱老婆和小優兒來,只曉的唱出來就罷了。偏他又說那一段兒唱的不是了,那一句兒唱的差了,又那一節兒稍了。但是他爹說出箇曲兒來,就和他白搽白亂,必須搽惱了纔罷。」孟玉樓在旁邊戲道:「姑奶奶你不知,我三四胎兒只存了這箇丫頭子,這般精靈古怪的。」金蓮笑向他打了一下,說道:「我到替你爭氣,你到沒規矩起來了。」楊姑娘道:「姐姐,你今後讓官人一句兒罷。常言一夜夫妻百夜恩,相隨百步也有箇徘徊之意。一箇熱突突人兒,指頭兒似的少了一箇,{\pangpi{妙譬。}}有箇不想不疼不題念的?」金蓮道:「想怎不想,也有箇常時兒。一般都是你的老婆,做什麼擡一箇滅一箇?只嗔俺們不替他戴孝,他又不是婆婆,胡亂戴過斷七罷了,{\pangpi{好過文。}}只顧戴幾時?」楊姑娘道:「姐姐每見一半不見一半兒罷。」大妗子道:「好快!斷七過了,這一向又早百日來了。」{\meipi{淡淡接去,天衣無縫。}}楊姑娘問:「幾時是百日?」月娘道:「早哩,臘月二十六日。」王姑子道:「少不的念箇經兒。」{\meipi{一感便應。}}月娘道:「挨年近節,念什麼經!他爹只好過年念罷了。」說着,只見小玉拿上一道茶來,每人一盞。

須臾吃畢。月娘洗手,向爐中炷了香,聽薛姑子講說佛法。薛姑子就先宣念偈言,講了一段五戒禪師破戒戲紅蓮女子,轉世為東坡、佛印的佛法。講說了良久方罷。只見玉樓房中蘭香,拿了兩方盒細巧素菜菓碟、茶食點心來,收了香爐,擺在桌上。又是一壺茶,與衆人陪三箇師父吃了。然後又拿葷下飯來,開啟一罈麻姑酒,衆人圍爐吃酒。月娘便與大妗子擲色搶紅。金蓮便與李嬌兒猜枚,玉簫在旁邊斟酒,便替金蓮打桌底下轉子兒。須臾,把李嬌兒贏了數盃。{\meipi{李嬌兒滯貨。}}玉樓道:「等我和你猜,你只顧贏他罷。」卻要金蓮拿出手來,不許褪在袖子裡,又不許玉簫近前。{\meipi{玉樓便心眼不同。}}一連反贏了金蓮幾大鍾。金蓮坐不住,去了。到前邊叫了半日,角門纔開,只見秋菊揉眼。婦人罵道:「賊奴才,你睡來?」秋菊道:「我沒睡。」{\meipi{明明揉眼,卻賴沒睡,此蠢人弄巧處。}}婦人道:「見睡起來,你哄我。你到自在,就不說往後來接我接兒去。」因問:「你爹睡了?」秋菊道:「爹睡了這一日了。」婦人走到炕房裡,摟起裙子來就在炕上烤火。婦人要茶吃,秋菊連忙傾了一盞茶來。婦人道:「賊奴才,好乾淨手兒,我不吃這陳茶,熬的怪泛湯氣。你叫春梅來,叫他另拿小銚兒頓些好甜水茶兒,多着些茶葉,頓的苦艷艷我吃。」秋菊道:「他在那邊床房裡睡哩,等我叫他來。」婦人道:「你休叫他,且教他睡罷。」{\meipi{博寵人必有受寵處,溺愛人必有系愛處。眞受寵、系愛處,必有一段冷暖苦心出於蠢人,萬萬不得單指其恃寵之言、溺愛之情,而遂為蠢人不平也。}}這秋菊不依,走在那邊屋裡,見春梅𢱉在西門慶脚頭睡得正好。{\pangpi{究竟是丫頭情景,人多異之,吾且憐之。}}被他搖推醒了,道:「娘來了,要吃茶,你還不起來哩。」這春梅噦他一口,罵道:「見鬼的奴才,娘來了罷了,平白唬人剌剌的!」一面起來,慢條厮禮、撒腰拉袴走來見婦人,只顧倚着炕兒揉眼。婦人反罵秋菊:「恁奴才,你睡的甜甜兒的,把你叫醒了。」因叫他:「你頭上汗巾子跳上去了,還不往下扯扯哩。」又問:「你耳朵上墜子怎的只戴着一隻?」這春梅摸了摸,果然只有一隻。便點燈往那邊床上尋去,尋不見。良久,不想落在那脚踏板上,拾起來。婦人問:「在那裡來?」春梅道:「都是他失驚打怪叫我起來,吃帳鉤子抓下來了,纔在踏板上拾起來。」{\meipi{春梅與西門慶狂淫情態,只暗暗摹寫。}}婦人道:「我那等說着,他還只當叫起你來。」春梅道:「他說娘要茶吃來。」婦人道:「我要吃口茶兒,嫌他那手不乾淨。」這春梅連忙舀了一小銚子水,坐在火上,使他撾了些炭在火內,須臾就是茶湯。滌盞乾淨,濃濃的點上去,遞與婦人。婦人問春梅:「你爹睡下多大回了?」春梅道:「我打發睡了這一日了。問娘來,我說娘在後邊還未來哩。」這婦人吃了茶,因問春梅:「我頭裡袖了幾箇菓子和蜜餞,是玉簫與你姥姥吃的,交付這奴才接進來,你收了?」春梅道:「我沒見,他知道放在那裡。」

婦人叫秋菊,問他菓子在那裡,秋菊道:「我放在揀粧內哩。」走去取來,婦人數了數兒,少了一箇柑子,{\pangpi{數角兒習氣未除。}}問他那裡去了。秋菊道:「我拿進來就放在揀粧內,那箇害饞癆、爛了口吃他不成!」婦人道:「賊奴才,還漲漒嘴!你不偷,那去了?我親手數了交與你的,怎就少了一箇?原來只孝順了你!」教春梅:「你與我把那奴才一邊臉上打與他十箇嘴巴子。」春梅道:「那臢臉蛋子,倒沒的齷齪了我的手。」婦人道:「你與我拉過他來。」春梅用雙手推顙到婦人跟前。婦人用手擰着他腮頰,罵道:「賊奴才,這箇柑子是你偷吃了不是?你寔寔說了,我就不打你。不然,取馬鞭子來,我這一旋剝就打箇不數。我難道醉了?你偷吃了,一徑裡鬼混我。」因問春梅:「我醉不醉?」{\pangpi{以不認醉寫醉,妙甚。}}那春梅道:「娘清省白醒,那討酒來?{\pangpi{湊趣。}}娘不信只掏他袖子,怕不的還有柑子皮兒在袖子裡哩。」婦人於是扯過他袖子來,用手去掏,秋菊慌用手撇着不教掏。春梅一面拉起手來,果然掏出些柑子皮兒來。{\meipi{人之憎惡一人,雖極偏極暴,亦必有繇。人因其偏暴,往往轉為蠢人護短。此果掏出皮來,可謂至公之筆,使淺人為之,定寫作金蓮、春梅冤秋菊矣。}}被婦人儘力臉上擰了兩把,打了兩下嘴巴,罵道:「賊奴才,你諸般兒不會,相這說舌偷嘴吃偏會。眞賍寔犯拿住,你還賴那箇?我如今茶前酒後且不打你,{\pangpi{前云不醉,此又云茶前酒後,模糊得妙。}}到明日清省白醒,和你算帳。」春梅道:「娘到明日,休要與他行行忽忽的,好生旋剝了,叫箇人把他寔辣辣打與他幾十板子,叫他忍疼,也懼怕些。甚麼逗猴兒似湯那幾棍兒,他纔不放在心上!」那秋菊被婦人擰得臉脹腫的,谷都着嘴往廚下去了。婦人把那一箇柑子平分兩半,又拿了箇蘋婆、石榴,遞與春梅,說道:「這箇與你吃,把那箇留與姥姥吃。」這春梅也不瞧,接過來似有如無,掠在抽替內。婦人把蜜餞也要分開,春梅道:「娘不要分,我懶得吃這甜行貨子,留與姥姥吃罷。」以此婦人不分,都留下了。

婦人走到桶子上小解了,叫春梅掇進坐桶來,澡了牝,又問春梅:「這咱天有多時分了?」春梅道:「睡了這半日,也有三更了。」婦人摘了頭面,走來那邊床房裡,見桌上銀燈已殘,從新剔了剔,向床上看西門慶正打鼾睡。於是解鬆羅帶,卸褪湘裙,上床鑽入被窩裡,與西門慶並枕而臥。睡下不多時,向他腰間摸他那話。弄了一回,白不起。原來西門慶與春梅纔行房不久,那話綿軟,急切捏弄不起來。這婦人酒在腹中,慾情如火,蹲身在被底,把那話用口吮咂。挑弄蛙口,吞裹龜頭,只顧往來不絕。西門慶猛然醒了,便道:「怪小淫婦兒,如何這咱纔來?」婦人道:「俺每在後邊吃酒,孟三兒又安排了兩大方盒酒菜,郁大姐唱着,俺每猜枚擲骰兒,又頑了這一日,被我把李嬌兒贏醉了。落後孟三兒和我五子三猜,俺到輸了好幾鍾酒。你到是便宜,睡這一覺兒來好熬我,你看我依你不依?」{\pangpi{趣甚,諧甚。}}西門慶道:「你整治那帶子有了?」婦人道:「在褥子底下不是?」一面探手取出來,與西門慶看了,替他紮在麈柄根下,系在腰間,拴的緊緊的。又問:「你吃了不曾?」西門慶道:「我吃了。」須臾,那話吃婦人一壁廂弄起來,{\meipi{修身為學肯如此,何患不造其極。}}只見奢稜跳腦,挺身直舒,比尋常更舒半寸有餘。婦人爬在身上,龜頭昂大,兩手搧着牝戶往裡放。須臾突入牝中,婦人兩手摟定西門慶脖項,令西門慶亦扳抱其腰,在上只顧揉搓,那話漸沒至根。婦人叫西門慶:「達達,你取我的柱腰子墊在你腰底下。」這西門慶便向床頭取過他大紅綾抹胸兒,四摺疊起墊着腰,婦人在他身上馬伏着,那消幾揉,那話盡入。婦人道:「達達,你把手摸摸,都全放進去了,撐的裡頭滿滿兒的。你自在不自在?」西門慶用手摸摸,見盡沒至根,間不容髮,止剩二卵在外,心中覺翕翕然暢美不可言。婦人道:「好急的慌,只是寒冷,咱不得拿燈兒照着幹,趕不上夏天好。」因問西門慶,說道:「這帶子比那銀托子好不好?又不格的陰門生痛的,又長出許多來。你不信,摸摸我小肚子,七八頂到奴心。」又道:「你摟着我,等我一發在你身上睡一覺。」西門慶道:「我的兒,你睡,達達摟着。」那婦人把舌頭放在他口裡含着,一面朦朧星眼,款抱香肩。睡不多時,怎禁那慾火燒身,芳心撩亂,於是兩手按着他肩膊,一舉一坐,抽徹至首,復送至根,叫:「親心肝罷了,六兒的死了。」往來抽捲,又三百回。比及精泄,婦人口中只叫:「我的親達達,把腰扳緊了。」一面把乳頭教西門慶咂,不覺一陣昏迷,淫水溢下,婦人心頭小鹿突突的跳。登時四肢困軟,香雲撩亂。那話拽出來猶剛勁如故,婦人用帕搽之,說道:「我的達達,你不過卻怎麼的?」西門慶道:「等睡起一覺來再耍罷。」婦人道:「我的身子已軟癱熱化的。」{\meipi{用得好蘇文。}}當下雲收雨散,兩箇並肩交股,相與枕籍於床上,不知東方之既白。正是:

\begin{myquote}
等閒試把銀缸照,一對天生連理人。
\end{myquote}

