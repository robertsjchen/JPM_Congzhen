\part*{{\titlename}卷之七}
\addcontentsline{toc}{part}{{\titlename}卷之七}


\includepdf[pages={121,122},fitpaper=false]{tst.pdf}
\chapter*{第六十一回 西門慶乘醉燒陰戶 李瓶兒帶病宴重陽}
\addcontentsline{toc}{chapter}{第六十一回 西門慶乘醉燒陰戶 李瓶兒帶病宴重陽}
\markboth{{\titlename}卷之七}{第六十一回 西門慶乘醉燒陰戶 李瓶兒帶病宴重陽}


詞曰:

\begin{myquote}
蛩聲泣露驚秋枕,淚濕鴛鴦錦。獨臥玉肌涼,殘更與恨長。陰風翻翠幌,雨澀燈花暗。畢竟不成眠,鴉啼金井寒。

\raggedleft{——右調《菩薩蠻》\rightquadmargin}
\end{myquote}

話說一日,韓道國鋪中回家,睡到半夜,他老婆王六兒與他商議道:「你我被他照顧,{\pangpi{語不忘本。}}掙了恁些錢,也該擺席酒兒請他來坐坐。況他又丟了孩兒,只當與他釋悶,他能吃多少!彼此好看。就是後生小郎看着,{\pangpi{伏得妙。}}到明日南邊去,也知財主和你我親厚,比別人不同。」韓道國道:「我心裡也是這等說。明日初五日是月忌,不好。到初六日,安排酒席,叫兩箇唱的,具箇柬帖,等我親自到宅內,請老爹散悶坐坐。我晚夕便往鋪子裡睡去。」{\meipi{下此一句,別不說緣故,兩下心照,道國固是解人。}}王六兒道:「平白又叫甚麼唱的?只怕他酒後要來這屋裡坐坐,不方便。隔壁樂三嫂家,常走的一箇女兒申二姐,年紀小小的,且會唱,他又是瞽目的,請將他來唱唱罷。要打發他過去還容易。」韓道國道:「你說的是。」{\meipi{人家依老婆說的,亦只為其說的是耳。}}一宿晚景題過。

到次日,韓道國走到鋪子裡,央及溫秀才寫了箇請柬兒,親見西門慶,聲喏畢,說道:「明日,小人家裡治了一盃水酒,無事請老爹貴步下臨,散悶坐一日。」因把請柬遞上去。西門慶看了,說道:「你如何又費此心。我明日倒沒事,衙門中回家就去。」韓道國作辭出門。到次早,拿銀子叫後生胡秀買嗄飯菜蔬,一面叫廚子整理,又拿轎子接了申二姐來,王六兒同丫鬟伺候下好茶好水,單等西門慶來到。等到午後,只見琴童兒先送了一罈葡萄酒來,{\meipi{這回不怕韓二要吃矣。}}然後西門慶坐着涼轎,玳安、王經跟隨,到門首下轎,頭戴忠靖冠,身穿青水緯羅直身,粉頭皁靴。韓道國迎接入內,見畢禮數,說道:「又多謝老爹賜將酒來。」正面獨獨安放一張交椅,西門慶坐下。

不一時,王六兒打扮出來,與西門慶磕了四箇頭,回後邊看茶去了。須臾,王經拿出茶來,韓道國先取一盞,舉的高高的奉與西門慶,然後自取一盞,旁邊相陪。吃畢,王經接了茶盞下去,韓道國便開言說道:「小人承老爹莫大之恩,一向在外,家中小媳婦承老爹看顧,{\meipi{照顧着那一件?不說之說,妙。西門慶何以措辭。}}王經又蒙擡舉,叫在宅中答應,感恩不淺。前日哥兒沒了,雖然小人在那裡,媳婦兒因感了些風寒,不曾往宅裡弔問的,恐怕老爹惱。今日,一者請老爹解解悶,二者就恕俺兩口兒罪。」西門慶道:「無事又教你兩口兒費心。」說着,只見王六兒也在旁邊坐下。因向韓道國道:「你和老爹說了不?」道國道:「我還不曾說哩。」西門慶問道:「是甚麼?」王六兒道:「他今日要內邊請兩位姐兒來伏侍老爹,我恐怕不方便,故不去請。隔壁樂家常走的一箇女兒,叫做申二姐,諸般大小時樣曲兒,連數落都會唱。我前日在宅裡,見那一位郁大姐唱的也中中的,還不如這申二姐唱的好。教我今日請了他來,唱與爹聽。未知你老人家心下何如?若好,到明日叫了宅裡去,唱與他娘每聽。」西門慶道:「既是有女兒,亦發好了。你請出來我看看。」

不一時,韓道國叫玳安上來:「替老爹寬去衣服。」一面安放桌席,胡秀拿菓菜案酒上來。王六兒把酒開啟,燙熱了,在旁執壺,道國把盞,與西門慶安席坐下,然後纔叫出申二姐來。西門慶睜眼觀看,見他高髻雲鬟,插着幾枝稀稀花翠,淡淡釵梳,綠襖紅裙,顯一對金蓮趫趫;桃腮粉臉,抽兩道細細春山。{\meipi{描寫處,獨不及秋波,作者用筆之妙。}}望上與西門慶磕了四箇頭。西門慶便道:「請起。你今青春多少?」申二姐道:「小的二十一歲了。」又問:「你記得多少唱?」申二姐道:「大小也記百十套曲子。」西門慶令韓道國旁邊安下箇坐兒與他坐。申二姐向前行畢禮,方纔坐下。先拿箏來唱了一套「秋香亭」,然後吃了湯飯,添換上來,又唱了一套「半萬賊兵」。落後酒闌上來,西門慶分付:「把箏拿過去,取琵琶與他,等他唱小詞兒我聽罷。」那申二姐一徑要施逞他能彈會唱。一面輕搖羅袖,款跨鮫綃,頓開喉音,把弦兒放得低低的,彈了箇四不應《山坡羊》。唱完了,韓道國教渾家滿斟一盞,遞與西門慶。王六兒因說:「申二姐,你還有好《鎖南枝》,唱兩箇與老爹聽。」那申二姐就改了調兒,唱《鎖南枝》道:

\begin{myquote}
初相會,可意人,年少青春,不上二旬。黑鬖鬖兩朵烏雲,紅馥馥一點朱唇,臉賽夭桃如嫩筍。若生在畫閣蘭堂,端的也有箇夫人分。可惜在章臺,出落做下品。但能勾改嫁從良,勝強似棄舊迎新。

初相會,可意嬌,月貌花容,風塵中最少。瘦腰肢一撚堪描,俏心腸百事難學,恨只恨和他相逢不早。常則怨席上樽前,淺斟低唱相偎抱。一覷一箇眞,一看一箇飽。雖然是半霎歡娛,權且將悶解愁消。
\end{myquote}

西門慶聽了這兩箇《鎖南枝》,正打着他初請了鄭月兒那一節事來,心中甚喜。王六兒滿滿的又斟上一盞,笑嘻嘻說道:「爹,你慢慢兒的飲,申二姐這箇纔是零頭兒,{\meipi{語致都媚。}}他還記的好些小令兒哩。到明日閑了,拿轎子接了,唱與他娘每聽,管情比郁大姐唱的高。」西門慶因說:「申二姐,我重陽那日,使人來接你,去不去?」申二姐道:「老爹說那裡話,但呼喚,怎敢違阻!」西門慶聽見他說話伶俐,心中大喜。

不一時,交盃換盞之間,王六兒恐席間說話不方便,{\meipi{應前。}}叫他唱了幾套,悄悄向韓道國說:「教小厮招弟兒,送過樂三嫂家歇去罷。」臨去拜辭,西門慶向袖中掏出一包兒三錢銀子,賞他買弦。申二姐連忙嗑頭謝了。西門慶約下:「我初八日使人請你去。」王六兒道:「爹只使王經來對我說,等我這裡教小厮請他去。」說畢,申二姐往隔壁去了。韓道國與老婆說知,也就往鋪子裡睡去了。{\meipi{一箇悄悄說向,一箇說知,都不說出,心事瞭然。}}只落下老婆在席上,陪西門慶擲骰飲酒。吃了一回,兩箇看看吃的涎將上來,西門慶推起身更衣,就走入婦人房裡,兩箇頂門頑耍。王經便把燈燭拿出來,在前半間和玳安、琴童兒做一處飲酒。

那後生胡秀,在廚下偷吃了幾碗酒,{\meipi{前說使後生看着,至此照出,作在針密如蝨。}}打發廚子去了,走在王六兒隔壁供養佛祖先堂內,地下鋪着一領席,就睡着了。睡了一覺起來,忽聽見婦人房裡聲喚,又見板壁縫裡透過燈亮來,只道西門慶去了,韓道國在房中宿歇。暗暗用頭上簪子刺破板縫中糊的紙,往那邊張看。見那邊房中亮騰騰點着燈燭,不想西門慶和老婆在屋裡正幹得好。{\meipi{道國與王經、玳安等拾收已過,此番光景卻幻出胡秀以作波瀾。淩空駕奇,文心靈異如此。}}伶伶俐俐看見,把老婆兩隻腿,卻是用脚帶弔在床頭上,西門慶上身止着一件綾襖兒,下身赤露,就在床沿上一來一往,一動一靜,搧打的連聲响喨,老婆口裡百般言語都叫將出來。良久,只聽老婆說:「我的親達!你要燒淫婦,隨你心裡揀着那塊只顧燒,淫婦不敢攔你。左右淫婦的身子屬了你,怕那些兒了!」西門慶道:「只怕你家裡的嗔是的。」{\pangpi{豈有此理!}}老婆道:「那忘八七箇頭八箇膽,他敢嗔!他靠着那裡過日子哩?」西門慶道:「你既一心在我身上,等這遭打發他和來保起身,亦發留他長遠在南邊,做箇買手置貨罷。」{\pangpi{道國十分大雅,此着似乎不必。}}老婆道:「等走過兩遭兒,卻教他去。省的閑着在家做甚麼?他說倒在外邊走慣了,一心只要外邊去。你若下顧他,可知好哩!等他回來,我房裡替他尋下一箇,我也不要他,一心撲在你身上,隨你把我安插在那裡就是了。我若說一句假,把淫婦不値錢身子就爛化了。」{\meipi{王六兒牢攏牽挽,技倆在金蓮之上。蓋金蓮地親,故用強;六兒地遠,故用柔,兩人心事異出而同揆也。}}西門慶道:「我兒,你快休賭誓!」兩箇一動一靜,都被胡秀聽了箇不亦樂乎。

韓道國先在家中不見胡秀,只說往鋪子裡睡去了。走到段子鋪裡,問王顯、榮海,說他沒來。韓道國一面又走回家,叫開門,前後尋胡秀,那裡得來,只見王經陪玳安、琴童三箇在前邊吃酒。胡秀聽見他的語音來家,連忙倒在席上,又推睡了。不一時,韓道國點燈尋到佛堂地下,看見他鼻口內打鼾睡,用脚踢醒,罵道:「賊野狗死囚,還不起來!我只說先往鋪子裡睡去,你原來在這裡挺得好覺兒。還不起來跟我去!」那胡秀起來,推揉了揉眼,楞楞睜睜跟道國往鋪子裡去了。

西門慶弄老婆,直弄勾有一箇時辰,方纔了事。燒了王六兒心口裡並𣬼蓋子上、尾亭骨兒上共三處香。老婆起來穿了衣服,教丫頭打發舀水淨了手,重篩煖酒,再上佳餚,情話攀盤。又吃了幾鍾,方纔起身上馬,玳安、王經、琴童三箇跟着。到家中已有二更天氣,走到李瓶兒房中。李瓶兒睡在床上,見他吃的酣酣兒的進來,說道:「你今日在誰家吃酒來?」西門慶道:「韓道國家請我。見我丟了孩子,與我釋悶。他叫了箇女先生申二姐來,年紀小小,好不會唱!又不說郁大姐。等到明日重陽,使小厮拿轎子接他來家,唱兩日你每聽,就與你解解悶。你緊心裡不好,休要只顧思想他了。」{\meipi{海棠着雨,楊柳經秋,冷致悽情,可憐可愛。}}說着,就要叫迎春來脫衣裳,和李瓶兒睡。李瓶兒道:「你沒的說,我下邊不住的長流,丫頭替我煎藥哩。你往別人屋裡睡去罷。你看着我成日好模樣兒罷了,只有一口遊氣兒在這裡,又來纏我起來。」西門慶道:「我的心肝!我心裡捨不的你。只要和你睡,如之奈何?」李瓶兒瞟了他一眼,笑了笑兒:「誰信你那虛嘴掠舌的。我倒明日死了,你也捨不的我罷!」又道:「亦發等我好好兒,你再進來和我睡也不遲。」西門慶坐了一回,說道:「罷,罷。你不留我,等我往潘六兒那邊睡去罷。」李瓶兒道:「原來你去,省的屈着你那心腸兒。他那裡正等的你火裡火發,你不去,卻忙惚兒來我這屋裡纏。」{\meipi{西門慶捨此則彼,微瓶兒則金蓮寵極矣。瓶兒蓋心知之而心傷之。「往潘六兒那邊去」一語,固瓶兒不忍聞而不欲聞者,宜乎瓶兒之不起也。}}西門慶道:「你恁說,我又不去了。」李瓶兒微笑道:「我哄你哩,你去罷。」於是打發西門慶過去了。李瓶兒起來,坐在床上,迎春伺候他吃藥。拿起那藥來,止不住撲簌簌香腮邊滾下淚來,長吁了一口氣,{\meipi{可憐。}}方纔吃了那盞藥。正是:

\begin{myquote}
心中無限傷心事,付與黃鸝叫幾聲。
\end{myquote}

不說李瓶兒吃藥睡了,單表西門慶到於潘金蓮房裡。金蓮纔叫春梅罩了燈上床睡下。忽見西門慶推開門進來便道:{\meipi{意中事若出之意外,全在忽見二字寫出。}}「我兒,又早睡了?」金蓮道:「稀幸!那陣風兒刮你到我這屋裡來!」因問:「你今日往誰家吃酒去來?」西門慶道:「韓夥計打南邊來,見我沒了孩子,一者與我釋悶,二者照顧他外邊走了這遭,請我坐坐。」金蓮道:「他便在外邊,你在家又照顧他老婆了。」西門慶道:「夥計家,那裡有這道理?」婦人道:「夥計家,有這箇道理!齊腰拴着根線兒,只怕㒲過界兒去了。你還搗鬼哄俺每哩,俺每知道的不耐煩了!你生日,賊淫婦他沒在這裡?你悄悄把李瓶兒壽字簪子,{\meipi{偏是他曉得,是固其性生天授也。}}黃貓黑尾偷與他,卻叫他戴了來施展。大娘、孟三兒,這一家子那箇沒看見?吃我問了一句,他把臉兒都紅了,他沒告訴你?今日又摸到那裡去,賊沒廉恥的貨,一箇大摔瓜長淫婦,喬眉喬樣,描的那水𩬆長長的,搽的那嘴唇鮮紅的,倒象人家那血𣬼。甚麼好老婆,一箇大紫腔色黑淫婦,我不知你喜歡他那些兒!{\meipi{極其醜詆。}}嗔道把忘八舅子也招惹將來,一早一晚教他好往回傳話兒。」西門慶堅執不認,笑道:「怪小奴才兒,單管只胡說,那裡有此勾當?今日他男子漢陪我坐,他又沒出來。」婦人道:「你拿這箇話兒來哄我?誰不知他漢子是箇明忘八,又放羊,又拾柴,一徑把老婆丟與你,圖你家買賣做,要賺你的錢使。{\meipi{慧心所照,如見肺肝。}}你這傻行貨子,只好四十里聽銃响罷了!」西門慶脫了衣裳,坐在床沿上,婦人探出手來,把褲子扯開,{\pangpi{着。}}摸見那話軟叮噹的,托子還帶在上面,說道:「可又來,你『臘鴨子煮到鍋裡——身子兒爛了,嘴頭兒還硬』。見放着不語先生在這裡,{\pangpi{好美號。}}強盜和那淫婦怎麼弄聳,聳到這咱晚纔來家?弄的恁箇樣兒,嘴頭兒還強哩!你賭箇誓,我叫春梅舀一甌子涼水,你只吃了,我就算你好膽子。{\meipi{先設以必不可逃之數,隱隱說着自己,機鋒尖穎,比勘精詳,直可折獄。}}論起來,鹽也是這般鹹,醋也是這般酸,『禿子包網巾——饒這一抿子兒也罷了』。若是信着你意兒,把天下老婆都耍遍了罷。賊沒羞的貨,一箇大眼裡火行貨子!你早是箇漢子,若是箇老婆,就養遍街,㒲遍巷。」幾句說的西門慶睜睜的,只是笑。上的床來,叫春梅篩熱了燒酒,把金穿心盒兒內藥拈了一粒,放在口裡嚥下去,仰臥在枕上,令婦人:「我兒,你下去替你達品,品起來是你造化。」

那婦人一徑做喬張致,便道:「好乾淨兒!你在那淫婦窟礲子裡鑽了來,教我替你咂,可不臢殺了我!」西門慶道:「怪小淫婦兒,單管胡說白道的,那裡有此勾當?」婦人道:「那裡有此勾當?你指着肉身子賭箇誓麼!」亂了一回,教西門慶下去使水,西門慶不肯下去,婦人旋向袖子裡掏出箇汗巾來,將那話抹展了一回,{\pangpi{妙用。}}方纔用朱唇裹沒。{\meipi{金蓮明知是從六兒箇中來,不敢不咂;西門慶亦知金蓮知其從六兒箇中來,而使之不敢不咂。十分妙用,金蓮入其範圍矣。}}嗚咂半晌,咂弄的那話奢稜跳腦,暴怒起來,乃騎在婦人身上,縱麈柄自後插入牝中,兩手兜其股,蹲踞而擺之,肆行搧打,連聲响喨。燈光之下,窺玩其出入之勢,婦人倒伏在枕畔,舉股迎湊者久之。西門慶興猶不愜,將婦人仰臥朝上,那話上使了粉紅藥兒,頂入去,執其雙足,又舉腰僅沒其稜掀騰者將二三百度。婦人禁受不的,瞑目顫聲,沒口子叫:「達達,你這遭兒只當將就我,不使上他也罷了。」西門慶口中呼叫道:「小淫婦兒,你怕我不怕?再敢無禮不敢?」婦人道:「我的達達,罷麼,你將就我些兒,我再不敢了!達達慢慢提,看提散了我的頭髮。」兩箇顛鴛倒鳳,足狂了半夜,方纔體倦而寢。

話休饒舌,又早到重陽令節。西門慶對吳月娘說:「韓夥計前日請我,一箇唱的申二姐,生的人材又好,又會唱。我使小厮接他來,留他兩日,教他唱與你每聽。」又分付廚下收拾餚饌菓酒,在花園大捲棚聚景堂內,安放大八仙桌,閤家宅眷,慶賞重陽。

不一時,王經轎子接的申二姐到了。入到後邊,與月娘衆人磕了頭。月娘見他年小,生的好模樣兒。問他套數,也會不多,諸般小曲兒倒記的有好些。一面打發他吃了茶食,先教在後邊唱了兩套,然後花園擺下酒席。那日,西門慶不曾往衙門中去,在家看着栽了菊花。請了月娘、李嬌兒、孟玉樓、潘金蓮、李瓶兒、孫雪娥並大姐,都在席上坐的。春梅、玉簫、迎春、蘭香在旁斟酒伏侍。申二姐先拿琵琶在旁彈唱。那李瓶兒在房中,因身上不方便,請了半日纔來。恰似風兒颳倒的一般,強打着精神陪西門慶坐,衆人讓他酒兒也不大吃。西門慶和月娘見他面帶憂容,眉頭不展,說道:「李大姐,你把心放開,教申二姐彈唱曲兒你聽。」玉樓道:「你說與他,教他唱甚麼曲兒,他好唱。」李瓶兒只顧不說。正飲酒中間,忽見王經走來說道:「應二爹、常二叔來了。」西門慶道:「請你應二爹、常二叔在小捲棚內坐,我就來。」王經道:「常二叔教人拿了兩箇盒子在外頭。」西門慶向月娘道:「此是他成了房子,買禮來謝我的意思。」月娘道:「少不的安排些甚麼管待他,怎好空了他去!你陪他坐去,我這裡分付看菜兒。」{\meipi{是當人口吻。}}西門慶臨出來,又叫申二姐:「你唱箇好曲兒,與你六娘聽。」一直往前邊去了。金蓮道:「也沒見這李大姐,隨你心裡說箇甚麼曲兒,教申二姐唱就是了,辜負他爹的心!為你叫將他來,你又不言語。」催逼的李瓶兒急了,半日纔說出來:「你唱箇『紫陌紅塵』罷。」那申二姐道:「這箇不打緊,我有。」於是取過箏來,頓開喉音,細細唱了一套。唱畢,吳月娘道:「李大姐,好甜酒兒,你吃上一鍾兒。」李瓶兒又不敢違阻,拿起鍾兒來嚥了一口兒,又放下了。坐不多時,下邊一陣熱熱的來,又往屋裡去了,不題。

且說西門慶到於小捲棚翡翠軒,只見應伯爵與常峙節在松墻下正看菊花。原來松墻兩邊,擺放二十盆,都是七尺高,各樣有名的菊花,{\meipi{政固不俗。}}也有大紅袍、狀元紅、紫袍金帶、白粉西、黃粉西、滿天星、醉楊妃、玉牡丹、鵝毛菊、鴛鴦花之類。西門慶出來,二人向前作揖。常峙節即喚跟來人,把盒兒掇進來。西門慶一見便問:「又是甚麼?」伯爵道:「常二哥蒙哥厚情,成了房子,無可酬答,教他娘子製造了這螃蠏鮮並兩隻爐燒鴨兒,邀我來和哥坐坐。」西門慶道:「常二哥,你又費這箇心做甚麼?你令正病纔好些,你又禁害他!」伯爵道:「我也是恁說。他說道別的東西兒來,恐怕哥不稀罕。」西門慶令左右開啟盒兒觀看:四十箇大螃蠏,都是剔剝淨了的,裡邊釀着肉,外用椒料姜蒜米兒糰粉裹就,香油煠,醬油醋造過,香噴噴,酥脆好食。又是兩大隻院中爐燒熟鴨。西門慶看了,即令春鴻、王經掇進去,分付拿五十文錢賞拿盒人,因向常峙節謝了。

琴童在旁掀簾,請入翡翠軒坐。伯爵只顧誇獎不盡好菊花,問:「哥是那裡尋的?」西門慶道:「是管磚廠劉太監送的。這二十盆,就連盆都送與我了。」伯爵道:「花到不打緊,這盆正是官窯雙箍鄧漿盆,{\meipi{前便有一番誇獎,鑿鑿可據。}}都是用絹羅打,用脚跐過泥,纔燒造這箇物兒,與蘇州鄧漿磚一箇樣兒做法。如今那裡尋去!」誇了一回。西門慶喚茶來吃了,因問:「常二哥幾時搬過去?」伯爵道:「從兌了銀子三日就搬過去了。昨見好日子,買了些雜貨兒,門首把鋪兒也開了。就是常二嫂兄弟,替他在鋪裡看銀子兒。」西門慶道:「俺每幾時買些禮兒,休要人多了,再邀謝子純你三四位,我家裡整理菜兒擡了去,休費煩常二哥一些東西,叫兩箇妓者,咱每替他煖煖房,耍一日。」常峙節道:「小弟有心也要請哥坐坐,算計來不敢請。地方兒窄狹,只怕褻瀆了哥。」西門慶道:「沒的扯淡,那裡又費你的事起來。如今使小厮請將謝子純來,和他說說。」即令琴童兒:「快請你謝爹去!」伯爵因問:「哥,你那日叫那兩箇去?」西門慶笑道:「叫將鄭月兒和洪四兒去罷。」伯爵道:「哥,你是箇人,你請他就不對我說聲,我怎的也知道了?比李桂兒風月如何?」西門慶道:「通色絲子女不可言!」伯爵道:「他怎的前日你生日時,那等不言語,扭扭的,也是箇肉佞賊小淫婦兒。」西門慶道:「等我到幾時再去着,也攜帶你走走。你月娘會打的好雙陸,你和他打兩貼雙陸。」伯爵道:「等我去混那小淫婦兒,休要放了他!」西門慶道:「你這歪狗才,不要惡識他便好。」正說着,謝希大到了,聲諾畢,坐下。西門慶道:「常二哥如此這般,新有了華居,瞞着俺每,已搬過去了。咱每人隨意出些分資,休要費煩他絲毫。我這裡整治停當,教小厮擡到他府上,我還叫兩箇妓者,咱耍一日何如?」謝希大道:「哥分付每人出多少分資,俺每都送到哥這裡來就是了。還有那幾位?」西門慶道:「再沒人,只這三四箇兒,每人二星銀子就勾了。」伯爵道:「十分人多了,他那裡沒地方兒。」

正說着,只見琴童來說:「吳大舅來了。」西門慶道:「請你大舅這裡來坐。」不一時,吳大舅進入軒內,先與三人作了揖,然後與西門慶叙禮坐下。小厮拿茶上來,同吃了茶,吳大舅起身說道:「請姐夫到後邊說句話兒。」西門慶連忙讓大舅到後邊月娘房裡。月娘還在捲棚內與衆姊妹吃酒聽唱,聽見說:「大舅來了,爹陪着在後邊說話哩。」一面走到上房,見大舅道了萬福,叫小玉遞上茶來。大舅向袖中取出十兩銀子遞與月娘,說道:「昨日府裡纔領了三錠銀子,姐夫且收了這十兩,餘者待後次再送來。」西門慶道:「大舅,你怎的這般計較?且使着,慌怎的!」大舅道:「我恐怕遲了姐夫的。」西門慶因問:「倉廒修理的也將完了?」大舅道:「還得一箇月終完。」西門慶道:「工完之時,已定撫按有些獎勵。」大舅道:「今年考選軍政在邇,還望姐夫扶持,大巡上替我說說。」西門慶道:「大舅之事,都在於我。」說畢話,月娘道:「請大舅前邊同坐罷。」大舅道:「我去罷,只怕他三位來有甚麼話說。」西門慶道:「沒甚麼話。常二哥新近問我借了幾兩銀子,買下了兩間房子,已搬過去了,今日買了些禮兒來謝我,節間留他每坐坐。大舅來的正好。」於是讓至前邊坐了。月娘連忙叫廚下打發菜兒上去。琴童與王經先安放八仙桌席端正,西門慶旋教開庫房,拿出一罈夏提刑家送的菊花酒來。開啟碧靛清,噴鼻香,未曾篩,先攙一瓶涼水,以去其蓼辣之性,然後貯於布甑內,篩出來醇厚好吃,又不說葡萄酒。叫王經用小金鐘兒斟一盃兒,先與吳大舅嚐了,然後,伯爵等每人都嘗訖,極口稱羨不已。須臾,大盤大碗擺將上來,衆人吃了一頓。然後纔拿上釀螃蠏並兩盤燒鴨子來,伯爵讓大舅吃。連謝希大也不知是甚麼做的,這般有味,酥脆好吃。西門慶道:「此是常二哥家送我的。」大舅道:「我空癡長了五十二歲,並不知螃蠏這般造作,委的好吃!」伯爵又問道:「後邊嫂子都嚐了嚐兒不曾?」西門慶道:「房下每都有了。」伯爵道:「也難為我這常嫂子,眞好手段兒!」常峙節笑道:「賤累還恐整理的不堪口,教列位哥笑話。」吃畢螃蠏,左右上來斟酒,西門慶令春鴻和書童兩箇,在旁一遞一箇歌唱南曲。

應伯爵忽聽大捲棚內彈箏歌唱之聲,便問道:「哥,今日李桂姐在這裡?不然,如何這等音樂之聲?」西門慶道:。「你再聽,看是不是?」伯爵道:「李桂姐不是,就是吳銀兒。」西門慶道:「你這花子單管只瞎謅。倒是箇女先生。」伯爵道:「不是郁大姐?」西門慶道:「不是他,這箇是申二姐。年小哩,好箇人材,又會唱。」伯爵道:「眞箇這等好?哥怎的不牽出來俺每瞧瞧?就唱箇兒俺每聽。」西門慶道:「今日你衆娘每大節間,叫他來賞重陽頑耍,偏你這狗才耳朵尖,聽的見!」伯爵道:「我便是千里眼,順風耳,隨他四十里有蜜蜂兒叫,我也聽見了。」謝希大道:「你這花子,兩耳朵似竹籤兒也似,愁聽不見!」兩箇又頑笑了一回,伯爵道:「哥,你好歹叫他出來,俺每見見兒,俺每不打緊,教他只當唱箇與老舅聽也罷了。休要就古執了。」西門慶吃他逼迫不過,一面使王經領申二姐出來唱與大舅聽。不一時,申二姐來,望上磕了頭起來,旁邊安放交床兒與他坐下。伯爵問申二姐:「青春多少?」申二姐回道:「屬牛的,二十一歲了。」又問:「會多少小唱?」申二姐道:「琵琶箏上套數小唱,也會百十來套。」伯爵道:「你會許多唱也勾了。」西門慶道:「申二姐,你拿琵琶唱小詞兒罷,省的勞動了你。說你會唱『四夢八空』,你唱與大舅聽。」分付王經、書童兒,席間斟上酒。那申二姐款跨鮫綃,微開檀口,慢慢唱着,衆人飲酒不題。

且說李瓶兒歸到房中,坐淨桶,下邊似尿的一般,只顧流將起來,登時流的眼黑了。起來穿裙子,忽然一陣旋暈,向前一頭撞倒在地。饒是迎春在旁搊扶着,還把額角上磕傷了皮。和奶子搊到炕上,半日不省人事。慌了迎春,忙使綉春:「快對大娘說去!」綉春走到席上,報與月娘衆人。月娘撇了酒席,與衆姐妹慌忙走來看視。見迎春、奶子兩箇搊扶着他坐在炕上,不省人事。便問:「他好好的進屋裡,端的怎麼來就不好了?」迎春揭開淨桶與月娘瞧,把月娘唬了一跳。說道:「他剛纔只怕吃了酒,助趕的他血旺了,流了這些。」玉樓、金蓮都說:「他幾曾大吃酒來!」一面煎燈心薑湯灌他。半晌甦醒過來,纔說出話兒來。月娘問:「李大姐,你怎的來?」李瓶兒道:「我不怎的。坐下桶子起來穿裙子,只見眼兒前黑黑的一塊子,就不覺天旋地轉起來,繇不的身子就倒了。」月娘便要使來安兒:「請你爹進來。對他說,教他請任醫官來看你。」李瓶兒又嗔教請去:「休要大驚小怪,打攪了他吃酒。」{\meipi{瓶兒、月娘一對婆心。月娘可敬,瓶兒可憐。}}月娘分付迎春:「打鋪教你娘睡罷。」月娘於是也就吃不成酒了,分付收拾了家伙,都歸後邊去了。

西門慶陪侍吳大舅衆人,至晚歸到後邊月娘房中。月娘告訴李瓶兒跌倒之事,西門慶慌走到前邊來看視。見李瓶兒睡在炕上,面色蠟查黃了,扯着西門慶衣袖哭泣。西門慶問其所以,李瓶兒道:「我到屋裡坐榪子,不知怎的,下邊只顧似尿也一般流將起來,不覺眼前一塊黑黑的。起來穿裙子,天旋地轉,就跌倒了。」西門慶見他額上磕傷一道油皮,說道,「丫頭都在那裡,不看你,怎的跌傷了面貌?」李瓶兒道:「還虧大丫頭都在跟前,和奶子搊扶着我,不然,還不知跌的怎樣的。」西門慶道:「我明早請任醫官來看你。」當夜就在李瓶兒對面床上睡了一夜。

次日早晨,往衙門裡去,旋使琴童請任醫官去了。直到晌午纔來。西門慶先在大廳上陪吃了茶,使小厮說進去。李瓶兒房裡收拾乾淨,薰下香,然後請任醫官進房中。診畢脈,走出外邊廳上,對西門慶說:「老夫人脈息,比前番甚加沉重,{\pangpi{細。}}七情傷肝,肺火太旺,以致木旺土虛,血熱妄行,猶如山崩而不能節制。若所下的血紫者,猶可以調理;若鮮紅者,乃新血也。學生撮過藥來,若稍止,則可有望;不然,難為矣。」{\meipi{此醫見理極明。}}西門慶道:「望乞老先生留神加減,學生必當重謝!」任醫官道:「是何言語!你我厚間,又是明用情分,學生無不盡心。」西門慶待畢茶,送出門,隨即具一疋杭絹、二兩白金,使琴童兒討將藥來,名曰「歸脾湯」,乘熱吃下去,其血越流之不止。西門慶越發慌了,又請大街口胡太醫來瞧。胡太醫說是氣冲血管,熱入血室,亦取將藥來。吃下去,如石沉大海一般。

月娘見前邊亂着請太醫,只留申二姐住了一夜,與了他五錢銀子、一件雲絹比甲兒並花翠,裝了箇盒於,就打發他坐轎子去了。花子繇自從那日開張吃了酒去,聽見李瓶兒不好,使了花大嫂,買了兩盒禮來看他。見他瘦的黃懨懨兒,不比往時,兩箇在屋裡大哭了一回。{\pangpi{亦是好人。}}月娘後邊擺茶請他吃了。

韓道國說:「東門外住的一箇看婦人科的趙太醫,指下明白,極看得好。前歲,小媳婦月經不通,{\meipi{此醫想善於打胎,與瓶兒固非對症之劑。}}是他看來。老爹請他來看看六娘,管情就好哩。」西門慶聽了,就使琴童和王經兩箇疊騎着頭口,往門外請趙太醫去了。

西門慶請了應伯爵來,和他商議道:「第六箇房下,甚是不好的重,如之奈何?」伯爵失驚道:「這箇嫂子貴恙說好些,怎的又不好起來?」西門慶道:「自從小兒沒了,着了憂戚,把病又發了。昨日重陽,我接了申二姐,與他散悶頑耍,他又沒好生吃酒,誰知走到屋中就暈起來,一交跌倒,把臉都磕破了。請任醫官來看,說脈息比前沉重。吃了藥,倒越發血盛了。」伯爵道:「你請胡太醫來看,怎的說?」西門慶道:「胡大醫說,是氣冲了血管,吃了他的,也不見動靜。今日韓夥計說,門外一箇趙太醫,名喚趙龍崗,專科看婦女,我使小厮請去了。把我焦愁的了不的。生生為這孩子不好,白日黑夜思慮起這病來了。婦女人家,又不知箇迴轉,勸着他,又不依你,叫我無法可處。」{\meipi{丈夫處此眞難。}}

正說着,平安來報:「喬親家爹來了。」西門慶一面讓進廳上,同伯爵叙禮坐下。喬大戶道:「聞得六親家母有些不安,特來候問。」西門慶道:「便是。一向因小兒沒了,着了憂戚,身上原有些不調,又發起來了。蒙親家掛念。」喬大戶道:「也曾請人來看不曾?」西門慶道:「常吃任後溪的藥,昨日又請大街胡先生來看,吃藥越發轉盛。今日又請門外專看婦人科趙龍崗去了。」喬大戶道:「咱縣門前住的何老人,大小方脈俱精。他兒子何歧軒,見今上了箇冠帶醫士。親家何不請他來看看親家母?」西門慶道:「既是好,等趙龍崗來,來過再請他來看看。」喬大戶道:「親家,依我愚見,不如先請了何老人來,再等趙龍崗來,叫他兩箇細講一講,就論出病原來了。{\meipi{何家積祖名醫。}}然後下藥,無有不效之理。」西門慶道:「親家說的是。」一面使玳安拿拜帖兒和喬通去請。

那消半晌,何老人到來,與西門慶、喬大戶等作了揖,讓於上面坐下。西門慶舉手道:「數年不見你老人家,不覺越發蒼髯皓首。」喬大戶又問:「令郎先生肄業盛行?」何老人道:「他逐日縣中迎送,也不得閑,倒是老拙常出來看病。」伯爵道:「你老人家高壽了,還這等健朗。」何老人道:「老拙今年癡長八十一歲。」叙畢話,看茶上來吃了,小厮說進去。須臾,請至房中,就床看李瓶兒脈息,旋搊扶起來,坐在炕上,形容瘦的十分狼狽了。但見他:

\begin{myquote}
面如金紙,體似銀條。看看減褪丰標,漸漸消磨精彩。隱隱耳虛聞磐響,昏昏眼暗覺螢飛。六脈細沉,一靈縹緲,䘮門弔客已臨身,扁鵲盧醫難下手。
\end{myquote}

何老人看了脈息,出到廳上,向西門慶、喬大戶說道:「這位娘子,乃是精冲了血管起,{\pangpi{病從胡僧藥起。}}然後着了氣惱。氣與血相搏,則血如崩。{\meipi{畢竟老醫,開口道破。}}不知當初起病之繇是也不是?」西門慶道:「是便是,卻如何治療?」正論間,忽報:「琴童和王經請了趙先生來了。」何老人便問:「是何人?」西門慶道:「也是夥計舉來一醫者,你老人家只推不知,待他看了脈息,你老人家和他講一講,好下藥。」不一時,趙太醫從外而入,西門慶與他叙禮畢,然後與衆人相見。何、喬二老居中,讓他在左,伯爵在右,西門慶主位相陪。吃了茶,趙太醫便問:「列位尊長貴姓?」喬大戶道:「俺二人一姓何,一姓喬。」伯爵道:「在下姓應。老先想就是趙龍崗先生了。」趙太醫答道:「龍崗是賤號。{\pangpi{開口便妙。}}在下以醫為業,家祖見為太醫院院判,家父見充汝府良醫,祖傳三輩,習學醫術。{\meipi{只少一副串鈴,已供出一箇眞方假藥面孔。}}每日攻習王叔和、東垣勿聽子《藥性賦》、《黃帝素問》、《難經》、《活人書》、《丹溪纂要》、《丹溪心法》、《潔古老脈訣》、《加減十三方》、《千金奇效良方》、《壽域神方》、《海上方》,無書不讀。藥用胸中活法,脈明指下玄機。六氣四時,辨陰陽之標格;七表八里,定關格之沉浮。{\meipi{擬信餘藩,便覺談吐風生,當今儒道,不少此輩。}}風虛寒熱之症候,一覽無餘;弦洪芤石之脈理,莫不通曉。小人拙口鈍吻,不能細陳。」何老人聽了,道:「敢問看病當以何者為先?」趙太醫道:「古人云,望聞問切,神聖功巧。學生先問病,後看脈,還要觀其氣色。就如子平兼五星一般,{\pangpi{引得不通,惟不通故妙。}}纔看得準,庶乎不差。」何老人道:「既是如此,請先生進去看看。」西門慶即令琴童:「後邊說去,又請了趙先生來了。」

不一時,西門慶陪他進入李瓶兒房中。那李瓶兒方纔睡下,安逸一回,{\pangpi{細。}}又搊扶起來,靠着枕褥坐着。這趙太醫先診其左手,次診右手,便教:「老夫人擡起頭來,看看氣色。」那李瓶兒眞箇把頭兒揚起來。趙太醫教西門慶:「老爹,你問聲老夫人,我是誰?」{\pangpi{奇。}}西門慶便教李瓶兒:「你看這位是誰?」那李瓶兒擡頭看了一眼,便低聲說道:「他敢是太醫?」趙先生道:「老爹,不妨事,還認的人哩。」{\meipi{妙極,一出幽閨旅病。}}西門慶道:「趙先生,你用心看,我重謝你。」一面看視了半日,說道:「老夫人此病,休怪我說,據看其面色,又診其脈息,非傷寒,只為雜症,不是產後,定然胎前。」西門慶道:「不是此疾。先生你再仔細診一診。」趙先生又沉吟了半晌道:「如此面色這等黃,多管是脾虛泄瀉,再不然定是經水不調。」西門慶道:「寔說與先生,房下如此這般,下邊月水淋漓不止,所以身上都瘦弱了。有甚急方妙藥,我重重謝你。」趙先生道:「如何?我就說是經水不調。不打緊處,小人有藥。」{\meipi{趁口敲來,近日醫人多得此活。}}西門慶一面同他來到前廳,喬大戶、何老人問他甚麼病源,趙先生道:「依小人講,只是經水淋漓。」何老人道:「當用何藥治之?」趙先生道:「我有一妙方,用着這幾味藥材,吃下去管情就好。聽我說:

\begin{myquote}
甘草甘遂與碙砂,黎蘆巴荳與芫花,薑汁調着生半夏,用烏頭杏仁天麻。這幾味兒齊加,蔥蜜和丸只一撾,清晨用燒酒送下。」
\end{myquote}

何老人聽了,便道:「這等藥恐怕太狠毒,吃不得。」趙先生道:「自古毒藥苦口利於病。怎麼吃不得?」西門慶見他滿口胡說,因是韓夥計舉保來,不好囂他,稱二錢銀子,也不送,就打發他去了。因向喬大戶說:「此人原來不知甚麼。」何老人道:「老拙適纔不敢說,此人東門外有名的趙搗鬼,專一在街上賣杖搖鈴,哄過往之人,他那裡曉的甚脈息病源!」因說:「老夫人此疾,老拙到家撮兩帖藥來,遇緣,若服畢經水少減,胸口稍開,就好用藥。只怕下邊不止,就難為矣。」說畢,起身。

西門慶封白金一兩,使玳安拿盒兒討將藥來,晚夕與李瓶兒吃了,並不見分毫動靜。吳月娘道:「你也省可與他藥吃。他飲食先阻住了,肚腹中有甚麼兒,只是拿藥淘碌他。{\meipi{此言若出自金蓮,吾便疑為妒心下石,出自月娘,當是聖人之心。}}前者,那吳神仙算他三九上有血光之災,今年卻不整二十七歲了。你還使人尋這吳神仙去,叫替他打算算那祿馬數上如何。只怕犯着甚麼星辰,替他禳保禳保。」西門慶聽了,旋差人拿帖兒往周守備府裡問去。那裡回說:「吳神仙雲遊之人,來去不定。但來,只在城南土地廟下。今歲從四月裡,往武當山去了。要打數算命,眞武廟外有箇黃先生打的好數,一數只要三錢銀子,不上人家門。」西門慶隨即使陳敬濟拿三錢銀子,逕到北邊眞武廟門首黃先生家。{\meipi{寫出匆忙混亂,一毫不能主持,當局人自是如是。}}門上貼着:「抄算先天易數,每命卦金三錢。」陳敬濟向前作揖,奉上卦金,說道:「有一命煩先生推算。」寫與他八字:女命,年二十七歲,正月十五日午時。這黃先生把運算元一打,就說:「這箇命,辛未年庚寅月辛卯日甲午時,理取印綏之格,借四歲行運。四歲己未,十四歲戊午,二十四歲丁巳,三十四歲丙辰。今年流年丁酉,比肩用事,歲傷日干,計都星照命,又犯䘮門五鬼,災殺作炒。夫計都者,陰晦之星也。其象猶如亂絲而無頭,變異無常。大運逢之,多主闇昧之事,引惹疾病,主正、二、三、七、九月病災有損,小口兇殃,小人所算,口舌是非,主失財物。或是陰人,大為不利。」{\meipi{如此術家,並不多得。}}抄畢數,敬濟拿來家。西門慶正和應伯爵、溫秀才坐的,見抄了數來,拿到後邊,解說與月娘聽。見命中多兇少吉,不覺:

\begin{myquote}
眉間搭上三黃鎖,腹內包藏一肚愁。
\end{myquote}

