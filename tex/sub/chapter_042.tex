\includepdf[pages={83,84},fitpaper=false]{tst.pdf}
\chapter*{第四十二回 逞豪華門前放烟火 賞元宵樓上醉花燈}
\addcontentsline{toc}{chapter}{第四十二回 逞豪華門前放烟火 賞元宵樓上醉花燈}
\markboth{{\titlename}卷之五}{第四十二回 逞豪華門前放烟火 賞元宵樓上醉花燈}


詩曰:

\begin{myquote}
星月當空萬燭燒,人間天上兩元宵。\\樂和春奏聲偏好,人蹈衣歸馬亦嬌。\\易老韶光休浪度,最公白髮不相饒。\\千金博得斯須刻,分付譙更仔細敲。
\end{myquote}

話說西門慶打發喬家去了,走來上房,和月娘、大妗子、李瓶兒商議。月娘道:「他家既先來與咱孩子送節,咱少不得也買禮過去,與他家長姐送節。就權為插定一般,庶不差了禮數。」大妗子道:「咱這裡,少不的立上箇媒人,往來方便些。」月娘道:「他家是孔嫂兒,咱家安上誰好?」西門慶道:「一客不煩二主,就安上老馮罷。」于是,連忙寫了請帖八箇,就叫了老馮來,同玳安拿請帖盒兒,十五日請喬老親家母、喬五太太並尚舉人娘子、朱序班娘子、崔親家母、段大姐、鄭三姐來赴席,與李瓶兒做生日,並吃看燈酒。一面分付來興兒,拿銀子早定下蒸酥點心並羹菓食物。又是兩套遍地錦羅段衣服,一件大紅小袍兒、一頂金絲縐紗冠兒、兩盞雲南羊角珠燈、一盒衣翠、一對小金手鐲、四箇金寶石戒指兒。十四日早裝盒担,叫女婿陳敬濟和賁四穿青衣服押送過去。喬大戶那邊,酒筵管待,重加答賀。回盒中,又回了許多生活鞋脚,俱不必細說。正亂着,應伯爵來講李智、黃四官銀子事,看見,問其所以。西門慶告訴與喬大戶結親之事:「十五日好歹請令正來陪親家坐坐。」伯爵道:「嫂子呼喚,房下必定來。」西門慶道:「今日請衆堂官娘子吃酒,咱每往獅子街房子內看燈去罷。」伯爵應諾去了,不題。

且說那日,院中吳銀兒先送了四盒禮來,又是兩方銷金汗巾,一雙女鞋,送與李瓶兒上壽,就拜乾女兒。月娘收了禮物,打發轎子回去。李桂姐只到次日纔來,見吳銀兒在這裡,便悄悄問月娘:「他多咱來的?」月娘如此這般告他說:「昨日送了禮來,拜認你六娘做乾女兒了。」李桂姐聽了,一聲兒沒言語。一日只和吳銀兒使性子,兩箇不說話。{\meipi{無謂,可笑。}}卻說前廳王皇親家二十名小厮,兩箇師父領着,挑了箱子來,先與西門慶磕頭。西門慶分付西廂房做戲房,管待酒飯。不一時,周守備娘子、荊都監母親荊太太與張團練娘子,都先到了。俱是大轎,排軍喝道,家人媳婦跟隨。月娘與衆姊妹,都穿着袍出來迎接,至後廳叙禮。與衆親相見畢,讓坐遞茶,等着夏提刑娘子到纔擺茶。不料等到日中,還不見來。小厮邀了兩三遍,約午後纔喝了道來,擡着衣匣,家人媳婦跟隨,許多僕從擁護。鼓樂接進後廳,與衆堂客見畢禮數,依次序坐下。先在捲棚內擺茶,然後大廳上坐。春梅、玉簫、迎春、蘭香,都是齊整粧束,席上捧茶斟酒。那日扮的是《西廂記》。不說畫堂深處,珠圍翠遶,歌舞吹彈飲酒。

單表西門慶打發堂客上了茶,就騎馬,約下應伯爵、謝希大,往獅子街房裡去了。分付四架烟火,拿一架那裡去。晚夕,堂客跟前放兩架。旋叫了箇廚子,家下擡了兩食盒下飯菜蔬,兩罈金華酒去。又叫了兩箇唱的——董嬌兒、韓玉釧兒。原來西門慶已先使玳安顧轎子,請王六兒同往獅子街房裡去。玳安見婦人道:「爹說請韓大嬸,那裡晚夕看放烟火。」婦人笑道:「我羞剌剌,怎麼好去的,你韓大叔知道不嗔?」玳安道:「爹對韓大叔說了,教你老人家快收拾哩。因叫了兩箇唱的,沒人陪他。」那婦人聽了,還不動身。一回,只見韓道國來家。玳安道:「這不是韓大叔來了。韓大嬸這裡,不信我說哩。」婦人向他漢子說,「眞箇叫我去?」韓道國道:「老爹再三說,兩箇唱的沒人陪他,請你過去,晚夕就看放烟火。你還不收拾哩!剛纔教我把鋪子也收了,就晚夕一搭兒裡坐坐。保官兒也往家去了,晚夕該他上宿哩。」婦人道:「不知多咱纔散,你到那裡坐回就來罷,家裡沒人,你又不該上宿。」說畢,打扮穿了衣服,玳安跟隨,逕到獅子街房裡來。來昭妻一丈青早在房裡收拾下床炕、帳幔、褥被,安息沉香薰的噴鼻香。{\meipi{請客來看烟火,卻收拾床鋪,妙甚。}}房裡弔着一對紗燈,籠着一盆炭火。婦人走到裡面炕上坐下。一丈青走出來,道了萬福,拿茶吃了。

西門慶與應伯爵看了回燈,纔到房子裡。兩箇在樓上打雙陸。樓上除了六扇窻戶,掛着簾子,下邊就是燈市,十分鬧熱。打了回雙陸,收拾擺飯吃了,二人在簾裡觀看燈市。但見:

\begin{myquote}
萬井人烟錦綉圍,香車寶馬鬧如雷。\\鰲山聳出青雲上,何處遊人不看來?
\end{myquote}

二人看了一回,西門慶忽見人叢裡謝希大、祝實念,同一箇戴方巾的在燈棚下看燈,指與伯爵瞧。因問:「那戴方巾的,你可認的他?」伯爵道:「此人眼熟,不認的他。」西門慶便叫玳安:「你去下邊,悄悄請了謝爹來。休教祝麻子和那人看見。」玳安小厮賊,一直走下樓來,捱到人鬧裡,待祝實念和那人先過去了,從旁邊出來,把謝希大拉了一把。慌的希大回身觀看,卻是玳安。玳安道:「爹和應二爹在這樓上,請謝爹說話。」希大道:「你去,我知道了。等我陪他兩箇到粘梅花處,就來見你爹。」玳安便一道烟去了。希大到了粘梅花處,向人鬧處,就叉過一邊,繇着祝實念和那一箇人只顧尋。他便走來樓上,{\meipi{不過一杯酒,無大利害,便東藏西躲,半路拋人,寫出交情之薄。}}見西門慶、應伯爵兩箇作揖,因說道:「哥來此看燈,早晨就不呼喚兄弟一聲?」西門慶道:「我早晨對衆人,不好邀你每的。已托應二哥到你家請你去,說你不在家。剛纔,祝麻子沒看見麼?」因問:「那戴方巾的是誰?」希大道:「那戴方巾的,是王昭宣府裡王三官兒。今日和祝麻子到我家,要問許不與先生那裡借三百兩銀子。央我和老孫、祝麻子作保。要幹前程,入武學肄業。我那裡管他這閑帳!剛纔陪他燈市裡走了走,聽見哥呼喚,我只伴他到粘梅花處,交我乘人亂,就叉開了走來見哥。」因問伯爵:「你來多大回了?」伯爵道:「哥使我先到你家,你不在,我就來了,和哥在這裡打了這回雙陸。」西門慶問道:「你吃了飯不曾?」謝希大道:「早晨從哥那裡出來,和他兩箇搭了這一日,誰吃飯來!」西門慶分付玳安:「廚下安排飯來,與你謝爹吃。」不一時,就是春盤小菜、兩碗稀爛下飯、一碗𤆑肉粉湯、兩碗白米飯。希大獨自一箇,吃的裡外乾淨,剩下些汁湯兒,還泡了碗吃了。玳安收下家活去。希大在旁看着兩箇打雙陸。

只見兩箇唱的門首下了轎子,擡轎的提着衣裳包兒,笑進來。伯爵在窻裡看見,說道:「兩箇小淫婦兒,這咱纔來。」分付玳安:「且別教他往後邊去,先叫他樓上來見我。」希大道:「今日叫的是那兩箇?」玳安道:「是董嬌兒、韓玉釧兒。」忙下樓說道:「應二爹叫你說話。」兩箇那裡肯來,一直往後走了。見了一丈青,拜了,引他入房中。看見王六兒頭上戴着時樣扭心鬏髻兒,身上穿紫潞紬襖兒,玄色披襖兒、白挑線絹裙子,下邊露兩隻金蓮,拖的水鬢長長的,紫膛色,不十分搽鉛粉,學箇中人打扮,耳邊帶着丁香兒。進門只望着他拜了一拜,都在炕邊頭坐了。小鐵棍拿茶來,王六兒陪着吃了。兩箇唱的,上上下下把眼只看他身上。看一回,兩箇笑一回,更不知是什麼人。落後,玳安進來,兩箇悄悄問他道:「房裡那一位是誰?」玳安沒的回答,只說是:「俺爹大姨人家,接來看燈的。」兩箇聽的,從新到房中說道:「俺每頭裡不知是大姨,沒曾見的禮,休怪。」于是插燭磕了兩箇頭。慌的王六兒連忙還下半禮。落後,擺上湯飯來,陪着同吃。兩箇拿樂器,又唱與王六兒聽。伯爵打了雙陸,下樓來小解淨手,聽見後邊唱,點手兒叫玳安,{\pangpi{畫。}}問道:「你告我說,兩箇唱的在後邊唱與誰聽?」玳安只是笑,不做聲,說道:「你老人家『曹州兵備——管事寬』。唱不唱,管他怎的?」伯爵道:「好賊小油嘴,你不說,愁我不知道?」玳安笑道:「你老人家知道罷了,又問怎的?」說畢,一直往後走了。{\pangpi{賊。}}伯爵上的樓來,西門慶又與謝希大打了三貼雙陸。只見李銘、吳惠兩箇驀地上樓來磕頭。伯爵道:「好呀!你兩箇來的正好,怎知道俺每在這裡?」李銘跪下說道:「小的和吳惠先到宅裡來,宅裡說爹在這邊擺酒。特來伏侍爹每。」西門慶道:「也罷,你起來伺候。玳安,快往對門請你韓大叔去。」不一時,韓道國到了,作了揖,坐下。一面放桌兒,擺上春盤案酒來,琴童在旁邊篩酒。伯爵與希大居上,西門慶主位,韓道國打橫,坐下把酒來篩;一面使玳安後邊請唱的去。

少頃,韓玉釧兒、董嬌兒兩箇,慢條斯禮上樓來。望上不當不正磕下頭去。伯爵罵道:「我道是誰來,原來是這兩箇小淫婦兒。頭裡我叫着,怎的不先來見我?這等大膽!到明日,不與你箇功德,你也不怕。」董嬌兒笑道:「哥兒,那裡隔墻掠箇鬼臉兒,可不把我諕殺!」韓玉釧兒道:「你知道,『愛奴兒掇着獸頭城往裡掠——好箇丟醜兒的孩兒』!」伯爵道:「哥,你今日忒多餘了。有了李銘、吳惠在這裡唱罷了,又要這兩箇小淫婦做什麼?還不趁早打發他去。大節夜,還趕幾箇錢兒,等住回晚了,越發沒人要了。」韓玉釧兒道:「哥兒,你怎麼沒羞?大爹叫了俺每來答應,又不伏侍你,你怎的閑出氣?」伯爵道:「傻小𢱉剌骨兒,你見在這裡,不伏侍我,你說伏侍誰?」韓玉釧道:「『唐胖子弔在醋缸裡——把你撅酸了』。」伯爵道:「賊小淫婦兒,是撅酸了我。等住回散了家去時,我和你答話。我左右有兩箇法兒,你原出得我手!」董嬌兒問道:「哥兒,那兩箇法兒?說來我聽。」伯爵道:「我頭一箇,是對巡捕說了,拿你犯夜,教他拿了去,拶你一頓好拶子。十分不巧,只消三分銀子燒酒,把擡轎的灌醉了,隨你這小淫婦兒去,天晚到家沒錢,不怕鴇子不打。」韓玉釧道:「十分晚了,俺每不去,在爹這房子裡睡。再不,叫爹差人送俺每,王媽媽支錢一百文,不在於你。好淡嘴女又十撇兒。」{\pangpi{好罵。}}伯爵道:「我是奴才,如今年程反了,拿三道三。」說笑回,兩箇唱的在旁彈唱春景之詞。

衆人纔拿起湯飯來吃,只見玳安兒走來,報道:「祝爹來了。」{\pangpi{偏又來尋來。}}衆人都不言語。{\pangpi{傳神。}}不一時,祝實念上的樓來,看見伯爵和謝希大在上面,說道:「你兩箇好吃,可成箇人。」因說:「謝子純,哥這裡請你,也對我說一聲兒,三不知就走的來了,叫我只顧在粘梅花處尋你。」希大道:「我也是誤行,纔撞見哥在樓上和應二哥打雙陸。走上來作揖,被哥留住了。」西門慶因令玳安兒:「拿椅兒來,我和祝兄弟在下邊坐罷。」于是安放鍾筯,在下席坐了。廚下拿了湯飯上來,一齊同吃。西門慶只吃了一箇包兒,呷了一口湯,因見李銘在旁,都遞與李銘下去吃了。那應伯爵、謝希大、祝實念、韓道國,每人吃一大深碗八寶攢湯,三箇大包子,還零四箇桃花燒賣,只留了一箇包兒壓碟兒。左右收下湯碗去,斟上酒來飲酒。希大因問祝實念道:「你陪他到那裡纔拆開了?怎知道我在這裡?」祝實念如此這般告說:「我因尋了你一回尋不着,就同王三官到老孫家會了,往許不與先生{\pangpi{放債美名。}}那裡,借三百兩銀子去,吃孫寡嘴老油嘴把借契寫差了。」希大道:「你每休寫上我,我不管。左右是你與老孫作保,討保頭錢使。」因問:「怎的寫差了?」祝實念道:「我那等分付他,文書寫滑着些,立與他三限纔還。他不依我,教我從新把文書又改了。」希大道:「你立的是那三限?」祝實念道:「頭一限,風吹轆軸打孤雁;第二限,水底魚兒跳上岸;第三限,水裡石頭泡得爛。這三限交還他。」謝希大道:「你這等寫着,還說不滑哩。」祝實念道:「你到說的好,儻或一朝天旱水淺,朝廷挑河,把石頭吃做工的兩三鐝頭砍得稀爛,怎了?那時少不的還他銀子。」{\meipi{今人借銀子只約在明日、後日,偏能不還,比此更妙。}}衆人說笑了一回,看看天晚,西門慶分付樓上點燈,又樓簷前一邊一盞羊角玲燈,甚是奇巧。家中,月娘又使棋童兒和排軍,擡送了四箇攢盒,都是美口糖食、細巧菓品。西門慶叫棋童兒問道:「家中衆奶奶們散了不曾?誰使你送來?」棋童道:「大娘使小的來,與爹這邊下酒。衆奶奶們還未散哩。戲文扮了四折,大娘留在大門首吃酒,看放烟火哩。」西門慶問:「有人看沒有?」棋道:「擠圍着滿街人看。「西門慶道:「我分付留下四名青衣排軍,拿杆欄攔人伺候,休放閑雜人挨擠。」棋童道:「小的與平安兒兩箇,同排軍都看放了烟火,並沒閑雜人攪擾。」西門慶聽了,分付把桌上飲饌都搬下去,將攢盒擺上,廚下又拿上一道菓餡元宵來。兩箇唱的在席前遞酒。西門慶分付棋童回家看去。一面重篩美酒,再設珍羞,叫李銘、吳惠席前彈唱了一套燈詞。唱畢,吃了元宵,韓道國先往家去了。{\pangpi{知局。}}少頃,西門慶分付來昭將樓下開下兩間,吊掛上簾子,把烟火架擡出去。西門慶與衆人在樓上看,教王六兒陪兩箇粉頭和一丈青在樓下觀看。玳安和來昭將烟火安放在街心裡。須臾,點着。那兩邊圍看的,挨肩擦膀,不知其數。都說西門大官府在此放烟火,誰人不來觀看?果然紮得停當好烟火。但見:

\begin{myquote}
一丈五高花樁,四周下山棚熱鬧。最高處一隻仙鶴,口裡啣着一封丹書,乃是一枝起火,一道寒光,直鑽透斗牛邊。然後,正當中一箇西瓜砲迸開,四下裡人物皆着,觱剝剝萬箇轟雷皆燎徹。彩蓮舫,賽月明,一箇趕一箇,猶如金燈冲散碧天星;紫葡萄,萬架千株,好似驪珠倒掛水晶簾。霸王鞭,到處响喨;地老鼠,串遶人衣。瓊盞玉臺,端的旋轉得好看;銀蛾金彈,施逞巧妙難移。八仙捧壽,名顯中通;七聖降妖,通身是火。黃烟兒,綠烟兒,氤氳籠罩萬堆霞;緊吐蓮,慢吐蓮,燦爛爭開十段錦。一丈菊與烟蘭相對,火梨花共落地桃爭春。樓臺殿閣,頃刻不見巍峨之勢;村坊社鼓,彷彿難聞歡鬧之聲。貨郎担兒,上下光焰齊明;鮑老車兒,首尾迸得粉碎。五鬼鬧判,焦頭爛額見猙獰;十面埋伏,馬到人馳無勝負。總然費卻萬般心,只落得火滅烟消成煨燼。
\end{myquote}

應伯爵見西門慶有酒了,剛看罷烟火下樓來,因見王六兒在這裡,推小淨手,拉着謝希大、祝實念,也不辭西門慶就走了。玳安便道:「二爹那裡去?」伯爵向他耳邊說道:「傻孩子,我頭裡說的那本帳,我若不起身,別人也只顧坐着,顯的就不趣了。等你爹問,你只說俺每都跑了。」落後,西門慶見烟火放了,問伯爵等那裡去了,玳安道:「應二爹和謝爹都一路去了。小的攔不回來,多上覆爹。」西門慶就不再問了。因叫過李銘、吳惠來,每人賞了一大巨杯酒與他吃。分付:「我且不與你唱錢,你兩箇到十六日早來答應。還是應二爹三箇並衆夥計當家兒,晚夕在門首吃酒。」李銘跪下道:「小的告稟爹:十六日和吳惠、左順、鄭奉三箇,都往東平府,新陞的胡爺那裡到任,官身去,只到後晌纔得來。」西門慶道:「左右俺每晚夕纔吃酒哩。你只休誤了就是了。」二人道:「小的並不敢誤。」兩箇唱的也就來拜辭出門。西門慶分付:「明日,家中堂客擺酒,李桂姐、吳銀姐都在這裡,你兩箇好歹來走一走。」二人應諾了,一同出門,不在話下。西門慶分付來昭、玳安、琴童收家活。滅息了燈燭,就往後邊房裡去了。

且說來昭兒子小鐵棍兒,正在外邊看放了烟火,見西門慶進去了,就來樓上。見他爹老子收了一盤子雜合的肉菜、一甌子酒和些元宵,拿到屋裡,就問他娘一丈青討,被他娘打了兩下。不防他走在後邊院子裏頑耍,只聽正面房子裡笑聲,只說唱的還沒去哩,見房門關着,就在門縫裡張看,見房裡掌着燈燭。原來西門慶和王六兒兩箇,在床沿子上行房。西門慶已有酒的人,把老婆倒按在床沿上,褪去小衣,那話上使着托子幹後庭花。一進一退往來𢵞打,何止數百回,𢵞打的連聲响喨,其喘息之聲,往來之勢,猶賽折床一般,無處不聽見。這小孩子正在那裡張看,不防他娘一丈青走來看見,揪着頭角兒拖到前邊,鑿了兩箇栗爆,罵道:「賊禍根子,小奴才兒,你還少第二遭死?又往那裡張他去!」于是,與了他幾箇元宵吃了,不放他出來,就諕住他上炕睡了。西門慶和老婆足幹搗有兩頓飯時纔了事。玳安打發擡轎的酒飯吃了,跟送他到家,然後纔來同琴童兩箇打着燈兒跟西門慶家去。正是:

\begin{myquote}
不愁明月盡,自有夜珠來。
\end{myquote}

